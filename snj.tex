%%%%%%%%%%%%%%%%%%%%%%%%%%%%%%%%%%%%%%%%%%%%%%%%%
%                                               %
%          =====  Root Program  =====           %
%                                               %
%===============================================%
%====================  for  ====================%
%===============================================%
%                                               %
%            非対称作用素のスペクトル           %
%                                               %
%               Ichiro SHIGEKAWA                %
%                                               %
%===============================================%
%====================== in =====================%
%===============================================%
%                                               %
%                    日本語版                   %
%                                               %
%%%%%%%%%%%%%%%%%%%%%%%%%%%%%%%%%%%%%%%%%%%%%%%%%
%
% This file is to be run with the LaTeX-2e Version *
%
%\batchmode
\documentclass[12pt,leqno,a4paper]{jarticle}
% Information can be obtained from
% C:\ptex\texm\tex\latex\amslates\inputs\amsart.cls
% (c:/ptex/texmf/tex/latex/base/article.cls
\usepackage[utf8]{inputenc}
\usepackage[ipaex]{pxchfon}
\usepackage{amsmath,amsopn,amscd,amssymb,amsthm,verbatim,ascmac,latexsym,url}
%  amsmath.sty	古くは amstex.sty であったが、
%               こちらは互換性のみで凍結されている。
%  ascmac.sty	これは ASKII  独自のマクロなので一般的ではない。
%		個人使用に限るべし。  \itembox が使える利点がある。
% amsthm に proof 環境がある。  amsart, amsbook には 同様な pf 環境がある。
% ただし Proof. と後に period が来る。\begin{proof}[証明] とすれば、
% 日本語に出来るが、period が残る。尚証明は後で定義しなおしてあるので参照
%
%\usepackage[usenames]{color} % PSTrick を使うときは color 
%\usepackage[colorinlistoftodos]{todonotes}
\usepackage[colorlinks=true, allcolors=blue]{hyperref}
%%%%%%%%%%%%%%%%%%%%   Draft と Final のきりかえ %%%%%%%%%%%%%%%%%%%%%%%%%%
\newcount\draftversion      %\draftversion=1
\ifodd\draftversion \relax \else \usepackage{showkeys} \fi
% 最後の仕上げの段階で、1 にする。通常は percent out して default の 0。
%
%%%%%%%%%%%%%%%%%%%%   PDF と PS のきりかえ  グラフィックの設定 %%%%%%%%%%%%
%
% underscore _ を使うために url package で \path{} を使っている.
% \verb!#2! の形をマクロの中に入れるとうまく機能しない.
%
\newcount\pdfflag    \pdfflag=1
\ifodd\pdfflag % PDF の場合 \pdfflag=1 のとき
\def\includePdfEps#1#2{\includegraphics#1{eps/#2.eps}
	\ifodd\draftversion \relax
	\else \par File name: \path{#2.eps}
	\fi
}
\usepackage[dvipdfmx]{graphicx} %\usepackage{pst-all}
%\usepackage[usenames]{color}

\else % EPS の場合 \pdfflag=0 のとき
%
\def\includePdfEps#1#2{\includegraphics#1{#2.eps}
	\ifodd\draftversion \relax
	\else \par File name: \path{#2.eps}
	\fi
}
\usepackage[dvips]{graphicx}

%\usepackage{pst-all}

%\usepackage{color}
%\newpsobject{showgrid}{psgrid}{subgriddiv=1,griddots=10,gridlabels=6pt}
% PSTrick の中で \colorが使える. 
%\usepackage[dvips,usenames]{color} % PSTrick を使うときはcolor は使えない.
\fi
% PDF にするとき 1(すなわち TRUE) にする.
% 通常は EPS なのでパーセントアウトして default の 0 (すなわち FALSE).

%%%%%%%%%%%%%%%%%%%%   Packages  %%%%%%%%%%%%%%%%%%%%%%%%%%%%%%%%%%%%%%%%%%%
\usepackage{fancybox}
% \usepackage{boxedminipage}
% \usepackage{nruby}	%  ルビをつける : \ruby{漢字}{よみ}
\usepackage{ulem}	%  波線、取り消し線をつける。 \uwave{}、 \sout{} 
\usepackage{time}	%  時刻の挿入   \now 
%\usepackage{cancel}	%  斜線で cancel を入れる. \cancel{},\bcancel{}... 
%%%%%%%%%%%%%%%%%%%%   スクリプトのフォント   %%%%%%%%%%%%%%%%%%%%%%%%%%%%%%
% \def\mathscr#1{\mathcal{#1}}
%\usepackage{rsfs}  % Now you can use \mathscr in PC
\usepackage{mathrsfs}  % Now you can use \mathscr in UNIX
\renewcommand{\cal}{\mathscr}  % mathscr が使えない場合がある。
                                % そのためにすぐ \cal に戻せるように
                                % しておいた方がいいのだ。
% \def\cal{\mathcal}
%%%%%%%%%%%%%%%%%%%%      グラフィックス      %%%%%%%%%%%%%%%%%%%%%%%%%%%%%%
%
% \usepackage{righttag,array}
%\usepackage[dvips]{graphicx}	\usepackage{pst-all}
%\newpsobject{showgrid}{psgrid}{subgriddiv=1,griddots=10,gridlabels=6pt}
%%%%%%%%%%%%%%%%%%%%      floting objects     %%%%%%%%%%%%%%%%%%%%%%%%%%%%%%
%
%\usepackage{floatfig}
\usepackage{wrapfig}
% 図の左側にテキストを流し込む。
%
%%%%%%%%%%%%%%%%%%%%   連番 コメント (参考用で後で消す)   %%%%%%%%%%%%%%%%%%
\ifodd\draftversion \long\def\hide#1\endhide{\relax} \def\endhide{}
\else
\newcounter{commentno}
\long\def\hide#1\endhide{\stepcounter{commentno} \begin{center}\begin{itembox}[l]{コメント: No.~\arabic{commentno}} \parindent=1zw \small #1 \end{itembox} \end{center}}   \def\endhide{}
\fi
%\def\hide#1\endhide{\stepcounter{commentno} \begin{center} \begin{boxedminipage}[t]{\textwidth}{コメント: No.~\arabic{commentno}} \\ \noindent \small #1 \end{boxedminipage} \end{center}}   \def\endhide{}
%\def\hide#1\endhide{\relax} \def\endhide{}
%%%%%%%%%%%%%%%%%%%%   メモ書き (参考用で後で消す)   %%%%%%%%%%%%%%%%%%%%%%%
%
\ifodd\draftversion \newcommand{\memo}[1]{\relax}
\else
\newcommand{\memo}[1]{\ifmmode \quad\mbox{{\bf [}MEMO: #1 \ {\bf ]}} \else {\bf [}MEMO: #1 \ {\bf ]}\fi}
\fi



% 科研費の補助金の英語表記
% 科学研究費補助金 Grant-in-Aid for Scientific Research (略称「KAKENHI」)
% 特別推進研究 Grant-in-Aid for Specially Promoted Research
% 特定領域研究 Grant-in-Aid for Scientific Research on Priority Areas
% 新学術領域研究 Grant-in-Aid for Scientific Research on Innovative Areas
% 基盤研究(S),(A),(B),(C) Grant-in-Aid for Scientific Research (S)or (A)or(B)or(C)
% 萌芽研究 Grant-in-Aid for Exploratory Research
% 挑戦的萌芽研究 Grant-in-Aid for Challenging Exploratory Research
% 若手研究(S),(A),(B) Grant-in-Aid for Young Scientists (S)or(A)or(B)
% 若手研究(スタートアップ) Grant-in-Aid for Young Scientists (Start-up)
% 研究活動スタート支援 Grant-in-Aid for Research Activity Start-up
% 特別研究促進費 Grant-in-Aid for Special Purposes
% 研究成果公開促進費 Grant-in-Aid for Publication of Scientific Research Results
% 特別研究員奨励費 Grant-in-Aid for JSPS Fellows
% 学術創成研究費 Grant-in-Aid for Creative Scientific Research
% ※ 文部科学省  The Ministry of Education,Culture,Sports,Science and Technology (MEXT)
% ※ 独立行政法人日本学術振興会 Japan Society for the Promotion of Science(JSPS)

%%%%%%%%%%%%%%%%%%%%   参考資料 (参考用でdraft でのみ残す)   %%%%%%%%%%%%%%%
%
\ifodd\draftversion \long\def\sankou#1\endsankou{\relax} \def\endsankou{}
\else
\long\def\sankou#1\endsankou{%
%
%\renewcommand{\thepage}{\arabic{page} \rput[b]{45}(2,5){\color[rgb]{0.85,0.85,0.85}{\rotatebox{50}{\scalebox{8}{Draft}}}} }
%\def\thepage{\rput[b]{45}(2,5){Draft}  \arabic{page} }
%
\centerline{\fbox{ここから後は参考用にのみ残してある}} \medskip \par
%\pagecolor{gray}
%\color[rgb]{0.5,0.5,0.5}
\color[rgb]{0.2,0.2,1}
%\color{gray}
\parindent=1zw %\small
#1
\medskip \par
%\pagecolor{white}
\color{Black}
\centerline{\fbox{これ以前は参考用にのみ残してある}} }
\def\endhide{}
\fi



%%%%%%%%%%%%%%%%%%%%   全体にわたる Box   %%%%%%%%%%%%%%%%%%%%%%%%%%%%%%%%%%
\newlength{\IchiroLength}
\long\def\widebox#1\endwidebox{ \setlength{\fboxsep}{4pt} \setlength{\IchiroLength}{\linewidth} \addtolength{\IchiroLength}{-2\fboxsep} \addtolength{\IchiroLength}{-2\fboxrule} \begin{center} \framebox[\linewidth]{\parbox{\IchiroLength}{\setlength{\abovedisplayskip}{-0pt} \setlength{\belowdisplayskip}{4pt}#1}}\unskip \end{center} \setlength{\fboxsep}{2pt}} \def\endwidebox{}
%\def\widebox#1\endwidebox{#1} \def\endwidebox{}

%======================= Beginning of TeX ==================================
\usepackage{secdot}  % 節の番号の跡にピリオドを打つ
%
% (c:/ptex/texmf/tex/latex/base/article.cls から変更
\makeatletter
\renewcommand{\section}{\@startsection{section}{1}{\z@}%
%   {1.5\Cvs \@plus.5\Cdp \@minus.2\Cdp}%
   {1\Cvs \@plus.5\Cdp \@minus.2\Cdp}%
   {.5\Cvs \@plus.3\Cdp}%
   {\reset@font\large\bfseries}}
%  \large は元は \Large だった 
\renewcommand{\subsection}{\@startsection{subsection}{2}{\z@}%
%   {1.5\Cvs \@plus.5\Cdp \@minus.2\Cdp}%
   {.7\Cvs \@plus.5\Cdp \@minus.2\Cdp}%
%   {.5\Cvs \@plus.3\Cdp}%
   {.3\Cvs \@plus.3\Cdp}%
%   {\reset@font\large\bfseries}}
%   {\reset@font\normalsize\bfseries\noindent\underline}}
   {\reset@font\normalsize\bfseries\boldmath\noindent\underline}}
%  \normalsize は元は \large だった 
\makeatother
%\newcommand{\SS}[2]{\section{#2}\setcounter{equation}{0}\label{Sec.#1}}
\renewcommand{\SS}[2]{\section{#2}\label{Sec.#1}}
\newcommand{\StartNewSection}{
\ifodd\draftversion \relax \else \cleardoublepage \fi}
\newcommand{\Sec}[1]{第 \ref{Sec.#1}節}
%\newcommand{\subsec}[1]{\subsection{\boldmath #1}}

\newcommand{\subsec}[1]{\subsection{#1}}
%\newcommand{\subsec}[1]{\underline{\subsection{#1}}}
%
%======================= Setting of Length =================================
\setlength{\topmargin}{0cm}
%\setlength{\headheight}{0pt}		\setlength{\headsep}{0pt}
\setlength{\oddsidemargin}{-.2cm}	\setlength{\evensidemargin}{-.2cm}
\setlength{\textheight}{23cm}		\setlength{\textwidth}{16.5cm}
%\setlength{\textheight}{21cm}		\setlength{\textwidth}{14.5cm}
\hfuzz=.00001pt			
\ifodd\draftversion \overfullrule=0pt
\else \overfullrule=8pt %(plain TeX command)
\fi
%\renewcommand{\baselinestretch}{1.2}
\raggedbottom
%
\parindent=1zw
%\jintercharskip=.5pt plus .3pt minus .4pt
%
%=======================  Setting of Header, Footer  =======================
%\pagestyle{plain}
%\pagestyle{headings}
%\markboth{\hfill Weitzenbock formula \hfill \thepage}%
%{\thepage\hfill T. Kazumi and I. Shigekawa \hfill}
%
%=======================  Setting of Counters  =============================
\setcounter{secnumdepth}{1}
\numberwithin{equation}{section}
%======================= 定理・証明の設定 ==================================
%
\makeatletter     % 以下は amsthm.sty から持ってきた
\renewenvironment{proof}[1][\proofname]{\par\normalfont
 \topsep6\p@\@plus6\p@ \trivlist
%  \item[\hskip\labelsep\itshape
  \item[\hskip\labelsep
%    #1\@addpunct{.}]\ignorespaces
    #1 ]\ignorespaces}{\qed\endtrivlist}
\renewcommand{\proofname}{{\bf 証明}}
%\renewcommand{\qedsymbol}{\rule{3pt}{12pt}} % default は \openbox
\makeatother
\newcommand{\Proof}{\begin{proof}}		\newcommand{\QED}{\end{proof}}


\newenvironment{demonst}[1]{\begin{trivlist}%
\item[]{\bf #1}\ }{\end{trivlist}}
\newcommand{\qbd}[1]{\begin{demonst}{#1\ }}
\newcommand{\myqed}{\hfill\qedsymbol\end{demonst}}
%\newsavebox{\toybox}
%\savebox{\toy}{\framebox[0.65em]{\rule{0cm}{1ex}}}
%
\newenvironment{property}%
{\begin{list}{}{\setlength{\rightmargin}{0pt}%
\setlength{\itemsep}{0pt}}}{\end{list}}
\newlength{\templength}
\newcommand{\bp}{\setlength{\templength}{\labelwidth}%
\setlength{\labelwidth}{2em}\begin{property}}
\newcommand{\ep}{\end{property}\setlength{\labelwidth}{\templength}}
%
\newtheorem{theorem}{定理}[section]
\newtheorem{lemma}[theorem]{補題}
\newtheorem{proposition}[theorem]{命題}
\newtheorem{corollary}[theorem]{系}
%\newtheorem{assumption}{仮定}
\newtheorem{definition}[theorem]{定義}
\newtheorem{problem}{問題}[section]
\newtheorem{example}{例}[section]
\newtheorem{remark}{注意}[section]
%
\newcommand{\Thm}[1]{定理 \ref{Thm.#1}}
\newcommand{\Lem}[1]{補題 \ref{Lem.#1}}
\newcommand{\Cor}[1]{系 \ref{Cor.#1}}
\newcommand{\Prop}[1]{命題 \ref{Prop.#1}}
%\newcommand{\Ass}[1]{仮定 \ref{Ass.#1}}
\newcommand{\Def}[1]{定義 \ref{Def.#1}}
\newcommand{\Prob}[1]{問題 \ref{Prob.#1}}
\newcommand{\Exam}[1]{例 \ref{Exam.#1}}
\newcommand{\Rem}[1]{注意 \ref{Rem.#1}}
\newcommand{\Eq}[1]{(\ref{Eq.#1})}
%
\newcommand{\Theorem}[1]{\begin{theorem}\label{Thm.#1}\rm}
\newcommand{\Lemma}[1]{\begin{lemma}\label{Lem.#1}\rm}
\newcommand{\Proposition}[1]{\begin{proposition}\label{Prop.#1}\rm}
\newcommand{\Corollary}[1]{\begin{corollary}\label{Cor.#1}\rm}
%\newcommand{\Assumption}[1]{\begin{assumption}\label{Ass.#1}\rm}
\newcommand{\Definition}[1]{\begin{definition}\label{Def.#1}\rm}
%\newcommand{\Problem}[1]{\begin{problem}\label{Prob.#1}\rm}
\newcommand{\Problem}[1]{\begin{problem}\label{Prob.#1}\rm}
\newcommand{\Example}[1]{\begin{example}\label{Exam.#1}\rm}
\newcommand{\Remark}[1]{\begin{remark}\label{Rem.#1}\rm}
%
%=======================  Abbriviation of Greek letters  ===================
\newcommand{\al}{\alpha}
\newcommand{\be}{\beta}
\newcommand{\gm}{\gamma}
\newcommand{\Gm}{\Gamma}
\renewcommand{\d}{\delta}
% \newcommand{\D}{\Delta}
% \newcommand{\vD}{\varDelta}
\newcommand{\e}{\varepsilon}
% \newcommand{\ve}{\epsilon}
% \newcommand{\z}{\zeta}
\newcommand{\h}{\eta}
% \renewcommand{\th}{\theta} 	% 正体不明のコマンドとして定義済み	
\newcommand{\Th}{\Theta}
% \newcommand{\vTh}{\varTheta}
% \newcommand{\vth}{\vartheta}
\newcommand{\io}{\iota}
% \newcommand{\k}{\kappa}
% \newcommand{\vk}{\varkappa}
\newcommand{\lm}{\lambda}
\newcommand{\Lm}{\Lambda}
% \newcommand{\vLm}{\varLambda}
% \newcommand{\m}{\mu}
% \newcommand{\n}{\nu}
% \newcommand{\vX}{\varXi}
% \newcommand{\vPi}{\varPi}
% \renewcommand{\r}{\rho}		% アクセント記号上に小さな円として定義済み
% \newcommand{\vr}{\varrho}
\newcommand{\s}{\sigma}
% \newcommand{\S}{\Sigma}
% \newcommand{\vS}{\varSigma}
% \renewcommand{\t}{\tau}		% 上に丸いアクセント記号として定義済み
% \renewcommand{\u}{\upsilon}	% u 型のアクセント記号として定義済み
% \newcommand{\U}{\Upsilon}
% \newcommand{\vU}{\varUpsilon}
\newcommand{\ph}{\phi}
% \newcommand{\Ph}{\Phi}
% \newcommand{\vPh}{\varPhi}
\newcommand{\vph}{\varphi}
% \newcommand{\ch}{\chi}
% \newcommand{\ps}{\psi}
% \newcommand{\Ps}{\Psi}
% \newcommand{\vPs}{\varPsi}
\newcommand{\w}{\omega}
\newcommand{\W}{\Omega}
% \newcommand{\vW}{\varOmega}
%
%=======================  Abbriviation of Calligraphic letters  ============
\newcommand{\B}{{\cal B}}
%\newcommand{\D}{{\cal D}}
\newcommand{\F}{{\cal F}}
\newcommand{\sA}{{\cal A}}
\newcommand{\sB}{{\cal B}}
\newcommand{\sC}{{\cal C}}
\newcommand{\sD}{{\cal D}}
\newcommand{\sE}{{\cal E}} 
\newcommand{\sF}{{\cal F}}
\newcommand{\sH}{{\cal H}}
\newcommand{\sJ}{{\cal J}}
\newcommand{\sL}{{\cal L}}
\newcommand{\sP}{{\cal P}}
\newcommand{\sS}{{\cal S}}
\newcommand{\sV}{{\cal V}}
\newcommand{\sY}{{\cal Y}}
\newcommand{\sW}{{\cal W}}
%
%=======================  Abbriviation of German letters  ==================
\newcommand{\g}{{\mathfrak g}}
%\newcommand{\gg}{{\mathfrak g}}
\newcommand{\gG}{{\mathfrak G}}
\newcommand{\gS}{{\mathfrak S}}
%
%=======================  Abbriviation of Bold letters  ====================
\newcommand{\bX}{{\bold X}}
\newcommand{\bY}{{\bold Y}}
%\newcommand{\bH}{\mbox{\boldmath $H$}}
\newcommand{\bH}{{\bold H}}
\newcommand{\boldf}{{\bold f}}
\newcommand{\bh}{{\boldsymbol\eta}}

\newcommand{\bal}{{\boldsymbol \alpha}}
\newcommand{\bbe}{{\boldsymbol \beta}}
\newcommand{\bga}{{\boldsymbol \gamma}}
\newcommand{\bde}{{\boldsymbol \delta}}
%
%=======================  Common Macro Commands  ===========================
\newcommand{\1}{\sqrt{-1}}
\newcommand{\del}{\partial}
\newcommand{\delb}{\bar{\partial}}
\newcommand{\8}{\infty}
\newcommand{\ds}{\displaystyle}
\newcommand{\R}{{\mathbb R}}
\newcommand{\C}{{\mathbb C}}
\newcommand{\Z}{{\mathbb Z}}
\newcommand{\N}{{\mathbb N}}
\newcommand{\Q}{{\mathbb Q}}
\newcommand{\la}{\langle}
\newcommand{\ra}{\rangle}
\newcommand\tn{|\!\|}
%\newcommand{\injection}{\mathrel{\subset\kern-.82em\lower.28em\hbox{$\to$}}}
\newcommand{\injection}{\;{\subset\kern-.82em\lower.28em\hbox{$\to$}}\;}
\newcommand{\grad}{\operatorname{grad}\nolimits}
\renewcommand{\div}{\mathop{\rm div}\nolimits} % \div は本来 ÷ の記号
\newcommand{\Dom}{\operatorname{Dom}\nolimits}
\newcommand{\Ran}{\operatorname{Ran}\nolimits}
\newcommand{\Ker}{\operatorname{Ker}\nolimits}
\newcommand{\sgn}{\operatorname{sgn}\nolimits}
\newcommand{\supp}{\operatorname{supp}\nolimits}
\newcommand{\inflim}{\mathop{\underline{\operatorname{lim}}}}
\newcommand{\suplim}{\mathop{\overline{\operatorname{lim}}}}
\newcommand{\maru}{\raise.1em\hbox{\scriptsize $\circ$}}
\newcommand{\ol}[1]{\overline{#1}}
\newcommand{\interior}[1]{\operatorname{Dom}\nolimitsi(#1)}
\newcommand{\inte}{\operatorname{int}\nolimits}
\newcommand{\ext}{\operatorname{ext}\nolimits}
\newcommand{\plus}[1]{{#1}_{\raise.1em\hbox{$\scriptscriptstyle \,+$}}}
\newcommand{\minus}[1]{{#1}_{\raise.1em\hbox{$\scriptscriptstyle \,-$}}}
\newcommand{\plusminus}[1]{{#1}_{\raise.1em\hbox{$\scriptscriptstyle \,\pm$}}}
\newcommand{\delete}[2]{\stackrel{\scriptstyle #1\atop\vee}{#2}}
\newcommand{\ddelete}[3]{\stackrel{{\scriptstyle #1\atop\vee}\;{\scriptstyle #2\atop\vee}}{#3}}
\newcommand{\define}{\mbox{$\overset{\textrm{def}}{\Longleftrightarrow}$}}
\newcommand{\bd}{\[}		\newcommand{\ed}{\]}
\def\bdn#1\edn{\begin{equation}#1\end{equation}}	\def\edn{}
\def\bdm#1\edm{\begin{align*}#1\end{align*}}		\def\edm{}
%\def\bdm#1\edm{\begin{eqnarray*}#1\end{eqnarray*}}
\def\bdmn#1\edmn{\begin{align}#1\end{align}}		\def\edmn{}
%\def\bdmn#1\edmn{\begin{eqnarray}#1\end{eqnarray}}
\def\bg#1\eg{\begin{gather*}#1\end{gather*}}		\def\eg{}
\def\bgn#1\egn{\begin{gather}#1\end{gather}}		\def\egn{}
%
\newlength{\FrameLength}
\def\bframe#1\eframe{\begin{center}\setlength{\fboxsep}{6pt}\setlength{\FrameLength}{\linewidth}\addtolength{\FrameLength}{-2\fboxsep}\addtolength{\FrameLength}{-2\fboxrule}\fbox{\begin{minipage}{\FrameLength}{#1\unskip}\end{minipage}}\end{center}}
\def\eframe{}
%\def\bframe#1\eframe{\noindent\fbox{\parbox{16.05cm}{#1}}}
\newcommand{\Eqn}[1]{\label{Eq.#1}}
\newcommand{\nn}{\nonumber}
\newcommand{\squad}{\hspace{1.2em}}
\newcommand{\ul}[1]{\underline{#1}}
\newcommand{\Laplace}{\triangle}

%
%=======================  Special Macro Commands  ==========================

\newcommand{\FC}{{\cal F} C^\infty_0}
\newcommand{\hs}[1]{{\cal L}_{(2)}^#1(H;\R)}
\newcommand{\gh}{g^{\scriptscriptstyle \rightarrow}}
\newcommand{\Gh}{G^{\scriptscriptstyle \rightarrow}}
\newcommand{\Hh}{H^{\scriptscriptstyle \rightarrow}}
\newcommand{\Ph}{{{\cal P}^{\scriptscriptstyle \rightarrow}}}
\newcommand{\gv}{g^{\scriptscriptstyle \uparrow}}
\newcommand{\Gv}{G^{\scriptscriptstyle \uparrow}}
\newcommand{\Hv}{H^{\scriptscriptstyle \uparrow}}
\newcommand{\Pv}{{\cal P}^{\mbox{\tiny $\uparrow$}}}
\newcommand{\Pt}{{\cal P}}
\newcommand{\HPv}{H{\cal P}^{\mbox{\tiny $\uparrow$}}}
%\newcommand{\HPh}{H{\cal P}}
\newcommand{\GPv}{G{\cal P}^{\mbox{\tiny $\uparrow$}}}
%\newcommand{\GPh}{G{\cal P}}
\newcommand{\PV}{P^V}
\newcommand{\Ent}{\operatorname{Ent}\nolimits}
\newcommand{\Var}{\operatorname{Var}\nolimits}
\newcommand{\D}[2]{\mbox{D[$#1, #2$]}}
\newcommand{\E}[1]{E\left[ #1 \right]}
\newcommand{\Diri}{\cal{E}}
\newcommand{\tDiri}{\tilde{\cal{E}}}
\newcommand{\sDiri}{\tilde{\cal{E}}}
\newcommand{\aDiri}{\check{\cal{E}}}
\newcommand{\hDiri}{\hat{\cal{E}}}
\newcommand{\vDiri}{\cal{E}^V}
\newcommand{\DiriV}{\cal{E}^V}
\newcommand{\m}{m}
\newcommand{\tmu}{{\tilde{\mu}}}
\newcommand{\tb}{{\tilde{b}}}
\newcommand{\tw}{{\tilde{\omega}}}
\newcommand{\Proj}[1]{P}
\newcommand{\J}{J} % 複素共役をとる作用。 C とか c にしもよいかも

%\renewcommand{\Pr}[1]{\hat{P}_{#1}}   % already defined as 'Pr'
\newcommand{\gen}{{\mathfrak A}}
\newcommand{\vgen}{{\mathfrak A}^V}
\newcommand{\genV}{{\mathfrak A}^V}
%\newcommand{\vgen}{{\vec{\mathfrak A}}}
\newcommand{\sem}[1]{T_{#1}}
\newcommand{\tsem}[1]{\tilde{T}_{#1}}
\newcommand{\vsem}[1]{{T}^V_{#1}}
\newcommand{\semV}[1]{{T}^V_{#1}}
\newcommand{\res}[1]{G_{#1}}

\newcommand{\Er}[1]{\hat{E}_{#1}}
%\newcommand{\lt}{\mathrel{\raise.3ex\hbox{$<$}\kern-.78em\lower.85ex\hbox{$\sim$}}}
\newcommand{\lt}{\lesssim}
%\newcommand{\gth}{\mathrel{\raise.2ex\hbox{$>$}\kern-.75em\lower.32em\hbox{$\sim$}}}
\newcommand{\gth}{\gtrsim}
\newcommand\op{{\operatorname{op}}}
\newcommand\tr{{\operatorname{tr}}}
\newcommand\co{{\overline{\operatorname{co}}}}
\newcommand\HS{{\operatorname{HS}}}
\newcommand\Hom{{\operatorname{Hom}}}
\newcommand\bwedge{{\textstyle \bigwedge}}
\newcommand{\Ric}{\operatorname{Ric}}

\setlength{\parsep}{.2pt plus .2pt minus .1pt}
\setlength{\itemsep}{-23pt plus .2pt minus .1pt}
\setlength{\labelwidth}{15mm}



%
%=======================  End Of Beginning Of TeX  =========================
%
%\includeonly{snj,unj,unj_bibliography}
%
\begin{document}
\allowdisplaybreaks
%=======================     漢字間隔      =================================
%\jintercharskip=.5pt plus .4pt minus .46pt    % UNIX 用のコマンド
%\kanjiskip=.5pt plus .4pt minus .46pt % PC 用のコマンド
% \jintercharskip=.5pt plus .5pt minus .4pt   % PC 用
%
\thispagestyle{empty}
%\vspace*{30mm}
\begin{center}
\scalebox{4}{\bf
\begin{tabular}{c}
非対称作用素の\\[.7ex] スペクトル  \end{tabular}}
\end{center}
\setcounter{tocdepth}{2}
\tableofcontents
%
%
\ifodd\draftversion
	%%%%%%%%%%%%%%%%%%%%%%%%%%%%%%%%%%%%%%%%%%%%%%%%%
%                                               %
%        =======  Program  No.1  =======        %
%                                               %
%===============================================%
%%%%%%%%%%%%%%%%%%%%%%%%%%%%%%%%%%%%%%%%%%%%%%%%%
%
%=======================  Title  ===========================================
\title{非対称作用素のスペクトル}
%
%=======================  Dedication  ======================================
%\dedicatory{Dedicated to Professor XX YY on his 70th birthday}
%=======================  Author  ==========================================
\author{重川 一郎
\thanks{e-mail: {\tt ichiro@math.kyoto-u.ac.jp},\quad
           URL: {\tt http://www.math.kyoto-u.ac.jp/\~{}ichiro/}}
\\ (京都大学大学院理学研究科)}
% \author{Ichiro Shigekawa
% \footnote{This research was partially supported
% by the Ministry of Education, Culture, Sports, Science and Technology,
% Grant-in-Aid for  Scientific Research (B), No.~11440045, 1999}
% }
\date{}
\maketitle
%=======================    Contents   =====================================
\setcounter{tocdepth}{2}
% \tableofcontents
% \address{Department of Mathematics, %\\
% Graduate School of Science, %\\
%%Faculty of Science, %\\
% Kyoto University, %\\
% Kyoto 606-8502, %\\
% Japan}
% \email{ichiro@kusm.kyoto-u.ac.jp}
% \urladdr{http://www.kusm.kyoto-u.ac.jp/\~{}ichiro/}
%\date{}
%====================  Scientific Research Fund Classification  ============
% This research was partially supported by the Ministry of Education, 
% Culture, Sports, Science and Technology, Grant-in-Aid for XXX,
% ZZZZZZZZ, 19YY
% XXX is
% Tokubetsu Suuisin Kenkyu     Specially Promoted Research 
% Juuten ryouiki Kenkyu        Scientific Research on Priority Areas
%                              (Area Name)
% Kiban Kenkyu (A), (B), (C)   Scientific Research (A),(B),(C) 
% Hougateki Kenkyu             Exploratory Research 
% Syourei Kenkyu (A)           Encouragement of Young Scientists 
% ZZZ is Kadai Bangou
%=======================  Mathematics Subject Classification  ==============
% \subjclass{60J60, 58G32}
% 31-XX POTENTIAL THEORY
% 31Cxx   Other generalization
% 31C15     Potentials and capacities
% 31C25     Dirichlet spaces
% 47Dxx Groups and semigroups of linear operators, their generalizations and applications 
% 47D03 Groups and semigroups of linear operators
% 47D06 One-parameter semigroups and linear evolution equations
% 47D08 Schrodinger and Feynman-Kac semigroups
% 47D09 Operator sine and cosine functions and higher-order Cauchy problems
% 47D60 $C$-semigroups
% 47D62 Integrated semigroups
% 47D99 None of the above, but in this section
% 58-XX GLOBAL ANALYSIS, ANALYSIS ON MANIFOLDS
% 58Bxx   Infinite dimensional manifolds
% 58B10     Differentiability questions
% 58Gxx   Partial differential equations on manifolds;differential operators
% 58G32     Diffusion processes and stochastic analysis on manifolds
% 60-XX PROBABILITY THEORY AND STOCHASTIC PROCESSES
% 60Hxx   Stochastic Analysis
% 60H07     Stochastic calculus of variation and the Malliavin calculus
% 60Jxx   Markov Process
% 60J05 Markov processes with discrete parameter
% 60J10 Markov chains with discrete parameter
% 60J20 Applications of discrete Markov processes
%       (social mobility, learning theory, industrial processes, etc.)
% 60J22 Computational methods in Markov chains
% 60J25 Markov processes with continuous parameter
% 60J27 Markov chains with continuous parameter
% 60J35 Transition functions, generators and resolvents
% 60J40 Right processes
% 60J45 Probabilistic potential theory [See also 31Cxx, 31D05]
% 60J50 Boundary theory
% 60J55 Local time and additive functionals
% 60J57 Multiplicative functionals
% 60J60 Diffusion processes [See also 58J65]
% 60J65 Brownian motion [See also 58J65]
% 60J70 Applications of diffusion theory
%       (population genetics, absorption problems, etc.) [See also 92Dxx]
% 60J75 Jump processes
% 60J80 Branching processes (Galton-Watson, birth-and-death, etc.)
% 60J85 Applications of branching processes [See also 92Dxx]
% 60J99 None of the above, but in this section
%=======================  Abstract  ========================================
%\begin{abstract} \end{abstract}
\hide
\vspace{-4mm}
\begin{itemize} \itemsep=-2mm \parsep=0mm
\item 非対称作用素のスペクトルについて.
\item Total file name: snj01 snj02 $\dots $ snj?, snj\_bibliography
\item File name: unj01.tex \hfill 印刷日: \today \ \now
\end{itemize}
\endhide
%=======================  Text  ============================================
%
\SS{DUI}{導入} %////////////////////////////////////////////////////////////
% Dual ultracontractivity introduction
\hide
非対称作用素のスペクトルについて述べる。
具体的にスペクトルが完全に決定できる場合を例としていくつか列挙する。
池野君が修論で纏めてくれたものに補足を加えたものである。
\hfill 2010年12月23日(木)
\endhide

作用素のスペクトルを完全に決定することは一般には難しい問題である。
対称な作用素については比較的よく調べられてきたといえるが、
非対称な場合のスペクトルの解析はまだ十分になされていない。
この論文では、非対称な作用素でスペクトルが完全に決定できるものの例を
いくつか見ていく。
非対称な作用素の中でも、正規作用素はスペクトル分解の理論がつかえ、
解析が易しくなることもあり、正規作用素を中心に論じる。

また正規作用素については、加藤の平方作用素の問題と言われるものについても
考察する。
この問題は作用素の平方根の定義域と、対応する双線型形式の定義域とが一致するかどうかという問題である。
正規作用素の場合はこの問題は容易に解くことができることを注意する。
	%%%%%%%%%%%%%%%%%%%%%%%%%%%%%%%%%%%%%%%%%%%%%%%%%
%                                               %
%        =======  Program  No.2  =======        %
%                                               %
%===============================================%
%%%%%%%%%%%%%%%%%%%%%%%%%%%%%%%%%%%%%%%%%%%%%%%%%
%
\hide
\vspace{-4mm}
\begin{itemize} \itemsep=-2mm \parsep=0mm
\item Total file name: snj01 snj02 $\dots $ snj?, snj\_bibliography
\item File name: snj02.tex \hfill コンパイル日: \today \ \now
\end{itemize}
\endhide

\SS{SFH}{Hilbert 空間の上の sesquilinear form} %////////////////////////////
% Sesquilinear forms in a Hilbert space

Hilbert 空間上の sesquilinear form ついて,後で必要となることをまとめておく.$\Dom(a)\times\Dom(a)$ 上の関数 $a(u,v)$ で $u$ に関して線型,
$v$ に関して共役線型な関数を sesquilinear form という.
さらに $a(u)=a(u,u)$ を quadratic form という.
$a(u,v)$ は次の関係で $a(u)$ から一意的に定まる.
\bdn %----------------------------------------------------------------------
a(u,v)
= \frac{1}{4}(a(u+v) - a(u-v) + ia(u+iv) - ia(u-iv)).
\Eqn{SFH.4}
\edn %----------------------------------------------------------------------
sesquilinear form $a$ が
\bdm %----------------------------------------------------------------------
a(u,v) = \ol{a(v,u)}
\edm %----------------------------------------------------------------------
をみたすとき,対称であるという.
$a$ が対称であるための必要十分条件は,
対応する quadratic form が実数値であることである.

一般の sesquilinear form $a$ に対して adjoint form を
\bdn %----------------------------------------------------------------------
a^*(u,v)
= \ol{a(v,u)}
\Eqn{SFH.6}
\edn %----------------------------------------------------------------------
で定義する.
$a$ が対称であることは $a=a^*$ が成り立つことに他ならない.
また sesquilinear form $a$ に対して
\bdn %----------------------------------------------------------------------
b
= \frac{1}{2}(a+a^*), \quad
c
= \frac{1}{2i}(a-a^*),
\Eqn{SFH.8}
\edn %----------------------------------------------------------------------
とおくと
\bdn %----------------------------------------------------------------------
a
= b + ic
\Eqn{SFH.10}
\edn %----------------------------------------------------------------------
が成り立つ.
$b$ を実部,$c$ を虚部と呼ぶ.
これは quadratic form で考えると
\bdn %----------------------------------------------------------------------
b(u)
= \Re a(u), \quad
c(u)
= \Im a(u)
\Eqn{SFH.12}
\edn %----------------------------------------------------------------------
の関係が成り立っている.

$a$ がsymmetric のとき
\bdn %----------------------------------------------------------------------
a(u)
\ge 0, \quad \forall u\in\Dom(a)
\Eqn{SFH.14}
\edn %----------------------------------------------------------------------
が成り立つとき,非負であるという.
正値性 から次の Schwarz の不等式が成り立つ:
\bdn %----------------------------------------------------------------------
a(u,v)
\le a(u)^{1/2}a(v)^{1/2}.
\Eqn{SFH.16}
\edn %----------------------------------------------------------------------
さて,$a$ を非負として,$t$ を一般の symmetric form とする.
このとき $\Dom(a) \subseteq \Dom(t) $ で quadratic form として
\bdm %----------------------------------------------------------------------
|t(u)| \le K a(u), \quad \forall u\in\Dom(a)
\edm %----------------------------------------------------------------------
が成り立つとき,
\bdm %----------------------------------------------------------------------
|t(u,v)| \le K a(u)^{1/2} a(v)^{1/2}, \quad \forall u,v\in\Dom(a)
\edm %----------------------------------------------------------------------
が成立する.
これを見るには,$u$ の代わりに $e^{i\theta}u$ を考えることにより
$t(u,v)$ は実数としてよい.
すると \Eq{SFH.4} の表示は
\bdm %----------------------------------------------------------------------
t(u,v)
= \frac{1}{4}(t(u+v) - t(u-v))
\edm %----------------------------------------------------------------------
となる.
ここで quadratic form $t$ が実数値であることを使っている.
よって仮定から
\bdm %----------------------------------------------------------------------
|t(u,v)|
\le \frac{K}{4}(a(u+v) + a(u-v))
=   \frac{K}{2}(a(u) + a(v)).
\edm %----------------------------------------------------------------------
ここで $u$, $v$ の代わりに $\e u$, $\e^{-1}v$ をとれば
\bdm %----------------------------------------------------------------------
|t(u,v)|
\le \frac{K}{2}(\e a(u) + \e^{-1} a(v)).
\edm %----------------------------------------------------------------------
$a(u)=0$ ならば $\e\to\infty$ として $t(u,v)=0$ を得る.
$a(u)\not=0$ のときは $\e=\frac{a(v)^1/2}{a(u)^{1/2}}$ と取れば
\bdm %----------------------------------------------------------------------
|t(u,v)|
\le K a(u)^{1/2} a(v)^{1/2}
\edm %----------------------------------------------------------------------
が成り立つことが分かる.

さて $a$ を sesquilinear form とする.  
$a = b+ic$ を実部と虚部に分ける.
$b$ が非負であるとき $a$ も非負 (または accretive) という.
\bdm %----------------------------------------------------------------------
\Re a(u,u) \ge 0, \quad \forall u\in\Dom(a)
\edm %----------------------------------------------------------------------
また $a$ の numerical range を
\bdn %----------------------------------------------------------------------
\Theta(a)
= \{ a(u,u);\, u\in\Dom(a),\ |u|=1\}
\Eqn{SFH.20}
\edn %----------------------------------------------------------------------
で定義する.
$a$ が非負であることは
$\Theta(a) \subseteq \{\zeta\in\C;\, \Re \zeta \ge 0\}$ と同値である.
さらに $a$ が扇形条件をみたす(sectorial)ということを
\bdm %----------------------------------------------------------------------
\Theta(a)
\subseteq S_\theta
\edm %----------------------------------------------------------------------
で定義する.
ここで $S\theta$ 次で定まる角領域である:
\bdn %----------------------------------------------------------------------
S_\theta
= \{z\in\C;\, |\arg z| \le \theta \}.
\Eqn{SFH.22}
\edn %----------------------------------------------------------------------
但し $\theta$ は $0\le \theta<\frac{\pi}{2}$ にとる.

扇形条件が成り立つと,実部 $b$ は非負である.
この条件は $a(u,u)= b(u)+ic(u)$ に注意すれば
$c(u) \le \tan\theta b(u)$ とも同値である.
さらにこの条件は次の条件とも同値になる:
$b$ が非負で
ある定数 $K \ge 0$ が存在して
\bdn %----------------------------------------------------------------------
|a(u,v)|
\le (1+K) b(u)^{1/2} b(v)^{1/2}
\Eqn{SFH.24}
\edn %----------------------------------------------------------------------
が成り立つ.
このことを証明しておこう.

\Proposition{SFH-6} %*******************************************************
$a$ が \Eq{SFH.24} をみたすことと,
ある $\theta\in [0,\frac{\pi}{2})$ が存在して
$\Theta(a) \subseteq S_\theta$ となることは同値である.
\end{proposition} %*********************************************************

\Proof
まず \Eq{SFH.24} が成り立っているとする.
すると $v=u$ として両辺を2乗して
\bdm %----------------------------------------------------------------------
|a(u,u)|^2
\le (1+K)^2 b(u)^2.
\edm %----------------------------------------------------------------------
ところで,$a(u,u) = b(u) + ic(u)$ だから左辺は $b(u)^2 + c(u)^2$ であり
\bdm %----------------------------------------------------------------------
c(u)^2 + b(u)^2
\le (1+2K + K^2) b(u)^2
\edm %----------------------------------------------------------------------
から
\bdm %----------------------------------------------------------------------
c(u) \le \sqrt{K(2 + K)} b(u)
\edm %----------------------------------------------------------------------
が成り立つ.
即ち $\theta=\arctan \sqrt{K(2 + K)}$ として
$\Theta(a) \subseteq S_\theta$ が成り立つ.

逆に $\Theta(a) \subseteq S_\theta$ が成り立てば
$c(u) \le \tan\theta b(u)$ なので
\bdm %----------------------------------------------------------------------
|c(u,v)| \le \tan\theta b(u)^{1/2}b(v)^{1/2}
\edm %----------------------------------------------------------------------
が成り立つ.
従って
\bdm %----------------------------------------------------------------------
|a(u,v)|
&= |b(u,v) + ic(u,v)| \\
&\le |b(u,v)| + |c(u,v)| \\
&\le b(u)^{1/2}b(v)^{1/2} + \tan\theta b(u)^{1/2}b(v)^{1/2} \\
&= (1+\tan \theta) b(u)^{1/2}b(v)^{1/2}.
\edm %----------------------------------------------------------------------
これで \Eq{SFH.24} が $K=\tan\theta$ として成立することが分かった.
\QED %======================================================================

非負性はもう少し緩めることが出来る.
$a$ が下に有界であることを,ある $\gm\in\R$ が存在して
\bdn %----------------------------------------------------------------------
S_\theta
\subseteq \{\zeta \in\C;\, \Re \zeta \ge - \gm \}.
\Eqn{SFH.28}
\edn %----------------------------------------------------------------------
さらに扇形的に下に有界 (sectorially bounded from below) であることを
ある $\gm\in\R$ が存在して
\bdm %----------------------------------------------------------------------
\Theta(a)
\subseteq S_\theta - \gm 
\edm %----------------------------------------------------------------------
が成り立つと定義する.
これは $a_\gm$ を
\bdm %----------------------------------------------------------------------
a_\gm(u,v)
= a(u,v) + \gm(u,v)
\edm %----------------------------------------------------------------------
で定義したとき $a_\gm$ が 扇形条件をみたすことに他ならない.


\subsec{Notes}
\begin{itemize}
\item ここの話は Kato \cite{Kato76} Chapter VI, \S1,\S2 に元づく.
扇形条件という言葉をどう定義するかは迷うところである.
ここでは numerical range を用いて定義した.
Ma-R\"ockner とは用語法がずれる.
[2011年1月15日]
\end{itemize}


	%%%%%%%%%%%%%%%%%%%%%%%%%%%%%%%%%%%%%%%%%%%%%%%%%
%                                               %
%        =======  Program  No.3  =======        %
%                                               %
%===============================================%
%%%%%%%%%%%%%%%%%%%%%%%%%%%%%%%%%%%%%%%%%%%%%%%%%
%
\hide
\vspace{-4mm}
\begin{itemize} \itemsep=-2mm \parsep=0mm
\item Total file name: snj01 snj02 $\dots $ snj?, snj\_bibliography
\item File name: snj03.tex \hfill コンパイル日: \today \ \now
\end{itemize}
\endhide

\SS{OHS}{Hilbert 空間の上の作用素} %////////////////////////////////////////
% Operators on a Hilbert spaces

Hilbert 空間上の作用素のついて,後で必要となることをまとめておく.
$T$ をHilbert 空間 $H$ 上の作用素とする.
$T$ の numerical range $\Theta(T)$ を
\bdn %----------------------------------------------------------------------
\Theta(T)
:= \{(Tu,u);\, u\in\Dom(T)\}. 
\Eqn{OHS.4}
\edn %----------------------------------------------------------------------
ここでは証明しないが $\Theta(T)$ は凸集合であることが知られている
(Stone \cite{Stone32} を参照).
$\Delta=\C\setminus \ol{\Theta(T)}$ とする.
$\Theta(T)$ は凸集合だから,$\Delta$ の連結成分は高々二つである.
連結成分が2つある場合は $\Delta_1$, $\Delta_2$ とかく.
  

\Theorem{OHS-4} %***********************************************************
$T$ を閉作用素とする.
$\zeta\in\Delta$ のとき $T-\zeta$ の像は閉集合となり,
$\Ker(T-\zeta)=\{0\}$ で $T-\zeta$ の指数は $\Delta$ の各連結成分で
一定である.
もし $\Delta$ ($\Delta_1$ or $\Delta_2$) で $T-\zeta$ の指数が $0$
であるならば $\Delta$ ($\Delta_1$ or $\Delta_2$) はレゾルベント集合
$\rho(T)$ に含まれる.
\end{theorem} %*************************************************************

\Proof
まず $u\in\Dom(T)$, $|u|=1$ のとき
\bdm %----------------------------------------------------------------------
|(Tu,u)-\zeta|
= |((T-\zeta u),u)-\zeta|
\le |(T-\zeta)u|
\edm %----------------------------------------------------------------------
が成り立つことに注意しよう.
これから $\zeta\in\Delta$ のとき $\d=d(\zeta, \ol{\Theta(T)})>0$ とおくと
$|(T-\zeta)u|\ge \d$, $|u|=1$ だから
\bdm %----------------------------------------------------------------------
|(T-\zeta)u|
\ge \d |u|, \quad \forall u\in\Dom(T).
\edm %----------------------------------------------------------------------
これから $\Ker(T-\zeta)=0$ で $\Ran(T-\zeta)$ が閉集合であることが従う.
また各連結成分で $T-\zeta$ の指数は指数の安定性から一定である.
特に指数が 0 のときは $\Ran(T-\zeta)=H$ であるから有界な逆が存在する.
従って $\rho(T)$ に含まれていることが分かる.
\QED %======================================================================

さて閉作用素 $A$ が accretive であることを
$\Re (Au,u)\ge 0$, $\forall u\in\Dom(A)$
で定義する.
これは $\Theta(A)$ が右半平面に含まれることを意味する.
さらにある $\Re \zeta<0$ となる $\zeta$ が存在して $\Ran(A-\zeta)=H$
が成り立っているとき,$\gen$ を $m$-accretive と呼ぶ.
このとき $\Re \lm<0$ となる任意の $\lm$ に対して $\lm\in\rho(A)$ で
\bdm %----------------------------------------------------------------------
\|(A-\lm)^{-1}\| \le |\Re \lm|^{-1}
\edm %----------------------------------------------------------------------
が成り立っている.
$A$ の定義域が稠密であることは $m$-accretive の条件から必然的に従う.
実際,$((A-\lm)^{-1}u,v)$, $\forall u\in H$ が成り立てば
$u-v$, $w=(A-\lm)^{-1}v$ として
\bdm %----------------------------------------------------------------------
0
= \Re((A-\lm)^{-1}v,v)
= \Re(w,(A-\lm)w)
\ge -\Re\lm |w|^2
\edm %----------------------------------------------------------------------
から $w=0$ が従う.

さて $A$ が accretive のとき
$\Theta(A) \subseteq S_\theta$, $\theta\in[0,\frac{\pi}{2})$
であるとき,扇形条件が成り立つ (sectorial) という.
ここで $S_\theta$ は次で定まる角領域である:
$S_\theta = \{z\in\C;\, |\arg z| \le \theta \}$.
これらの概念は sesquilinear form のときに既に出てきている.
実際 sesquilinear form $a$ を
\bdn %----------------------------------------------------------------------
a(u,v)
= (Au,v)
\Eqn{OHS.8}
\edn %----------------------------------------------------------------------
で定めればよい.
定義域は $\Dom(A)\times \Dom(A)$ とする.
$a$ の実部,虚部を $b$, $c$ とすれば
\bdmn %---------------------------------------------------------------------
b(u,v)
&= \frac{1}{2}\{ (Au,v) + (u,Av)\}
\Eqn{OHS.10} \\
c(u,v)
&= \frac{1}{2i}\{ (Au,v) - (u,Av)\}
\Eqn{OHS.12}
\edmn %---------------------------------------------------------------------
\Prop{SFH-6} から扇形条件は次の条件と同値である:
ある定数 $K$ が存在して
\bdn %----------------------------------------------------------------------
|(Au,v)|
\le K (\Re(Au,u))^{1/2} (\Re(Av,v))^{1/2}.
\Eqn{OHS.16}
\edn %----------------------------------------------------------------------

$T$ を閉作用素とする.
ある実数 $\gm$ が存在して $T+\gm$ が accretive のとき
 $T$ を quasi-accretive という.
また $T+\gm$ が扇形条件をみたすとき,quasi-sectorial という.
さらに $T+\gm$ が $m$-accretive のとき quasi-$m$-accretive といい,
これに加えて扇形条件をみたすとき quasi-$m$-sectorial であるという.


\bigskip
今後 Hilbert 空間上の作用素のスペクトルのことを問題にしていくわけだが,
そのためには複素Hilbert 空間で調べることが自然である.
そこで,実 Hilbert 空間の複素化の話をまずまとめておく.

$H$ を実Hilbert 空間とする.
複素化を $H\oplus i H$ で定める.
$H\oplus i H$ の元は $u+iv$ ($u$,$v\in H$) と表される.
スカラー倍は $x+iy\in\C$ に対し
\bdn %----------------------------------------------------------------------
(x+iy)(u+iv)
= (x u -y v) + i(x v + y u) 
\Eqn{OHS.20}
\edn %----------------------------------------------------------------------
で定め,内積は
\bdn %----------------------------------------------------------------------
(u+iv, r+is)
=(u,r) + (v,s) + i\{(v,r) - (u,s)\}
\Eqn{OHS.22}
\edn %----------------------------------------------------------------------
で定める.
また $H\oplus i H$ には自然な共役作用素 $\J$ が次で定義される.
\bdn %----------------------------------------------------------------------
\J(u+iv)
= u-iv.
\Eqn{OHS.24}
\edn %----------------------------------------------------------------------
$\J$ は次の関係をみたす.
\bdmn %---------------------------------------------------------------------
\J^2(f)
&= f,
\Eqn{OHS.26} \\
\J(\al f + \be g)
&= \ol{\al} \J(f) + \ol{\be}\J(g).
\Eqn{OHS.28}
\edmn %---------------------------------------------------------------------
複素数に対しては $\ol{\phantom{f}}$ は 共役複素数を表す.
従って $\J$ は共役線型であることに注意しよう.
また内積については
\bdn %----------------------------------------------------------------------
(\J(f), \J(g))
= \ol{(f,g)}
\Eqn{OHS.30}
\edn %----------------------------------------------------------------------
が成り立っている.
以後 $\J(f)$ を $\ol{f}$ ともかく.

逆に複素 Hilbert 空間 $H$ に,\Eq{OHS.26}, \Eq{OHS.28}, \Eq{OHS.30}
をみたす共役作用素 $\J$ が与えられているとき,
\bdm %----------------------------------------------------------------------
H_r
= \{u\in H;\, \J(u)= u\}
\edm %----------------------------------------------------------------------
で定めると,$H$ は $H_r$ の複素化になる.
$H$ の作用素 $A$ が
\bdm %----------------------------------------------------------------------
A\J
= \J A
\edm %----------------------------------------------------------------------
をみたすとき実作用素と呼ぶ.
$A$ が実作用素のとき,$A$ は $H_r$ の作用素になる.
もう少し正確に言うと $f\in \Dom(A)\cap H_r$ ならば $Af\in H_r$ で,
さらに $\Dom(A) = (\Dom(A)\cap H_r)\oplus i(\Dom(A)\cap H_r)$ が成り立つ.
逆に実 Hilbert 空間 $H_r$ の作用素は,複素化すると実作用素になる.

さて $A$ を実作用素とする.
$\J$ の定義から $\J (A-\zeta)\J=A-\ol{\zeta}$ である.
これから $\Ker(A-\zeta)=\{0\}$ $\Leftrightarrow$ $\Ker(A-\ol{\zeta})=\{0\}$
であり $\Ran(A-\zeta)=H$ $\Leftrightarrow$ $\Ran(A-\ol{\zeta})=H$ である.
これから $(A-\zeta)^{-1}$ が $H$ 全体で定義された有界作用素になるときには
$(A-\ol{\zeta})^{-1}$ も $H$ 全体で定義された有界作用素になる.
従ってレゾルベント集合は共役をとる操作で不変になる.
スペクトル集合も同じ性質を持つことが分かる.

実の Hilbert 空間で双一次形式 $a$ が与えられたとき,
その複素化について考えよう.
$a$ は自然に複素化された Hilbert 空間で sesquilinear form に拡張できる.
従って $a$ は第2変数について共役線型だから
\bdm %----------------------------------------------------------------------
a(u+iv,u+iv)
&= a(u,u)+a(v,v) -i a(u,v) + ia(v,u) \\
&= a(u,u)+a(v,v) + i \{a(v,u) - a(u,v)\}
\edm %----------------------------------------------------------------------
が成り立つ.
従って $a$ の実部,虚部を $b$, $c$ とすると
\bdm %----------------------------------------------------------------------
b(u+iv) = a(u,u)+a(v,v), \quad
c(u+iv) = a(v,u)-a(u,v)
\edm %----------------------------------------------------------------------
これから扇形条件 $|c| \le \tan\theta b$ は
$|a(v,u)-a(u,v)| \le \tan\theta\{a(u,u)+a(v,v)\}$ となる.
$u$ の代わりに $\e u$, $v$ の代わりに $\e^{-1} v$ をとって
\bdm %----------------------------------------------------------------------
|a(v,u)-a(u,v)| \le \tan\theta\{\e^2 a(u,u)+\e^{-2} a(v,v)\}
\edm %----------------------------------------------------------------------
特に $\e^2= \frac{a(v,v)^{1/2}}{a(u,u)^{1/2}}$ をとれば
\bdn %----------------------------------------------------------------------
|a(v,u)-a(u,v)| \le 2 \tan\theta a(u,u)^{1/2} a(v,v)^{1/2}
\Eqn{OHS.32}
\edn %----------------------------------------------------------------------
が成り立つ.
逆にこの関係から
\bdm %----------------------------------------------------------------------
|a(v,u)-a(u,v)| \le 2 \tan\theta a(u,u)^{1/2} a(v,v)^{1/2}
\le  \tan\theta \{a(u,u) + a(v,v)\}
\edm %----------------------------------------------------------------------
が成り立つ.
従って扇形条件は上の式と同値になる.

\subsec{Notes}
\begin{itemize}
\item accretive な作用素についてのまとめ.
\item 実 Hilbert 空間の複素化と実作用素.
\end{itemize}




	%%%%%%%%%%%%%%%%%%%%%%%%%%%%%%%%%%%%%%%%%%%%%%%%%
%                                               %
%        =======  Program  No.4  =======        %
%                                               %
%===============================================%
%%%%%%%%%%%%%%%%%%%%%%%%%%%%%%%%%%%%%%%%%%%%%%%%%
%
\hide
\vspace{-4mm}
\begin{itemize} \itemsep=-2mm \parsep=0mm
\item Total file name: snj01 snj02 $\dots $ snj?, snj\_bibliography
\item File name: snj04.tex \hfill コンパイル日: \today \ \now
\end{itemize}
\endhide

\SS{NSS}{正規作用素} %//////////////////////////////////////////////////////
% Non-symmetric semigroup
\hide
論文としてまとめようと思いながら,とうとう2011年となってしまった.
前に進むのが相変わらず遅い.
\hfill 2011年1月3日(月)
\endhide

正規作用素を考察しよう
作用素 $A$ が正規作用素とは,$A$ が次の条件をみたすときをいう.
\bdn %----------------------------------------------------------------------
A^*A
= AA^*.
\Eqn{NSS.20}
\edn %----------------------------------------------------------------------
自己共役作用素は正規作用素であるが,一般に作用素の正規性を確かめるのは
難しい問題である.
応用上現れる作用素は,定義域を特徴付けることが難しく,
ある作用素の閉包,というような形で定義されることが多いからである.
この問題はまた後で考えることにする.
$A$ が正規作用素なら複素数 $\lm$ に対し $\lm+A$ も正規であることは
容易に分かる.
実際
\bdm %----------------------------------------------------------------------
(\lm+A)^* (\lm+A)
&= (\ol{\lm} +A^*) (\lm+A)
= \lm^2 + \ol{\lm}A + \lm A^* + A^*A \\
&= \lm^2 + \ol{\lm}A + \lm A^* + AA^*
= (\lm+A) (\lm+A)^*.
\edm %----------------------------------------------------------------------
正規作用素に $A$ 対しては スペクトル分解定理が成立する.
即ち,単位の分解 $E(dz)$ が存在し
\bdn %----------------------------------------------------------------------
A
= \int_\C z E(dz)
\Eqn{NSS.22}
\edn %----------------------------------------------------------------------
と表現される.
$E$ は射影作用素に値をとる $\C$ 上の測度である.
$\s(A)$ を $A$ のスペクトル集合とすると $E(dz)$ の台は $\s(A)$ と一致する.
ここで
\bdm %----------------------------------------------------------------------
(Au,u)
= \int_\C z (E(dz)u,u).
\edm %----------------------------------------------------------------------
が成立する.
右辺の積分は $\C$ 全体ではなく $\s(A)$ に制限してよい.
$|u|=1$ ならば $(E(dz)u,u)$ は 確率測度である.
つまり $(Au,u)$ は $\s(A)$ の convex combination で表されていることになる.
$\s(A)$ の閉凸包を $\co(\s(A))$ で表すと numerical range $\Theta(A)$
に対しては $\Theta(A) \subseteq \co(\s(A))$ が成り立つ.
一方 $\ol{\Theta(A)}$ はスペクトルを含む閉凸集合であった.
従って
\bdn %----------------------------------------------------------------------
\ol{\Theta(A)} = \co(\s(A))
\Eqn{NSS.24}
\edn %----------------------------------------------------------------------
が得られた.
これで sectorial かどうかはスペクトル $\s(A)$ を見れば
完全に分かることになる.

さて,後で縮小半群を生成する作用素を考えたいので,
ここで扱う正規作用素は次の増大性を仮定する
\bdn %----------------------------------------------------------------------
\Re (Au,u)
\ge 0, \quad \forall u\in \Dom(A).
\Eqn{NSS.26}
\edn %----------------------------------------------------------------------
従って縮小半群との対応を考えるときは $-A$ が縮小半群の生成作用素となる.
\Eq{NSS.22} から $A^*$ に対しては
\bdn %----------------------------------------------------------------------
A^*
= \int_\C \ol{z} E(dz)
\Eqn{NSS.28}
\edn %----------------------------------------------------------------------
が成り立つ.
これは
\bdm %----------------------------------------------------------------------
(A^*u,v)
= (u,Av)
= (u,\int_\C z E(dz)v)
= \int_\C \ol{z} (u,E(dz)v)
= \int_\C \ol{z} (E(dz)u,v)
= (\int_\C \ol{z}E(dz)u,v)
\edm %----------------------------------------------------------------------
から分かる.


さらに $A$ はスペクトル分解を持つから作用素の平方根 $\sqrt{A}$ が
次で定義できる.
\bdn %----------------------------------------------------------------------
\sqrt{A}
= \int_\C \sqrt{z} E(dz).
\Eqn{NSS.30}
\edn %----------------------------------------------------------------------
ここで $\sqrt{z}$ の分岐は実軸上で $\sqrt{z}$ となるものをとる.
この作用素の定義域は $|\sqrt{z}|^2 = |z|$ であるから
\bdn %----------------------------------------------------------------------
\Dom(\sqrt{A})
= \{u\in H;\, \int_\C |z| (u,E(dz)u) < \infty \}
\Eqn{NSS.32}
\edn %----------------------------------------------------------------------
である. 
同様に $A^*$ に対しても
\bdn %----------------------------------------------------------------------
\sqrt{A^*}
= \int_\C \sqrt{\ol{z}} E(dz)
\Eqn{NSS.34}
\edn %----------------------------------------------------------------------
および
\bdn %----------------------------------------------------------------------
\Dom(\sqrt{A^*})
= \{u\in H;\, \int_\C |z| (u,E(dz)u) < \infty \}
\Eqn{NSS.36}
\edn %----------------------------------------------------------------------
が成り立つ.
両者から正規作用素に対しては 
\bdn %----------------------------------------------------------------------
\Dom(\sqrt{A})
= \Dom(\sqrt{A^*})
\Eqn{NSS.38}
\edn %----------------------------------------------------------------------
が常に成り立っていることが分かる.
これはある意味で Kato の平方根問題が肯定的に解けていることを意味する.
ここでは sesquilinear form との関連を調べよう.

さて, $A$ に対応する sesquilinear form $a$ を定義するなら
\bdn %----------------------------------------------------------------------
a(u,v)
= (Au,v), \quad u,v\in\Dom(A)
\Eqn{NSS.40}
\edn %----------------------------------------------------------------------
とするのが自然である.
さらにこの $a$ の実部 (対称な部分) $b$ は
\bdn %----------------------------------------------------------------------
b(u,v)
= \frac{(Au,v) + (A^*u,v)}{2}, \quad u,v\in\Dom(A)
\Eqn{NSS.42}
\edn %----------------------------------------------------------------------
で定義される.
増大性の仮定から 
\bdm %----------------------------------------------------------------------
b(u,u)
= \frac{(Au,u) + (A^*u,u)}{2}
= \Re(Au,u)
\ge 0
\edm %----------------------------------------------------------------------
が成り立つ.
さらにスペクトル分解を用いて
\bdm %----------------------------------------------------------------------
b(u,v)
= \int_\C \Re{z} (u,E(dz)v)
\edm %----------------------------------------------------------------------
が成り立つ.
これから定義域を次で定めると $b$ は閉になる:
\bdn %----------------------------------------------------------------------
\Dom(b)
= \{u\in H;\, \int_\C \Re{z} (u,E(dz)u) < \infty \}.
\Eqn{NSS.44}
\edn %----------------------------------------------------------------------

これから $\Dom(\sqrt{A}) \subseteq \Dom(b)$ であることは明らかである.
二つが一致するための必要十分条件を与えよう.
$0<\theta<\pi$ に対し複素平面内の角領域 $S_\theta$ を
\bdn %----------------------------------------------------------------------
S_\theta
= \{z\in\C;, |\arg z| \le \theta \}
\Eqn{NSS.50}
\edn %----------------------------------------------------------------------
で定義する.
このとき次が成り立つ.
\Theorem{NSS-4} %***********************************************************
$\Dom(\sqrt{A}) = \Dom(b)$ が成り立つための必要十分条件は
ある $\theta\in (0,\pi/2)$ が存在して $1+\s(A) \subseteq S_\theta$
となることである.
\end{theorem} %*************************************************************

\Proof
ある $\theta\in(0,\pi/2)$ が存在して $1+\s(A) \subseteq S_\theta$
が成立しているとしよう.
すると $u\in \Dom(\sDiri)$ ならば
$z=x+iy\in S_\theta-1$ ならば $|y| \le (x+1)\tan\theta$ だから
\bdm %----------------------------------------------------------------------
\int_\C |z| (E(dz)u,u)
&= \int_{S_\theta-1} |z| (E(dz)u,u) \\
&\le \int_{S_\theta-1} (x+ |y|) (E(dz)u,u) \\
&\le \int_{S_\theta-1} (x+ (x+1)\tan\theta) (E(dz)u,u) \\
&\le (1+\tan\theta) \int_{S_\theta-1} x (E(dz)u,u)
    + \tan\theta \int_{S_\theta-1} (E(dz)u,u)
< \infty
\edm %----------------------------------------------------------------------
となり $u\in \Dom(\sqrt{A})$ が従う.

逆に,いくら $\theta$ を $\frac{\pi}{2}$ に近くとっても
$1+\s(A) \subseteq S_\theta$ とならないときは,任意の $n\in\N$ に対し
$z_n=x_n+iy_n\in \s(A)$ を
\bdm %----------------------------------------------------------------------
|y_n|
\ge n(x_n+1) + 1
\edm %----------------------------------------------------------------------
が成り立つように出来る.
\memo{条件からは $|y_n| \ge n(x_n+1)$ だけど,これから $|y_n|-1 \ge (n-1)(x_n+1) +1$ が成立するから,番号を一つずらせばよい.}
さらに $|z_n-z_m|>2$ ($n\not=m$)としてよい.
$B_n$ を $z_n$ を中心とする半径 $1$ の閉円板とする.
そして $u_n\in\Ran E(B_n)$ を $|u_n|=1$ となるようにとる.
とり方から $\{u_n\}$ は正規直交系になる.
これを用いて $u = \sum_n \frac{u_n}{n\sqrt{x_n+1}}$ と定める.
$u\in H$ は明らかである.
また 
\bdm %----------------------------------------------------------------------
\int_\C \Re z(E(dz)u,u)
&=   \sum_n \int_\C \Re z
     (E(dz)\frac{u_n}{n\sqrt{x_n+1}},\frac{u_n}{n\sqrt{x_n+1}}) \\
&=   \sum_n \int_{B_n} \Re z
     (E(dz)\frac{u_n}{n\sqrt{x_n+1}},\frac{u_n}{n\sqrt{x_n+1}}) \\
&\le \sum_n \int_{B_n} (x_n+1) \frac{1}{n^2(x_n+1)} (E(dz)u_n,u_n) \\
&= \sum_n \frac{1}{n^2}
<  \infty
\edm %----------------------------------------------------------------------
より $u\in\Dom(\sDiri)$ である.

一方
\bdm %----------------------------------------------------------------------
\int_\C |z|(E(dz)u,u)
&=   \sum_n \int_\C |z|
     (E(dz)\frac{u_n}{n\sqrt{x_n+1}},\frac{u_n}{n\sqrt{x_n+1}}) \\
&=   \sum_n \int_{B_n} |z|
     (E(dz)\frac{u_n}{n\sqrt{x_n+1}},\frac{u_n}{n\sqrt{x_n+1}}) \\
&\ge \sum_n \int_{B_n} (|y_n|-1) \frac{1}{n^2(x_n+1)} (E(dz)u_n,u_n) \\
&\ge \sum_n \frac{n(x_n+1)}{n^2(x_n+1)}
=    \infty
\edm %----------------------------------------------------------------------
となるから $u\not\in\Dom(\sqrt{A})$ である.
これで主張が示せた.
\QED %======================================================================

最後に実作用素の場合の注意を与えておこう.
正規作用素 $A$ が実作用素であるとする.
単位の分解 $E(dz)$ が存在して
\bdm %----------------------------------------------------------------------
A
= \int_\C z E(dz)
\edm %----------------------------------------------------------------------
と表される.
従って
\bdm %----------------------------------------------------------------------
\J A \J
= \J \int_\C z E(dz) \J
=  \int_\C \ol{z} \J E(dz)\J
=  \int_\C z \J E(d\ol{z})\J. \quad (\because \text{ 変数変換})
\edm %----------------------------------------------------------------------
$\J E(d\ol{z})\J$ も一つの単位の分解を与えている.
ここで $A=\J A \J$ だから,分解の一意性から二つの単位の分解は一致する.
従って
\bdn %----------------------------------------------------------------------
\J E(d\ol{z})\J
=  E(dz)
\Eqn{NSS.58}
\edn %----------------------------------------------------------------------
あるいは
\bdn %----------------------------------------------------------------------
E(d\ol{z})
= \J  E(dz)\J
\Eqn{NSS.60}
\edn %----------------------------------------------------------------------
成立していることが示せた.



\subsec{Notes}
\begin{itemize}
\item
\end{itemize}


	%%%%%%%%%%%%%%%%%%%%%%%%%%%%%%%%%%%%%%%%%%%%%%%%%
%                                               %
%         ====== Program  No.5 =======          %
%                                               %
%             file name snj05.tex               %
%                                               %
%===============================================%
%===================  for  =====================%
%===============================================%
%
\hide
\vspace{-4mm}
\begin{itemize} \itemsep=-2mm \parsep=0mm
\item Total file name: snj01 snj02 $\dots $ snj?, snj\_bibliography
\item File name: snj05.tex \hfill 印刷日: \today \ \now
\item 生成作用素が正規であるための条件を求める.
[2009年8月19日]
\item この節は 初め lnj にあったのを移動させてきた.
Riemannian manifold の上の作用素も一般的な形で論じてある.
単に趣味的なだけではあるが.
[2011年1月5日]
\end{itemize}
\endhide
\SS{CNO}{正規作用素であるための条件} %======================================
% Condition for normal operator
生成作用素が正規であるための条件を求めよう.
まず一般的な枠組みを準備し,その後で Riemannian manifold の場合を考える.

\Theorem{CNO-1} %**********************************************************
$H$ を Hilbert 空間,$A$, $B$ を $\sD$ を定義域とする消散作用素,
$\ol{A}$, $\ol{B}$ をその閉方とし,
$\ol{A}$, $\ol{B}$ が $m$-dissipative であることを仮定する.
さらに $A\sD\subseteq \sD$, $B\sD\subseteq \sD$ で
\bdmn %---------------------------------------------------------------------
AB &= BA \quad \text{on $\sD$}
\Eqn{CNO.2} \\
(Au,v) &= (u,Bv), \quad  u,v\in\sD
\Eqn{CNO.4}
\edmn %---------------------------------------------------------------------
が成り立っているとする.
このとき $\ol{A}$ は正規作用素で $\ol{A}^*=\ol{B}$ である.
\end{theorem} %*************************************************************

\Proof
条件 \Eq{CNO.2}, \Eq{CNO.4} から $u$, $v\in\sD$ のとき
\bdn %----------------------------------------------------------------------
(Au, Av)
= (u,BAv)
= (u,ABv)
= (Bu,Bv)
\Eqn{CNO.6}
\edn %----------------------------------------------------------------------
が成り立つから $|Au|=|Bu|$ が任意の $u\in\sD$ に対して成立する.
従って $u\in\Dom(\ol{A})$ とすると $\{u_n\}\subseteq \sD$ で $u_n\to u$,
$\{Au_n\}$ は Cauchy 列となるものが存在する.
従って $Bu_n$ も Cauchy 列となり,$\Dom(\ol{A})=\Dom(\ol{B})$ が成り立つ.
$\sE=\Dom(\ol{A})$ とおく.
さらに \Eq{CNO.4} で極限をとることにより
\bdm %----------------------------------------------------------------------
(\ol{A} u, v)
= (u,\ol{B}v), \quad u,v\in \sE
\edm %----------------------------------------------------------------------
が成り立つ.
従って $\ol{A}^* \supseteq \ol{B}$ となる.
ところで $\ol{A}$ は $m$-dissipative だから $\ol{A}^*$ も dissipative
になる.
$\ol{B}$ が $m$-dissipative
であるから極大性から $\ol{A}^* = \ol{B}$ となる.
\hide
ここで $\ol{A}$ が $m$- dissipative であることも使っている.
実際 $\ol{A}$ が $m$-dissipative だと 
$\ol{A}^*$ も dissipative になることが田辺の定理 2.1.3 に書いてある.
このことがなければ $\ol{A}^*$ が dissipative かどうか分からなくなるので,
$\ol{A}^* = \ol{B}$ を出せなくなる.
$m$-dissipative の仮定は片方だけでよいと思ったが,やはり両方必要である.
\hfill [2011年1月8日]
\endhide

一方上の  \Eq{CNO.6} で極限をとることにより
\bdm %----------------------------------------------------------------------
(\ol{A} u, \ol{A}v)
= (\ol{B}u,\ol{B}v), \quad u,v\in \sE
\edm %----------------------------------------------------------------------
が成り立つ.
さて $u\in\Dom(\ol{A}^*\, \ol{A})$ をとる.
つまり $u\in \Dom(\ol{A})$ かつ $\ol{A}u\in\Dom(\ol{A}^*)=\sE$
が成り立っているとする.
すると上の式から
\bdm %----------------------------------------------------------------------
(\ol{A}^*\, \ol{A} u, v)
= (\ol{A}u, \ol{A}v)
= (\ol{B}u,\ol{B}v), \quad g\in \sE
\edm %----------------------------------------------------------------------
が成り立つ.
これは $\ol{B}u\in \Dom(\ol{B}^*)$ で
 $\ol{B}^*\,\ol{B}u=\ol{A}^*\, \ol{A}u$ を意味する.
同様に $u\in\Dom(\ol{B}^*\, \ol{B})$ をとると $\ol{B}u\in \Dom(\ol{A}^*)$
で  $\ol{A}^*\,\ol{A}u=\ol{B}^* \,\ol{B}u$ が成り立っていることを意味する.
従って $\ol{A}^*\,\ol{A} = \ol{B}^*\, \ol{B}$ であるが,
$\ol{A}^*=\ol{B}$ であったから $\ol{A}^* \,\ol{A} = \ol{A}\, \ol{A}^*$
が成立する.
これは即ち $\ol{A}$ が正規作用素であることを意味している.
\QED %======================================================================

この結果を使って Riemannian manifold の場合を考えよう.
$M$ を Riemannian manifold として,完備性を常に仮定しておく.
作用素の正規性を言うには可換性を調べる必要があるが,
そのためには Killing vector field
の概念が必要になるので,まずそのことの準備をする.
ベクトル場 $X$ が Killing field であることは $L_X g = 0$ が
成り立つことであった.
このとき $X$ は等距離変換群を定義する.
$L_X$ は Lie 微分を表す.

\Proposition{CNO-2} %*******************************************************
$X$ が Killing vector field であるための必要十分条件は
$v\mapsto \nabla_v X$ が歪対称であること,即ち
\bdn %----------------------------------------------------------------------
g(\nabla_v X, w) + g(\nabla_w X, v)
= 0, \quad \forall v,\, w
\Eqn{CNO.12}
\edn %----------------------------------------------------------------------
が成り立つことである.
\end{proposition} %*********************************************************

\Proof
$\xi=X^\flat$ とするとき,$V$, $W$ を任意のベクトル場として
\bdn %----------------------------------------------------------------------
d\xi(V,W) + (L_X g)(V, W)
= 2g(\nabla_V X, W)
\Eqn{CNO.14}
\edn %----------------------------------------------------------------------
が成り立っている.
これは \cite{Petersen06} の Chapter~2, \S 1 に出ている.
\memo{むしろ Petersen ではこれを定義にして議論を進めている.}
これから $L_Xg=0$ ならば $(V,W) \mapsto g(\nabla_V X, W)$ が歪対称になる.
逆に $(V,W) \mapsto g(\nabla_V X, W)$ が歪対称ならば,$L_X g$ が
歪対称になるが,$L_X g$ は対称でもあるので $L_Xg=0$ となる.
\QED %======================================================================

\Theorem{CNO-6} %***********************************************************
$X$ が Killing vector fileld とする.
$\xi=X^\flat$ とおくと次が成り立つ:
\bdmn %---------------------------------------------------------------------
\nabla^* \nabla \xi
&= \Ric(\xi),
\Eqn{CNO.18} \\
\nabla^* \xi
&= 0.
\Eqn{CNO.20}
\edmn %----------------------------------------------------------------------
\end{theorem} %*************************************************************

\Proof
\Eq{CNO.14} の関係式から $d\xi= 2\nabla\xi$ である.
一方
\bdm %----------------------------------------------------------------------
\nabla^* \xi
= - \tr \nabla\xi
= -\frac{1}{2} \tr d\xi
\edm %----------------------------------------------------------------------
で,$d\xi$ は skew symmetric だから $\nabla^* \xi=0$ が成り立つ.
さらに,一般に微分形式に対して $d^*\w = \nabla^* \w$ が成り立つので
\bdm %----------------------------------------------------------------------
2 \nabla^* \nabla \xi
&= 2 d^* \nabla \xi \\
&= d^* d \xi \quad (\because\ 2\nabla\xi=d\xi) \\
&= (d^* d + d d^*) \xi \quad (\because\ d^*\xi=0) \\
&= (\nabla^*\nabla + \Ric) \theta.
\edm %----------------------------------------------------------------------
よって $\nabla^* \nabla \xi = \Ric(\theta)$ が成り立つ.
\QED %======================================================================

多様体がコンパクトだと上の逆が成立する.
これを示すには更に準備が必要になる.
次の結果は Petersen \cite{Petersen06} のp.~230 に述べてある.
証明がないのでつけておく.

\Proposition{CNO-8} %*******************************************************
$M$ がコンパクトな多様体であるとき,ベクトル場 $X$ に対して
次の等式が成立する.
\bdn %----------------------------------------------------------------------
\int_M (\Ric(X),X) + \tr\{(\nabla X)^2\} - (\div X)^2) d\m
= 0.
\Eqn{CNO.22}
\edn %----------------------------------------------------------------------
ここで $\nabla X$ は $\Hom(T(M))$ の元とみている.
$\tr$ もその意味である.
\end{proposition} %*********************************************************

\Proof
いくつか予備的な等式が必要である.
まず $\div \nabla_X X$ を計算する.
\bdm %----------------------------------------------------------------------
\div(\nabla_X X)
&= \div(X^i \nabla_i X) \\
&= - \la \nabla_j(X^i \nabla_i X), dx^j\ra \\
&= - \la \nabla_j X^i \nabla_i X, dx^j \ra 
   - \la X^i  \nabla_j\nabla_i X, dx^j\ra \\
&= - \nabla_j X^i \la \nabla_i X, dx^j \ra 
   - X^i \la \nabla_j\nabla_i X, dx^j\ra \\
&= - \nabla_j X^i \la \nabla_i X, dx^j \ra 
   - X^i \la (\nabla_j\nabla_i - \nabla_i\nabla_j) X, dx^j\ra
   - X^i \la \nabla_i\nabla_j X, dx^j\ra \\
&= - \nabla_j X^i \la \nabla_i X, dx^j \ra 
   - X^i \la R(\del_j,\del_i)X, dx^j\ra
   - X^i \la \nabla_i\nabla_j X, dx^j\ra \\
&= - \nabla_j X^i \la \nabla_i X, dx^j \ra 
   - \Ric(X,X) - X^i \la \nabla_i\nabla_j X, dx^j\ra.
\edm %----------------------------------------------------------------------
次に $\div((\div X)X)$ を計算する.
\bdm %----------------------------------------------------------------------
\div((\div X)X)
&= - \la \nabla_i((\div X)X), dx^i\ra \\
&= \la \nabla_i((\la \nabla_j X, dx^j\ra X), dx^i\ra \\
&= \la \del_i \la \nabla_j X, dx^j\ra X, dx^i\ra
   + \la \la \nabla_j X, dx^j\ra \nabla_i X, dx^i\ra \\
&= \la (\la \nabla_i \nabla_j X
         + \la \nabla_j X,\nabla_i dx^j \ra) X, dx^i\ra
   + \la \nabla_j X, dx^j\ra \la \nabla_i X, dx^i\ra \\
&= X^i \la \nabla_i \nabla_j X dx^j \ra
         + X^i \la \nabla_j X,\nabla_i dx^j \ra + (\div X)^2.
\edm %----------------------------------------------------------------------
ここで
\bdm %----------------------------------------------------------------------
\nabla_j X^i = \la \nabla_j X,dx^i\ra + \la X, \nabla_j dx^i\ra
\edm %----------------------------------------------------------------------
であることに注意して
\bdm %----------------------------------------------------------------------
\div(\nabla_X X) + \div((\div X)X)
&= -(\la \nabla_j X, dx^i\ra
   + \la X, \nabla_j dx^i\ra) \la\nabla_i X,dx^j\ra
   - \Ric(X,X) \\
&\squad
   + X^i \la \nabla_j X,\nabla_i dx^j \ra + (\div X)^2 \\
&= - \la \nabla_j X, dx^i\ra \la\nabla_i X,dx^j\ra - \Ric(X,X) \\
&\squad
   + (\div X)^2 - \la X, \nabla_j dx^i\ra \la \nabla_i X,dx^j\ra
   + X^i \la \nabla_j X, \nabla_i dx^j\ra.
\edm %----------------------------------------------------------------------
ここで上の最後の2項が $0$ になることを示せば,求める結果に成る.
最後の2項は
\bdm %----------------------------------------------------------------------
- \la X, \nabla_j dx^i\ra \la \nabla_i X,dx^j\ra
  + X^i \la \nabla_j X, \nabla_i dx^j\ra
&= \la X, \Gm_{jk}^i dx^k\ra \la \nabla_i X, dx^j\ra
  - X^i\la\nabla_j X, \Gm_{ik}^j dx^k\ra \\
&= X^k \Gm_{jk}^i \la \nabla_i X, dx^j\ra
  - X^i \Gm_{ik}^j \la\nabla_j X,  dx^k\ra \\
&=0. 
\edm %----------------------------------------------------------------------
これで証明できた.
\QED %======================================================================
\hide
$\div((\div X)X)$ の方は $\nabla^*(\nabla^*\xi,\xi))$ であることに注意して
\bdm %----------------------------------------------------------------------
\nabla^*(\nabla^*\xi,\xi))
&= - \la \nabla\nabla^* \xi,\xi\ra + (\nabla^*\xi)^2 \\
&= - \la dd^* \xi,\xi\ra + (\nabla^*\xi)^2 \\
&= - \la (dd^* + d^*d)\xi,\xi\ra + \la d^*d\xi,\xi\ra
   + (\nabla^*\xi)^2
\edm %----------------------------------------------------------------------
と表現できる.
他方も同じような考えで変形できないか.
そうすれば計算が簡単になるのではないか.
\endhide

\hide
Kobayashi \cite{Kobayashi95}(こちらは \cite{Kobayashi72} のreprint版である) の p.~155 に対応する証明が載っているが,follow 出来ない.
間違っているように思うのだが.
Yono-Bochner \cite{YB53} p.~57 にはテンソル計算できちんと示してある.
\endhide

$X \in \Gm(T(M))$  に対して $\nabla X\in \Gm(\Hom(T(M),T(M))$
と見ているわけであるが,この転置を ${}^t\nabla X\in \Gm(\Hom(T^*(M),T^*(M))$
と表すことにする.
$\nabla X$ と ${}^t\nabla X$ の合成を考えたいが,
このままでは意味を成さないので,同型
$\sharp\colon T^*(M) \to T(M)$,
$\flat\colon T(M) \to T^*(M)$ を用いて $T(M)$ と $T^*(M)$ を同一視して
${}^t\nabla X$ を $\Gm(\Hom(T(M),T(M))$ のように考える.
より正確に言えば,${}^t\nabla X$ の代わりに,次の合成
\bdm %----------------------------------------------------------------------
\begin{CD}
T(M) @>\flat>> T^*(M) @>{{}^t\nabla X}>> T^*(M) @>\sharp>> T(M)
\end{CD}
\edm %----------------------------------------------------------------------
$\sharp\maru {}^t\nabla X \maru \flat$ を考えることである.
煩瑣ではあるが,数学的に厳密な $\sharp\maru {}^t\nabla X \maru \flat$ 
の方を用いることにして,${}^t\nabla X$ は本来の $\Gm(\Hom(T^*(M),T^*(M))$
の元を表すものとする.
basis として $\del_i$, $dx^j$ を取って成分表示すると
\bdm %----------------------------------------------------------------------
\flat_{ij} = g_{ij},\quad
\sharp^{ij} = g^{ij},\quad
(\nabla X)_i^j = \la \nabla_i X, dx^j\ra
\edm %----------------------------------------------------------------------
であるから
\bdm %----------------------------------------------------------------------
(\sharp\maru {}^t\nabla X\maru \flat)_i^l
= g_{ij} (\nabla X)_k^j g^{kl}
\edm %----------------------------------------------------------------------
という成分表示が得られる.

\Proposition{CNO-10} %******************************************************
ベクトル場 $X$, $Y$ に対して次が成立する:
\bdn %----------------------------------------------------------------------
(\nabla X, \nabla Y)
= \tr(\nabla X\maru \sharp \maru {}^t\nabla Y \maru \flat)
\Eqn{CNO.24}
\edn %----------------------------------------------------------------------
\end{proposition} %*********************************************************

\Proof
$(\nabla X)_i^j = \la \nabla_i X,dx^j\ra$ であったから
\bdm %----------------------------------------------------------------------
(\nabla X, \nabla Y)
&= ((\nabla X)_i^j dx^i \otimes \del_j,
    (\nabla Y)_k^l dx^k \otimes \del_l) \\
&= (\nabla X)_i^j (\nabla Y)_k^l g^{ik} g_{jl} \\
&= (\nabla X)_i^j (\sharp \maru \nabla Y \maru \flat)_j^i \\
&= \tr( \nabla X \maru \sharp \maru \nabla Y \maru \flat).
\edm %----------------------------------------------------------------------
これが示すべきことである.
\QED %======================================================================

以上で逆を示す準備が出来た.

\Theorem{CNO-12} %**********************************************************
$M$ をコンパクトな Riemannian manifold で,ベクトル場 $X$ に対して
$\xi=X^\flat$ とおくとき次が成り立っているとする.
\bdmn %---------------------------------------------------------------------
\nabla^* \nabla X
&= \Ric(X), 
\Eqn{CNO.30} \\
\div X
&= 0.
\Eqn{CNO.32}
\edmn %----------------------------------------------------------------------
このとき $X$ は Killing vector field になる.
\end{theorem} %*************************************************************

\Proof
\Prop{CNO-8} から
\bdm %----------------------------------------------------------------------
\int_M (\Ric(X),X) + \tr\{(\nabla X)^2\} - (\div X)^2) d\m
= 0
\edm %----------------------------------------------------------------------
が成り立つ.
ここで \Eq{CNO.30} を使うと
\bdm %----------------------------------------------------------------------
0
&= \int_M (\nabla^*\nabla X, X) + \tr\{(\nabla X)^2\} \,d\m \\
&= \int_M (\nabla X, \nabla^X) + \tr\{(\nabla X)^2\} \,d\m.
\edm %----------------------------------------------------------------------
さらに \Eq{CNO.32} を使えば
\bdm %----------------------------------------------------------------------
0
&= \int_M [\tr( \nabla X\maru \sharp\maru {}^t\nabla X\maru\flat
           + \tr\{(\nabla X)^2\}] \,d\m \\
&= \frac{1}{2} \int_M
   \tr\{ (\nabla X + \sharp\maru {}^t\nabla X\maru\flat)^2\}\,d\m.
\edm %----------------------------------------------------------------------
ここで $\nabla X + \sharp\maru {}^t\nabla X\maru\flat$ が
対称であることに注意しよう.
実際
\bdm %----------------------------------------------------------------------
g(\sharp\maru {}^t\nabla X\maru\flat(v), w)
= \la  {}^t\nabla X\maru\flat(v), w\ra
= \la  \flat(v), \nabla_w X \ra
=g(v,\nabla_w X)
\edm %----------------------------------------------------------------------
から対称であることが分かる.
従って2乗すれば非定値となり,積分して $0$ になることが上で示せているので
$\tr\{ (\nabla X + \sharp\maru {}^t\nabla X\maru\flat)^2\}=0$,
さらには
\bdm %----------------------------------------------------------------------
\nabla X + \sharp\maru {}^t\nabla X\maru\flat = 0
\edm %----------------------------------------------------------------------
が示せる.
これから上の対称性と合わせて
$(v,w) \mapsto g(\nabla_v X,w)$ が skew-symmetric
であることが従う.
よって \Prop{CNO-2} から $X$ は Killing vector field であることが分かる.
\QED %======================================================================

$M$ がコンパクトでなければこの定理は成り立たない.
$\R^2$ の場合に簡単に見ておこう.
$\R^2$ の座標を $(x,y)$ とかく.
$\C=\R^2$ とみて, $\C$ で正則な関数 $f$ をとり,$f=u+iv$ と
実部と虚部に分解しておく.
Cauchy-Riemann の関係式から $u_x=v_y$, $u_y=-v_x$ が成り立つ.
そこで ベクトル場 $X$ を $X= u\frac{\del}{\del x}- v\frac{\del}{\del y}$
と定めると,$\div X=0$ はすぐに分かる.
また
$\nabla^*\nabla X
= - \Laplace u\frac{\del}{\del x} + \Laplace v\frac{\del}{\del y}=0$
もすぐに分かる.
これで \Eq{CNO.30}, \Eq{CNO.32} を満たすが,Killing vector field
でないものが容易に作れる.

\bigskip
さて,正規作用素の話に戻ろう.
この場合に必要となるのは \Prop{CNO-2} と \Thm{CNO-6} だけである.
\memo{その意味では,やや余分の話をしたことになる.}

% \tb を \tilde{b} としていたがここでは特に tilde をつける意味はないから
% はずしておく.別の文脈ではつけたほうが自然であろうから,そこでも使える
% ようにしておく.
\renewcommand{\tb}{b}


\bigskip
$M$ を完備なリーマン多様体とする.
$\m$ を Riemannian volume として,測度 $\nu=e^{-U}\m$ のもとで
 $\gen = -\frac{1}{2} \nabla_\nu^* \nabla + \tb$
という作用素を $L^2(\nu)$ で考える.
$\nabla_\nu^*$ は,測度 $\nu$ に対する $\nabla$ の共役で
Riemannian volume に対する共役 $\nabla^*$ を用いて
$\nabla_\nu^* = \nabla^* + (\nabla U, \cdot)$ と表される.
この作用素が正規である条件を求めたいわけである.

\hide
$\div_\nu\tb=0$ を始めは仮定していた.
それで以下ではこの条件をつけないでやってみたが,条件が綺麗にならない.
$\div_\nu\tb=0$ が出てくると思ったのだが.
この条件は $\gen^*(e^{-U}) = 0$ と同値であった.
これと正規性と関連付けるとかできないのだろうか.
とにかく今のところうまく行っていない.
\hfill [2011年1月6日]
\endhide
$L^2(\nu)$ での(形式的な)共役作用素は
$\gen_\nu^*= -\frac{1}{2} \nabla_\nu^* \nabla - \tb -\div_\nu\tb$ となる.
$\gen_\nu^*$ と $\nu$ を添え字につけたのは
 $\nu$ に関する共役という意味である.
$\div_\nu$ も $\nu$ に対する共役という意味である.
$\nu$ をつけないときは Riemannian volume $\m$ に対する共役を表すものとする.
\bdn %----------------------------------------------------------------------
\Laplace_\nu
= - \nabla_\nu^*\nabla
= - \nabla^*\nabla - \nabla U \cdot \nabla
\Eqn{CNO.66}
\edn %----------------------------------------------------------------------
とおくと,
\bdmn %---------------------------------------------------------------------
\gen
&= \frac{1}{2} \Laplace_\nu + \nabla_\tb,
\Eqn{CNO.68} \\
\gen_\nu^*
&= \frac{1}{2} \Laplace_\nu - \nabla_\tb - \div_\nu\tb.
\Eqn{CNO.70}
\edmn %---------------------------------------------------------------------
であるから,
\bdm %----------------------------------------------------------------------
\gen_\nu^* \gen - \gen \gen_\nu^* 
&= (\frac{1}{2} \Laplace_\nu - \nabla_\tb -\div_\nu\tb)
  (\frac{1}{2} \Laplace_\nu + \nabla_\tb)
  -(\frac{1}{2} \Laplace_\nu + \nabla_\tb)
   (\frac{1}{2} \Laplace_\nu - \nabla_\tb -\div_\nu\tb) \\
&= \Laplace_\nu \nabla_\tb - \nabla_\tb \Laplace_\nu
   + [\frac{1}{2}\Laplace_\nu + \nabla_\tb, \div_\nu\tb] \\
&= [\Laplace_\nu, \nabla_\tb]
   + [\frac{1}{2}\Laplace_\nu+\nabla_\tb, \div_\nu\tb].
\edm %----------------------------------------------------------------------
従って可換である条件は
$[\Laplace_\nu,\nabla_\tb]+[\frac{1}{2}\Laplace_\nu+\nabla_\tb, \div_\nu\tb]=0$
である.
このための条件を求めればよい.
\Theorem{CNO-10} %**********************************************************
$\gen$ と $\gen_\nu^*$ が可換であるための必要十分条件は $\tb$ が
Killing vector field であり,次の等式が成立することである.
\bdmn %---------------------------------------------------------------------
(\frac{1}{2}\Laplace_\nu + \nabla_\tb)\div_\nu \tb
= 0,
\Eqn{CNO.74} \\
[(\nabla U)^\sharp,\tb] + \nabla\div_\nu\tb
= 0.
\Eqn{CNO.76}
\edmn %---------------------------------------------------------------------
\end{theorem} %*************************************************************

\Proof
$\tb$ に対応する 1-form を $\tw$ とする.

まず一般のテンソル $\theta$, $\eta$ に対し
\bdm %----------------------------------------------------------------------
\Laplace(\theta,\eta)
= 2(\nabla \theta, \nabla \eta) - (\nabla^*\nabla\theta, \eta)
-(\theta, \nabla^*\nabla\eta)
\edm %----------------------------------------------------------------------
が成り立つことに注意しよう
(例えば,\cite{Petersen06} の Chapter 7, \S 3 か同じ章の \S 7, Exercise 13
を見よ.)
 特に 1-form の場合,$\theta=\tw$, $\eta=\nabla f$ として
\bdm %----------------------------------------------------------------------
\Laplace(\tw, \nabla f)
= 2(\nabla \tw, \nabla^2 f) - (\nabla^*\nabla\tw, \nabla f)
-(\tw, \nabla^*\nabla\nabla f).
\edm %----------------------------------------------------------------------
ここで 1-form に対して $dd^*+d^*d = \nabla^*\nabla + \Ric$ であることと
\bdm %----------------------------------------------------------------------
(dd^*+d^*d)\nabla f = \nabla (dd^*+d^*d)f = \nabla \nabla^*\nabla
\edm %----------------------------------------------------------------------
を使うと
\bdm %----------------------------------------------------------------------
\Laplace(\tw, \nabla f)
&= 2(\nabla \tw, \nabla^2 f) - (\nabla^*\nabla\tw, \nabla f)
  -(\tw, (dd^*+d^*d)\nabla f) +(\tw, \Ric(\nabla f)) \\
&= 2(\nabla \tw, \nabla^2 f) - (\nabla^*\nabla\tw, \nabla f)
  -(\tw, \nabla \nabla^*\nabla f) +(\tw, \Ric(\nabla f))
\edm %----------------------------------------------------------------------
$(\tw, \nabla f) = \nabla_\tb f$ であるから
\bdm %----------------------------------------------------------------------
[\Laplace, \nabla_\tb]f
= 2(\nabla \tw, \nabla^2 f) - (\nabla^*\nabla\tw - \Ric(\tw), \nabla f).
\edm %----------------------------------------------------------------------
$\Laplace_\nu = \Laplace + \nabla U\cdot\nabla$ だから
\bdm %----------------------------------------------------------------------
[\Laplace_\nu, \nabla_\tb]f
&= [\Laplace + \nabla U\cdot\nabla, \nabla_\tb]f \\
&= [\Laplace, \nabla_\tb]f
  + \nabla_{[\nabla U^{\sharp}, \tb]}f \\
&= 2 (\nabla\tw, \nabla^2 f)
  + (-\nabla^*\nabla\tw + \Ric(\tw) + [\nabla U^\sharp,\tb]^\flat, \nabla f)
\edm %----------------------------------------------------------------------
また $[\frac{1}{2}\Laplace_\nu+\nabla_\tb, \div_\nu\tb]$ の方は
\bdm %----------------------------------------------------------------------
[\frac{1}{2}\Laplace_\nu+\nabla_\tb, \div_\nu\tb]f
&= (\frac{1}{2}\Laplace_\nu+\nabla_\tb)(\div_\nu\tb f)
  - \div_\nu\tb(\frac{1}{2}\Laplace_\nu+\nabla_\tb)f \\
&= \frac{1}{2}\{ (\Laplace_\nu\div_\nu\tb)f+2\nabla\div_\nu\tb\cdot\nabla f
   + \div_\nu \tb\Laplace_\nu f\} \\
&\squad
   + (\nabla_\tb\div_\nu \tb)f + \div_\nu \tb \nabla_\tb f)
   - \div_\nu\tb(\frac{1}{2}\Laplace_\nu+\nabla_\tb)f \\
&= (\frac{1}{2}\Laplace_\nu\div_\nu\tb + \nabla_\tb\div_\nu \tb)f
   + \nabla\div_\nu\tb\cdot\nabla f.
\edm %----------------------------------------------------------------------
両者をあわせて
\bdm %----------------------------------------------------------------------
[\Laplace_\nu, \nabla_\tb]f
 + [\frac{1}{2}\Laplace_\nu+\nabla_\tb, \div_\nu\tb]f
&= 2 (\nabla\tw, \nabla^2 f) \\
&\squad
  + (-\nabla^*\nabla\tw + \Ric(\tw) + [\nabla U^\sharp,\tb]^\flat
      + \nabla\div_\nu\tb, \nabla f) \\
&\squad
  + (\frac{1}{2}\Laplace_\nu\div_\nu\tb + \nabla_\tb\div_\nu \tb)f.
\edm %----------------------------------------------------------------------
これが全ての $f$ に対して 0 であればよいことが,可換性の必要十分条件である.
$f=1$ として 
\bdn %----------------------------------------------------------------------
(\frac{1}{2}\Laplace_\nu + \nabla_\tb)\div_\nu \tb
= 0
\Eqn{CNO.80}
\edn %----------------------------------------------------------------------
が従う.
さらに,点 $x$ をとめれば,$\nabla f(x)=0$ で $\nabla^2 f$ が
任意の対称行列を取るように出来る.
従って可換性の条件は \Eq{CNO.70} と
\bdmn %---------------------------------------------------------------------
&\widehat{\nabla\tw}= 0,
\Eqn{CNO.82} \\
&-\nabla^*\nabla\tw + \Ric(\tw) + [\nabla U^\sharp,\tb]^\flat
 + \nabla\div_\nu\tb= 0
\Eqn{CNO.84}
\edmn %---------------------------------------------------------------------
である.
ここで $\widehat{\nabla\tw}$ は $\nabla\tw$ の対称部分を表す.
まず $\widehat{\nabla\tw}= 0$ は $\nabla\tw$ が skew symmetric
であることを意味している.
従って \Prop{CNO-2} から $\tb$ は Killing field である.
ここで Killing field の性質 \Eq{CNO.18} と条件 \Eq{CNO.84} を使えば
\bdn %----------------------------------------------------------------------
[(\nabla U)^\sharp,\tb] + \nabla\div_\nu\tb
= 0
\Eqn{CNO.86}
\edn %----------------------------------------------------------------------
が従う.

逆に $\tb$ が Killing vector fileld であり,
\Eq{CNO.80}, \Eq{CNO.86} が成り立っているとする. 
すると,$\tb$ がKilling であることから $\nabla\tw$ がskey symmetricとなり
また \Eq{CNO.18} が成り立っている.
よって \Eq{CNO.84} が成り立っていることが分かる.
先の計算により $\gen$ と $\gen_\nu^*$ が可換になる.
\QED %======================================================================

$M$ がコンパクトのときは
$\gen 1=0$, $\gen_\nu^* 1=-\div_\nu\tb$ であるが,可換性が成り立てば 
$\|\gen 1\|_2=\|\gen_\nu^*1\|_2$ が成り立つから $\div_\nu\tb=0$ が成立する.
従って \Eq{CNO.76} の条件は $[(\nabla U)^\sharp, \tb]=0$ と少し簡単になる.
\hide
non-compact の場合も $\div_\nu\tb=0$ が出せるといいのだが.
$\nu$ が有限測度の場合も同じだと思ったが,$1\in L^2(\nu)$
は成り立つが $1\in\Dom(\gen)$ は成り立つのだろうか.
成り立つとしても $\gen 1=0$ となるのだろうか.
定義としては $C_0^\infty(M)$ の元で近似することになるから
$\gen 1=0$ も決して自明ではないことになる.
$1$ に収束する近似列を作らなければならない.
そうすると $\tb$ の増大条件を課さなければならなくなるのではないか.
結局あまり単純な条件ではなくなるような気がする.[2011年1月6日]
\endhide

non-compact の場合は必ずしも $\div_\nu\tb=0$ は成り立たない.
$M=\R$, $U=cx$, $\nu(dx) = e^{-cx}dx$ の場合を考えよう.
このとき
\bdm %----------------------------------------------------------------------
(\frac{d}{dx})_\nu^*
= -\frac{d}{dx} + c
\edm %----------------------------------------------------------------------
である.
$\gen = -\frac{1}{2}(\frac{d}{dx})_\nu^* \frac{d}{dx}+ k\frac{d}{dx}$
とする.
$b= k\frac{d}{dx}$ で $\div_\nu b = \div b - bU = -kc$
また
\bdm %----------------------------------------------------------------------
[\nabla U^{\sharp},b]
= [c\frac{d}{dx}, k\frac{d}{dx}]
= 0
\edm %----------------------------------------------------------------------
だから \Thm{CNO-10} の条件はすべてみたしている.
従ってこの作用素は(閉包をとれば)正規作用素になる.
しかし $\div_\nu b = 0$ は成立していない.
後の例でこの作用素のスペクトルを求めてみる.

\bigskip
上の定理は計算が自由に行えるところでの話である.
例えば滑らかな関数のクラス $C_0^\infty(M)$ では正当化できる.
\Thm{CNO-1} によれば,正規性を示すには $C_0^\infty(M)$ で定義された
$\gen$ の閉包が $m$-dissipative を示さなければならない.
以下ではこの条件を付加して $m$-dissipative であることを示そう.

\Theorem{CNO-14} %**********************************************************
$\tb$ を Killing vector field で $\div_\nu\tb$ が下に有界を仮定する.
このとき $C_0^\infty(M)$ を定義域とする $\gen$, $\gen_\nu^*$ の閉包は
$m$-dissipative である.
\end{theorem} %*************************************************************

\Proof
十分条件が Shigekawa \cite{Shigekawa10} に与えられているからそれを確かめる.
基準の点 $x_0\in M$ を取り固定する.
また $\rho(x)=d(x,x_0)$ を $x_0$ からの距離とする.
ベクトル場 $\tb$ で生成される1-パラメーター変換群を $\vph_t$ で表す.
$\vph_t$ は次の微分方程式をみたす.
\bdm %----------------------------------------------------------------------
\frac{d}{dt}\vph_t(x) = \tb(\vph_t(x)).
\edm %----------------------------------------------------------------------
$|t| \le 1$ で考えると $\tb(\vph_t(x_0))$ は有界であるから
\bdm %----------------------------------------------------------------------
|d(\vph_t(x_0),x_0)|
\le \int_0^t |\tb(\vph_s(x_0))|ds
\le K|t|
\edm %----------------------------------------------------------------------
をみたす $K$ が存在する.
これから
\bdm %----------------------------------------------------------------------
\rho(\vph_t(x))
&= d(\vph_t(x),x_0)
\le d(\vph_t(x),\vph_t(x_0)) + d(\vph_t(x_0),x_0) \\
&\le d(x,x_0) + K|t|
\le \rho(x) + K|t|.
\edm %----------------------------------------------------------------------
同様に $\rho(\vph_t(x)) \ge \rho(x) - K|t|$ も示せるから
\bdm %----------------------------------------------------------------------
|\rho(\vph_t(x)) - \rho(x)|
\le K|t|.
\edm %----------------------------------------------------------------------
ここで 
\bdm %----------------------------------------------------------------------
\tb\rho(x)
= \lim_{t\to 0} \frac{\rho(\vph_t(x)) - \rho(x)}{t}
\edm %----------------------------------------------------------------------
であるから $|\tb\rho(x)|\le K$ が成立する.

さて,閉包が $m$-dissipative であるための十分条件が \cite{Shigekawa10}
で次のように与えられている.
非増加関数 $\kappa:[0,\infty) \to [0,1]$ で
$\int_0^\infty \kappa(x) = \infty$ をみたす関数 $\kappa$ が存在して
$\kappa(\rho(x))\tb\rho(x) \ge -1$ をみたす.

ここでは $\kappa=\frac{1}{K}$ とすればよい.
$\gen_\nu^*$ のときは $-\tb$ を考えればいいので同様に示せる.
\QED %======================================================================

以上纏めれば次が得られた.

\Theorem{CNO-16} %**********************************************************
$\gen$ は $\tb$ が Killing vector field で $\div_\nu\tb$ は下に有界であるとする.
このとき,$\gen$ が正規であるための必要十分条件は $\tb$ が
Killing vector field であり \Eq{CNO.74}, \Eq{CNO.76} が成り立つことである.
\end{theorem} %*************************************************************


最後に簡単な正規作用素の例を挙げておこう.
$\R^2$ で $\nu = \frac{1}{2\pi} e^{-(x^2+y^2)/2} dxdy$
とし,$\tb = c(y\frac{\del}{\del x} - x\frac{\del}{\del y})$
とする.
すると $\tb$ はKilling vector field である.
$U=\frac{x^2+y^2}{2} + \log 2\pi$ とおくと $\nu=e^{-U}dxdy$ である.
このとき $\div\tb=0$ で $\tb U = c(yx - xy)=0$ であるから
$\div_\nu\tb = 0$ である.
さらに $\nabla U^\sharp = x\frac{\del}{\del x} + y \frac{\del}{\del y}$
であるから,
\bdm %----------------------------------------------------------------------
[\tb, \nabla U^\sharp]
&= c[y\frac{\del}{\del x} - x\frac{\del}{\del y},
    x\frac{\del}{\del x} + y \frac{\del}{\del y}] \\
&= c[y\frac{\del}{\del x},x\frac{\del}{\del x}]
  + c[y\frac{\del}{\del x},y \frac{\del}{\del y}]
  - c[x\frac{\del}{\del y},x\frac{\del}{\del x}]
  - c[x\frac{\del}{\del y},y\frac{\del}{\del y}] \\
&= cy\frac{\del}{\del x} - cy\frac{\del}{\del x}
  + cx\frac{\del}{\del y} - cx\frac{\del}{\del y}
= 0.
\edm %----------------------------------------------------------------------
よって,$\gen = -\frac{1}{2} \nabla_\nu^*\nabla + \tb$ は
$L^2(\nu)$ での正規作用素である.
\hide
では,スペクトルはどうなっているのかというのが次の問題になる.
この問題は池野君に解いてもらった.
複素 Hermite 多項式を用いるとスペクトルが完全に決定できる.
それはとても面白かった.また Laguerre 多項式との繋がりも分かってきたし.
[2011年1月5日]
\endhide

\subsec{Notes}
\begin{itemize}
\item 正規作用素の例では $\div_\nu\tb = 0$ の条件を付加して
考えたことになるが,この条件がなければどうなるのだろう.
なんとか $\div_\nu\tb = 0$ の条件を出してきたいのだが.
[2011年1月6日]
\item 本文中に追加してあるが $\div_\nu\tb = 0$ が成り立たない
正規作用素の例が存在する.
但しまだ $\div_\nu\tb = {\rm const.}$ である.
constant 以外の例は思いつかない.
[2011年1月15日]
\end{itemize}

	%%%%%%%%%%%%%%%%%%%%%%%%%%%%%%%%%%%%%%%%%%%%%%%%%
%                                               %
%         ====== Program  No.6 =======          %
%                                               %
%             file name snj06.tex               %
%                                               %
%===============================================%
%===================  for  =====================%
%===============================================%
%
\hide
\vspace{-4mm}
\begin{itemize} \itemsep=-2mm \parsep=0mm
\item Total file name: snj01 snj02 $\dots $ snj?, snj\_bibliography
\item File name: snj06.tex \hfill 印刷日: \today \ \now
\item 単調作用素のことを纏めておく.Lions に従う.
それにしても Lions は凄い.
非線型の場合が念頭にあるが,後でStannat の generalized Dirichlet form の
話をするのに,非線型の話が必要になる.
Stannat の本には証明を書いていないので,ノートとして纏めておく.
Lions の本はフランス語だけれど,英語の文献が何かあると思うのだが分からない.
[2011年2月24日]
\end{itemize}
\endhide
\SS{FPM}{単調作用素の基本的性質} %==========================================
% FPM
後で必要になる単調作用素に関する定理を準備しておく.
$V$ を実 Banach 空間とする.
双対空間を $V^*$ とかく.
\Definition{FPM-2} %********************************************************
写像 $T\colon V\to V^*$ が,すべての $u$, $v\in V$ に対し
\bdn %----------------------------------------------------------------------
(Tu-Tv, u-v) \ge 0
\Eqn{FPM.6}
\edn %----------------------------------------------------------------------
が成り立つとき{\bf 単調}であるという.
さらに $(Tu-Tv, u-v) =0$ となるのは $u=v$ のときに限るとき,
{\bf 狭義単調}であるという.
\end{definition} %**********************************************************

\Remark{FPM-4} %************************************************************
複素 Banach 空間の場合は \Eq{FPM.6} を $\Re(Tu-Tv, u-v) \ge 0$
の条件に替える.
また $T$ も一般には多価関数に対して定義するが,
ここではそこまで一般にはしない.
さらに深入理論は田辺に述べてあるが,ここでは Lions に従って
以下の議論を進める.
\end{remark} %**********************************************************


一般にBanach 空間から Banach 空間への写像が任意の有界集合を有界集合に
移すとき,{\bf 有界}であるという.
さらに後で必要になる連続性について述べておこう.


\Definition{FPM-6} %********************************************************
写像 $T\colon V\to V^*$ が線分上弱連続であることを,
すべての $u$, $v$, $w\in V$ に対し
\bdn %----------------------------------------------------------------------
[0,1]\ni t\mapsto (T((1-t)u+tv),w)
\Eqn{FPM.8}
\edn %----------------------------------------------------------------------
が連続であることと定義する.
\end{definition} %**********************************************************

さて単調写像の連続性に関して次が成り立つ.

\Proposition{FPM-10} %******************************************************
$V$ を反射的な Banach 空間とする.
$T\colon V\to V^*$ が有界単調写像で線分上弱連続であるとき,
任意の $v\in V$ に対して $u\mapsto (Tu,v)$ は連続である.
即ち $T$ は $V$(強位相) から $V^*$(弱位相) への連続な写像である.
\end{proposition} %*********************************************************

\Proof
$u_n$ が $u$ に強収束しているとする.
$\{Tu_n\}$ は有界集合だから弱収束する部分列を持つ.
部分列のとり方によらず極限が $Tu$ であることを示せばよい.
簡単のために $\{Tu_n\}$ が $f$ に弱収束しているとする.

さて,単調性から任意の $w$ に対し
\bdm %----------------------------------------------------------------------
(Tu_n - Tw, u_n-w)
\ge 0
\edm %----------------------------------------------------------------------
が成り立つ.
$w=(1-t)u+tv$ $t\in(0,1)$ ととると.
\bdm %----------------------------------------------------------------------
0
&\le (Tu_n - Tw, u_n-w)
= (Tu_n - Tw, u_n-(1-t)u -tv) \\
&= (Tu_n - Tw,u_n-u) + t(Tu_n - Tw, u-v).
\edm %----------------------------------------------------------------------
ここで $n\to\infty$ として
\bdm %----------------------------------------------------------------------
0
\le  t(f - Tw, u-v).
\edm %----------------------------------------------------------------------
よって $t$ で割って
\bdm %----------------------------------------------------------------------
(f, u-v)
\ge (Tw, u-v).
\edm %----------------------------------------------------------------------
さらに $t\to0$ とすると,右辺は $T$ の線分上弱連続性から
$(Tu, u-v)$ に収束する.
よって
\bdm %----------------------------------------------------------------------
(f, u-v)
\ge (Tu, u-v).
\edm %----------------------------------------------------------------------
これが任意の $v$ について成り立つので $Tu=f$ が従う.
これが示すべきことであった.
\QED %======================================================================

\hide
P.-L.~Lions \cite{Lions69} p.~179, Propposition~2.2.5 ではもう
少し弱い擬単調性からこのことを導いている.
さらに田辺 \cite{Tanabe75} では第6章 \S 5 で単調作用素のことが
より一般の設定で議論されている.
そこまで一般的にする必要はないので,強い条件の下で証明しておいた.
\endhide

最後にもう一つ定義をしておく.

\Definition{FPM-12} %*******************************************************
$T$ を Banach 空間 $V$ からその双対空間 $V^*$ への写像とする.
ある $u_0\in V$ が存在して
\bdn %----------------------------------------------------------------------
\lim_{\|u\|\to\infty} \frac{(Tu,u-u_0)}{\|u\|} = \infty
\Eqn{FPM.10}
\edn %----------------------------------------------------------------------
がなりたつとき $T$ は統御的 (coercive) であるという.
\end{definition} %**********************************************************

さて,ここでは単調作用素の全射性について調べたいのであるが,
そのためにまず有限次元の結果から述べる.

\Proposition{FPM-16} %******************************************************
$P$ を $\R^n$ から $\R^n$ への連続写像で,ある $\rho>0$ が存在して
\bdn %----------------------------------------------------------------------
(P(\xi),\xi) \ge 0, \quad \forall \xi \text{ with $|\xi|=\rho$} 
\Eqn{FPM.12}
\edn %----------------------------------------------------------------------
が成り立っている.
このときある $\xi$, $|\xi| \le \rho$ が存在して $P(\xi)=0$ とできる.
\end{proposition} %*********************************************************

\Proof
背理法で示す.
$K=\{\xi ; \|\xi|\le \rho\}$ で $P(\xi)\not=0$ とする.
さらに
\bdm %----------------------------------------------------------------------
\xi \mapsto -P(\xi)\frac{\rho}{|P(\xi)|}
\edm %----------------------------------------------------------------------
という写像を考えるとこれは $K$ から $K$ への連続写像である.
すると Brouwer の不動点定理から不動点 $\xi$ が存在する.
即ち
\bdm %----------------------------------------------------------------------
\xi =  -P(\xi)\frac{\rho}{|P(\xi)|}.
\edm %----------------------------------------------------------------------
この $\xi$ は $|\xi|=\rho$ をみたし
\bdm %----------------------------------------------------------------------
(P(\xi),\xi)
= \biggl(P(\xi), -P(\xi)\frac{\rho}{|P(\xi)|}\biggr)
= -\rho |P(\xi)|
<0
\edm %----------------------------------------------------------------------
となり,\Eq{FPM.12} に矛盾する.
\QED %======================================================================

\hide
これは Lions \cite{Lions69} p.53, Lemme~1.4.3 に述べてある.
確かに証明は簡単だが.
\endhide

さて,これを使って本題の単調関数に関する全射性を導く定理を示す.

\Theorem{FPM-18} %**********************************************************
$V$ を反射的可分 Banach 空間,
$T\colon V \to V^*$ が有界単調写像で統御的であるとする.
このとき $T$ は全射である.
さらに $T$ が狭義単調であれば $T$ は単射でもある.
\end{theorem} %*************************************************************

\Proof
統御的の条件 \Eq{FPM.10} は $u_0=0$ とすることが出来る.
実際 $Tu$ の代わりに $T(u+u_0)$ を考えればよいからである.
以下 \Eq{FPM.10} が $u_0=0$ として成り立っているとする.

さて示すべきは任意に $f\in V^*$ を与えて $Tu=f$ となる
$u$ の存在を示すことである.
$V$ は可分だから $w_1$, $w_2,\dots$ を1次独立で,一次結合が  $V$ で
稠密なものをとる.
$m$ を固定して $\{w_1,\dots,w_m\}$ で張られる線型空間を
$V_m$ とする.
$V_m$ から $\R^m$ への写像 $P$ を
\bdm %----------------------------------------------------------------------
Pu
= ((Tu-f,w_1),\dots,(Tu-f,w_m))
\edm %----------------------------------------------------------------------
で定める.
$\vph^j$ を $w_i$ の dual basis とする.
即ち,
\bdm %----------------------------------------------------------------------
(\vph^i,w_j)
= \d^i_j
\edm %----------------------------------------------------------------------
を満たすものとする.
$V_m$ は次の対応で $\R^m$ と同一視する:
$u\mapsto ((\vph^1,u),\dots,(\vph^m,u))$.
また $u=\sum_j(\vph^j) w_j$ だから
\bdm %----------------------------------------------------------------------
(Tu-f,u)
= \sum_j (Tu-f,w_j)(\vph^j,u)
= (Pu,u)_{\R^n}
\edm %----------------------------------------------------------------------
となる.
さらに
\bdm %----------------------------------------------------------------------
(Pu,u)
= (Tu-f,u)
\ge (Tu,u) - \|f\|\,\|u\|
\edm %----------------------------------------------------------------------
であり,統御的の条件から $\|u\|$ が十分大きいと
\bdm %----------------------------------------------------------------------
(Tu,u)
\ge \|f\|\,\|u\|
\edm %----------------------------------------------------------------------
とできる.
また $V_m$ は有限次元だからノルム $\|u\|$ は $\sqrt{\sum_j (\vph^j,u)^2}$
と同値になる(定数倍で上下から抑えられる).
よって $\rho$ を十分大きくとると $\sqrt{\sum_j (\vph^j,u)^2}=\rho$
のとき $Pu,u)\ge 0$ が成り立つ.
また \Prop{FPM-10} から $P$ は $V_m$ 上で連続である.
よって \Prop{FPM-16} から $Pu_m=0$ となる$u_m\in V_m$ が存在する.
これは
\bdm %----------------------------------------------------------------------
(Tu_m,w_j)
= (f,w_j), \quad j=1,2,\dots,m
\edm %----------------------------------------------------------------------
を意味している.
これから
\bdm %----------------------------------------------------------------------
(Tu_m,u_m)
= (f,u_m)
\le \|f\| \, \|u_m\|
\edm %----------------------------------------------------------------------
であるから
\bdm %----------------------------------------------------------------------
\frac{(Tu_m,u_m)}{\|u_m\|}
\le \|f\|
\edm %----------------------------------------------------------------------
が成り立つ.

ここから $m$ を動かして考えよう.
統御的の条件から $\|u_m\|$ は有界である.
さらに $T$ の有界性を使えば $Tu_m$ も有界である.
そこで部分列 $\{u_\mu\}$ を
\bg %----------------------------------------------------------------------
u_\mu \to u \text{ weakly in $V$} \\
Tu_\mu \to \chi \text{ weakly in $V^*$}
\eg %----------------------------------------------------------------------
ととる.
$j$ を固定すると $(Tu_\mu,w_j) = (f,w_j)$ が十分大きな $\mu$ に対して
成り立っているから $\mu\to\infty$ として
\bdm %----------------------------------------------------------------------
(\chi,w_j) = (f,w_j).
\edm %----------------------------------------------------------------------
これが任意の $j$ について成り立つから $\chi=f$ である.
一方
\bdm %----------------------------------------------------------------------
(Tu_\mu, u_\mu) = (f,u_\mu) \to (f,u)
\edm %----------------------------------------------------------------------
より
\bdn %----------------------------------------------------------------------
\lim_{\mu\to\infty} (Tu_\mu, u_\mu) = (f,u).
\Eqn{FPM.14}
\edn %----------------------------------------------------------------------

あとは $Tu=\chi$ を示せばよい.
そこで単調性から
\bdm %----------------------------------------------------------------------
(Tu_\mu-Tv,u_\mu-v) \ge 0 \quad \forall v\in V
\edm %----------------------------------------------------------------------
即ち
\bdm %----------------------------------------------------------------------
(Tu_\mu, u_\mu) - (Tu_\mu, v) - (Tv,u_\mu-v) \ge 0.
\edm %----------------------------------------------------------------------
ここで \Eq{FPM.14} に注意して $\mu\to\infty$ とすれば
\bdm %----------------------------------------------------------------------
(f,u) - (\chi, v) - (Tv,u-v) \ge 0.
\edm %----------------------------------------------------------------------
従って
\bdm %----------------------------------------------------------------------
(\chi-Tv,u-v) \ge 0.
\edm %----------------------------------------------------------------------
ここで $v=(1-t)u + tw$ $t\in(0,1)$ とすると
\bdm %----------------------------------------------------------------------
0
\le (\chi-Tv,u-(1-t)u - tv)
= t(\chi-Tv,u+w).
\edm %----------------------------------------------------------------------
$t$ で割って
\bdm %----------------------------------------------------------------------
(\chi-Tv,u+w) \ge 0.
\edm %----------------------------------------------------------------------
ここで $t\to 0$ とすると $T$ の線分上弱連続性から
\bdm %----------------------------------------------------------------------
(\chi-Tu,u+w) \ge 0.
\edm %----------------------------------------------------------------------
$w$ は任意にとれるから $Tu=\chi$ が示せる.

狭義単調性があれば $Tu=Tv$ ならば
\bdm %----------------------------------------------------------------------
(Tu-Tv,u-v) = 0
\edm %----------------------------------------------------------------------
なので $u=v$ が成り立ち $T$ は単射である.
\QED %======================================================================

\subsec{Notes}
\begin{itemize}
\item 単調写像の全射性について纏めた.
P.L.~Lions \cite{Lions69} に従っている.
[2011年2月24日]
\end{itemize}


	%%%%%%%%%%%%%%%%%%%%%%%%%%%%%%%%%%%%%%%%%%%%%%%%%
%                                               %
%         ====== Program  No.7 =======          %
%                                               %
%             file name snj07.tex               %
%                                               %
%===============================================%
%===================  for  =====================%
%===============================================%
%
\StartNewSection
\hide
\vspace{-4mm}
\begin{itemize} \itemsep=-2mm \parsep=0mm
\item Total file name: snj01 snj02 $\dots $ snj?, snj\_bibliography
\item File name: snj07.tex \hfill 印刷日: \today \ \now
\item 正規作用素が Generalized Dirichlet form の枠組みに入ることを証明する.
2008年の日独のとき R\"ockner が別の話で Generalized Dirichlet form
の枠組みに入るのではと言っていたが,正規作用素だったらいいように思う.
何か感じたんでしょうね,彼は.
[2011年1月16日]
\end{itemize}
\endhide
\SS{GDF}{正規作用素と Generalized Dirichlet form} %=========================
% Generalized Dirichlet form
正規作用素で生成される Markov 半群は Generalized Diriclet form
の枠組みに入る.
そのことを示していく.
\hide
昔 Stanatt がやっていたことだが,僕は彼の話を全然理解していなかった.
かれの話は実は結構使えるわけだ.
[2011年1月16日]
\endhide
$(M,\m)$ を $\s$-有限な測度空間とし,$H=L^2(\m)$ として,以下 Hilbert 空間
$H$ で考える.
$\gen$ を正規作用素とする.
$m$-dissipative を仮定する.
spectral represntation により
\bdn %----------------------------------------------------------------------
- \gen
= \int_\C z E(dz)
\Eqn{GDF.6}
\edn %----------------------------------------------------------------------
と表現できる.
対称部分 $(\gen+\gen^*)/2$ は
\bdn %----------------------------------------------------------------------
\frac{\gen+\gen^*}{2}
= \int_\C \Re z E(dz)
\Eqn{GDF.8}
\edn %----------------------------------------------------------------------
歪対称部分 $(\gen-\gen^*)/2$ は
\bdn %----------------------------------------------------------------------
\frac{\gen-\gen^*}{2}
= \int_\C i \Im z E(dz)
\Eqn{GDF.10}
\edn %----------------------------------------------------------------------
で表される.
これはユニタリーグループを生成する.
ここで
\bdn %----------------------------------------------------------------------
L
= \int_\C \Re z E(dz),\quad
\Lm
= \int_\C i \Im z E(dz)
\Eqn{GDF.14}
\edn %----------------------------------------------------------------------
とおく.
$L$ は self-adjoint で $\Lm$ は $i\Lm$ が self-adjointになる.
また定義域は
\bdn %----------------------------------------------------------------------
\Dom(L)
= \{ f;\, \int_\C |\Re z|^2 (f,E(dz)f) <\infty \},\quad
\Dom(\Lm)
= \{ f;\, \int_\C |\Im z|^2 (f,E(dz)f) <\infty \}
\Eqn{GDF.16}
\edn %----------------------------------------------------------------------
となる.
$L$ は $m$-dissipative を仮定したので,半群を生成する.
また $L$ から対称な双線型形式 $\tDiri$ が定まる.
実際 $\tDiri$ は次で定義される:
\bdn %----------------------------------------------------------------------
\tDiri(f,g)
= \int_\C \Re z (f,E(dz)g).
\Eqn{GDF.18}
\edn %----------------------------------------------------------------------
またその定義域も
\bdn %----------------------------------------------------------------------
\Dom(\tDiri)
= \{f;\, \int_\C |\Re z| (f,E(dz)f) <\infty\}
\Eqn{GDF.20}
\edn %----------------------------------------------------------------------
で与えられる.
$\sV =\Dom(\tDiri)$ とおく.

さて,$\Lm$ も半群を生成する.
実際は 1変数 unitary 群を生成するが,半群の部分だけ使う.
この半群を $\{U_t\}_{t\ge0}$ とかく.
このとき次が成り立つ.


\Proposition{GDF-4} %*******************************************************
$\{U_t\}$ は $\sV$ の $C_0$-縮小半群である.
(実際は1変数 unitary 群である.)
\end{proposition} %*********************************************************

さて $\Lm\colon \Dom(\Lm)\cap \sV \to \sV'$ を $\sV$ から $\sV'$
への作用素とみて,そのの閉包を $(\Lm,\sF)$ とする.
この閉包が存在することは Stanatt Lemma 2.3 にある.

\Proposition{GDF-6} %*******************************************************
$f\in \sF$ であるための必要十分条件は
\bdn %----------------------------------------------------------------------
\int_\C (\frac{|\Im(z)|^2}{\Re z+1} + \Re z) \,(f,E(dz)f) < \infty
\Eqn{GDF.22}
\edn %----------------------------------------------------------------------
である.
\end{proposition} %*********************************************************

\Proof
まず $\sV$ のノルムとして
\bdm %----------------------------------------------------------------------
\|f\|_\sV^2
= \int_C (\Re z + 1)(f,E(dz)f)
\edm %----------------------------------------------------------------------
を取ることができる.
\bdm %----------------------------------------------------------------------
|(\Lm f,g)_H|
&=   |\int_\C \Im z (f,E(dz)g)| \\
&=   |\int_\C \frac{\Im z}{\sqrt{\Re z + 1}} \sqrt{\Re z + 1} (f,E(dz)g)| \\
&\le \biggl\{\int_\C \frac{|\Im z|^2}{\Re z + 1} (f,E(dz)f)\biggr\}^{1/2}
     \biggl\{\int_\C (\Re z + 1) (g,E(dz)g) \biggl\}^{1/2} \\
&\le \biggl\{\int_\C \frac{|\Im z|^2}{\Re z + 1} (f,E(dz)f)\biggr\}^{1/2}
     \|g\|_\sV
\edm %----------------------------------------------------------------------
これから
\bdm %----------------------------------------------------------------------
\|\Lm f\|_{\sV'}
\le \biggl\{\int_\C \frac{|\Im z|^2}{\Re z + 1} (f,E(dz)f)\biggr\}^{1/2}
\edm %----------------------------------------------------------------------
が示せる.
逆向きの不等式も $g$ をうまくとれば成立することが分かる.
よって
\bdm %----------------------------------------------------------------------
\|\Lm f\|_{\sV'}
= \biggl\{\int_\C \frac{|\Im z|^2}{\Re z + 1} (f,E(dz)f)\biggr\}^{1/2}
\edm %----------------------------------------------------------------------
また
\bdm %----------------------------------------------------------------------
\|f\|_\sF^2
= \|f\|_\sV^2 + \|\Lm f\|_{\sV'}
\edm %----------------------------------------------------------------------
であるから求める結果を得る.
\QED %======================================================================

\Remark{GDF-8} %************************************************************
弱扇形条件が成り立つときはスペクトルが扇形領域に含まれるから
そこでは $|\Im(z)| \le C(\Re z+1)$ となる定数 $C$ が存在する.
従って,$\sF$ の定義域は $\sV$ と同じになる.
\end{remark} %**************************************************************

同様の議論を $U_t$ の双対半群 $\hat{U}_t$ についても行って
$\hat{\sF}$ を定める.
$\hat{U}_t$ の生成作用素は
\bdn %----------------------------------------------------------------------
\hat{\Lm}
= - \int_\C i \Im z E(dz)
\Eqn{GDF.26}
\edn %----------------------------------------------------------------------

さて generalized Dirichlet form の枠組みでは双線型形式を
\bdn %----------------------------------------------------------------------
\Diri(f,g)
=
\begin{cases}
\tDiri(f,g)-\la \Lm f,g\ra, \quad \text{if $f\in\sF$, $g\in \sV$} \\
\tDiri(f,g)-\la \hat{\Lm} g, f\ra, \quad \text{if $f\in\sV$, $g\in \hat{\sF}$}
\end{cases}
\Eqn{GDF.24}
\edn %----------------------------------------------------------------------
で定める.

容量の概念も定義され,次の定理を適用することにより確率過程の存在が保証される.

\Theorem{GDF-10} %**********************************************************
{\bf (Stanatt 1994) }
条件 (D3) のもとで,擬正則一般 Dirichlet 形式に対して,
$\m$-thght special standard 過程が存在する.
\end{theorem} %*************************************************************

\begin{description}
\item[(D3)]
ある線型部分空間 $\sY \subseteq L^2(\m)\cap L^\infty(\m)$ で
$\sY\cap \sF$ が $\sF$ 稠密であり,
$\lim_{\al\to\infty} e_{\al G_\al u-u}=0$ が $u\in\sY$ に対して成り立つ.
さらに
$\sY$ の $L^\infty$ での閉包を $\ol{\sY}$ とすると
$u\wedge \al\in \ol{\sY}$ が $u\in\sY$, $\al>0$ に対して成立する.
\end{description}







\subsec{Notes}
\begin{itemize}
\item 
\end{itemize}

	%\input{snj08}
	%\input{snj09}	\input{snj10} \input{snj11}
    %\input{snj12}	\input{snj13} %\input{snj14}
	%\input{snj15}
	%%%%%%%%%%%%%%%%%%%%%%%%%%%%%%%%%%%%%%%%%%%%%%%%%
%                                               %
%         ====== Program  No.16 =======          %
%                                               %
%             file name snj16.tex               %
%                                               %
%===============================================%
%===================  for  =====================%
%===============================================%
%
\hide
\vspace{-4mm}
\begin{itemize} \itemsep=-2mm \parsep=0mm
\item Total file name: snj01 snj02 $\dots $ snj?, snj\_bibliography
\item File name: snj16.tex \hfill 印刷日: \today \ \now
\item Ornstein-Uhlenbeck に回転を加えた作用素のスペクトルを調べる.
[2011年8月26日(金)]
\end{itemize}
\endhide
\SS{OUR}{$\R^2$ 上の回転を加えた Ornstein-Uhlenbeck 作用素} %==================
% Ornstein-Uhlenbeck operator with rotation
この節では Ornstein-Uhlenbeck に回転を加えた作用素のスペクトルを調べる.
$\al\in\R$ に対し $L_{\al}$ を
\bdn %----------------------------------------------------------------------
L_\al
= \frac{\del^2}{\del x^2} + \frac{\del }{\del y^2} 
  - x \frac{\del}{\del x} - y \frac{\del }{\del y}
  \alpha \biggl( x \frac{\del }{\del y} - y \frac{\del }{\del x} \biggr)
\Eqn{OUR.6}
\edn %----------------------------------------------------------------------
で定め,$L^2(\R^2, \mu)$ 上の作用素としてスペクトルを調べる.
ここで測度 $\mu$ は
\bdn %----------------------------------------------------------------------
\mu
= \frac{1}{2\pi} e^{-(x^2+y^2)/2}\,dx\,dy
\Eqn{OUR.8}
\edn %----------------------------------------------------------------------
で定まるGauss 測度である.

通常の Ornstein-Uhlenbeck 作用素 $L_0$ のスペクトルは $\{0,-1,-2,\dots\}$
であることはよく知られている.
固有関数は,まず Hermite 多項式を
\bdn %----------------------------------------------------------------------
H_n(x)
= (-1)^n e^{x^2/2}\frac{d^n}{dx^n}e^{-x^2/2}.
\Eqn{OUR.10}
\edn %----------------------------------------------------------------------
で定めると,
\bdm %----------------------------------------------------------------------
L_0 H_k(x)H_{n-k}(y)
= - n H_k(x)H_{n-k}(y)
\edm %----------------------------------------------------------------------
が成り立つ.
回転を加えた場合は,固有値は複素 Hermite 多項式になる.

\subsec{複素 Hermite 多項式}
複素 Hermite 多項式について復習しておく.
$\R^2$ は $\C$ と同一視,$z=x+iy$ とする.
複素 Hermite 多項式は $p$, $q\in\plus{\Z}$ に対し
\bdn %----------------------------------------------------------------------
H_{p,q}(z , \bar{z})
= (-1)^{p+q} e^{z\bar{z}/2} \biggl( \frac{\del}{\del\bar{z}} \biggr)^p
 \biggl(\frac{\del }{\del z} \biggr)^q e^{-z\bar{z}/2}
\Eqn{OUR.14}
\edn %----------------------------------------------------------------------
で定義される.
通常の定義とは定数が異なっていることに注意しよう.
これは Gauss 測度 $\mu$ の取り方合わせたためである.
ここで
\bdm %----------------------------------------------------------------------
\frac{\del}{\del z}
= \frac{1}{2}\biggl( \frac{\del}{\del x} - i \frac{\del}{\del y}\biggr), \quad
\frac{\del}{\del \bar{z}}
= \frac{1}{2}\biggl( \frac{\del}{\del x} + i \frac{\del}{\del y}\biggr)
\edm %----------------------------------------------------------------------
である.
以下では
\bdm %----------------------------------------------------------------------
\del
= \frac{\del}{\del z}, \quad
\delb
= \frac{\del}{\del \bar{z}}, \quad
\edm %----------------------------------------------------------------------
と略記する.
また
\bdn %----------------------------------------------------------------------
\del^*
= -\del + \frac{\bar{z}}{2}, \quad
\delb^*
= -\delb  + \frac{\bar{z}}{2}
\Eqn{OUR.18}
\edn %----------------------------------------------------------------------
が成り立っている.

\Proposition{OUR-2} %*******************************************************
上の \Eq{OUR.18} の作用素には次の交換関係が成立している.
\bdmn %---------------------------------------------------------------------
\del H_{p,q}
&= \frac{p}{2} H_{p-1,q}, 
\Eqn{OUR.20} \\
\delb H_{p,q}
&= \frac{q}{2} H_{p,q-1}, 
\Eqn{OUR.22} \\
\del^* H_{p,q}
&= H_{p+1,q}, 
\Eqn{OUR.24} \\
\delb^* H_{p,q}
&= H_{p,q+1}, 
\Eqn{OUR.26} \\
2\del\delb H_{p,q} - z\del H_{p,q}
&= - p H_{p,q}
\Eqn{OUR.28} \\
2\del\delb H_{p,q} - \bar{z}\delb H_{p,q}
&= - q H_{p,q}
\Eqn{OUR.30} \\
(z\del - \bar{z}\delb)H_{p,q}
&= (p-q) H_{p,q}
\Eqn{OUR.32}
\edmn %----------------------------------------------------------------------
\end{proposition} %*********************************************************

\Proof
\QED %======================================================================

さて $L_\al$ のスペクトルを求めるために
$L_\al$ を $\del$, $\delb$ を用いて表そう.
まず
\bdm %----------------------------------------------------------------------
\del \delb
&= \frac{1}{4} \biggl(\frac{\del^2}{\del x^2}+\frac{\del}{\del y^2}\biggr), \\
z\del
&=  x\frac{\del}{\del x} + y\frac{\del}{\del y}
    -i\biggl(  x \frac{\del}{\del y} - y \frac{\del }{\del x}\biggr), \\
\bar{z}\delb
&=  x \frac{\del}{\del x} + y \frac{\del }{\del y}
    +i\biggl(  x \frac{\del}{\del y} - y \frac{\del }{\del x}\biggr)
\edm %----------------------------------------------------------------------
が成り立つので
\bdm %----------------------------------------------------------------------
L_\al
&= 4 \del\delb - z\del - \bar{z}\delb +\al i (z\del - \bar{z}\delb) \\
&= (2\del\delb - z\del) + (2\del\delb- \bar{z}\delb)
   + \al i (z\del - \bar{z}\delb).
\edm %----------------------------------------------------------------------
従って
\bdm %----------------------------------------------------------------------
L_\al H_{p,q}
&= (2\del\delb - z\del)  H_{p,q} + (2\del\delb- \bar{z}\delb)H_{p,q}
   + \al i (z\del - \bar{z}\delb)H_{p,q} \\
&= - p H_{p,q} + q H_{p,q} + (p-q) \al i H_{p,q} \\
&= (-p-q + (p-q) \al i) H_{p,q}.
\edm %----------------------------------------------------------------------

よって次の定理が得られた.
\Theorem{OUR-6} %**********************************************************
$-L\al$ の固有値は $\{ - (p+q) + (p-q) \alpha i \}_{p,q = 0}^{\infty}$
であり対応する固有関数は $H_{p,q}$ である.
\end{theorem} %*************************************************************


\begin{center}
\includePdfEps{}{snj_OU_spectrum}
\includePdfEps{}{snj_OU_rotate_spectrum}
\end{center}


\bdm %----------------------------------------------------------------------
V_n := \{ L_0 f = n f\}
\edm %----------------------------------------------------------------------
とすれば直交分解
\bdm %----------------------------------------------------------------------
L^2(\C,\mu)
= \bigoplus_{n=0}^\infty V_n
\edm %----------------------------------------------------------------------
が成り立つ.
$V_n$ をさらに
\bdm %----------------------------------------------------------------------
V_n = \bigoplus_{p + q = n} {\bf{C}}H_{p,q}
\edm %----------------------------------------------------------------------
と分解したことになる.
これは回転群 $U(1)$ の規約分解を与えている.
固有値 $2n$ ($n\in\plus{\Z}$)に対応する固有関数は回転方向の微分
$(z\del - \bar{z}\delb)$ が $0$ になるので,半径方向の関数であり
\bdm %----------------------------------------------------------------------
\biggl(\frac{d^2}{d r^2} + \frac{1}{r}\frac{d}{d r}
       - r \frac{d}{d r} \biggr) H_{n,n}
= -2n H_{n,n}
\edm %----------------------------------------------------------------------
が成り立っている.
ここで $r=\sqrt{2u}$ と変数変換すると
\bdm %----------------------------------------------------------------------
\frac{d^2}{d r^2} + \frac{1}{r}\frac{d}{d r}
       - r \frac{d}{d r}
= 2u \frac{d^2}{du^2} + 2(1-u) \frac{d}{du}
\edm %----------------------------------------------------------------------
となる.
$F(u)= H_{n,n}(r)$ は微分方程式
\bdm %----------------------------------------------------------------------
2u \frac{d^2}{du^2}F + 2(1-u) \frac{d}{du}F +nF=0
\edm %----------------------------------------------------------------------
をみたしている.
これは Laguerre 多項式が満たす微分方程式である.
ここで Laguerre 多項式は
\bdn %----------------------------------------------------------------------
L_{n} = \frac{e^x}{n!} \frac{d^n}{dx^n} (e^{-x}x^n)
\Eqn{OUR.40}
\edn %----------------------------------------------------------------------
で定義される.
実際次が成立する.

\Theorem{OUR-10} %**********************************************************
複素 Hermite polynomials $H_{n,n}$ は次をみたす.
\bdn %----------------------------------------------------------------------
H_{n,n}(z,\bar{z}) = c L_n \left(\frac{|z|^2}{2} \right),
\Eqn{OUR.42}
\edn %----------------------------------------------------------------------
$c$ は定数である.
\end{theorem} %*************************************************************

\hide
上の定数 $c$ がはっきりしないのは問題である.

\endhide

さらに複素 Hermite 多項式は,実 Hermite 多項式を用いて
\bdn %----------------------------------------------------------------------
H_{n,n}(z,\bar{z})
\frac{1}{4^n} \sum_{k=0}^n H_{2k}(x)H_{2n-k}(y)
\Eqn{OUR.44}
\edn %----------------------------------------------------------------------
と表されるから次を得る.

\Corollary{OUR-12} %********************************************************
Laguerre 多項式は Hermite 多項式と次の関係で結ばれている.
\bdn %----------------------------------------------------------------------
L_n(\frac{x^2+y^2}{2})
= \frac{c}{4^n} \sum_{k=0}^n H_{2k}(x)H_{2n-k}(y).
\Eqn{OUR.46}
\edn %----------------------------------------------------------------------
\end{corollary} %***********************************************************

\subsec{Notes}
\begin{itemize}
\item 2011年9月12日 に Jan Van Neerven からメールが来た.OU のスペクトルの研究をしている.Bonn のISSAA で会って,話を聞いた.この節で論じたスペクトルの話に関連したことをやっている.
\item 2012年7月に中国に行ったとき,同じ結果を得ていることを聞いた.
さっさと書かないからこういうことになる・
\end{itemize}

 %\input{snj17}
	%%%%%%%%%%%%%%%%%%%%%%%%%%%%%%%%%%%%%%%%%%%%%%%%%
%                                               %
%         ====== Program  No.18 =======          %
%                                               %
%             file name snj18.tex               %
%                                               %
%===============================================%
%===================  for  =====================%
%===============================================%
%
\hide
\vspace{-4mm}
\begin{itemize} \itemsep=-2mm \parsep=0mm
\item Total file name: snj01 snj02 $\dots $ snj?, snj\_bibliography
\item File name: snj18.tex \hfill 印刷日: \today \ \now
\item 正規作用素の例を挙げる.
さらにそのスペクトルを調べる.
[2011年8月16日(火)]
\end{itemize}
\endhide
\SS{BWD}{ドリフトのある1次元ブラウン運動} %===================================
% On-dimensional brownian motion with a drift
ここからいくつか正規作用素の例を与える.
またスペクトルを完全な形で求めることを目標とする.

$\R$ 上で作用素 $\gen = \frac{d^2}{dx^2} - c\frac{d}{dx}$ を考える.
但しここでは基礎の測度を
\bdn %----------------------------------------------------------------------
\nu_1(dx)
= e^{-cx}dx
\Eqn{BWD.4}
\edn %----------------------------------------------------------------------
にとる.
このとき
\bdm %----------------------------------------------------------------------
\int_\R (\frac{d^2}{dx^2}f - c\frac{d}{dx}f)g(x) e^{-cx}\,dx
&= \int_\R \frac{d}{dx}(\frac{d}{dx}fe^{-cx})\, g(x)\,dx
&= - \int_\R (\frac{d}{dx}f e^{-cx}) \frac{d}{dx}g(x)\,dx
\edm %----------------------------------------------------------------------
であるから,$\gen$ は自己共役作用素となる.
実際この作用素は次の対称双線型形式 $\Diri$ に対応する自己共役作用素である:
\bdn %----------------------------------------------------------------------
\Diri(f,g)
= \int_\R \frac{df}{dx} \frac{dg}{dx} e^{-cx}\,dx
\Eqn{BWD.6}
\edn %----------------------------------------------------------------------
また $\nu_1$ に関する $\frac{d}{dx}$ の共役は
\bdn %----------------------------------------------------------------------
\Bigl(\frac{d}{dx}\Bigr)_{\nu_1}^*
= -\frac{d}{dx} + c
\Eqn{BWD.8}
\edn %----------------------------------------------------------------------
であり,生成作用素 $\gen$ は
\bdm %----------------------------------------------------------------------
\gen
= \Bigl(\frac{d}{dx}\Bigr)_{\nu_1}^* \frac{d}{dx}
\edm %----------------------------------------------------------------------
の形でも表される.
$\gen$ のスペクトルを調べるには変換したほうがよい.
次の写像 $I\colon L^2(\nu_1)\longrightarrow L^2(dx)$ は等距離作用素である:
\bdn %----------------------------------------------------------------------
If(x)=e^{-cx/2}f(x).
\Eqn{BWD.12}
\edn %----------------------------------------------------------------------
ここで
\bdm %----------------------------------------------------------------------
I\circ \gen \circ I^{-1} f
&= e^{-cx/2}(\frac{d^2}{dx^2} - c\frac{d}{dx})(e^{cx/2}f) \\
&= e^{-cx/2}(\frac{c^2}{4} e^{cx/2} f + c e^{cx/2}\frac{df}{dx}
    + e^{cx/2} \frac{d^2f}{dx^2} - c\frac{1}{2}ce^{cx/2} f
    - ce^{cx/2}\frac{df}{dx}) \\
&= - \frac{c^2}{4} f + \frac{d^2f}{dx^2}
\edm %----------------------------------------------------------------------
であるから次の図式が可換となる:
\bdn %----------------------------------------------------------------------
\begin{CD}
L^2(\nu_1)	@>{\gen}>>	L^2(\nu_1)		\\
@V{I}VV             @VV{I}V	\\
L^2(dx)		@>{\frac{d^2}{dx^2} - \frac{c^2}{4}}>>	L^2(dx)
\end{CD}
\Eqn{BWD.14}
\edn %----------------------------------------------------------------------
従ってこの場合のスペクトル集合を $\s_1$ とすると
\bdn %----------------------------------------------------------------------
\s_1
= [\frac{c^2}{4},\infty)
\Eqn{BWD.16}
\edn %----------------------------------------------------------------------
となる.

さて,作用素 $\gen$ に摂動を加えて,スペクトルの変化を見ていく.
ベクトル場 $b$ を
\bdn %----------------------------------------------------------------------
b
= k\frac{d}{dx}
\Eqn{BWD.18}
\edn %----------------------------------------------------------------------
で定める.
すると $\nu_1$ に関する発散は
\bdm %----------------------------------------------------------------------
\div_{\nu_1} b
= -ck
\edm %----------------------------------------------------------------------
となる.
従って\Thm{CNO-16} から $\gen+b$ は正規作用素になる.
このスペクトルを求めよう.
この場合もやはり $I$ で変換して $L^2(dx)$ の話に持っていく.
\bdm %----------------------------------------------------------------------
I\circ b \circ I^{-1}f
&= e^{-cx/2}(k\frac{d}{dx})(e^{cx/2}f) \\
&= e^{-cx/2} (k\frac{1}{2}ce^{cx/2} f + ke^{cx/2}\frac{df}{dx}) \\
&= \frac{kc}{2} f + k \frac{df}{dx}
\edm %----------------------------------------------------------------------
であるから
\bdm %----------------------------------------------------------------------
I\circ (\gen+ b) \circ I^{-1}f
&= (\frac{d^2}{dx^2} - \frac{c^2}{4} + \frac{kc}{2} + k \frac{d}{dx})f \\
&= (\frac{d^2}{dx^2} + k \frac{d}{dx} - \frac{c(c-2k)}{4} )f.
\edm %----------------------------------------------------------------------
よって $\frac{d^2}{dx^2} + k \frac{d}{dx}$ の
スペクトルを調べればよいことになる. 
そのために Fourier 変換を利用する.
Fourier 変換は次で定義される.
\bdm %----------------------------------------------------------------------
\hat{f}(\xi)
= \frac{1}{\sqrt{2\pi}}\int_\R f(x) e^{-\xi x}\,dx.
\edm %----------------------------------------------------------------------
これは $L^2(dx)$ から $L^2(d\xi)$ への等長変換を与える.
さらに
\bdm %----------------------------------------------------------------------
\int_\R (\frac{d^2}{dx^2} + k\frac{d}{dx})f(x)\ol{g(x)}\,dx
= \int_\R (-\xi^2 + ik\xi) \hat{f}(\xi) \ol{\hat{g}(\xi)}\,d\xi
\edm %----------------------------------------------------------------------
が成り立つ.
即ち,Fourier 変換した空間では $\frac{d^2}{dx^2} + k\frac{d}{dx}$ に対応する
写像は $-\xi^2 + ik\xi$ をかけるという掛け算作用素である.
よって $\frac{d^2}{dx^2} + k\frac{d}{dx}$ のスペクトルは
\bdm %----------------------------------------------------------------------
\{ - \xi^2 + ik\xi; \xi\in\R\}
\edm %----------------------------------------------------------------------
という放物線である.

元に戻って,$L^2(\nu_1)$ での
$-\gen -b = -\frac{d^2}{dx^2} + c \frac{d}{dx} - k\frac{d}{dx} - $
のスペクトルは
\bdm %----------------------------------------------------------------------
 \{ \frac{c(c-k)}{2} +  \xi^2 + ik\xi;\, \xi\in\R\}
\edm %----------------------------------------------------------------------
である.


\begin{center}
\includePdfEps{}{snj_BM_drift1}
\includePdfEps{}{snj_BM_drift2}
\end{center}




\begin{comment}
\begin{center}
\begin{pspicture}(-1,-4)(5,3) %\showgrid
%\psset{arcangle=8, linewidth=.8pt}  % default
%\psset{arrowsize=1.5pt 2} % default
\psset{arrowsize=3pt 2, arrowlength=1.5}
\psset{xunit=8mm, yunit=12mm}
\psaxes[labels=none,showorigin=true,linewidth=.2pt,ticks=none]{->}(0,0)(-1,-2)(4.5,2)
\psset{linewidth=1pt,arrows=c-c,linecolor=red}
\psline(.5,0)(4.2,0)

%\psecurve(9,3)(4,2)(1,1)(0,0)(1,-1)(4,-2)(9,-3)
%\psecurve(9,3)(4,2)(0,0)(4,-2)(9,-3)

\uput[90](.5,0){$\ds \frac{c^2}{4}$}
\uput{5pt}[45](1,-3){$-\gen$}
%\showgrid
\end{pspicture} 
%
\begin{pspicture}(-1,-4)(5,3) %\showgrid
%\psset{arcangle=8, linewidth=.8pt}  % default
%\psset{arrowsize=1.5pt} % default
\psset{arrowsize=3pt 2, arrowlength=1.5}
\psset{xunit=8mm, yunit=12mm}
\psaxes[labels=y,showorigin=true,linewidth=.2pt,ticks=none]{->}(0,0)(-1,-2)(4.5,2)
\psset{linewidth=1pt,arrows=c-c,linecolor=red}
%\psecurve(6.25,2.5)(4,2)(2.25,1.5)(1,1)(.25,.5)(0,0)(1,-1)(2.25,-1.5)(4,-2)(6.25,-2.5)
\pscurve(5,2)(1,0)(5,-2)
\uput{5pt}[90](1,0){$\frac{c(c-2k)}{4}$}
\uput{5pt}[45](1,-3){$-\gen - k\frac{d}{dx}$}
\end{pspicture} 
\end{center}
\end{comment}


これは $k$ とともにスペクトルが連続的に変化している様を表している.
これを違った観点から見てみよう.
作用素を $\gen = \frac{d^2}{dx^2} - c\frac{d}{dx}$ とする.
測度を $\nu_0=dx$, $\nu_1=e^{-cx}dx$ として $L^2(\nu_0)$ でのスペクトルと
$L^2(\nu_1)$ でのスペクトルを比べてみよう.
同じ作用素を異なった空間で考えていることになる.
上の計算は $-\gen$ のスペクトルは $L^2(\nu_0)$ では $\{\xi^2 - ic\xi\}$
であり $L^2(\nu_1)$ では $[\frac{c^2}{4},\infty)$ となっていることが分かる.


\begin{center}
\includePdfEps{}{snj_BM_exp1}
\includePdfEps{}{snj_BM_exp2}
\end{center}


\begin{comment}
\begin{center}
\begin{pspicture}(-1,-4)(5,3) %\showgrid
%\psset{arcangle=8, linewidth=.8pt}  % default
%\psset{arrowsize=1.5pt} % default
\psset{arrowsize=3pt 2, arrowlength=1.5}
\psset{xunit=8mm, yunit=12mm}
\psaxes[labels=y,showorigin=true,linewidth=.2pt,ticks=none]{->}(0,0)(-1,-2)(4.5,2)
\psset{linewidth=1pt,arrows=c-c,linecolor=red}
%\psecurve(6.25,2.5)(4,2)(2.25,1.5)(1,1)(.25,.5)(0,0)(1,-1)(2.25,-1.5)(4,-2)(6.25,-2.5)
\pscurve(4,2)(0,0)(4,-2)
\uput{5pt}[45](1,-3){w.r.t. $\nu_0=dx$}

%\showgrid
\end{pspicture} 
%
\begin{pspicture}(-1,-4)(5,3) %\showgrid
%\psset{arcangle=8, linewidth=.8pt}  % default
%\psset{arrowsize=1.5pt 2} % default
\psset{arrowsize=3pt 2, arrowlength=1.5}
\psset{xunit=8mm, yunit=12mm}
\psaxes[labels=none,showorigin=true,linewidth=.2pt,ticks=none]{->}(0,0)(-1,-2)(4.5,2)
\psset{linewidth=1pt,arrows=c-c,linecolor=red}
\psline(.5,0)(4.2,0)

%\psecurve(9,3)(4,2)(1,1)(0,0)(1,-1)(4,-2)(9,-3)
%\psecurve(9,3)(4,2)(0,0)(4,-2)(9,-3)

\uput[90](.5,0){$\ds \frac{c^2}{4}$}
\uput{5pt}[45](1,-3){w.r.t. $\nu_1=e^{-cx}dx$}
%\showgrid
\end{pspicture} 
\end{center}
\end{comment}


今度は測度を連続的に動かしてスペクトルの変化を見よう.
$\theta\in[0,1]$ に対し測度 $\nu_\theta$ を
\bdn %----------------------------------------------------------------------
\nu_\theta(dx) = (1-\theta)dx  + \theta e^{-cx} dx
\Eqn{BWD.22}
\edn %----------------------------------------------------------------------
で定める.
$L^2(\nu_\theta)$ でのスペクトルはどのように変化するであろうか.
次のように連続的に変化することも考えられる.

\begin{center}
\includePdfEps{}{snj_BM_measure1}
\end{center}

\begin{comment}
\begin{center}
\begin{pspicture}(-1,-4)(5,3) %\showgrid
%\psset{arcangle=8, linewidth=.8pt}  % default
%\psset{arrowsize=1.5pt} % default
\psset{arrowsize=3pt 2, arrowlength=1.5}
\psset{xunit=8mm, yunit=12mm}
\psaxes[labels=y,showorigin=true,linewidth=.2pt,ticks=none]{->}(0,0)(-1,-2)(4.5,2)
%\psset{linewidth=1pt,arrows=c-c}
\psset{linewidth=1pt,arrows=c-c,linecolor=red}
%\psecurve(6.25,2.5)(4,2)(2.25,1.5)(1,1)(.25,.5)(0,0)(1,-1)(2.25,-1.5)(4,-2)(6.25,-2.5)
\psline(.5,0)(4.2,0)
\pscurve(4,2)(0,0)(4,-2)
\pscurve[linecolor=blue](4,.9)(.25,0)(4,-.9)
\psline[linecolor=black]{->}(2.3,1.2)(3.1,.3)
\uput{5pt}[45](1,-3){w.r.t. $\nu_\theta$}
\uput[45](2.5,1){?}

%\showgrid
\end{pspicture} 
\end{center}
\end{comment}

しかし,これは正しくない.
実際は二つの和集合になる.
\Theorem{BWD-4} %**********************************************************
$\theta\in (0,1)$ のとき $-\gen$ の $L^2(\nu_\theta)$ におけるスペクトルは
\bdm %----------------------------------------------------------------------
\{\xi^2 - ik\xi; \xi\in\R\} \cup [\frac{c^2}{4},\infty)
\edm %----------------------------------------------------------------------
である.
\end{theorem} %*************************************************************



\begin{center}
\includePdfEps{}{snj_BM_measure2}
\end{center}

\begin{comment}
\begin{center}
\begin{pspicture}(-1,-4)(5,3) %\showgrid
%\psset{arcangle=8, linewidth=.8pt}  % default
%\psset{arrowsize=1.5pt} % default
\psset{arrowsize=3pt 2, arrowlength=1.5}
\psset{xunit=8mm, yunit=12mm}
\psaxes[labels=y,showorigin=true,linewidth=.2pt,ticks=none]{->}(0,0)(-1,-2)(4.5,2)
\psset{linewidth=1pt,arrows=c-c,linecolor=red}
%\psecurve(6.25,2.5)(4,2)(2.25,1.5)(1,1)(.25,.5)(0,0)(1,-1)(2.25,-1.5)(4,-2)(6.25,-2.5)
\psline(.5,0)(4.2,0)
\pscurve(4,2)(0,0)(4,-2)
\uput{5pt}[45](1,-3){w.r.t. $\nu_\theta$}
%\showgrid
\end{pspicture} 
\end{center}
\end{comment}















\subsec{Notes}
\begin{itemize}
\item 
\end{itemize}

 %%%%%%%%%%%%%%%%%%%%%%%%%%%%%%%%%%%%%%%%%%%%%%%%%
%                                               %
%         ====== Program  No.19 =======          %
%                                               %
%             file name snj19.tex               %
%                                               %
%===============================================%
%===================  for  =====================%
%===============================================%
%
\hide
\vspace{-4mm}
\begin{itemize} \itemsep=-2mm \parsep=0mm
\item Total file name: snj01 snj02 $\dots $ snj?, snj\_bibliography
\item File name: snj19.tex \hfill 印刷日: \today \ \now
\item 球面上で正規作用素の例を挙げる.
\end{itemize}
\endhide
\SS{LBS}{Laplace-Beltrami 作用素に回転を加えた作用素} %=====================
% Laplace-Beltrami operator with rotation on a sphier
幾何学的な構造が反映するような作用素の例を与えよう.
球面は最も単純な例である.

\subsec{$S^2$ 上の正規作用素} 
球面 $S^2$ 上の Laplace-Berltrami は次のように表される.
\bdm %----------------------------------------------------------------------
\Laplace
= \frac{1}{\sin \theta} \frac{\del}{\del \theta}
  \biggl(\sin \theta \frac{\del}{\del \theta}\biggr)
  + \frac{1}{\sin^2 \theta} \frac{\del^2}{\del \vph^2}.
\edm %----------------------------------------------------------------------
ここで $(\theta,\vph)$ は次のような球面極座標である.

\begin{figure}[h]
\begin{center}
\includePdfEps{}{snj_polar_coordinate}
\caption{球面極座標}
\end{center}
\end{figure}

Laplace-Berltrami 作用素の固有値はよく知られているように
$n(n+1)$, $n=0,1,2,\dots$ である.

対応する固有関数を記述するには次のものが必要である.
\begin{itemize}
\item Legendre 多項式:
\bdm %----------------------------------------------------------------------
P_n(x)
= \frac{(-1)^n}{2^n n!} \frac{d^n}{dx^n}(1-x^2)^n.
\edm %----------------------------------------------------------------------

\item Legendre 多項式の満たす微分方程式
\bdm %----------------------------------------------------------------------
(1-x^2)P_n'' - 2x P_n' = -n(n+1) P_n.
\edm %----------------------------------------------------------------------
\item Legendre の陪関数:
\bdm %----------------------------------------------------------------------
P_n^m(x)
= (-1)^m(1-x^2)^{m/2}\frac{d^m}{dx^m} P_n(x).
\edm %----------------------------------------------------------------------
\item Legendre の陪関数の満たす微分方程式
\bdm %----------------------------------------------------------------------
(1-x^2)\frac{d^2}{dx^2} P_n^m(x) - 2x \frac{d}{dx} P_n^m(x) 
  + \biggl[ n(n+1)-\frac{m^2}{1-x^2} \biggr] P_n^m(x)
= 0.
\edm %----------------------------------------------------------------------
\end{itemize}

これらの関数を用いると,Laplace-Berltrami 作用素の固有関数は
\bdm %----------------------------------------------------------------------
P_n^m(\cos \theta) e^{im\vph}, \quad
n=0,1,\dots, \ m=-n, -n+1, \dots, -1, 0,1,\dots, n
\edm %----------------------------------------------------------------------
で与えられる.

これらはよく知られた結果であるが,Laplace-Beltrami 作用素に回転 $\frac{\del}{\del\vph}$ が加わったものもこれが固有関数になる.
実際
\bdm %----------------------------------------------------------------------
\frac{\del}{\del \vph} [ P_n^m(\cos \theta) e^{im\vph}]
= im P_n^m(\cos \theta) e^{im\vph}
\edm %----------------------------------------------------------------------
に注意すれば,$-\Laplace-\frac{\del}{\del \vph}$ の固有値は $n(n+1)+im$ で
対応する固有関数は 
\bdm %----------------------------------------------------------------------
P_n^m(\cos \theta) e^{im\vph}
\edm %----------------------------------------------------------------------
であることが容易に分かる.
図示すれば次のようになる.

\begin{figure}[h]
\begin{center}
\includePdfEps{}{snj_LB}
\includePdfEps{}{snj_LB_rotate}
\caption{回転を加えた Laplace-Beltrami 作用素のスペクトル}
\end{center}
\end{figure}

この作用素は正規作用素で実の作用素でもあるので,固有値が実軸に対して
対称に配置されていることが読み取れるだろう.










\subsec{Notes}
\begin{itemize}
\item 
\end{itemize}


	\makeatletter
\def\section{\@startsection {section}{1}{\z@}{-3.5ex plus -1ex minus 
    -.2ex}{2.3ex plus .2ex}{\center\bf}}
\makeatother
\begin{thebibliography}{99}
\small
\itemsep .5pt plus .2pt minus .1pt
\baselineskip=11.5pt
%
\bibitem{Kato76} T.~Kato, % Tosio Kato
``{\it Perturbation theory for linear operators. Second edition,\/}''
Second edition, % Grundlehren der Mathematischen Wissenschaften, Band 132,
Springer-Verlag, Berlin-New York, 1976.
%
\bibitem{Kobayashi72} S.~Kobayashi, % Shoshichi Kobayash,
Transformation groups in differential geometry,
Ergebnisse der Mathematik und ihrer Grenzgebiete, Band 70,
Springer-Verlag, New York-Heidelberg, 1972.
%
\bibitem{Kobayashi95} S.~Kobayashi, % Shoshichi Kobayash,
Transformation groups in differential geometry,
Reprint of the 1972 edition. Classics in Mathematics,
Springer-Verlag, Berlin, 1995.
%
\bibitem{Lions69} J.-L.~Lions, 
``{\it Quelques m\'ethodes de r\'esolution des probl\`emes aux limites non
lin\'eaires\/},'' Dunod, Gauthier-Villars, Paris 1969.
%
\bibitem{Petersen06} P.~Petersen, % Peter Petersen, 
``{\it Riemannian geometry\/},'' Second edition,
Graduate Texts in Mathematics, 171, Springer, New York, 2006.
%
\bibitem{Shigekawa10} I.~Shigekawa, % Ichiro Shigekawa
Non-symmetric diffusions on a Riemannian manifold,
``{\it Probabilistic approach to geometry\/},'',
Adv.\ Stud.\ Pure Math., 57, pp.~437--461, Math. Soc. Japan, Tokyo, 2010.
%
\bibitem{Stone32} M.~H.~Stone,
Linear transformations in Hilbert space and their applications to analysis,
Amer.\ Math.\ Soc.\ Colloq.\ Publi., {\bf 15}, Providence, 1932.
R. S. Strichartz,
Analysis of Laplacian on the complete Riemannian manifold,
{\it J. Funct. Anal.}, {\bf 52} (1983), 48--79.
%
\bibitem{Tanabe75} 田辺 広城、
発展方程式、岩波書店、東京、1975.
%
\bibitem{YB53} K.~Yano and S.~Bochner,
``{\it Curvature and Betti numbers\/},''
Annals of Mathematics Studies, No. 32,
Princeton University Press, Princeton, N. J., 1953.
%
\end{thebibliography}

\else
	%%%%%%%%%%%%%%%%%%%%%%%%%%%%%%%%%%%%%%%%%%%%%%%%%
%                                               %
%        =======  Program  No.1  =======        %
%                                               %
%===============================================%
%%%%%%%%%%%%%%%%%%%%%%%%%%%%%%%%%%%%%%%%%%%%%%%%%
%
%=======================  Title  ===========================================
\title{非対称作用素のスペクトル}
%
%=======================  Dedication  ======================================
%\dedicatory{Dedicated to Professor XX YY on his 70th birthday}
%=======================  Author  ==========================================
\author{重川 一郎
\thanks{e-mail: {\tt ichiro@math.kyoto-u.ac.jp},\quad
           URL: {\tt http://www.math.kyoto-u.ac.jp/\~{}ichiro/}}
\\ (京都大学大学院理学研究科)}
% \author{Ichiro Shigekawa
% \footnote{This research was partially supported
% by the Ministry of Education, Culture, Sports, Science and Technology,
% Grant-in-Aid for  Scientific Research (B), No.~11440045, 1999}
% }
\date{}
\maketitle
%=======================    Contents   =====================================
\setcounter{tocdepth}{2}
% \tableofcontents
% \address{Department of Mathematics, %\\
% Graduate School of Science, %\\
%%Faculty of Science, %\\
% Kyoto University, %\\
% Kyoto 606-8502, %\\
% Japan}
% \email{ichiro@kusm.kyoto-u.ac.jp}
% \urladdr{http://www.kusm.kyoto-u.ac.jp/\~{}ichiro/}
%\date{}
%====================  Scientific Research Fund Classification  ============
% This research was partially supported by the Ministry of Education, 
% Culture, Sports, Science and Technology, Grant-in-Aid for XXX,
% ZZZZZZZZ, 19YY
% XXX is
% Tokubetsu Suuisin Kenkyu     Specially Promoted Research 
% Juuten ryouiki Kenkyu        Scientific Research on Priority Areas
%                              (Area Name)
% Kiban Kenkyu (A), (B), (C)   Scientific Research (A),(B),(C) 
% Hougateki Kenkyu             Exploratory Research 
% Syourei Kenkyu (A)           Encouragement of Young Scientists 
% ZZZ is Kadai Bangou
%=======================  Mathematics Subject Classification  ==============
% \subjclass{60J60, 58G32}
% 31-XX POTENTIAL THEORY
% 31Cxx   Other generalization
% 31C15     Potentials and capacities
% 31C25     Dirichlet spaces
% 47Dxx Groups and semigroups of linear operators, their generalizations and applications 
% 47D03 Groups and semigroups of linear operators
% 47D06 One-parameter semigroups and linear evolution equations
% 47D08 Schrodinger and Feynman-Kac semigroups
% 47D09 Operator sine and cosine functions and higher-order Cauchy problems
% 47D60 $C$-semigroups
% 47D62 Integrated semigroups
% 47D99 None of the above, but in this section
% 58-XX GLOBAL ANALYSIS, ANALYSIS ON MANIFOLDS
% 58Bxx   Infinite dimensional manifolds
% 58B10     Differentiability questions
% 58Gxx   Partial differential equations on manifolds;differential operators
% 58G32     Diffusion processes and stochastic analysis on manifolds
% 60-XX PROBABILITY THEORY AND STOCHASTIC PROCESSES
% 60Hxx   Stochastic Analysis
% 60H07     Stochastic calculus of variation and the Malliavin calculus
% 60Jxx   Markov Process
% 60J05 Markov processes with discrete parameter
% 60J10 Markov chains with discrete parameter
% 60J20 Applications of discrete Markov processes
%       (social mobility, learning theory, industrial processes, etc.)
% 60J22 Computational methods in Markov chains
% 60J25 Markov processes with continuous parameter
% 60J27 Markov chains with continuous parameter
% 60J35 Transition functions, generators and resolvents
% 60J40 Right processes
% 60J45 Probabilistic potential theory [See also 31Cxx, 31D05]
% 60J50 Boundary theory
% 60J55 Local time and additive functionals
% 60J57 Multiplicative functionals
% 60J60 Diffusion processes [See also 58J65]
% 60J65 Brownian motion [See also 58J65]
% 60J70 Applications of diffusion theory
%       (population genetics, absorption problems, etc.) [See also 92Dxx]
% 60J75 Jump processes
% 60J80 Branching processes (Galton-Watson, birth-and-death, etc.)
% 60J85 Applications of branching processes [See also 92Dxx]
% 60J99 None of the above, but in this section
%=======================  Abstract  ========================================
%\begin{abstract} \end{abstract}
\hide
\vspace{-4mm}
\begin{itemize} \itemsep=-2mm \parsep=0mm
\item 非対称作用素のスペクトルについて.
\item Total file name: snj01 snj02 $\dots $ snj?, snj\_bibliography
\item File name: unj01.tex \hfill 印刷日: \today \ \now
\end{itemize}
\endhide
%=======================  Text  ============================================
%
\SS{DUI}{導入} %////////////////////////////////////////////////////////////
% Dual ultracontractivity introduction
\hide
非対称作用素のスペクトルについて述べる。
具体的にスペクトルが完全に決定できる場合を例としていくつか列挙する。
池野君が修論で纏めてくれたものに補足を加えたものである。
\hfill 2010年12月23日(木)
\endhide

作用素のスペクトルを完全に決定することは一般には難しい問題である。
対称な作用素については比較的よく調べられてきたといえるが、
非対称な場合のスペクトルの解析はまだ十分になされていない。
この論文では、非対称な作用素でスペクトルが完全に決定できるものの例を
いくつか見ていく。
非対称な作用素の中でも、正規作用素はスペクトル分解の理論がつかえ、
解析が易しくなることもあり、正規作用素を中心に論じる。

また正規作用素については、加藤の平方作用素の問題と言われるものについても
考察する。
この問題は作用素の平方根の定義域と、対応する双線型形式の定義域とが一致するかどうかという問題である。
正規作用素の場合はこの問題は容易に解くことができることを注意する。
 %%%%%%%%%%%%%%%%%%%%%%%%%%%%%%%%%%%%%%%%%%%%%%%%%
%                                               %
%        =======  Program  No.2  =======        %
%                                               %
%===============================================%
%%%%%%%%%%%%%%%%%%%%%%%%%%%%%%%%%%%%%%%%%%%%%%%%%
%
\hide
\vspace{-4mm}
\begin{itemize} \itemsep=-2mm \parsep=0mm
\item Total file name: snj01 snj02 $\dots $ snj?, snj\_bibliography
\item File name: snj02.tex \hfill コンパイル日: \today \ \now
\end{itemize}
\endhide

\SS{SFH}{Hilbert 空間の上の sesquilinear form} %////////////////////////////
% Sesquilinear forms in a Hilbert space

Hilbert 空間上の sesquilinear form ついて,後で必要となることをまとめておく.$\Dom(a)\times\Dom(a)$ 上の関数 $a(u,v)$ で $u$ に関して線型,
$v$ に関して共役線型な関数を sesquilinear form という.
さらに $a(u)=a(u,u)$ を quadratic form という.
$a(u,v)$ は次の関係で $a(u)$ から一意的に定まる.
\bdn %----------------------------------------------------------------------
a(u,v)
= \frac{1}{4}(a(u+v) - a(u-v) + ia(u+iv) - ia(u-iv)).
\Eqn{SFH.4}
\edn %----------------------------------------------------------------------
sesquilinear form $a$ が
\bdm %----------------------------------------------------------------------
a(u,v) = \ol{a(v,u)}
\edm %----------------------------------------------------------------------
をみたすとき,対称であるという.
$a$ が対称であるための必要十分条件は,
対応する quadratic form が実数値であることである.

一般の sesquilinear form $a$ に対して adjoint form を
\bdn %----------------------------------------------------------------------
a^*(u,v)
= \ol{a(v,u)}
\Eqn{SFH.6}
\edn %----------------------------------------------------------------------
で定義する.
$a$ が対称であることは $a=a^*$ が成り立つことに他ならない.
また sesquilinear form $a$ に対して
\bdn %----------------------------------------------------------------------
b
= \frac{1}{2}(a+a^*), \quad
c
= \frac{1}{2i}(a-a^*),
\Eqn{SFH.8}
\edn %----------------------------------------------------------------------
とおくと
\bdn %----------------------------------------------------------------------
a
= b + ic
\Eqn{SFH.10}
\edn %----------------------------------------------------------------------
が成り立つ.
$b$ を実部,$c$ を虚部と呼ぶ.
これは quadratic form で考えると
\bdn %----------------------------------------------------------------------
b(u)
= \Re a(u), \quad
c(u)
= \Im a(u)
\Eqn{SFH.12}
\edn %----------------------------------------------------------------------
の関係が成り立っている.

$a$ がsymmetric のとき
\bdn %----------------------------------------------------------------------
a(u)
\ge 0, \quad \forall u\in\Dom(a)
\Eqn{SFH.14}
\edn %----------------------------------------------------------------------
が成り立つとき,非負であるという.
正値性 から次の Schwarz の不等式が成り立つ:
\bdn %----------------------------------------------------------------------
a(u,v)
\le a(u)^{1/2}a(v)^{1/2}.
\Eqn{SFH.16}
\edn %----------------------------------------------------------------------
さて,$a$ を非負として,$t$ を一般の symmetric form とする.
このとき $\Dom(a) \subseteq \Dom(t) $ で quadratic form として
\bdm %----------------------------------------------------------------------
|t(u)| \le K a(u), \quad \forall u\in\Dom(a)
\edm %----------------------------------------------------------------------
が成り立つとき,
\bdm %----------------------------------------------------------------------
|t(u,v)| \le K a(u)^{1/2} a(v)^{1/2}, \quad \forall u,v\in\Dom(a)
\edm %----------------------------------------------------------------------
が成立する.
これを見るには,$u$ の代わりに $e^{i\theta}u$ を考えることにより
$t(u,v)$ は実数としてよい.
すると \Eq{SFH.4} の表示は
\bdm %----------------------------------------------------------------------
t(u,v)
= \frac{1}{4}(t(u+v) - t(u-v))
\edm %----------------------------------------------------------------------
となる.
ここで quadratic form $t$ が実数値であることを使っている.
よって仮定から
\bdm %----------------------------------------------------------------------
|t(u,v)|
\le \frac{K}{4}(a(u+v) + a(u-v))
=   \frac{K}{2}(a(u) + a(v)).
\edm %----------------------------------------------------------------------
ここで $u$, $v$ の代わりに $\e u$, $\e^{-1}v$ をとれば
\bdm %----------------------------------------------------------------------
|t(u,v)|
\le \frac{K}{2}(\e a(u) + \e^{-1} a(v)).
\edm %----------------------------------------------------------------------
$a(u)=0$ ならば $\e\to\infty$ として $t(u,v)=0$ を得る.
$a(u)\not=0$ のときは $\e=\frac{a(v)^1/2}{a(u)^{1/2}}$ と取れば
\bdm %----------------------------------------------------------------------
|t(u,v)|
\le K a(u)^{1/2} a(v)^{1/2}
\edm %----------------------------------------------------------------------
が成り立つことが分かる.

さて $a$ を sesquilinear form とする.  
$a = b+ic$ を実部と虚部に分ける.
$b$ が非負であるとき $a$ も非負 (または accretive) という.
\bdm %----------------------------------------------------------------------
\Re a(u,u) \ge 0, \quad \forall u\in\Dom(a)
\edm %----------------------------------------------------------------------
また $a$ の numerical range を
\bdn %----------------------------------------------------------------------
\Theta(a)
= \{ a(u,u);\, u\in\Dom(a),\ |u|=1\}
\Eqn{SFH.20}
\edn %----------------------------------------------------------------------
で定義する.
$a$ が非負であることは
$\Theta(a) \subseteq \{\zeta\in\C;\, \Re \zeta \ge 0\}$ と同値である.
さらに $a$ が扇形条件をみたす(sectorial)ということを
\bdm %----------------------------------------------------------------------
\Theta(a)
\subseteq S_\theta
\edm %----------------------------------------------------------------------
で定義する.
ここで $S\theta$ 次で定まる角領域である:
\bdn %----------------------------------------------------------------------
S_\theta
= \{z\in\C;\, |\arg z| \le \theta \}.
\Eqn{SFH.22}
\edn %----------------------------------------------------------------------
但し $\theta$ は $0\le \theta<\frac{\pi}{2}$ にとる.

扇形条件が成り立つと,実部 $b$ は非負である.
この条件は $a(u,u)= b(u)+ic(u)$ に注意すれば
$c(u) \le \tan\theta b(u)$ とも同値である.
さらにこの条件は次の条件とも同値になる:
$b$ が非負で
ある定数 $K \ge 0$ が存在して
\bdn %----------------------------------------------------------------------
|a(u,v)|
\le (1+K) b(u)^{1/2} b(v)^{1/2}
\Eqn{SFH.24}
\edn %----------------------------------------------------------------------
が成り立つ.
このことを証明しておこう.

\Proposition{SFH-6} %*******************************************************
$a$ が \Eq{SFH.24} をみたすことと,
ある $\theta\in [0,\frac{\pi}{2})$ が存在して
$\Theta(a) \subseteq S_\theta$ となることは同値である.
\end{proposition} %*********************************************************

\Proof
まず \Eq{SFH.24} が成り立っているとする.
すると $v=u$ として両辺を2乗して
\bdm %----------------------------------------------------------------------
|a(u,u)|^2
\le (1+K)^2 b(u)^2.
\edm %----------------------------------------------------------------------
ところで,$a(u,u) = b(u) + ic(u)$ だから左辺は $b(u)^2 + c(u)^2$ であり
\bdm %----------------------------------------------------------------------
c(u)^2 + b(u)^2
\le (1+2K + K^2) b(u)^2
\edm %----------------------------------------------------------------------
から
\bdm %----------------------------------------------------------------------
c(u) \le \sqrt{K(2 + K)} b(u)
\edm %----------------------------------------------------------------------
が成り立つ.
即ち $\theta=\arctan \sqrt{K(2 + K)}$ として
$\Theta(a) \subseteq S_\theta$ が成り立つ.

逆に $\Theta(a) \subseteq S_\theta$ が成り立てば
$c(u) \le \tan\theta b(u)$ なので
\bdm %----------------------------------------------------------------------
|c(u,v)| \le \tan\theta b(u)^{1/2}b(v)^{1/2}
\edm %----------------------------------------------------------------------
が成り立つ.
従って
\bdm %----------------------------------------------------------------------
|a(u,v)|
&= |b(u,v) + ic(u,v)| \\
&\le |b(u,v)| + |c(u,v)| \\
&\le b(u)^{1/2}b(v)^{1/2} + \tan\theta b(u)^{1/2}b(v)^{1/2} \\
&= (1+\tan \theta) b(u)^{1/2}b(v)^{1/2}.
\edm %----------------------------------------------------------------------
これで \Eq{SFH.24} が $K=\tan\theta$ として成立することが分かった.
\QED %======================================================================

非負性はもう少し緩めることが出来る.
$a$ が下に有界であることを,ある $\gm\in\R$ が存在して
\bdn %----------------------------------------------------------------------
S_\theta
\subseteq \{\zeta \in\C;\, \Re \zeta \ge - \gm \}.
\Eqn{SFH.28}
\edn %----------------------------------------------------------------------
さらに扇形的に下に有界 (sectorially bounded from below) であることを
ある $\gm\in\R$ が存在して
\bdm %----------------------------------------------------------------------
\Theta(a)
\subseteq S_\theta - \gm 
\edm %----------------------------------------------------------------------
が成り立つと定義する.
これは $a_\gm$ を
\bdm %----------------------------------------------------------------------
a_\gm(u,v)
= a(u,v) + \gm(u,v)
\edm %----------------------------------------------------------------------
で定義したとき $a_\gm$ が 扇形条件をみたすことに他ならない.


\subsec{Notes}
\begin{itemize}
\item ここの話は Kato \cite{Kato76} Chapter VI, \S1,\S2 に元づく.
扇形条件という言葉をどう定義するかは迷うところである.
ここでは numerical range を用いて定義した.
Ma-R\"ockner とは用語法がずれる.
[2011年1月15日]
\end{itemize}


	%%%%%%%%%%%%%%%%%%%%%%%%%%%%%%%%%%%%%%%%%%%%%%%%%
%                                               %
%        =======  Program  No.3  =======        %
%                                               %
%===============================================%
%%%%%%%%%%%%%%%%%%%%%%%%%%%%%%%%%%%%%%%%%%%%%%%%%
%
\hide
\vspace{-4mm}
\begin{itemize} \itemsep=-2mm \parsep=0mm
\item Total file name: snj01 snj02 $\dots $ snj?, snj\_bibliography
\item File name: snj03.tex \hfill コンパイル日: \today \ \now
\end{itemize}
\endhide

\SS{OHS}{Hilbert 空間の上の作用素} %////////////////////////////////////////
% Operators on a Hilbert spaces

Hilbert 空間上の作用素のついて,後で必要となることをまとめておく.
$T$ をHilbert 空間 $H$ 上の作用素とする.
$T$ の numerical range $\Theta(T)$ を
\bdn %----------------------------------------------------------------------
\Theta(T)
:= \{(Tu,u);\, u\in\Dom(T)\}. 
\Eqn{OHS.4}
\edn %----------------------------------------------------------------------
ここでは証明しないが $\Theta(T)$ は凸集合であることが知られている
(Stone \cite{Stone32} を参照).
$\Delta=\C\setminus \ol{\Theta(T)}$ とする.
$\Theta(T)$ は凸集合だから,$\Delta$ の連結成分は高々二つである.
連結成分が2つある場合は $\Delta_1$, $\Delta_2$ とかく.
  

\Theorem{OHS-4} %***********************************************************
$T$ を閉作用素とする.
$\zeta\in\Delta$ のとき $T-\zeta$ の像は閉集合となり,
$\Ker(T-\zeta)=\{0\}$ で $T-\zeta$ の指数は $\Delta$ の各連結成分で
一定である.
もし $\Delta$ ($\Delta_1$ or $\Delta_2$) で $T-\zeta$ の指数が $0$
であるならば $\Delta$ ($\Delta_1$ or $\Delta_2$) はレゾルベント集合
$\rho(T)$ に含まれる.
\end{theorem} %*************************************************************

\Proof
まず $u\in\Dom(T)$, $|u|=1$ のとき
\bdm %----------------------------------------------------------------------
|(Tu,u)-\zeta|
= |((T-\zeta u),u)-\zeta|
\le |(T-\zeta)u|
\edm %----------------------------------------------------------------------
が成り立つことに注意しよう.
これから $\zeta\in\Delta$ のとき $\d=d(\zeta, \ol{\Theta(T)})>0$ とおくと
$|(T-\zeta)u|\ge \d$, $|u|=1$ だから
\bdm %----------------------------------------------------------------------
|(T-\zeta)u|
\ge \d |u|, \quad \forall u\in\Dom(T).
\edm %----------------------------------------------------------------------
これから $\Ker(T-\zeta)=0$ で $\Ran(T-\zeta)$ が閉集合であることが従う.
また各連結成分で $T-\zeta$ の指数は指数の安定性から一定である.
特に指数が 0 のときは $\Ran(T-\zeta)=H$ であるから有界な逆が存在する.
従って $\rho(T)$ に含まれていることが分かる.
\QED %======================================================================

さて閉作用素 $A$ が accretive であることを
$\Re (Au,u)\ge 0$, $\forall u\in\Dom(A)$
で定義する.
これは $\Theta(A)$ が右半平面に含まれることを意味する.
さらにある $\Re \zeta<0$ となる $\zeta$ が存在して $\Ran(A-\zeta)=H$
が成り立っているとき,$\gen$ を $m$-accretive と呼ぶ.
このとき $\Re \lm<0$ となる任意の $\lm$ に対して $\lm\in\rho(A)$ で
\bdm %----------------------------------------------------------------------
\|(A-\lm)^{-1}\| \le |\Re \lm|^{-1}
\edm %----------------------------------------------------------------------
が成り立っている.
$A$ の定義域が稠密であることは $m$-accretive の条件から必然的に従う.
実際,$((A-\lm)^{-1}u,v)$, $\forall u\in H$ が成り立てば
$u-v$, $w=(A-\lm)^{-1}v$ として
\bdm %----------------------------------------------------------------------
0
= \Re((A-\lm)^{-1}v,v)
= \Re(w,(A-\lm)w)
\ge -\Re\lm |w|^2
\edm %----------------------------------------------------------------------
から $w=0$ が従う.

さて $A$ が accretive のとき
$\Theta(A) \subseteq S_\theta$, $\theta\in[0,\frac{\pi}{2})$
であるとき,扇形条件が成り立つ (sectorial) という.
ここで $S_\theta$ は次で定まる角領域である:
$S_\theta = \{z\in\C;\, |\arg z| \le \theta \}$.
これらの概念は sesquilinear form のときに既に出てきている.
実際 sesquilinear form $a$ を
\bdn %----------------------------------------------------------------------
a(u,v)
= (Au,v)
\Eqn{OHS.8}
\edn %----------------------------------------------------------------------
で定めればよい.
定義域は $\Dom(A)\times \Dom(A)$ とする.
$a$ の実部,虚部を $b$, $c$ とすれば
\bdmn %---------------------------------------------------------------------
b(u,v)
&= \frac{1}{2}\{ (Au,v) + (u,Av)\}
\Eqn{OHS.10} \\
c(u,v)
&= \frac{1}{2i}\{ (Au,v) - (u,Av)\}
\Eqn{OHS.12}
\edmn %---------------------------------------------------------------------
\Prop{SFH-6} から扇形条件は次の条件と同値である:
ある定数 $K$ が存在して
\bdn %----------------------------------------------------------------------
|(Au,v)|
\le K (\Re(Au,u))^{1/2} (\Re(Av,v))^{1/2}.
\Eqn{OHS.16}
\edn %----------------------------------------------------------------------

$T$ を閉作用素とする.
ある実数 $\gm$ が存在して $T+\gm$ が accretive のとき
 $T$ を quasi-accretive という.
また $T+\gm$ が扇形条件をみたすとき,quasi-sectorial という.
さらに $T+\gm$ が $m$-accretive のとき quasi-$m$-accretive といい,
これに加えて扇形条件をみたすとき quasi-$m$-sectorial であるという.


\bigskip
今後 Hilbert 空間上の作用素のスペクトルのことを問題にしていくわけだが,
そのためには複素Hilbert 空間で調べることが自然である.
そこで,実 Hilbert 空間の複素化の話をまずまとめておく.

$H$ を実Hilbert 空間とする.
複素化を $H\oplus i H$ で定める.
$H\oplus i H$ の元は $u+iv$ ($u$,$v\in H$) と表される.
スカラー倍は $x+iy\in\C$ に対し
\bdn %----------------------------------------------------------------------
(x+iy)(u+iv)
= (x u -y v) + i(x v + y u) 
\Eqn{OHS.20}
\edn %----------------------------------------------------------------------
で定め,内積は
\bdn %----------------------------------------------------------------------
(u+iv, r+is)
=(u,r) + (v,s) + i\{(v,r) - (u,s)\}
\Eqn{OHS.22}
\edn %----------------------------------------------------------------------
で定める.
また $H\oplus i H$ には自然な共役作用素 $\J$ が次で定義される.
\bdn %----------------------------------------------------------------------
\J(u+iv)
= u-iv.
\Eqn{OHS.24}
\edn %----------------------------------------------------------------------
$\J$ は次の関係をみたす.
\bdmn %---------------------------------------------------------------------
\J^2(f)
&= f,
\Eqn{OHS.26} \\
\J(\al f + \be g)
&= \ol{\al} \J(f) + \ol{\be}\J(g).
\Eqn{OHS.28}
\edmn %---------------------------------------------------------------------
複素数に対しては $\ol{\phantom{f}}$ は 共役複素数を表す.
従って $\J$ は共役線型であることに注意しよう.
また内積については
\bdn %----------------------------------------------------------------------
(\J(f), \J(g))
= \ol{(f,g)}
\Eqn{OHS.30}
\edn %----------------------------------------------------------------------
が成り立っている.
以後 $\J(f)$ を $\ol{f}$ ともかく.

逆に複素 Hilbert 空間 $H$ に,\Eq{OHS.26}, \Eq{OHS.28}, \Eq{OHS.30}
をみたす共役作用素 $\J$ が与えられているとき,
\bdm %----------------------------------------------------------------------
H_r
= \{u\in H;\, \J(u)= u\}
\edm %----------------------------------------------------------------------
で定めると,$H$ は $H_r$ の複素化になる.
$H$ の作用素 $A$ が
\bdm %----------------------------------------------------------------------
A\J
= \J A
\edm %----------------------------------------------------------------------
をみたすとき実作用素と呼ぶ.
$A$ が実作用素のとき,$A$ は $H_r$ の作用素になる.
もう少し正確に言うと $f\in \Dom(A)\cap H_r$ ならば $Af\in H_r$ で,
さらに $\Dom(A) = (\Dom(A)\cap H_r)\oplus i(\Dom(A)\cap H_r)$ が成り立つ.
逆に実 Hilbert 空間 $H_r$ の作用素は,複素化すると実作用素になる.

さて $A$ を実作用素とする.
$\J$ の定義から $\J (A-\zeta)\J=A-\ol{\zeta}$ である.
これから $\Ker(A-\zeta)=\{0\}$ $\Leftrightarrow$ $\Ker(A-\ol{\zeta})=\{0\}$
であり $\Ran(A-\zeta)=H$ $\Leftrightarrow$ $\Ran(A-\ol{\zeta})=H$ である.
これから $(A-\zeta)^{-1}$ が $H$ 全体で定義された有界作用素になるときには
$(A-\ol{\zeta})^{-1}$ も $H$ 全体で定義された有界作用素になる.
従ってレゾルベント集合は共役をとる操作で不変になる.
スペクトル集合も同じ性質を持つことが分かる.

実の Hilbert 空間で双一次形式 $a$ が与えられたとき,
その複素化について考えよう.
$a$ は自然に複素化された Hilbert 空間で sesquilinear form に拡張できる.
従って $a$ は第2変数について共役線型だから
\bdm %----------------------------------------------------------------------
a(u+iv,u+iv)
&= a(u,u)+a(v,v) -i a(u,v) + ia(v,u) \\
&= a(u,u)+a(v,v) + i \{a(v,u) - a(u,v)\}
\edm %----------------------------------------------------------------------
が成り立つ.
従って $a$ の実部,虚部を $b$, $c$ とすると
\bdm %----------------------------------------------------------------------
b(u+iv) = a(u,u)+a(v,v), \quad
c(u+iv) = a(v,u)-a(u,v)
\edm %----------------------------------------------------------------------
これから扇形条件 $|c| \le \tan\theta b$ は
$|a(v,u)-a(u,v)| \le \tan\theta\{a(u,u)+a(v,v)\}$ となる.
$u$ の代わりに $\e u$, $v$ の代わりに $\e^{-1} v$ をとって
\bdm %----------------------------------------------------------------------
|a(v,u)-a(u,v)| \le \tan\theta\{\e^2 a(u,u)+\e^{-2} a(v,v)\}
\edm %----------------------------------------------------------------------
特に $\e^2= \frac{a(v,v)^{1/2}}{a(u,u)^{1/2}}$ をとれば
\bdn %----------------------------------------------------------------------
|a(v,u)-a(u,v)| \le 2 \tan\theta a(u,u)^{1/2} a(v,v)^{1/2}
\Eqn{OHS.32}
\edn %----------------------------------------------------------------------
が成り立つ.
逆にこの関係から
\bdm %----------------------------------------------------------------------
|a(v,u)-a(u,v)| \le 2 \tan\theta a(u,u)^{1/2} a(v,v)^{1/2}
\le  \tan\theta \{a(u,u) + a(v,v)\}
\edm %----------------------------------------------------------------------
が成り立つ.
従って扇形条件は上の式と同値になる.

\subsec{Notes}
\begin{itemize}
\item accretive な作用素についてのまとめ.
\item 実 Hilbert 空間の複素化と実作用素.
\end{itemize}




	%%%%%%%%%%%%%%%%%%%%%%%%%%%%%%%%%%%%%%%%%%%%%%%%%
%                                               %
%        =======  Program  No.4  =======        %
%                                               %
%===============================================%
%%%%%%%%%%%%%%%%%%%%%%%%%%%%%%%%%%%%%%%%%%%%%%%%%
%
\hide
\vspace{-4mm}
\begin{itemize} \itemsep=-2mm \parsep=0mm
\item Total file name: snj01 snj02 $\dots $ snj?, snj\_bibliography
\item File name: snj04.tex \hfill コンパイル日: \today \ \now
\end{itemize}
\endhide

\SS{NSS}{正規作用素} %//////////////////////////////////////////////////////
% Non-symmetric semigroup
\hide
論文としてまとめようと思いながら,とうとう2011年となってしまった.
前に進むのが相変わらず遅い.
\hfill 2011年1月3日(月)
\endhide

正規作用素を考察しよう
作用素 $A$ が正規作用素とは,$A$ が次の条件をみたすときをいう.
\bdn %----------------------------------------------------------------------
A^*A
= AA^*.
\Eqn{NSS.20}
\edn %----------------------------------------------------------------------
自己共役作用素は正規作用素であるが,一般に作用素の正規性を確かめるのは
難しい問題である.
応用上現れる作用素は,定義域を特徴付けることが難しく,
ある作用素の閉包,というような形で定義されることが多いからである.
この問題はまた後で考えることにする.
$A$ が正規作用素なら複素数 $\lm$ に対し $\lm+A$ も正規であることは
容易に分かる.
実際
\bdm %----------------------------------------------------------------------
(\lm+A)^* (\lm+A)
&= (\ol{\lm} +A^*) (\lm+A)
= \lm^2 + \ol{\lm}A + \lm A^* + A^*A \\
&= \lm^2 + \ol{\lm}A + \lm A^* + AA^*
= (\lm+A) (\lm+A)^*.
\edm %----------------------------------------------------------------------
正規作用素に $A$ 対しては スペクトル分解定理が成立する.
即ち,単位の分解 $E(dz)$ が存在し
\bdn %----------------------------------------------------------------------
A
= \int_\C z E(dz)
\Eqn{NSS.22}
\edn %----------------------------------------------------------------------
と表現される.
$E$ は射影作用素に値をとる $\C$ 上の測度である.
$\s(A)$ を $A$ のスペクトル集合とすると $E(dz)$ の台は $\s(A)$ と一致する.
ここで
\bdm %----------------------------------------------------------------------
(Au,u)
= \int_\C z (E(dz)u,u).
\edm %----------------------------------------------------------------------
が成立する.
右辺の積分は $\C$ 全体ではなく $\s(A)$ に制限してよい.
$|u|=1$ ならば $(E(dz)u,u)$ は 確率測度である.
つまり $(Au,u)$ は $\s(A)$ の convex combination で表されていることになる.
$\s(A)$ の閉凸包を $\co(\s(A))$ で表すと numerical range $\Theta(A)$
に対しては $\Theta(A) \subseteq \co(\s(A))$ が成り立つ.
一方 $\ol{\Theta(A)}$ はスペクトルを含む閉凸集合であった.
従って
\bdn %----------------------------------------------------------------------
\ol{\Theta(A)} = \co(\s(A))
\Eqn{NSS.24}
\edn %----------------------------------------------------------------------
が得られた.
これで sectorial かどうかはスペクトル $\s(A)$ を見れば
完全に分かることになる.

さて,後で縮小半群を生成する作用素を考えたいので,
ここで扱う正規作用素は次の増大性を仮定する
\bdn %----------------------------------------------------------------------
\Re (Au,u)
\ge 0, \quad \forall u\in \Dom(A).
\Eqn{NSS.26}
\edn %----------------------------------------------------------------------
従って縮小半群との対応を考えるときは $-A$ が縮小半群の生成作用素となる.
\Eq{NSS.22} から $A^*$ に対しては
\bdn %----------------------------------------------------------------------
A^*
= \int_\C \ol{z} E(dz)
\Eqn{NSS.28}
\edn %----------------------------------------------------------------------
が成り立つ.
これは
\bdm %----------------------------------------------------------------------
(A^*u,v)
= (u,Av)
= (u,\int_\C z E(dz)v)
= \int_\C \ol{z} (u,E(dz)v)
= \int_\C \ol{z} (E(dz)u,v)
= (\int_\C \ol{z}E(dz)u,v)
\edm %----------------------------------------------------------------------
から分かる.


さらに $A$ はスペクトル分解を持つから作用素の平方根 $\sqrt{A}$ が
次で定義できる.
\bdn %----------------------------------------------------------------------
\sqrt{A}
= \int_\C \sqrt{z} E(dz).
\Eqn{NSS.30}
\edn %----------------------------------------------------------------------
ここで $\sqrt{z}$ の分岐は実軸上で $\sqrt{z}$ となるものをとる.
この作用素の定義域は $|\sqrt{z}|^2 = |z|$ であるから
\bdn %----------------------------------------------------------------------
\Dom(\sqrt{A})
= \{u\in H;\, \int_\C |z| (u,E(dz)u) < \infty \}
\Eqn{NSS.32}
\edn %----------------------------------------------------------------------
である. 
同様に $A^*$ に対しても
\bdn %----------------------------------------------------------------------
\sqrt{A^*}
= \int_\C \sqrt{\ol{z}} E(dz)
\Eqn{NSS.34}
\edn %----------------------------------------------------------------------
および
\bdn %----------------------------------------------------------------------
\Dom(\sqrt{A^*})
= \{u\in H;\, \int_\C |z| (u,E(dz)u) < \infty \}
\Eqn{NSS.36}
\edn %----------------------------------------------------------------------
が成り立つ.
両者から正規作用素に対しては 
\bdn %----------------------------------------------------------------------
\Dom(\sqrt{A})
= \Dom(\sqrt{A^*})
\Eqn{NSS.38}
\edn %----------------------------------------------------------------------
が常に成り立っていることが分かる.
これはある意味で Kato の平方根問題が肯定的に解けていることを意味する.
ここでは sesquilinear form との関連を調べよう.

さて, $A$ に対応する sesquilinear form $a$ を定義するなら
\bdn %----------------------------------------------------------------------
a(u,v)
= (Au,v), \quad u,v\in\Dom(A)
\Eqn{NSS.40}
\edn %----------------------------------------------------------------------
とするのが自然である.
さらにこの $a$ の実部 (対称な部分) $b$ は
\bdn %----------------------------------------------------------------------
b(u,v)
= \frac{(Au,v) + (A^*u,v)}{2}, \quad u,v\in\Dom(A)
\Eqn{NSS.42}
\edn %----------------------------------------------------------------------
で定義される.
増大性の仮定から 
\bdm %----------------------------------------------------------------------
b(u,u)
= \frac{(Au,u) + (A^*u,u)}{2}
= \Re(Au,u)
\ge 0
\edm %----------------------------------------------------------------------
が成り立つ.
さらにスペクトル分解を用いて
\bdm %----------------------------------------------------------------------
b(u,v)
= \int_\C \Re{z} (u,E(dz)v)
\edm %----------------------------------------------------------------------
が成り立つ.
これから定義域を次で定めると $b$ は閉になる:
\bdn %----------------------------------------------------------------------
\Dom(b)
= \{u\in H;\, \int_\C \Re{z} (u,E(dz)u) < \infty \}.
\Eqn{NSS.44}
\edn %----------------------------------------------------------------------

これから $\Dom(\sqrt{A}) \subseteq \Dom(b)$ であることは明らかである.
二つが一致するための必要十分条件を与えよう.
$0<\theta<\pi$ に対し複素平面内の角領域 $S_\theta$ を
\bdn %----------------------------------------------------------------------
S_\theta
= \{z\in\C;, |\arg z| \le \theta \}
\Eqn{NSS.50}
\edn %----------------------------------------------------------------------
で定義する.
このとき次が成り立つ.
\Theorem{NSS-4} %***********************************************************
$\Dom(\sqrt{A}) = \Dom(b)$ が成り立つための必要十分条件は
ある $\theta\in (0,\pi/2)$ が存在して $1+\s(A) \subseteq S_\theta$
となることである.
\end{theorem} %*************************************************************

\Proof
ある $\theta\in(0,\pi/2)$ が存在して $1+\s(A) \subseteq S_\theta$
が成立しているとしよう.
すると $u\in \Dom(\sDiri)$ ならば
$z=x+iy\in S_\theta-1$ ならば $|y| \le (x+1)\tan\theta$ だから
\bdm %----------------------------------------------------------------------
\int_\C |z| (E(dz)u,u)
&= \int_{S_\theta-1} |z| (E(dz)u,u) \\
&\le \int_{S_\theta-1} (x+ |y|) (E(dz)u,u) \\
&\le \int_{S_\theta-1} (x+ (x+1)\tan\theta) (E(dz)u,u) \\
&\le (1+\tan\theta) \int_{S_\theta-1} x (E(dz)u,u)
    + \tan\theta \int_{S_\theta-1} (E(dz)u,u)
< \infty
\edm %----------------------------------------------------------------------
となり $u\in \Dom(\sqrt{A})$ が従う.

逆に,いくら $\theta$ を $\frac{\pi}{2}$ に近くとっても
$1+\s(A) \subseteq S_\theta$ とならないときは,任意の $n\in\N$ に対し
$z_n=x_n+iy_n\in \s(A)$ を
\bdm %----------------------------------------------------------------------
|y_n|
\ge n(x_n+1) + 1
\edm %----------------------------------------------------------------------
が成り立つように出来る.
\memo{条件からは $|y_n| \ge n(x_n+1)$ だけど,これから $|y_n|-1 \ge (n-1)(x_n+1) +1$ が成立するから,番号を一つずらせばよい.}
さらに $|z_n-z_m|>2$ ($n\not=m$)としてよい.
$B_n$ を $z_n$ を中心とする半径 $1$ の閉円板とする.
そして $u_n\in\Ran E(B_n)$ を $|u_n|=1$ となるようにとる.
とり方から $\{u_n\}$ は正規直交系になる.
これを用いて $u = \sum_n \frac{u_n}{n\sqrt{x_n+1}}$ と定める.
$u\in H$ は明らかである.
また 
\bdm %----------------------------------------------------------------------
\int_\C \Re z(E(dz)u,u)
&=   \sum_n \int_\C \Re z
     (E(dz)\frac{u_n}{n\sqrt{x_n+1}},\frac{u_n}{n\sqrt{x_n+1}}) \\
&=   \sum_n \int_{B_n} \Re z
     (E(dz)\frac{u_n}{n\sqrt{x_n+1}},\frac{u_n}{n\sqrt{x_n+1}}) \\
&\le \sum_n \int_{B_n} (x_n+1) \frac{1}{n^2(x_n+1)} (E(dz)u_n,u_n) \\
&= \sum_n \frac{1}{n^2}
<  \infty
\edm %----------------------------------------------------------------------
より $u\in\Dom(\sDiri)$ である.

一方
\bdm %----------------------------------------------------------------------
\int_\C |z|(E(dz)u,u)
&=   \sum_n \int_\C |z|
     (E(dz)\frac{u_n}{n\sqrt{x_n+1}},\frac{u_n}{n\sqrt{x_n+1}}) \\
&=   \sum_n \int_{B_n} |z|
     (E(dz)\frac{u_n}{n\sqrt{x_n+1}},\frac{u_n}{n\sqrt{x_n+1}}) \\
&\ge \sum_n \int_{B_n} (|y_n|-1) \frac{1}{n^2(x_n+1)} (E(dz)u_n,u_n) \\
&\ge \sum_n \frac{n(x_n+1)}{n^2(x_n+1)}
=    \infty
\edm %----------------------------------------------------------------------
となるから $u\not\in\Dom(\sqrt{A})$ である.
これで主張が示せた.
\QED %======================================================================

最後に実作用素の場合の注意を与えておこう.
正規作用素 $A$ が実作用素であるとする.
単位の分解 $E(dz)$ が存在して
\bdm %----------------------------------------------------------------------
A
= \int_\C z E(dz)
\edm %----------------------------------------------------------------------
と表される.
従って
\bdm %----------------------------------------------------------------------
\J A \J
= \J \int_\C z E(dz) \J
=  \int_\C \ol{z} \J E(dz)\J
=  \int_\C z \J E(d\ol{z})\J. \quad (\because \text{ 変数変換})
\edm %----------------------------------------------------------------------
$\J E(d\ol{z})\J$ も一つの単位の分解を与えている.
ここで $A=\J A \J$ だから,分解の一意性から二つの単位の分解は一致する.
従って
\bdn %----------------------------------------------------------------------
\J E(d\ol{z})\J
=  E(dz)
\Eqn{NSS.58}
\edn %----------------------------------------------------------------------
あるいは
\bdn %----------------------------------------------------------------------
E(d\ol{z})
= \J  E(dz)\J
\Eqn{NSS.60}
\edn %----------------------------------------------------------------------
成立していることが示せた.



\subsec{Notes}
\begin{itemize}
\item
\end{itemize}


	%%%%%%%%%%%%%%%%%%%%%%%%%%%%%%%%%%%%%%%%%%%%%%%%%
%                                               %
%         ====== Program  No.5 =======          %
%                                               %
%             file name snj05.tex               %
%                                               %
%===============================================%
%===================  for  =====================%
%===============================================%
%
\hide
\vspace{-4mm}
\begin{itemize} \itemsep=-2mm \parsep=0mm
\item Total file name: snj01 snj02 $\dots $ snj?, snj\_bibliography
\item File name: snj05.tex \hfill 印刷日: \today \ \now
\item 生成作用素が正規であるための条件を求める.
[2009年8月19日]
\item この節は 初め lnj にあったのを移動させてきた.
Riemannian manifold の上の作用素も一般的な形で論じてある.
単に趣味的なだけではあるが.
[2011年1月5日]
\end{itemize}
\endhide
\SS{CNO}{正規作用素であるための条件} %======================================
% Condition for normal operator
生成作用素が正規であるための条件を求めよう.
まず一般的な枠組みを準備し,その後で Riemannian manifold の場合を考える.

\Theorem{CNO-1} %**********************************************************
$H$ を Hilbert 空間,$A$, $B$ を $\sD$ を定義域とする消散作用素,
$\ol{A}$, $\ol{B}$ をその閉方とし,
$\ol{A}$, $\ol{B}$ が $m$-dissipative であることを仮定する.
さらに $A\sD\subseteq \sD$, $B\sD\subseteq \sD$ で
\bdmn %---------------------------------------------------------------------
AB &= BA \quad \text{on $\sD$}
\Eqn{CNO.2} \\
(Au,v) &= (u,Bv), \quad  u,v\in\sD
\Eqn{CNO.4}
\edmn %---------------------------------------------------------------------
が成り立っているとする.
このとき $\ol{A}$ は正規作用素で $\ol{A}^*=\ol{B}$ である.
\end{theorem} %*************************************************************

\Proof
条件 \Eq{CNO.2}, \Eq{CNO.4} から $u$, $v\in\sD$ のとき
\bdn %----------------------------------------------------------------------
(Au, Av)
= (u,BAv)
= (u,ABv)
= (Bu,Bv)
\Eqn{CNO.6}
\edn %----------------------------------------------------------------------
が成り立つから $|Au|=|Bu|$ が任意の $u\in\sD$ に対して成立する.
従って $u\in\Dom(\ol{A})$ とすると $\{u_n\}\subseteq \sD$ で $u_n\to u$,
$\{Au_n\}$ は Cauchy 列となるものが存在する.
従って $Bu_n$ も Cauchy 列となり,$\Dom(\ol{A})=\Dom(\ol{B})$ が成り立つ.
$\sE=\Dom(\ol{A})$ とおく.
さらに \Eq{CNO.4} で極限をとることにより
\bdm %----------------------------------------------------------------------
(\ol{A} u, v)
= (u,\ol{B}v), \quad u,v\in \sE
\edm %----------------------------------------------------------------------
が成り立つ.
従って $\ol{A}^* \supseteq \ol{B}$ となる.
ところで $\ol{A}$ は $m$-dissipative だから $\ol{A}^*$ も dissipative
になる.
$\ol{B}$ が $m$-dissipative
であるから極大性から $\ol{A}^* = \ol{B}$ となる.
\hide
ここで $\ol{A}$ が $m$- dissipative であることも使っている.
実際 $\ol{A}$ が $m$-dissipative だと 
$\ol{A}^*$ も dissipative になることが田辺の定理 2.1.3 に書いてある.
このことがなければ $\ol{A}^*$ が dissipative かどうか分からなくなるので,
$\ol{A}^* = \ol{B}$ を出せなくなる.
$m$-dissipative の仮定は片方だけでよいと思ったが,やはり両方必要である.
\hfill [2011年1月8日]
\endhide

一方上の  \Eq{CNO.6} で極限をとることにより
\bdm %----------------------------------------------------------------------
(\ol{A} u, \ol{A}v)
= (\ol{B}u,\ol{B}v), \quad u,v\in \sE
\edm %----------------------------------------------------------------------
が成り立つ.
さて $u\in\Dom(\ol{A}^*\, \ol{A})$ をとる.
つまり $u\in \Dom(\ol{A})$ かつ $\ol{A}u\in\Dom(\ol{A}^*)=\sE$
が成り立っているとする.
すると上の式から
\bdm %----------------------------------------------------------------------
(\ol{A}^*\, \ol{A} u, v)
= (\ol{A}u, \ol{A}v)
= (\ol{B}u,\ol{B}v), \quad g\in \sE
\edm %----------------------------------------------------------------------
が成り立つ.
これは $\ol{B}u\in \Dom(\ol{B}^*)$ で
 $\ol{B}^*\,\ol{B}u=\ol{A}^*\, \ol{A}u$ を意味する.
同様に $u\in\Dom(\ol{B}^*\, \ol{B})$ をとると $\ol{B}u\in \Dom(\ol{A}^*)$
で  $\ol{A}^*\,\ol{A}u=\ol{B}^* \,\ol{B}u$ が成り立っていることを意味する.
従って $\ol{A}^*\,\ol{A} = \ol{B}^*\, \ol{B}$ であるが,
$\ol{A}^*=\ol{B}$ であったから $\ol{A}^* \,\ol{A} = \ol{A}\, \ol{A}^*$
が成立する.
これは即ち $\ol{A}$ が正規作用素であることを意味している.
\QED %======================================================================

この結果を使って Riemannian manifold の場合を考えよう.
$M$ を Riemannian manifold として,完備性を常に仮定しておく.
作用素の正規性を言うには可換性を調べる必要があるが,
そのためには Killing vector field
の概念が必要になるので,まずそのことの準備をする.
ベクトル場 $X$ が Killing field であることは $L_X g = 0$ が
成り立つことであった.
このとき $X$ は等距離変換群を定義する.
$L_X$ は Lie 微分を表す.

\Proposition{CNO-2} %*******************************************************
$X$ が Killing vector field であるための必要十分条件は
$v\mapsto \nabla_v X$ が歪対称であること,即ち
\bdn %----------------------------------------------------------------------
g(\nabla_v X, w) + g(\nabla_w X, v)
= 0, \quad \forall v,\, w
\Eqn{CNO.12}
\edn %----------------------------------------------------------------------
が成り立つことである.
\end{proposition} %*********************************************************

\Proof
$\xi=X^\flat$ とするとき,$V$, $W$ を任意のベクトル場として
\bdn %----------------------------------------------------------------------
d\xi(V,W) + (L_X g)(V, W)
= 2g(\nabla_V X, W)
\Eqn{CNO.14}
\edn %----------------------------------------------------------------------
が成り立っている.
これは \cite{Petersen06} の Chapter~2, \S 1 に出ている.
\memo{むしろ Petersen ではこれを定義にして議論を進めている.}
これから $L_Xg=0$ ならば $(V,W) \mapsto g(\nabla_V X, W)$ が歪対称になる.
逆に $(V,W) \mapsto g(\nabla_V X, W)$ が歪対称ならば,$L_X g$ が
歪対称になるが,$L_X g$ は対称でもあるので $L_Xg=0$ となる.
\QED %======================================================================

\Theorem{CNO-6} %***********************************************************
$X$ が Killing vector fileld とする.
$\xi=X^\flat$ とおくと次が成り立つ:
\bdmn %---------------------------------------------------------------------
\nabla^* \nabla \xi
&= \Ric(\xi),
\Eqn{CNO.18} \\
\nabla^* \xi
&= 0.
\Eqn{CNO.20}
\edmn %----------------------------------------------------------------------
\end{theorem} %*************************************************************

\Proof
\Eq{CNO.14} の関係式から $d\xi= 2\nabla\xi$ である.
一方
\bdm %----------------------------------------------------------------------
\nabla^* \xi
= - \tr \nabla\xi
= -\frac{1}{2} \tr d\xi
\edm %----------------------------------------------------------------------
で,$d\xi$ は skew symmetric だから $\nabla^* \xi=0$ が成り立つ.
さらに,一般に微分形式に対して $d^*\w = \nabla^* \w$ が成り立つので
\bdm %----------------------------------------------------------------------
2 \nabla^* \nabla \xi
&= 2 d^* \nabla \xi \\
&= d^* d \xi \quad (\because\ 2\nabla\xi=d\xi) \\
&= (d^* d + d d^*) \xi \quad (\because\ d^*\xi=0) \\
&= (\nabla^*\nabla + \Ric) \theta.
\edm %----------------------------------------------------------------------
よって $\nabla^* \nabla \xi = \Ric(\theta)$ が成り立つ.
\QED %======================================================================

多様体がコンパクトだと上の逆が成立する.
これを示すには更に準備が必要になる.
次の結果は Petersen \cite{Petersen06} のp.~230 に述べてある.
証明がないのでつけておく.

\Proposition{CNO-8} %*******************************************************
$M$ がコンパクトな多様体であるとき,ベクトル場 $X$ に対して
次の等式が成立する.
\bdn %----------------------------------------------------------------------
\int_M (\Ric(X),X) + \tr\{(\nabla X)^2\} - (\div X)^2) d\m
= 0.
\Eqn{CNO.22}
\edn %----------------------------------------------------------------------
ここで $\nabla X$ は $\Hom(T(M))$ の元とみている.
$\tr$ もその意味である.
\end{proposition} %*********************************************************

\Proof
いくつか予備的な等式が必要である.
まず $\div \nabla_X X$ を計算する.
\bdm %----------------------------------------------------------------------
\div(\nabla_X X)
&= \div(X^i \nabla_i X) \\
&= - \la \nabla_j(X^i \nabla_i X), dx^j\ra \\
&= - \la \nabla_j X^i \nabla_i X, dx^j \ra 
   - \la X^i  \nabla_j\nabla_i X, dx^j\ra \\
&= - \nabla_j X^i \la \nabla_i X, dx^j \ra 
   - X^i \la \nabla_j\nabla_i X, dx^j\ra \\
&= - \nabla_j X^i \la \nabla_i X, dx^j \ra 
   - X^i \la (\nabla_j\nabla_i - \nabla_i\nabla_j) X, dx^j\ra
   - X^i \la \nabla_i\nabla_j X, dx^j\ra \\
&= - \nabla_j X^i \la \nabla_i X, dx^j \ra 
   - X^i \la R(\del_j,\del_i)X, dx^j\ra
   - X^i \la \nabla_i\nabla_j X, dx^j\ra \\
&= - \nabla_j X^i \la \nabla_i X, dx^j \ra 
   - \Ric(X,X) - X^i \la \nabla_i\nabla_j X, dx^j\ra.
\edm %----------------------------------------------------------------------
次に $\div((\div X)X)$ を計算する.
\bdm %----------------------------------------------------------------------
\div((\div X)X)
&= - \la \nabla_i((\div X)X), dx^i\ra \\
&= \la \nabla_i((\la \nabla_j X, dx^j\ra X), dx^i\ra \\
&= \la \del_i \la \nabla_j X, dx^j\ra X, dx^i\ra
   + \la \la \nabla_j X, dx^j\ra \nabla_i X, dx^i\ra \\
&= \la (\la \nabla_i \nabla_j X
         + \la \nabla_j X,\nabla_i dx^j \ra) X, dx^i\ra
   + \la \nabla_j X, dx^j\ra \la \nabla_i X, dx^i\ra \\
&= X^i \la \nabla_i \nabla_j X dx^j \ra
         + X^i \la \nabla_j X,\nabla_i dx^j \ra + (\div X)^2.
\edm %----------------------------------------------------------------------
ここで
\bdm %----------------------------------------------------------------------
\nabla_j X^i = \la \nabla_j X,dx^i\ra + \la X, \nabla_j dx^i\ra
\edm %----------------------------------------------------------------------
であることに注意して
\bdm %----------------------------------------------------------------------
\div(\nabla_X X) + \div((\div X)X)
&= -(\la \nabla_j X, dx^i\ra
   + \la X, \nabla_j dx^i\ra) \la\nabla_i X,dx^j\ra
   - \Ric(X,X) \\
&\squad
   + X^i \la \nabla_j X,\nabla_i dx^j \ra + (\div X)^2 \\
&= - \la \nabla_j X, dx^i\ra \la\nabla_i X,dx^j\ra - \Ric(X,X) \\
&\squad
   + (\div X)^2 - \la X, \nabla_j dx^i\ra \la \nabla_i X,dx^j\ra
   + X^i \la \nabla_j X, \nabla_i dx^j\ra.
\edm %----------------------------------------------------------------------
ここで上の最後の2項が $0$ になることを示せば,求める結果に成る.
最後の2項は
\bdm %----------------------------------------------------------------------
- \la X, \nabla_j dx^i\ra \la \nabla_i X,dx^j\ra
  + X^i \la \nabla_j X, \nabla_i dx^j\ra
&= \la X, \Gm_{jk}^i dx^k\ra \la \nabla_i X, dx^j\ra
  - X^i\la\nabla_j X, \Gm_{ik}^j dx^k\ra \\
&= X^k \Gm_{jk}^i \la \nabla_i X, dx^j\ra
  - X^i \Gm_{ik}^j \la\nabla_j X,  dx^k\ra \\
&=0. 
\edm %----------------------------------------------------------------------
これで証明できた.
\QED %======================================================================
\hide
$\div((\div X)X)$ の方は $\nabla^*(\nabla^*\xi,\xi))$ であることに注意して
\bdm %----------------------------------------------------------------------
\nabla^*(\nabla^*\xi,\xi))
&= - \la \nabla\nabla^* \xi,\xi\ra + (\nabla^*\xi)^2 \\
&= - \la dd^* \xi,\xi\ra + (\nabla^*\xi)^2 \\
&= - \la (dd^* + d^*d)\xi,\xi\ra + \la d^*d\xi,\xi\ra
   + (\nabla^*\xi)^2
\edm %----------------------------------------------------------------------
と表現できる.
他方も同じような考えで変形できないか.
そうすれば計算が簡単になるのではないか.
\endhide

\hide
Kobayashi \cite{Kobayashi95}(こちらは \cite{Kobayashi72} のreprint版である) の p.~155 に対応する証明が載っているが,follow 出来ない.
間違っているように思うのだが.
Yono-Bochner \cite{YB53} p.~57 にはテンソル計算できちんと示してある.
\endhide

$X \in \Gm(T(M))$  に対して $\nabla X\in \Gm(\Hom(T(M),T(M))$
と見ているわけであるが,この転置を ${}^t\nabla X\in \Gm(\Hom(T^*(M),T^*(M))$
と表すことにする.
$\nabla X$ と ${}^t\nabla X$ の合成を考えたいが,
このままでは意味を成さないので,同型
$\sharp\colon T^*(M) \to T(M)$,
$\flat\colon T(M) \to T^*(M)$ を用いて $T(M)$ と $T^*(M)$ を同一視して
${}^t\nabla X$ を $\Gm(\Hom(T(M),T(M))$ のように考える.
より正確に言えば,${}^t\nabla X$ の代わりに,次の合成
\bdm %----------------------------------------------------------------------
\begin{CD}
T(M) @>\flat>> T^*(M) @>{{}^t\nabla X}>> T^*(M) @>\sharp>> T(M)
\end{CD}
\edm %----------------------------------------------------------------------
$\sharp\maru {}^t\nabla X \maru \flat$ を考えることである.
煩瑣ではあるが,数学的に厳密な $\sharp\maru {}^t\nabla X \maru \flat$ 
の方を用いることにして,${}^t\nabla X$ は本来の $\Gm(\Hom(T^*(M),T^*(M))$
の元を表すものとする.
basis として $\del_i$, $dx^j$ を取って成分表示すると
\bdm %----------------------------------------------------------------------
\flat_{ij} = g_{ij},\quad
\sharp^{ij} = g^{ij},\quad
(\nabla X)_i^j = \la \nabla_i X, dx^j\ra
\edm %----------------------------------------------------------------------
であるから
\bdm %----------------------------------------------------------------------
(\sharp\maru {}^t\nabla X\maru \flat)_i^l
= g_{ij} (\nabla X)_k^j g^{kl}
\edm %----------------------------------------------------------------------
という成分表示が得られる.

\Proposition{CNO-10} %******************************************************
ベクトル場 $X$, $Y$ に対して次が成立する:
\bdn %----------------------------------------------------------------------
(\nabla X, \nabla Y)
= \tr(\nabla X\maru \sharp \maru {}^t\nabla Y \maru \flat)
\Eqn{CNO.24}
\edn %----------------------------------------------------------------------
\end{proposition} %*********************************************************

\Proof
$(\nabla X)_i^j = \la \nabla_i X,dx^j\ra$ であったから
\bdm %----------------------------------------------------------------------
(\nabla X, \nabla Y)
&= ((\nabla X)_i^j dx^i \otimes \del_j,
    (\nabla Y)_k^l dx^k \otimes \del_l) \\
&= (\nabla X)_i^j (\nabla Y)_k^l g^{ik} g_{jl} \\
&= (\nabla X)_i^j (\sharp \maru \nabla Y \maru \flat)_j^i \\
&= \tr( \nabla X \maru \sharp \maru \nabla Y \maru \flat).
\edm %----------------------------------------------------------------------
これが示すべきことである.
\QED %======================================================================

以上で逆を示す準備が出来た.

\Theorem{CNO-12} %**********************************************************
$M$ をコンパクトな Riemannian manifold で,ベクトル場 $X$ に対して
$\xi=X^\flat$ とおくとき次が成り立っているとする.
\bdmn %---------------------------------------------------------------------
\nabla^* \nabla X
&= \Ric(X), 
\Eqn{CNO.30} \\
\div X
&= 0.
\Eqn{CNO.32}
\edmn %----------------------------------------------------------------------
このとき $X$ は Killing vector field になる.
\end{theorem} %*************************************************************

\Proof
\Prop{CNO-8} から
\bdm %----------------------------------------------------------------------
\int_M (\Ric(X),X) + \tr\{(\nabla X)^2\} - (\div X)^2) d\m
= 0
\edm %----------------------------------------------------------------------
が成り立つ.
ここで \Eq{CNO.30} を使うと
\bdm %----------------------------------------------------------------------
0
&= \int_M (\nabla^*\nabla X, X) + \tr\{(\nabla X)^2\} \,d\m \\
&= \int_M (\nabla X, \nabla^X) + \tr\{(\nabla X)^2\} \,d\m.
\edm %----------------------------------------------------------------------
さらに \Eq{CNO.32} を使えば
\bdm %----------------------------------------------------------------------
0
&= \int_M [\tr( \nabla X\maru \sharp\maru {}^t\nabla X\maru\flat
           + \tr\{(\nabla X)^2\}] \,d\m \\
&= \frac{1}{2} \int_M
   \tr\{ (\nabla X + \sharp\maru {}^t\nabla X\maru\flat)^2\}\,d\m.
\edm %----------------------------------------------------------------------
ここで $\nabla X + \sharp\maru {}^t\nabla X\maru\flat$ が
対称であることに注意しよう.
実際
\bdm %----------------------------------------------------------------------
g(\sharp\maru {}^t\nabla X\maru\flat(v), w)
= \la  {}^t\nabla X\maru\flat(v), w\ra
= \la  \flat(v), \nabla_w X \ra
=g(v,\nabla_w X)
\edm %----------------------------------------------------------------------
から対称であることが分かる.
従って2乗すれば非定値となり,積分して $0$ になることが上で示せているので
$\tr\{ (\nabla X + \sharp\maru {}^t\nabla X\maru\flat)^2\}=0$,
さらには
\bdm %----------------------------------------------------------------------
\nabla X + \sharp\maru {}^t\nabla X\maru\flat = 0
\edm %----------------------------------------------------------------------
が示せる.
これから上の対称性と合わせて
$(v,w) \mapsto g(\nabla_v X,w)$ が skew-symmetric
であることが従う.
よって \Prop{CNO-2} から $X$ は Killing vector field であることが分かる.
\QED %======================================================================

$M$ がコンパクトでなければこの定理は成り立たない.
$\R^2$ の場合に簡単に見ておこう.
$\R^2$ の座標を $(x,y)$ とかく.
$\C=\R^2$ とみて, $\C$ で正則な関数 $f$ をとり,$f=u+iv$ と
実部と虚部に分解しておく.
Cauchy-Riemann の関係式から $u_x=v_y$, $u_y=-v_x$ が成り立つ.
そこで ベクトル場 $X$ を $X= u\frac{\del}{\del x}- v\frac{\del}{\del y}$
と定めると,$\div X=0$ はすぐに分かる.
また
$\nabla^*\nabla X
= - \Laplace u\frac{\del}{\del x} + \Laplace v\frac{\del}{\del y}=0$
もすぐに分かる.
これで \Eq{CNO.30}, \Eq{CNO.32} を満たすが,Killing vector field
でないものが容易に作れる.

\bigskip
さて,正規作用素の話に戻ろう.
この場合に必要となるのは \Prop{CNO-2} と \Thm{CNO-6} だけである.
\memo{その意味では,やや余分の話をしたことになる.}

% \tb を \tilde{b} としていたがここでは特に tilde をつける意味はないから
% はずしておく.別の文脈ではつけたほうが自然であろうから,そこでも使える
% ようにしておく.
\renewcommand{\tb}{b}


\bigskip
$M$ を完備なリーマン多様体とする.
$\m$ を Riemannian volume として,測度 $\nu=e^{-U}\m$ のもとで
 $\gen = -\frac{1}{2} \nabla_\nu^* \nabla + \tb$
という作用素を $L^2(\nu)$ で考える.
$\nabla_\nu^*$ は,測度 $\nu$ に対する $\nabla$ の共役で
Riemannian volume に対する共役 $\nabla^*$ を用いて
$\nabla_\nu^* = \nabla^* + (\nabla U, \cdot)$ と表される.
この作用素が正規である条件を求めたいわけである.

\hide
$\div_\nu\tb=0$ を始めは仮定していた.
それで以下ではこの条件をつけないでやってみたが,条件が綺麗にならない.
$\div_\nu\tb=0$ が出てくると思ったのだが.
この条件は $\gen^*(e^{-U}) = 0$ と同値であった.
これと正規性と関連付けるとかできないのだろうか.
とにかく今のところうまく行っていない.
\hfill [2011年1月6日]
\endhide
$L^2(\nu)$ での(形式的な)共役作用素は
$\gen_\nu^*= -\frac{1}{2} \nabla_\nu^* \nabla - \tb -\div_\nu\tb$ となる.
$\gen_\nu^*$ と $\nu$ を添え字につけたのは
 $\nu$ に関する共役という意味である.
$\div_\nu$ も $\nu$ に対する共役という意味である.
$\nu$ をつけないときは Riemannian volume $\m$ に対する共役を表すものとする.
\bdn %----------------------------------------------------------------------
\Laplace_\nu
= - \nabla_\nu^*\nabla
= - \nabla^*\nabla - \nabla U \cdot \nabla
\Eqn{CNO.66}
\edn %----------------------------------------------------------------------
とおくと,
\bdmn %---------------------------------------------------------------------
\gen
&= \frac{1}{2} \Laplace_\nu + \nabla_\tb,
\Eqn{CNO.68} \\
\gen_\nu^*
&= \frac{1}{2} \Laplace_\nu - \nabla_\tb - \div_\nu\tb.
\Eqn{CNO.70}
\edmn %---------------------------------------------------------------------
であるから,
\bdm %----------------------------------------------------------------------
\gen_\nu^* \gen - \gen \gen_\nu^* 
&= (\frac{1}{2} \Laplace_\nu - \nabla_\tb -\div_\nu\tb)
  (\frac{1}{2} \Laplace_\nu + \nabla_\tb)
  -(\frac{1}{2} \Laplace_\nu + \nabla_\tb)
   (\frac{1}{2} \Laplace_\nu - \nabla_\tb -\div_\nu\tb) \\
&= \Laplace_\nu \nabla_\tb - \nabla_\tb \Laplace_\nu
   + [\frac{1}{2}\Laplace_\nu + \nabla_\tb, \div_\nu\tb] \\
&= [\Laplace_\nu, \nabla_\tb]
   + [\frac{1}{2}\Laplace_\nu+\nabla_\tb, \div_\nu\tb].
\edm %----------------------------------------------------------------------
従って可換である条件は
$[\Laplace_\nu,\nabla_\tb]+[\frac{1}{2}\Laplace_\nu+\nabla_\tb, \div_\nu\tb]=0$
である.
このための条件を求めればよい.
\Theorem{CNO-10} %**********************************************************
$\gen$ と $\gen_\nu^*$ が可換であるための必要十分条件は $\tb$ が
Killing vector field であり,次の等式が成立することである.
\bdmn %---------------------------------------------------------------------
(\frac{1}{2}\Laplace_\nu + \nabla_\tb)\div_\nu \tb
= 0,
\Eqn{CNO.74} \\
[(\nabla U)^\sharp,\tb] + \nabla\div_\nu\tb
= 0.
\Eqn{CNO.76}
\edmn %---------------------------------------------------------------------
\end{theorem} %*************************************************************

\Proof
$\tb$ に対応する 1-form を $\tw$ とする.

まず一般のテンソル $\theta$, $\eta$ に対し
\bdm %----------------------------------------------------------------------
\Laplace(\theta,\eta)
= 2(\nabla \theta, \nabla \eta) - (\nabla^*\nabla\theta, \eta)
-(\theta, \nabla^*\nabla\eta)
\edm %----------------------------------------------------------------------
が成り立つことに注意しよう
(例えば,\cite{Petersen06} の Chapter 7, \S 3 か同じ章の \S 7, Exercise 13
を見よ.)
 特に 1-form の場合,$\theta=\tw$, $\eta=\nabla f$ として
\bdm %----------------------------------------------------------------------
\Laplace(\tw, \nabla f)
= 2(\nabla \tw, \nabla^2 f) - (\nabla^*\nabla\tw, \nabla f)
-(\tw, \nabla^*\nabla\nabla f).
\edm %----------------------------------------------------------------------
ここで 1-form に対して $dd^*+d^*d = \nabla^*\nabla + \Ric$ であることと
\bdm %----------------------------------------------------------------------
(dd^*+d^*d)\nabla f = \nabla (dd^*+d^*d)f = \nabla \nabla^*\nabla
\edm %----------------------------------------------------------------------
を使うと
\bdm %----------------------------------------------------------------------
\Laplace(\tw, \nabla f)
&= 2(\nabla \tw, \nabla^2 f) - (\nabla^*\nabla\tw, \nabla f)
  -(\tw, (dd^*+d^*d)\nabla f) +(\tw, \Ric(\nabla f)) \\
&= 2(\nabla \tw, \nabla^2 f) - (\nabla^*\nabla\tw, \nabla f)
  -(\tw, \nabla \nabla^*\nabla f) +(\tw, \Ric(\nabla f))
\edm %----------------------------------------------------------------------
$(\tw, \nabla f) = \nabla_\tb f$ であるから
\bdm %----------------------------------------------------------------------
[\Laplace, \nabla_\tb]f
= 2(\nabla \tw, \nabla^2 f) - (\nabla^*\nabla\tw - \Ric(\tw), \nabla f).
\edm %----------------------------------------------------------------------
$\Laplace_\nu = \Laplace + \nabla U\cdot\nabla$ だから
\bdm %----------------------------------------------------------------------
[\Laplace_\nu, \nabla_\tb]f
&= [\Laplace + \nabla U\cdot\nabla, \nabla_\tb]f \\
&= [\Laplace, \nabla_\tb]f
  + \nabla_{[\nabla U^{\sharp}, \tb]}f \\
&= 2 (\nabla\tw, \nabla^2 f)
  + (-\nabla^*\nabla\tw + \Ric(\tw) + [\nabla U^\sharp,\tb]^\flat, \nabla f)
\edm %----------------------------------------------------------------------
また $[\frac{1}{2}\Laplace_\nu+\nabla_\tb, \div_\nu\tb]$ の方は
\bdm %----------------------------------------------------------------------
[\frac{1}{2}\Laplace_\nu+\nabla_\tb, \div_\nu\tb]f
&= (\frac{1}{2}\Laplace_\nu+\nabla_\tb)(\div_\nu\tb f)
  - \div_\nu\tb(\frac{1}{2}\Laplace_\nu+\nabla_\tb)f \\
&= \frac{1}{2}\{ (\Laplace_\nu\div_\nu\tb)f+2\nabla\div_\nu\tb\cdot\nabla f
   + \div_\nu \tb\Laplace_\nu f\} \\
&\squad
   + (\nabla_\tb\div_\nu \tb)f + \div_\nu \tb \nabla_\tb f)
   - \div_\nu\tb(\frac{1}{2}\Laplace_\nu+\nabla_\tb)f \\
&= (\frac{1}{2}\Laplace_\nu\div_\nu\tb + \nabla_\tb\div_\nu \tb)f
   + \nabla\div_\nu\tb\cdot\nabla f.
\edm %----------------------------------------------------------------------
両者をあわせて
\bdm %----------------------------------------------------------------------
[\Laplace_\nu, \nabla_\tb]f
 + [\frac{1}{2}\Laplace_\nu+\nabla_\tb, \div_\nu\tb]f
&= 2 (\nabla\tw, \nabla^2 f) \\
&\squad
  + (-\nabla^*\nabla\tw + \Ric(\tw) + [\nabla U^\sharp,\tb]^\flat
      + \nabla\div_\nu\tb, \nabla f) \\
&\squad
  + (\frac{1}{2}\Laplace_\nu\div_\nu\tb + \nabla_\tb\div_\nu \tb)f.
\edm %----------------------------------------------------------------------
これが全ての $f$ に対して 0 であればよいことが,可換性の必要十分条件である.
$f=1$ として 
\bdn %----------------------------------------------------------------------
(\frac{1}{2}\Laplace_\nu + \nabla_\tb)\div_\nu \tb
= 0
\Eqn{CNO.80}
\edn %----------------------------------------------------------------------
が従う.
さらに,点 $x$ をとめれば,$\nabla f(x)=0$ で $\nabla^2 f$ が
任意の対称行列を取るように出来る.
従って可換性の条件は \Eq{CNO.70} と
\bdmn %---------------------------------------------------------------------
&\widehat{\nabla\tw}= 0,
\Eqn{CNO.82} \\
&-\nabla^*\nabla\tw + \Ric(\tw) + [\nabla U^\sharp,\tb]^\flat
 + \nabla\div_\nu\tb= 0
\Eqn{CNO.84}
\edmn %---------------------------------------------------------------------
である.
ここで $\widehat{\nabla\tw}$ は $\nabla\tw$ の対称部分を表す.
まず $\widehat{\nabla\tw}= 0$ は $\nabla\tw$ が skew symmetric
であることを意味している.
従って \Prop{CNO-2} から $\tb$ は Killing field である.
ここで Killing field の性質 \Eq{CNO.18} と条件 \Eq{CNO.84} を使えば
\bdn %----------------------------------------------------------------------
[(\nabla U)^\sharp,\tb] + \nabla\div_\nu\tb
= 0
\Eqn{CNO.86}
\edn %----------------------------------------------------------------------
が従う.

逆に $\tb$ が Killing vector fileld であり,
\Eq{CNO.80}, \Eq{CNO.86} が成り立っているとする. 
すると,$\tb$ がKilling であることから $\nabla\tw$ がskey symmetricとなり
また \Eq{CNO.18} が成り立っている.
よって \Eq{CNO.84} が成り立っていることが分かる.
先の計算により $\gen$ と $\gen_\nu^*$ が可換になる.
\QED %======================================================================

$M$ がコンパクトのときは
$\gen 1=0$, $\gen_\nu^* 1=-\div_\nu\tb$ であるが,可換性が成り立てば 
$\|\gen 1\|_2=\|\gen_\nu^*1\|_2$ が成り立つから $\div_\nu\tb=0$ が成立する.
従って \Eq{CNO.76} の条件は $[(\nabla U)^\sharp, \tb]=0$ と少し簡単になる.
\hide
non-compact の場合も $\div_\nu\tb=0$ が出せるといいのだが.
$\nu$ が有限測度の場合も同じだと思ったが,$1\in L^2(\nu)$
は成り立つが $1\in\Dom(\gen)$ は成り立つのだろうか.
成り立つとしても $\gen 1=0$ となるのだろうか.
定義としては $C_0^\infty(M)$ の元で近似することになるから
$\gen 1=0$ も決して自明ではないことになる.
$1$ に収束する近似列を作らなければならない.
そうすると $\tb$ の増大条件を課さなければならなくなるのではないか.
結局あまり単純な条件ではなくなるような気がする.[2011年1月6日]
\endhide

non-compact の場合は必ずしも $\div_\nu\tb=0$ は成り立たない.
$M=\R$, $U=cx$, $\nu(dx) = e^{-cx}dx$ の場合を考えよう.
このとき
\bdm %----------------------------------------------------------------------
(\frac{d}{dx})_\nu^*
= -\frac{d}{dx} + c
\edm %----------------------------------------------------------------------
である.
$\gen = -\frac{1}{2}(\frac{d}{dx})_\nu^* \frac{d}{dx}+ k\frac{d}{dx}$
とする.
$b= k\frac{d}{dx}$ で $\div_\nu b = \div b - bU = -kc$
また
\bdm %----------------------------------------------------------------------
[\nabla U^{\sharp},b]
= [c\frac{d}{dx}, k\frac{d}{dx}]
= 0
\edm %----------------------------------------------------------------------
だから \Thm{CNO-10} の条件はすべてみたしている.
従ってこの作用素は(閉包をとれば)正規作用素になる.
しかし $\div_\nu b = 0$ は成立していない.
後の例でこの作用素のスペクトルを求めてみる.

\bigskip
上の定理は計算が自由に行えるところでの話である.
例えば滑らかな関数のクラス $C_0^\infty(M)$ では正当化できる.
\Thm{CNO-1} によれば,正規性を示すには $C_0^\infty(M)$ で定義された
$\gen$ の閉包が $m$-dissipative を示さなければならない.
以下ではこの条件を付加して $m$-dissipative であることを示そう.

\Theorem{CNO-14} %**********************************************************
$\tb$ を Killing vector field で $\div_\nu\tb$ が下に有界を仮定する.
このとき $C_0^\infty(M)$ を定義域とする $\gen$, $\gen_\nu^*$ の閉包は
$m$-dissipative である.
\end{theorem} %*************************************************************

\Proof
十分条件が Shigekawa \cite{Shigekawa10} に与えられているからそれを確かめる.
基準の点 $x_0\in M$ を取り固定する.
また $\rho(x)=d(x,x_0)$ を $x_0$ からの距離とする.
ベクトル場 $\tb$ で生成される1-パラメーター変換群を $\vph_t$ で表す.
$\vph_t$ は次の微分方程式をみたす.
\bdm %----------------------------------------------------------------------
\frac{d}{dt}\vph_t(x) = \tb(\vph_t(x)).
\edm %----------------------------------------------------------------------
$|t| \le 1$ で考えると $\tb(\vph_t(x_0))$ は有界であるから
\bdm %----------------------------------------------------------------------
|d(\vph_t(x_0),x_0)|
\le \int_0^t |\tb(\vph_s(x_0))|ds
\le K|t|
\edm %----------------------------------------------------------------------
をみたす $K$ が存在する.
これから
\bdm %----------------------------------------------------------------------
\rho(\vph_t(x))
&= d(\vph_t(x),x_0)
\le d(\vph_t(x),\vph_t(x_0)) + d(\vph_t(x_0),x_0) \\
&\le d(x,x_0) + K|t|
\le \rho(x) + K|t|.
\edm %----------------------------------------------------------------------
同様に $\rho(\vph_t(x)) \ge \rho(x) - K|t|$ も示せるから
\bdm %----------------------------------------------------------------------
|\rho(\vph_t(x)) - \rho(x)|
\le K|t|.
\edm %----------------------------------------------------------------------
ここで 
\bdm %----------------------------------------------------------------------
\tb\rho(x)
= \lim_{t\to 0} \frac{\rho(\vph_t(x)) - \rho(x)}{t}
\edm %----------------------------------------------------------------------
であるから $|\tb\rho(x)|\le K$ が成立する.

さて,閉包が $m$-dissipative であるための十分条件が \cite{Shigekawa10}
で次のように与えられている.
非増加関数 $\kappa:[0,\infty) \to [0,1]$ で
$\int_0^\infty \kappa(x) = \infty$ をみたす関数 $\kappa$ が存在して
$\kappa(\rho(x))\tb\rho(x) \ge -1$ をみたす.

ここでは $\kappa=\frac{1}{K}$ とすればよい.
$\gen_\nu^*$ のときは $-\tb$ を考えればいいので同様に示せる.
\QED %======================================================================

以上纏めれば次が得られた.

\Theorem{CNO-16} %**********************************************************
$\gen$ は $\tb$ が Killing vector field で $\div_\nu\tb$ は下に有界であるとする.
このとき,$\gen$ が正規であるための必要十分条件は $\tb$ が
Killing vector field であり \Eq{CNO.74}, \Eq{CNO.76} が成り立つことである.
\end{theorem} %*************************************************************


最後に簡単な正規作用素の例を挙げておこう.
$\R^2$ で $\nu = \frac{1}{2\pi} e^{-(x^2+y^2)/2} dxdy$
とし,$\tb = c(y\frac{\del}{\del x} - x\frac{\del}{\del y})$
とする.
すると $\tb$ はKilling vector field である.
$U=\frac{x^2+y^2}{2} + \log 2\pi$ とおくと $\nu=e^{-U}dxdy$ である.
このとき $\div\tb=0$ で $\tb U = c(yx - xy)=0$ であるから
$\div_\nu\tb = 0$ である.
さらに $\nabla U^\sharp = x\frac{\del}{\del x} + y \frac{\del}{\del y}$
であるから,
\bdm %----------------------------------------------------------------------
[\tb, \nabla U^\sharp]
&= c[y\frac{\del}{\del x} - x\frac{\del}{\del y},
    x\frac{\del}{\del x} + y \frac{\del}{\del y}] \\
&= c[y\frac{\del}{\del x},x\frac{\del}{\del x}]
  + c[y\frac{\del}{\del x},y \frac{\del}{\del y}]
  - c[x\frac{\del}{\del y},x\frac{\del}{\del x}]
  - c[x\frac{\del}{\del y},y\frac{\del}{\del y}] \\
&= cy\frac{\del}{\del x} - cy\frac{\del}{\del x}
  + cx\frac{\del}{\del y} - cx\frac{\del}{\del y}
= 0.
\edm %----------------------------------------------------------------------
よって,$\gen = -\frac{1}{2} \nabla_\nu^*\nabla + \tb$ は
$L^2(\nu)$ での正規作用素である.
\hide
では,スペクトルはどうなっているのかというのが次の問題になる.
この問題は池野君に解いてもらった.
複素 Hermite 多項式を用いるとスペクトルが完全に決定できる.
それはとても面白かった.また Laguerre 多項式との繋がりも分かってきたし.
[2011年1月5日]
\endhide

\subsec{Notes}
\begin{itemize}
\item 正規作用素の例では $\div_\nu\tb = 0$ の条件を付加して
考えたことになるが,この条件がなければどうなるのだろう.
なんとか $\div_\nu\tb = 0$ の条件を出してきたいのだが.
[2011年1月6日]
\item 本文中に追加してあるが $\div_\nu\tb = 0$ が成り立たない
正規作用素の例が存在する.
但しまだ $\div_\nu\tb = {\rm const.}$ である.
constant 以外の例は思いつかない.
[2011年1月15日]
\end{itemize}

	%%%%%%%%%%%%%%%%%%%%%%%%%%%%%%%%%%%%%%%%%%%%%%%%%
%                                               %
%         ====== Program  No.6 =======          %
%                                               %
%             file name snj06.tex               %
%                                               %
%===============================================%
%===================  for  =====================%
%===============================================%
%
\hide
\vspace{-4mm}
\begin{itemize} \itemsep=-2mm \parsep=0mm
\item Total file name: snj01 snj02 $\dots $ snj?, snj\_bibliography
\item File name: snj06.tex \hfill 印刷日: \today \ \now
\item 単調作用素のことを纏めておく.Lions に従う.
それにしても Lions は凄い.
非線型の場合が念頭にあるが,後でStannat の generalized Dirichlet form の
話をするのに,非線型の話が必要になる.
Stannat の本には証明を書いていないので,ノートとして纏めておく.
Lions の本はフランス語だけれど,英語の文献が何かあると思うのだが分からない.
[2011年2月24日]
\end{itemize}
\endhide
\SS{FPM}{単調作用素の基本的性質} %==========================================
% FPM
後で必要になる単調作用素に関する定理を準備しておく.
$V$ を実 Banach 空間とする.
双対空間を $V^*$ とかく.
\Definition{FPM-2} %********************************************************
写像 $T\colon V\to V^*$ が,すべての $u$, $v\in V$ に対し
\bdn %----------------------------------------------------------------------
(Tu-Tv, u-v) \ge 0
\Eqn{FPM.6}
\edn %----------------------------------------------------------------------
が成り立つとき{\bf 単調}であるという.
さらに $(Tu-Tv, u-v) =0$ となるのは $u=v$ のときに限るとき,
{\bf 狭義単調}であるという.
\end{definition} %**********************************************************

\Remark{FPM-4} %************************************************************
複素 Banach 空間の場合は \Eq{FPM.6} を $\Re(Tu-Tv, u-v) \ge 0$
の条件に替える.
また $T$ も一般には多価関数に対して定義するが,
ここではそこまで一般にはしない.
さらに深入理論は田辺に述べてあるが,ここでは Lions に従って
以下の議論を進める.
\end{remark} %**********************************************************


一般にBanach 空間から Banach 空間への写像が任意の有界集合を有界集合に
移すとき,{\bf 有界}であるという.
さらに後で必要になる連続性について述べておこう.


\Definition{FPM-6} %********************************************************
写像 $T\colon V\to V^*$ が線分上弱連続であることを,
すべての $u$, $v$, $w\in V$ に対し
\bdn %----------------------------------------------------------------------
[0,1]\ni t\mapsto (T((1-t)u+tv),w)
\Eqn{FPM.8}
\edn %----------------------------------------------------------------------
が連続であることと定義する.
\end{definition} %**********************************************************

さて単調写像の連続性に関して次が成り立つ.

\Proposition{FPM-10} %******************************************************
$V$ を反射的な Banach 空間とする.
$T\colon V\to V^*$ が有界単調写像で線分上弱連続であるとき,
任意の $v\in V$ に対して $u\mapsto (Tu,v)$ は連続である.
即ち $T$ は $V$(強位相) から $V^*$(弱位相) への連続な写像である.
\end{proposition} %*********************************************************

\Proof
$u_n$ が $u$ に強収束しているとする.
$\{Tu_n\}$ は有界集合だから弱収束する部分列を持つ.
部分列のとり方によらず極限が $Tu$ であることを示せばよい.
簡単のために $\{Tu_n\}$ が $f$ に弱収束しているとする.

さて,単調性から任意の $w$ に対し
\bdm %----------------------------------------------------------------------
(Tu_n - Tw, u_n-w)
\ge 0
\edm %----------------------------------------------------------------------
が成り立つ.
$w=(1-t)u+tv$ $t\in(0,1)$ ととると.
\bdm %----------------------------------------------------------------------
0
&\le (Tu_n - Tw, u_n-w)
= (Tu_n - Tw, u_n-(1-t)u -tv) \\
&= (Tu_n - Tw,u_n-u) + t(Tu_n - Tw, u-v).
\edm %----------------------------------------------------------------------
ここで $n\to\infty$ として
\bdm %----------------------------------------------------------------------
0
\le  t(f - Tw, u-v).
\edm %----------------------------------------------------------------------
よって $t$ で割って
\bdm %----------------------------------------------------------------------
(f, u-v)
\ge (Tw, u-v).
\edm %----------------------------------------------------------------------
さらに $t\to0$ とすると,右辺は $T$ の線分上弱連続性から
$(Tu, u-v)$ に収束する.
よって
\bdm %----------------------------------------------------------------------
(f, u-v)
\ge (Tu, u-v).
\edm %----------------------------------------------------------------------
これが任意の $v$ について成り立つので $Tu=f$ が従う.
これが示すべきことであった.
\QED %======================================================================

\hide
P.-L.~Lions \cite{Lions69} p.~179, Propposition~2.2.5 ではもう
少し弱い擬単調性からこのことを導いている.
さらに田辺 \cite{Tanabe75} では第6章 \S 5 で単調作用素のことが
より一般の設定で議論されている.
そこまで一般的にする必要はないので,強い条件の下で証明しておいた.
\endhide

最後にもう一つ定義をしておく.

\Definition{FPM-12} %*******************************************************
$T$ を Banach 空間 $V$ からその双対空間 $V^*$ への写像とする.
ある $u_0\in V$ が存在して
\bdn %----------------------------------------------------------------------
\lim_{\|u\|\to\infty} \frac{(Tu,u-u_0)}{\|u\|} = \infty
\Eqn{FPM.10}
\edn %----------------------------------------------------------------------
がなりたつとき $T$ は統御的 (coercive) であるという.
\end{definition} %**********************************************************

さて,ここでは単調作用素の全射性について調べたいのであるが,
そのためにまず有限次元の結果から述べる.

\Proposition{FPM-16} %******************************************************
$P$ を $\R^n$ から $\R^n$ への連続写像で,ある $\rho>0$ が存在して
\bdn %----------------------------------------------------------------------
(P(\xi),\xi) \ge 0, \quad \forall \xi \text{ with $|\xi|=\rho$} 
\Eqn{FPM.12}
\edn %----------------------------------------------------------------------
が成り立っている.
このときある $\xi$, $|\xi| \le \rho$ が存在して $P(\xi)=0$ とできる.
\end{proposition} %*********************************************************

\Proof
背理法で示す.
$K=\{\xi ; \|\xi|\le \rho\}$ で $P(\xi)\not=0$ とする.
さらに
\bdm %----------------------------------------------------------------------
\xi \mapsto -P(\xi)\frac{\rho}{|P(\xi)|}
\edm %----------------------------------------------------------------------
という写像を考えるとこれは $K$ から $K$ への連続写像である.
すると Brouwer の不動点定理から不動点 $\xi$ が存在する.
即ち
\bdm %----------------------------------------------------------------------
\xi =  -P(\xi)\frac{\rho}{|P(\xi)|}.
\edm %----------------------------------------------------------------------
この $\xi$ は $|\xi|=\rho$ をみたし
\bdm %----------------------------------------------------------------------
(P(\xi),\xi)
= \biggl(P(\xi), -P(\xi)\frac{\rho}{|P(\xi)|}\biggr)
= -\rho |P(\xi)|
<0
\edm %----------------------------------------------------------------------
となり,\Eq{FPM.12} に矛盾する.
\QED %======================================================================

\hide
これは Lions \cite{Lions69} p.53, Lemme~1.4.3 に述べてある.
確かに証明は簡単だが.
\endhide

さて,これを使って本題の単調関数に関する全射性を導く定理を示す.

\Theorem{FPM-18} %**********************************************************
$V$ を反射的可分 Banach 空間,
$T\colon V \to V^*$ が有界単調写像で統御的であるとする.
このとき $T$ は全射である.
さらに $T$ が狭義単調であれば $T$ は単射でもある.
\end{theorem} %*************************************************************

\Proof
統御的の条件 \Eq{FPM.10} は $u_0=0$ とすることが出来る.
実際 $Tu$ の代わりに $T(u+u_0)$ を考えればよいからである.
以下 \Eq{FPM.10} が $u_0=0$ として成り立っているとする.

さて示すべきは任意に $f\in V^*$ を与えて $Tu=f$ となる
$u$ の存在を示すことである.
$V$ は可分だから $w_1$, $w_2,\dots$ を1次独立で,一次結合が  $V$ で
稠密なものをとる.
$m$ を固定して $\{w_1,\dots,w_m\}$ で張られる線型空間を
$V_m$ とする.
$V_m$ から $\R^m$ への写像 $P$ を
\bdm %----------------------------------------------------------------------
Pu
= ((Tu-f,w_1),\dots,(Tu-f,w_m))
\edm %----------------------------------------------------------------------
で定める.
$\vph^j$ を $w_i$ の dual basis とする.
即ち,
\bdm %----------------------------------------------------------------------
(\vph^i,w_j)
= \d^i_j
\edm %----------------------------------------------------------------------
を満たすものとする.
$V_m$ は次の対応で $\R^m$ と同一視する:
$u\mapsto ((\vph^1,u),\dots,(\vph^m,u))$.
また $u=\sum_j(\vph^j) w_j$ だから
\bdm %----------------------------------------------------------------------
(Tu-f,u)
= \sum_j (Tu-f,w_j)(\vph^j,u)
= (Pu,u)_{\R^n}
\edm %----------------------------------------------------------------------
となる.
さらに
\bdm %----------------------------------------------------------------------
(Pu,u)
= (Tu-f,u)
\ge (Tu,u) - \|f\|\,\|u\|
\edm %----------------------------------------------------------------------
であり,統御的の条件から $\|u\|$ が十分大きいと
\bdm %----------------------------------------------------------------------
(Tu,u)
\ge \|f\|\,\|u\|
\edm %----------------------------------------------------------------------
とできる.
また $V_m$ は有限次元だからノルム $\|u\|$ は $\sqrt{\sum_j (\vph^j,u)^2}$
と同値になる(定数倍で上下から抑えられる).
よって $\rho$ を十分大きくとると $\sqrt{\sum_j (\vph^j,u)^2}=\rho$
のとき $Pu,u)\ge 0$ が成り立つ.
また \Prop{FPM-10} から $P$ は $V_m$ 上で連続である.
よって \Prop{FPM-16} から $Pu_m=0$ となる$u_m\in V_m$ が存在する.
これは
\bdm %----------------------------------------------------------------------
(Tu_m,w_j)
= (f,w_j), \quad j=1,2,\dots,m
\edm %----------------------------------------------------------------------
を意味している.
これから
\bdm %----------------------------------------------------------------------
(Tu_m,u_m)
= (f,u_m)
\le \|f\| \, \|u_m\|
\edm %----------------------------------------------------------------------
であるから
\bdm %----------------------------------------------------------------------
\frac{(Tu_m,u_m)}{\|u_m\|}
\le \|f\|
\edm %----------------------------------------------------------------------
が成り立つ.

ここから $m$ を動かして考えよう.
統御的の条件から $\|u_m\|$ は有界である.
さらに $T$ の有界性を使えば $Tu_m$ も有界である.
そこで部分列 $\{u_\mu\}$ を
\bg %----------------------------------------------------------------------
u_\mu \to u \text{ weakly in $V$} \\
Tu_\mu \to \chi \text{ weakly in $V^*$}
\eg %----------------------------------------------------------------------
ととる.
$j$ を固定すると $(Tu_\mu,w_j) = (f,w_j)$ が十分大きな $\mu$ に対して
成り立っているから $\mu\to\infty$ として
\bdm %----------------------------------------------------------------------
(\chi,w_j) = (f,w_j).
\edm %----------------------------------------------------------------------
これが任意の $j$ について成り立つから $\chi=f$ である.
一方
\bdm %----------------------------------------------------------------------
(Tu_\mu, u_\mu) = (f,u_\mu) \to (f,u)
\edm %----------------------------------------------------------------------
より
\bdn %----------------------------------------------------------------------
\lim_{\mu\to\infty} (Tu_\mu, u_\mu) = (f,u).
\Eqn{FPM.14}
\edn %----------------------------------------------------------------------

あとは $Tu=\chi$ を示せばよい.
そこで単調性から
\bdm %----------------------------------------------------------------------
(Tu_\mu-Tv,u_\mu-v) \ge 0 \quad \forall v\in V
\edm %----------------------------------------------------------------------
即ち
\bdm %----------------------------------------------------------------------
(Tu_\mu, u_\mu) - (Tu_\mu, v) - (Tv,u_\mu-v) \ge 0.
\edm %----------------------------------------------------------------------
ここで \Eq{FPM.14} に注意して $\mu\to\infty$ とすれば
\bdm %----------------------------------------------------------------------
(f,u) - (\chi, v) - (Tv,u-v) \ge 0.
\edm %----------------------------------------------------------------------
従って
\bdm %----------------------------------------------------------------------
(\chi-Tv,u-v) \ge 0.
\edm %----------------------------------------------------------------------
ここで $v=(1-t)u + tw$ $t\in(0,1)$ とすると
\bdm %----------------------------------------------------------------------
0
\le (\chi-Tv,u-(1-t)u - tv)
= t(\chi-Tv,u+w).
\edm %----------------------------------------------------------------------
$t$ で割って
\bdm %----------------------------------------------------------------------
(\chi-Tv,u+w) \ge 0.
\edm %----------------------------------------------------------------------
ここで $t\to 0$ とすると $T$ の線分上弱連続性から
\bdm %----------------------------------------------------------------------
(\chi-Tu,u+w) \ge 0.
\edm %----------------------------------------------------------------------
$w$ は任意にとれるから $Tu=\chi$ が示せる.

狭義単調性があれば $Tu=Tv$ ならば
\bdm %----------------------------------------------------------------------
(Tu-Tv,u-v) = 0
\edm %----------------------------------------------------------------------
なので $u=v$ が成り立ち $T$ は単射である.
\QED %======================================================================

\subsec{Notes}
\begin{itemize}
\item 単調写像の全射性について纏めた.
P.L.~Lions \cite{Lions69} に従っている.
[2011年2月24日]
\end{itemize}


	%%%%%%%%%%%%%%%%%%%%%%%%%%%%%%%%%%%%%%%%%%%%%%%%%
%                                               %
%         ====== Program  No.7 =======          %
%                                               %
%             file name snj07.tex               %
%                                               %
%===============================================%
%===================  for  =====================%
%===============================================%
%
\StartNewSection
\hide
\vspace{-4mm}
\begin{itemize} \itemsep=-2mm \parsep=0mm
\item Total file name: snj01 snj02 $\dots $ snj?, snj\_bibliography
\item File name: snj07.tex \hfill 印刷日: \today \ \now
\item 正規作用素が Generalized Dirichlet form の枠組みに入ることを証明する.
2008年の日独のとき R\"ockner が別の話で Generalized Dirichlet form
の枠組みに入るのではと言っていたが,正規作用素だったらいいように思う.
何か感じたんでしょうね,彼は.
[2011年1月16日]
\end{itemize}
\endhide
\SS{GDF}{正規作用素と Generalized Dirichlet form} %=========================
% Generalized Dirichlet form
正規作用素で生成される Markov 半群は Generalized Diriclet form
の枠組みに入る.
そのことを示していく.
\hide
昔 Stanatt がやっていたことだが,僕は彼の話を全然理解していなかった.
かれの話は実は結構使えるわけだ.
[2011年1月16日]
\endhide
$(M,\m)$ を $\s$-有限な測度空間とし,$H=L^2(\m)$ として,以下 Hilbert 空間
$H$ で考える.
$\gen$ を正規作用素とする.
$m$-dissipative を仮定する.
spectral represntation により
\bdn %----------------------------------------------------------------------
- \gen
= \int_\C z E(dz)
\Eqn{GDF.6}
\edn %----------------------------------------------------------------------
と表現できる.
対称部分 $(\gen+\gen^*)/2$ は
\bdn %----------------------------------------------------------------------
\frac{\gen+\gen^*}{2}
= \int_\C \Re z E(dz)
\Eqn{GDF.8}
\edn %----------------------------------------------------------------------
歪対称部分 $(\gen-\gen^*)/2$ は
\bdn %----------------------------------------------------------------------
\frac{\gen-\gen^*}{2}
= \int_\C i \Im z E(dz)
\Eqn{GDF.10}
\edn %----------------------------------------------------------------------
で表される.
これはユニタリーグループを生成する.
ここで
\bdn %----------------------------------------------------------------------
L
= \int_\C \Re z E(dz),\quad
\Lm
= \int_\C i \Im z E(dz)
\Eqn{GDF.14}
\edn %----------------------------------------------------------------------
とおく.
$L$ は self-adjoint で $\Lm$ は $i\Lm$ が self-adjointになる.
また定義域は
\bdn %----------------------------------------------------------------------
\Dom(L)
= \{ f;\, \int_\C |\Re z|^2 (f,E(dz)f) <\infty \},\quad
\Dom(\Lm)
= \{ f;\, \int_\C |\Im z|^2 (f,E(dz)f) <\infty \}
\Eqn{GDF.16}
\edn %----------------------------------------------------------------------
となる.
$L$ は $m$-dissipative を仮定したので,半群を生成する.
また $L$ から対称な双線型形式 $\tDiri$ が定まる.
実際 $\tDiri$ は次で定義される:
\bdn %----------------------------------------------------------------------
\tDiri(f,g)
= \int_\C \Re z (f,E(dz)g).
\Eqn{GDF.18}
\edn %----------------------------------------------------------------------
またその定義域も
\bdn %----------------------------------------------------------------------
\Dom(\tDiri)
= \{f;\, \int_\C |\Re z| (f,E(dz)f) <\infty\}
\Eqn{GDF.20}
\edn %----------------------------------------------------------------------
で与えられる.
$\sV =\Dom(\tDiri)$ とおく.

さて,$\Lm$ も半群を生成する.
実際は 1変数 unitary 群を生成するが,半群の部分だけ使う.
この半群を $\{U_t\}_{t\ge0}$ とかく.
このとき次が成り立つ.


\Proposition{GDF-4} %*******************************************************
$\{U_t\}$ は $\sV$ の $C_0$-縮小半群である.
(実際は1変数 unitary 群である.)
\end{proposition} %*********************************************************

さて $\Lm\colon \Dom(\Lm)\cap \sV \to \sV'$ を $\sV$ から $\sV'$
への作用素とみて,そのの閉包を $(\Lm,\sF)$ とする.
この閉包が存在することは Stanatt Lemma 2.3 にある.

\Proposition{GDF-6} %*******************************************************
$f\in \sF$ であるための必要十分条件は
\bdn %----------------------------------------------------------------------
\int_\C (\frac{|\Im(z)|^2}{\Re z+1} + \Re z) \,(f,E(dz)f) < \infty
\Eqn{GDF.22}
\edn %----------------------------------------------------------------------
である.
\end{proposition} %*********************************************************

\Proof
まず $\sV$ のノルムとして
\bdm %----------------------------------------------------------------------
\|f\|_\sV^2
= \int_C (\Re z + 1)(f,E(dz)f)
\edm %----------------------------------------------------------------------
を取ることができる.
\bdm %----------------------------------------------------------------------
|(\Lm f,g)_H|
&=   |\int_\C \Im z (f,E(dz)g)| \\
&=   |\int_\C \frac{\Im z}{\sqrt{\Re z + 1}} \sqrt{\Re z + 1} (f,E(dz)g)| \\
&\le \biggl\{\int_\C \frac{|\Im z|^2}{\Re z + 1} (f,E(dz)f)\biggr\}^{1/2}
     \biggl\{\int_\C (\Re z + 1) (g,E(dz)g) \biggl\}^{1/2} \\
&\le \biggl\{\int_\C \frac{|\Im z|^2}{\Re z + 1} (f,E(dz)f)\biggr\}^{1/2}
     \|g\|_\sV
\edm %----------------------------------------------------------------------
これから
\bdm %----------------------------------------------------------------------
\|\Lm f\|_{\sV'}
\le \biggl\{\int_\C \frac{|\Im z|^2}{\Re z + 1} (f,E(dz)f)\biggr\}^{1/2}
\edm %----------------------------------------------------------------------
が示せる.
逆向きの不等式も $g$ をうまくとれば成立することが分かる.
よって
\bdm %----------------------------------------------------------------------
\|\Lm f\|_{\sV'}
= \biggl\{\int_\C \frac{|\Im z|^2}{\Re z + 1} (f,E(dz)f)\biggr\}^{1/2}
\edm %----------------------------------------------------------------------
また
\bdm %----------------------------------------------------------------------
\|f\|_\sF^2
= \|f\|_\sV^2 + \|\Lm f\|_{\sV'}
\edm %----------------------------------------------------------------------
であるから求める結果を得る.
\QED %======================================================================

\Remark{GDF-8} %************************************************************
弱扇形条件が成り立つときはスペクトルが扇形領域に含まれるから
そこでは $|\Im(z)| \le C(\Re z+1)$ となる定数 $C$ が存在する.
従って,$\sF$ の定義域は $\sV$ と同じになる.
\end{remark} %**************************************************************

同様の議論を $U_t$ の双対半群 $\hat{U}_t$ についても行って
$\hat{\sF}$ を定める.
$\hat{U}_t$ の生成作用素は
\bdn %----------------------------------------------------------------------
\hat{\Lm}
= - \int_\C i \Im z E(dz)
\Eqn{GDF.26}
\edn %----------------------------------------------------------------------

さて generalized Dirichlet form の枠組みでは双線型形式を
\bdn %----------------------------------------------------------------------
\Diri(f,g)
=
\begin{cases}
\tDiri(f,g)-\la \Lm f,g\ra, \quad \text{if $f\in\sF$, $g\in \sV$} \\
\tDiri(f,g)-\la \hat{\Lm} g, f\ra, \quad \text{if $f\in\sV$, $g\in \hat{\sF}$}
\end{cases}
\Eqn{GDF.24}
\edn %----------------------------------------------------------------------
で定める.

容量の概念も定義され,次の定理を適用することにより確率過程の存在が保証される.

\Theorem{GDF-10} %**********************************************************
{\bf (Stanatt 1994) }
条件 (D3) のもとで,擬正則一般 Dirichlet 形式に対して,
$\m$-thght special standard 過程が存在する.
\end{theorem} %*************************************************************

\begin{description}
\item[(D3)]
ある線型部分空間 $\sY \subseteq L^2(\m)\cap L^\infty(\m)$ で
$\sY\cap \sF$ が $\sF$ 稠密であり,
$\lim_{\al\to\infty} e_{\al G_\al u-u}=0$ が $u\in\sY$ に対して成り立つ.
さらに
$\sY$ の $L^\infty$ での閉包を $\ol{\sY}$ とすると
$u\wedge \al\in \ol{\sY}$ が $u\in\sY$, $\al>0$ に対して成立する.
\end{description}







\subsec{Notes}
\begin{itemize}
\item 
\end{itemize}

 % \include{snj08}
	%\include{snj09} \include{snj10} \include{snj11} 
	%\include{snj12} \include{snj13} %\include{snj14}
	%\include{snj15}
	%%%%%%%%%%%%%%%%%%%%%%%%%%%%%%%%%%%%%%%%%%%%%%%%%
%                                               %
%         ====== Program  No.16 =======          %
%                                               %
%             file name snj16.tex               %
%                                               %
%===============================================%
%===================  for  =====================%
%===============================================%
%
\hide
\vspace{-4mm}
\begin{itemize} \itemsep=-2mm \parsep=0mm
\item Total file name: snj01 snj02 $\dots $ snj?, snj\_bibliography
\item File name: snj16.tex \hfill 印刷日: \today \ \now
\item Ornstein-Uhlenbeck に回転を加えた作用素のスペクトルを調べる.
[2011年8月26日(金)]
\end{itemize}
\endhide
\SS{OUR}{$\R^2$ 上の回転を加えた Ornstein-Uhlenbeck 作用素} %==================
% Ornstein-Uhlenbeck operator with rotation
この節では Ornstein-Uhlenbeck に回転を加えた作用素のスペクトルを調べる.
$\al\in\R$ に対し $L_{\al}$ を
\bdn %----------------------------------------------------------------------
L_\al
= \frac{\del^2}{\del x^2} + \frac{\del }{\del y^2} 
  - x \frac{\del}{\del x} - y \frac{\del }{\del y}
  \alpha \biggl( x \frac{\del }{\del y} - y \frac{\del }{\del x} \biggr)
\Eqn{OUR.6}
\edn %----------------------------------------------------------------------
で定め,$L^2(\R^2, \mu)$ 上の作用素としてスペクトルを調べる.
ここで測度 $\mu$ は
\bdn %----------------------------------------------------------------------
\mu
= \frac{1}{2\pi} e^{-(x^2+y^2)/2}\,dx\,dy
\Eqn{OUR.8}
\edn %----------------------------------------------------------------------
で定まるGauss 測度である.

通常の Ornstein-Uhlenbeck 作用素 $L_0$ のスペクトルは $\{0,-1,-2,\dots\}$
であることはよく知られている.
固有関数は,まず Hermite 多項式を
\bdn %----------------------------------------------------------------------
H_n(x)
= (-1)^n e^{x^2/2}\frac{d^n}{dx^n}e^{-x^2/2}.
\Eqn{OUR.10}
\edn %----------------------------------------------------------------------
で定めると,
\bdm %----------------------------------------------------------------------
L_0 H_k(x)H_{n-k}(y)
= - n H_k(x)H_{n-k}(y)
\edm %----------------------------------------------------------------------
が成り立つ.
回転を加えた場合は,固有値は複素 Hermite 多項式になる.

\subsec{複素 Hermite 多項式}
複素 Hermite 多項式について復習しておく.
$\R^2$ は $\C$ と同一視,$z=x+iy$ とする.
複素 Hermite 多項式は $p$, $q\in\plus{\Z}$ に対し
\bdn %----------------------------------------------------------------------
H_{p,q}(z , \bar{z})
= (-1)^{p+q} e^{z\bar{z}/2} \biggl( \frac{\del}{\del\bar{z}} \biggr)^p
 \biggl(\frac{\del }{\del z} \biggr)^q e^{-z\bar{z}/2}
\Eqn{OUR.14}
\edn %----------------------------------------------------------------------
で定義される.
通常の定義とは定数が異なっていることに注意しよう.
これは Gauss 測度 $\mu$ の取り方合わせたためである.
ここで
\bdm %----------------------------------------------------------------------
\frac{\del}{\del z}
= \frac{1}{2}\biggl( \frac{\del}{\del x} - i \frac{\del}{\del y}\biggr), \quad
\frac{\del}{\del \bar{z}}
= \frac{1}{2}\biggl( \frac{\del}{\del x} + i \frac{\del}{\del y}\biggr)
\edm %----------------------------------------------------------------------
である.
以下では
\bdm %----------------------------------------------------------------------
\del
= \frac{\del}{\del z}, \quad
\delb
= \frac{\del}{\del \bar{z}}, \quad
\edm %----------------------------------------------------------------------
と略記する.
また
\bdn %----------------------------------------------------------------------
\del^*
= -\del + \frac{\bar{z}}{2}, \quad
\delb^*
= -\delb  + \frac{\bar{z}}{2}
\Eqn{OUR.18}
\edn %----------------------------------------------------------------------
が成り立っている.

\Proposition{OUR-2} %*******************************************************
上の \Eq{OUR.18} の作用素には次の交換関係が成立している.
\bdmn %---------------------------------------------------------------------
\del H_{p,q}
&= \frac{p}{2} H_{p-1,q}, 
\Eqn{OUR.20} \\
\delb H_{p,q}
&= \frac{q}{2} H_{p,q-1}, 
\Eqn{OUR.22} \\
\del^* H_{p,q}
&= H_{p+1,q}, 
\Eqn{OUR.24} \\
\delb^* H_{p,q}
&= H_{p,q+1}, 
\Eqn{OUR.26} \\
2\del\delb H_{p,q} - z\del H_{p,q}
&= - p H_{p,q}
\Eqn{OUR.28} \\
2\del\delb H_{p,q} - \bar{z}\delb H_{p,q}
&= - q H_{p,q}
\Eqn{OUR.30} \\
(z\del - \bar{z}\delb)H_{p,q}
&= (p-q) H_{p,q}
\Eqn{OUR.32}
\edmn %----------------------------------------------------------------------
\end{proposition} %*********************************************************

\Proof
\QED %======================================================================

さて $L_\al$ のスペクトルを求めるために
$L_\al$ を $\del$, $\delb$ を用いて表そう.
まず
\bdm %----------------------------------------------------------------------
\del \delb
&= \frac{1}{4} \biggl(\frac{\del^2}{\del x^2}+\frac{\del}{\del y^2}\biggr), \\
z\del
&=  x\frac{\del}{\del x} + y\frac{\del}{\del y}
    -i\biggl(  x \frac{\del}{\del y} - y \frac{\del }{\del x}\biggr), \\
\bar{z}\delb
&=  x \frac{\del}{\del x} + y \frac{\del }{\del y}
    +i\biggl(  x \frac{\del}{\del y} - y \frac{\del }{\del x}\biggr)
\edm %----------------------------------------------------------------------
が成り立つので
\bdm %----------------------------------------------------------------------
L_\al
&= 4 \del\delb - z\del - \bar{z}\delb +\al i (z\del - \bar{z}\delb) \\
&= (2\del\delb - z\del) + (2\del\delb- \bar{z}\delb)
   + \al i (z\del - \bar{z}\delb).
\edm %----------------------------------------------------------------------
従って
\bdm %----------------------------------------------------------------------
L_\al H_{p,q}
&= (2\del\delb - z\del)  H_{p,q} + (2\del\delb- \bar{z}\delb)H_{p,q}
   + \al i (z\del - \bar{z}\delb)H_{p,q} \\
&= - p H_{p,q} + q H_{p,q} + (p-q) \al i H_{p,q} \\
&= (-p-q + (p-q) \al i) H_{p,q}.
\edm %----------------------------------------------------------------------

よって次の定理が得られた.
\Theorem{OUR-6} %**********************************************************
$-L\al$ の固有値は $\{ - (p+q) + (p-q) \alpha i \}_{p,q = 0}^{\infty}$
であり対応する固有関数は $H_{p,q}$ である.
\end{theorem} %*************************************************************


\begin{center}
\includePdfEps{}{snj_OU_spectrum}
\includePdfEps{}{snj_OU_rotate_spectrum}
\end{center}


\bdm %----------------------------------------------------------------------
V_n := \{ L_0 f = n f\}
\edm %----------------------------------------------------------------------
とすれば直交分解
\bdm %----------------------------------------------------------------------
L^2(\C,\mu)
= \bigoplus_{n=0}^\infty V_n
\edm %----------------------------------------------------------------------
が成り立つ.
$V_n$ をさらに
\bdm %----------------------------------------------------------------------
V_n = \bigoplus_{p + q = n} {\bf{C}}H_{p,q}
\edm %----------------------------------------------------------------------
と分解したことになる.
これは回転群 $U(1)$ の規約分解を与えている.
固有値 $2n$ ($n\in\plus{\Z}$)に対応する固有関数は回転方向の微分
$(z\del - \bar{z}\delb)$ が $0$ になるので,半径方向の関数であり
\bdm %----------------------------------------------------------------------
\biggl(\frac{d^2}{d r^2} + \frac{1}{r}\frac{d}{d r}
       - r \frac{d}{d r} \biggr) H_{n,n}
= -2n H_{n,n}
\edm %----------------------------------------------------------------------
が成り立っている.
ここで $r=\sqrt{2u}$ と変数変換すると
\bdm %----------------------------------------------------------------------
\frac{d^2}{d r^2} + \frac{1}{r}\frac{d}{d r}
       - r \frac{d}{d r}
= 2u \frac{d^2}{du^2} + 2(1-u) \frac{d}{du}
\edm %----------------------------------------------------------------------
となる.
$F(u)= H_{n,n}(r)$ は微分方程式
\bdm %----------------------------------------------------------------------
2u \frac{d^2}{du^2}F + 2(1-u) \frac{d}{du}F +nF=0
\edm %----------------------------------------------------------------------
をみたしている.
これは Laguerre 多項式が満たす微分方程式である.
ここで Laguerre 多項式は
\bdn %----------------------------------------------------------------------
L_{n} = \frac{e^x}{n!} \frac{d^n}{dx^n} (e^{-x}x^n)
\Eqn{OUR.40}
\edn %----------------------------------------------------------------------
で定義される.
実際次が成立する.

\Theorem{OUR-10} %**********************************************************
複素 Hermite polynomials $H_{n,n}$ は次をみたす.
\bdn %----------------------------------------------------------------------
H_{n,n}(z,\bar{z}) = c L_n \left(\frac{|z|^2}{2} \right),
\Eqn{OUR.42}
\edn %----------------------------------------------------------------------
$c$ は定数である.
\end{theorem} %*************************************************************

\hide
上の定数 $c$ がはっきりしないのは問題である.

\endhide

さらに複素 Hermite 多項式は,実 Hermite 多項式を用いて
\bdn %----------------------------------------------------------------------
H_{n,n}(z,\bar{z})
\frac{1}{4^n} \sum_{k=0}^n H_{2k}(x)H_{2n-k}(y)
\Eqn{OUR.44}
\edn %----------------------------------------------------------------------
と表されるから次を得る.

\Corollary{OUR-12} %********************************************************
Laguerre 多項式は Hermite 多項式と次の関係で結ばれている.
\bdn %----------------------------------------------------------------------
L_n(\frac{x^2+y^2}{2})
= \frac{c}{4^n} \sum_{k=0}^n H_{2k}(x)H_{2n-k}(y).
\Eqn{OUR.46}
\edn %----------------------------------------------------------------------
\end{corollary} %***********************************************************

\subsec{Notes}
\begin{itemize}
\item 2011年9月12日 に Jan Van Neerven からメールが来た.OU のスペクトルの研究をしている.Bonn のISSAA で会って,話を聞いた.この節で論じたスペクトルの話に関連したことをやっている.
\item 2012年7月に中国に行ったとき,同じ結果を得ていることを聞いた.
さっさと書かないからこういうことになる・
\end{itemize}

 %\include{snj17}
	%%%%%%%%%%%%%%%%%%%%%%%%%%%%%%%%%%%%%%%%%%%%%%%%%
%                                               %
%         ====== Program  No.18 =======          %
%                                               %
%             file name snj18.tex               %
%                                               %
%===============================================%
%===================  for  =====================%
%===============================================%
%
\hide
\vspace{-4mm}
\begin{itemize} \itemsep=-2mm \parsep=0mm
\item Total file name: snj01 snj02 $\dots $ snj?, snj\_bibliography
\item File name: snj18.tex \hfill 印刷日: \today \ \now
\item 正規作用素の例を挙げる.
さらにそのスペクトルを調べる.
[2011年8月16日(火)]
\end{itemize}
\endhide
\SS{BWD}{ドリフトのある1次元ブラウン運動} %===================================
% On-dimensional brownian motion with a drift
ここからいくつか正規作用素の例を与える.
またスペクトルを完全な形で求めることを目標とする.

$\R$ 上で作用素 $\gen = \frac{d^2}{dx^2} - c\frac{d}{dx}$ を考える.
但しここでは基礎の測度を
\bdn %----------------------------------------------------------------------
\nu_1(dx)
= e^{-cx}dx
\Eqn{BWD.4}
\edn %----------------------------------------------------------------------
にとる.
このとき
\bdm %----------------------------------------------------------------------
\int_\R (\frac{d^2}{dx^2}f - c\frac{d}{dx}f)g(x) e^{-cx}\,dx
&= \int_\R \frac{d}{dx}(\frac{d}{dx}fe^{-cx})\, g(x)\,dx
&= - \int_\R (\frac{d}{dx}f e^{-cx}) \frac{d}{dx}g(x)\,dx
\edm %----------------------------------------------------------------------
であるから,$\gen$ は自己共役作用素となる.
実際この作用素は次の対称双線型形式 $\Diri$ に対応する自己共役作用素である:
\bdn %----------------------------------------------------------------------
\Diri(f,g)
= \int_\R \frac{df}{dx} \frac{dg}{dx} e^{-cx}\,dx
\Eqn{BWD.6}
\edn %----------------------------------------------------------------------
また $\nu_1$ に関する $\frac{d}{dx}$ の共役は
\bdn %----------------------------------------------------------------------
\Bigl(\frac{d}{dx}\Bigr)_{\nu_1}^*
= -\frac{d}{dx} + c
\Eqn{BWD.8}
\edn %----------------------------------------------------------------------
であり,生成作用素 $\gen$ は
\bdm %----------------------------------------------------------------------
\gen
= \Bigl(\frac{d}{dx}\Bigr)_{\nu_1}^* \frac{d}{dx}
\edm %----------------------------------------------------------------------
の形でも表される.
$\gen$ のスペクトルを調べるには変換したほうがよい.
次の写像 $I\colon L^2(\nu_1)\longrightarrow L^2(dx)$ は等距離作用素である:
\bdn %----------------------------------------------------------------------
If(x)=e^{-cx/2}f(x).
\Eqn{BWD.12}
\edn %----------------------------------------------------------------------
ここで
\bdm %----------------------------------------------------------------------
I\circ \gen \circ I^{-1} f
&= e^{-cx/2}(\frac{d^2}{dx^2} - c\frac{d}{dx})(e^{cx/2}f) \\
&= e^{-cx/2}(\frac{c^2}{4} e^{cx/2} f + c e^{cx/2}\frac{df}{dx}
    + e^{cx/2} \frac{d^2f}{dx^2} - c\frac{1}{2}ce^{cx/2} f
    - ce^{cx/2}\frac{df}{dx}) \\
&= - \frac{c^2}{4} f + \frac{d^2f}{dx^2}
\edm %----------------------------------------------------------------------
であるから次の図式が可換となる:
\bdn %----------------------------------------------------------------------
\begin{CD}
L^2(\nu_1)	@>{\gen}>>	L^2(\nu_1)		\\
@V{I}VV             @VV{I}V	\\
L^2(dx)		@>{\frac{d^2}{dx^2} - \frac{c^2}{4}}>>	L^2(dx)
\end{CD}
\Eqn{BWD.14}
\edn %----------------------------------------------------------------------
従ってこの場合のスペクトル集合を $\s_1$ とすると
\bdn %----------------------------------------------------------------------
\s_1
= [\frac{c^2}{4},\infty)
\Eqn{BWD.16}
\edn %----------------------------------------------------------------------
となる.

さて,作用素 $\gen$ に摂動を加えて,スペクトルの変化を見ていく.
ベクトル場 $b$ を
\bdn %----------------------------------------------------------------------
b
= k\frac{d}{dx}
\Eqn{BWD.18}
\edn %----------------------------------------------------------------------
で定める.
すると $\nu_1$ に関する発散は
\bdm %----------------------------------------------------------------------
\div_{\nu_1} b
= -ck
\edm %----------------------------------------------------------------------
となる.
従って\Thm{CNO-16} から $\gen+b$ は正規作用素になる.
このスペクトルを求めよう.
この場合もやはり $I$ で変換して $L^2(dx)$ の話に持っていく.
\bdm %----------------------------------------------------------------------
I\circ b \circ I^{-1}f
&= e^{-cx/2}(k\frac{d}{dx})(e^{cx/2}f) \\
&= e^{-cx/2} (k\frac{1}{2}ce^{cx/2} f + ke^{cx/2}\frac{df}{dx}) \\
&= \frac{kc}{2} f + k \frac{df}{dx}
\edm %----------------------------------------------------------------------
であるから
\bdm %----------------------------------------------------------------------
I\circ (\gen+ b) \circ I^{-1}f
&= (\frac{d^2}{dx^2} - \frac{c^2}{4} + \frac{kc}{2} + k \frac{d}{dx})f \\
&= (\frac{d^2}{dx^2} + k \frac{d}{dx} - \frac{c(c-2k)}{4} )f.
\edm %----------------------------------------------------------------------
よって $\frac{d^2}{dx^2} + k \frac{d}{dx}$ の
スペクトルを調べればよいことになる. 
そのために Fourier 変換を利用する.
Fourier 変換は次で定義される.
\bdm %----------------------------------------------------------------------
\hat{f}(\xi)
= \frac{1}{\sqrt{2\pi}}\int_\R f(x) e^{-\xi x}\,dx.
\edm %----------------------------------------------------------------------
これは $L^2(dx)$ から $L^2(d\xi)$ への等長変換を与える.
さらに
\bdm %----------------------------------------------------------------------
\int_\R (\frac{d^2}{dx^2} + k\frac{d}{dx})f(x)\ol{g(x)}\,dx
= \int_\R (-\xi^2 + ik\xi) \hat{f}(\xi) \ol{\hat{g}(\xi)}\,d\xi
\edm %----------------------------------------------------------------------
が成り立つ.
即ち,Fourier 変換した空間では $\frac{d^2}{dx^2} + k\frac{d}{dx}$ に対応する
写像は $-\xi^2 + ik\xi$ をかけるという掛け算作用素である.
よって $\frac{d^2}{dx^2} + k\frac{d}{dx}$ のスペクトルは
\bdm %----------------------------------------------------------------------
\{ - \xi^2 + ik\xi; \xi\in\R\}
\edm %----------------------------------------------------------------------
という放物線である.

元に戻って,$L^2(\nu_1)$ での
$-\gen -b = -\frac{d^2}{dx^2} + c \frac{d}{dx} - k\frac{d}{dx} - $
のスペクトルは
\bdm %----------------------------------------------------------------------
 \{ \frac{c(c-k)}{2} +  \xi^2 + ik\xi;\, \xi\in\R\}
\edm %----------------------------------------------------------------------
である.


\begin{center}
\includePdfEps{}{snj_BM_drift1}
\includePdfEps{}{snj_BM_drift2}
\end{center}




\begin{comment}
\begin{center}
\begin{pspicture}(-1,-4)(5,3) %\showgrid
%\psset{arcangle=8, linewidth=.8pt}  % default
%\psset{arrowsize=1.5pt 2} % default
\psset{arrowsize=3pt 2, arrowlength=1.5}
\psset{xunit=8mm, yunit=12mm}
\psaxes[labels=none,showorigin=true,linewidth=.2pt,ticks=none]{->}(0,0)(-1,-2)(4.5,2)
\psset{linewidth=1pt,arrows=c-c,linecolor=red}
\psline(.5,0)(4.2,0)

%\psecurve(9,3)(4,2)(1,1)(0,0)(1,-1)(4,-2)(9,-3)
%\psecurve(9,3)(4,2)(0,0)(4,-2)(9,-3)

\uput[90](.5,0){$\ds \frac{c^2}{4}$}
\uput{5pt}[45](1,-3){$-\gen$}
%\showgrid
\end{pspicture} 
%
\begin{pspicture}(-1,-4)(5,3) %\showgrid
%\psset{arcangle=8, linewidth=.8pt}  % default
%\psset{arrowsize=1.5pt} % default
\psset{arrowsize=3pt 2, arrowlength=1.5}
\psset{xunit=8mm, yunit=12mm}
\psaxes[labels=y,showorigin=true,linewidth=.2pt,ticks=none]{->}(0,0)(-1,-2)(4.5,2)
\psset{linewidth=1pt,arrows=c-c,linecolor=red}
%\psecurve(6.25,2.5)(4,2)(2.25,1.5)(1,1)(.25,.5)(0,0)(1,-1)(2.25,-1.5)(4,-2)(6.25,-2.5)
\pscurve(5,2)(1,0)(5,-2)
\uput{5pt}[90](1,0){$\frac{c(c-2k)}{4}$}
\uput{5pt}[45](1,-3){$-\gen - k\frac{d}{dx}$}
\end{pspicture} 
\end{center}
\end{comment}


これは $k$ とともにスペクトルが連続的に変化している様を表している.
これを違った観点から見てみよう.
作用素を $\gen = \frac{d^2}{dx^2} - c\frac{d}{dx}$ とする.
測度を $\nu_0=dx$, $\nu_1=e^{-cx}dx$ として $L^2(\nu_0)$ でのスペクトルと
$L^2(\nu_1)$ でのスペクトルを比べてみよう.
同じ作用素を異なった空間で考えていることになる.
上の計算は $-\gen$ のスペクトルは $L^2(\nu_0)$ では $\{\xi^2 - ic\xi\}$
であり $L^2(\nu_1)$ では $[\frac{c^2}{4},\infty)$ となっていることが分かる.


\begin{center}
\includePdfEps{}{snj_BM_exp1}
\includePdfEps{}{snj_BM_exp2}
\end{center}


\begin{comment}
\begin{center}
\begin{pspicture}(-1,-4)(5,3) %\showgrid
%\psset{arcangle=8, linewidth=.8pt}  % default
%\psset{arrowsize=1.5pt} % default
\psset{arrowsize=3pt 2, arrowlength=1.5}
\psset{xunit=8mm, yunit=12mm}
\psaxes[labels=y,showorigin=true,linewidth=.2pt,ticks=none]{->}(0,0)(-1,-2)(4.5,2)
\psset{linewidth=1pt,arrows=c-c,linecolor=red}
%\psecurve(6.25,2.5)(4,2)(2.25,1.5)(1,1)(.25,.5)(0,0)(1,-1)(2.25,-1.5)(4,-2)(6.25,-2.5)
\pscurve(4,2)(0,0)(4,-2)
\uput{5pt}[45](1,-3){w.r.t. $\nu_0=dx$}

%\showgrid
\end{pspicture} 
%
\begin{pspicture}(-1,-4)(5,3) %\showgrid
%\psset{arcangle=8, linewidth=.8pt}  % default
%\psset{arrowsize=1.5pt 2} % default
\psset{arrowsize=3pt 2, arrowlength=1.5}
\psset{xunit=8mm, yunit=12mm}
\psaxes[labels=none,showorigin=true,linewidth=.2pt,ticks=none]{->}(0,0)(-1,-2)(4.5,2)
\psset{linewidth=1pt,arrows=c-c,linecolor=red}
\psline(.5,0)(4.2,0)

%\psecurve(9,3)(4,2)(1,1)(0,0)(1,-1)(4,-2)(9,-3)
%\psecurve(9,3)(4,2)(0,0)(4,-2)(9,-3)

\uput[90](.5,0){$\ds \frac{c^2}{4}$}
\uput{5pt}[45](1,-3){w.r.t. $\nu_1=e^{-cx}dx$}
%\showgrid
\end{pspicture} 
\end{center}
\end{comment}


今度は測度を連続的に動かしてスペクトルの変化を見よう.
$\theta\in[0,1]$ に対し測度 $\nu_\theta$ を
\bdn %----------------------------------------------------------------------
\nu_\theta(dx) = (1-\theta)dx  + \theta e^{-cx} dx
\Eqn{BWD.22}
\edn %----------------------------------------------------------------------
で定める.
$L^2(\nu_\theta)$ でのスペクトルはどのように変化するであろうか.
次のように連続的に変化することも考えられる.

\begin{center}
\includePdfEps{}{snj_BM_measure1}
\end{center}

\begin{comment}
\begin{center}
\begin{pspicture}(-1,-4)(5,3) %\showgrid
%\psset{arcangle=8, linewidth=.8pt}  % default
%\psset{arrowsize=1.5pt} % default
\psset{arrowsize=3pt 2, arrowlength=1.5}
\psset{xunit=8mm, yunit=12mm}
\psaxes[labels=y,showorigin=true,linewidth=.2pt,ticks=none]{->}(0,0)(-1,-2)(4.5,2)
%\psset{linewidth=1pt,arrows=c-c}
\psset{linewidth=1pt,arrows=c-c,linecolor=red}
%\psecurve(6.25,2.5)(4,2)(2.25,1.5)(1,1)(.25,.5)(0,0)(1,-1)(2.25,-1.5)(4,-2)(6.25,-2.5)
\psline(.5,0)(4.2,0)
\pscurve(4,2)(0,0)(4,-2)
\pscurve[linecolor=blue](4,.9)(.25,0)(4,-.9)
\psline[linecolor=black]{->}(2.3,1.2)(3.1,.3)
\uput{5pt}[45](1,-3){w.r.t. $\nu_\theta$}
\uput[45](2.5,1){?}

%\showgrid
\end{pspicture} 
\end{center}
\end{comment}

しかし,これは正しくない.
実際は二つの和集合になる.
\Theorem{BWD-4} %**********************************************************
$\theta\in (0,1)$ のとき $-\gen$ の $L^2(\nu_\theta)$ におけるスペクトルは
\bdm %----------------------------------------------------------------------
\{\xi^2 - ik\xi; \xi\in\R\} \cup [\frac{c^2}{4},\infty)
\edm %----------------------------------------------------------------------
である.
\end{theorem} %*************************************************************



\begin{center}
\includePdfEps{}{snj_BM_measure2}
\end{center}

\begin{comment}
\begin{center}
\begin{pspicture}(-1,-4)(5,3) %\showgrid
%\psset{arcangle=8, linewidth=.8pt}  % default
%\psset{arrowsize=1.5pt} % default
\psset{arrowsize=3pt 2, arrowlength=1.5}
\psset{xunit=8mm, yunit=12mm}
\psaxes[labels=y,showorigin=true,linewidth=.2pt,ticks=none]{->}(0,0)(-1,-2)(4.5,2)
\psset{linewidth=1pt,arrows=c-c,linecolor=red}
%\psecurve(6.25,2.5)(4,2)(2.25,1.5)(1,1)(.25,.5)(0,0)(1,-1)(2.25,-1.5)(4,-2)(6.25,-2.5)
\psline(.5,0)(4.2,0)
\pscurve(4,2)(0,0)(4,-2)
\uput{5pt}[45](1,-3){w.r.t. $\nu_\theta$}
%\showgrid
\end{pspicture} 
\end{center}
\end{comment}















\subsec{Notes}
\begin{itemize}
\item 
\end{itemize}

 %%%%%%%%%%%%%%%%%%%%%%%%%%%%%%%%%%%%%%%%%%%%%%%%%
%                                               %
%         ====== Program  No.19 =======          %
%                                               %
%             file name snj19.tex               %
%                                               %
%===============================================%
%===================  for  =====================%
%===============================================%
%
\hide
\vspace{-4mm}
\begin{itemize} \itemsep=-2mm \parsep=0mm
\item Total file name: snj01 snj02 $\dots $ snj?, snj\_bibliography
\item File name: snj19.tex \hfill 印刷日: \today \ \now
\item 球面上で正規作用素の例を挙げる.
\end{itemize}
\endhide
\SS{LBS}{Laplace-Beltrami 作用素に回転を加えた作用素} %=====================
% Laplace-Beltrami operator with rotation on a sphier
幾何学的な構造が反映するような作用素の例を与えよう.
球面は最も単純な例である.

\subsec{$S^2$ 上の正規作用素} 
球面 $S^2$ 上の Laplace-Berltrami は次のように表される.
\bdm %----------------------------------------------------------------------
\Laplace
= \frac{1}{\sin \theta} \frac{\del}{\del \theta}
  \biggl(\sin \theta \frac{\del}{\del \theta}\biggr)
  + \frac{1}{\sin^2 \theta} \frac{\del^2}{\del \vph^2}.
\edm %----------------------------------------------------------------------
ここで $(\theta,\vph)$ は次のような球面極座標である.

\begin{figure}[h]
\begin{center}
\includePdfEps{}{snj_polar_coordinate}
\caption{球面極座標}
\end{center}
\end{figure}

Laplace-Berltrami 作用素の固有値はよく知られているように
$n(n+1)$, $n=0,1,2,\dots$ である.

対応する固有関数を記述するには次のものが必要である.
\begin{itemize}
\item Legendre 多項式:
\bdm %----------------------------------------------------------------------
P_n(x)
= \frac{(-1)^n}{2^n n!} \frac{d^n}{dx^n}(1-x^2)^n.
\edm %----------------------------------------------------------------------

\item Legendre 多項式の満たす微分方程式
\bdm %----------------------------------------------------------------------
(1-x^2)P_n'' - 2x P_n' = -n(n+1) P_n.
\edm %----------------------------------------------------------------------
\item Legendre の陪関数:
\bdm %----------------------------------------------------------------------
P_n^m(x)
= (-1)^m(1-x^2)^{m/2}\frac{d^m}{dx^m} P_n(x).
\edm %----------------------------------------------------------------------
\item Legendre の陪関数の満たす微分方程式
\bdm %----------------------------------------------------------------------
(1-x^2)\frac{d^2}{dx^2} P_n^m(x) - 2x \frac{d}{dx} P_n^m(x) 
  + \biggl[ n(n+1)-\frac{m^2}{1-x^2} \biggr] P_n^m(x)
= 0.
\edm %----------------------------------------------------------------------
\end{itemize}

これらの関数を用いると,Laplace-Berltrami 作用素の固有関数は
\bdm %----------------------------------------------------------------------
P_n^m(\cos \theta) e^{im\vph}, \quad
n=0,1,\dots, \ m=-n, -n+1, \dots, -1, 0,1,\dots, n
\edm %----------------------------------------------------------------------
で与えられる.

これらはよく知られた結果であるが,Laplace-Beltrami 作用素に回転 $\frac{\del}{\del\vph}$ が加わったものもこれが固有関数になる.
実際
\bdm %----------------------------------------------------------------------
\frac{\del}{\del \vph} [ P_n^m(\cos \theta) e^{im\vph}]
= im P_n^m(\cos \theta) e^{im\vph}
\edm %----------------------------------------------------------------------
に注意すれば,$-\Laplace-\frac{\del}{\del \vph}$ の固有値は $n(n+1)+im$ で
対応する固有関数は 
\bdm %----------------------------------------------------------------------
P_n^m(\cos \theta) e^{im\vph}
\edm %----------------------------------------------------------------------
であることが容易に分かる.
図示すれば次のようになる.

\begin{figure}[h]
\begin{center}
\includePdfEps{}{snj_LB}
\includePdfEps{}{snj_LB_rotate}
\caption{回転を加えた Laplace-Beltrami 作用素のスペクトル}
\end{center}
\end{figure}

この作用素は正規作用素で実の作用素でもあるので,固有値が実軸に対して
対称に配置されていることが読み取れるだろう.










\subsec{Notes}
\begin{itemize}
\item 
\end{itemize}


	\makeatletter
\def\section{\@startsection {section}{1}{\z@}{-3.5ex plus -1ex minus 
    -.2ex}{2.3ex plus .2ex}{\center\bf}}
\makeatother
\begin{thebibliography}{99}
\small
\itemsep .5pt plus .2pt minus .1pt
\baselineskip=11.5pt
%
\bibitem{Kato76} T.~Kato, % Tosio Kato
``{\it Perturbation theory for linear operators. Second edition,\/}''
Second edition, % Grundlehren der Mathematischen Wissenschaften, Band 132,
Springer-Verlag, Berlin-New York, 1976.
%
\bibitem{Kobayashi72} S.~Kobayashi, % Shoshichi Kobayash,
Transformation groups in differential geometry,
Ergebnisse der Mathematik und ihrer Grenzgebiete, Band 70,
Springer-Verlag, New York-Heidelberg, 1972.
%
\bibitem{Kobayashi95} S.~Kobayashi, % Shoshichi Kobayash,
Transformation groups in differential geometry,
Reprint of the 1972 edition. Classics in Mathematics,
Springer-Verlag, Berlin, 1995.
%
\bibitem{Lions69} J.-L.~Lions, 
``{\it Quelques m\'ethodes de r\'esolution des probl\`emes aux limites non
lin\'eaires\/},'' Dunod, Gauthier-Villars, Paris 1969.
%
\bibitem{Petersen06} P.~Petersen, % Peter Petersen, 
``{\it Riemannian geometry\/},'' Second edition,
Graduate Texts in Mathematics, 171, Springer, New York, 2006.
%
\bibitem{Shigekawa10} I.~Shigekawa, % Ichiro Shigekawa
Non-symmetric diffusions on a Riemannian manifold,
``{\it Probabilistic approach to geometry\/},'',
Adv.\ Stud.\ Pure Math., 57, pp.~437--461, Math. Soc. Japan, Tokyo, 2010.
%
\bibitem{Stone32} M.~H.~Stone,
Linear transformations in Hilbert space and their applications to analysis,
Amer.\ Math.\ Soc.\ Colloq.\ Publi., {\bf 15}, Providence, 1932.
R. S. Strichartz,
Analysis of Laplacian on the complete Riemannian manifold,
{\it J. Funct. Anal.}, {\bf 52} (1983), 48--79.
%
\bibitem{Tanabe75} 田辺 広城、
発展方程式、岩波書店、東京、1975.
%
\bibitem{YB53} K.~Yano and S.~Bochner,
``{\it Curvature and Betti numbers\/},''
Annals of Mathematics Studies, No. 32,
Princeton University Press, Princeton, N. J., 1953.
%
\end{thebibliography}

\fi

\end{document}
