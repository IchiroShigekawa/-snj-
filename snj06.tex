%%%%%%%%%%%%%%%%%%%%%%%%%%%%%%%%%%%%%%%%%%%%%%%%%
%                                               %
%         ====== Program  No.6 =======          %
%                                               %
%             file name snj06.tex               %
%                                               %
%===============================================%
%===================  for  =====================%
%===============================================%
%
\hide
\vspace{-4mm}
\begin{itemize} \itemsep=-2mm \parsep=0mm
\item Total file name: snj01 snj02 $\dots $ snj?, snj\_bibliography
\item File name: snj06.tex \hfill 印刷日: \today \ \now
\item 単調作用素のことを纏めておく.Lions に従う.
それにしても Lions は凄い.
非線型の場合が念頭にあるが,後でStannat の generalized Dirichlet form の
話をするのに,非線型の話が必要になる.
Stannat の本には証明を書いていないので,ノートとして纏めておく.
Lions の本はフランス語だけれど,英語の文献が何かあると思うのだが分からない.
[2011年2月24日]
\end{itemize}
\endhide
\SS{FPM}{単調作用素の基本的性質} %==========================================
% FPM
後で必要になる単調作用素に関する定理を準備しておく.
$V$ を実 Banach 空間とする.
双対空間を $V^*$ とかく.
\Definition{FPM-2} %********************************************************
写像 $T\colon V\to V^*$ が,すべての $u$, $v\in V$ に対し
\bdn %----------------------------------------------------------------------
(Tu-Tv, u-v) \ge 0
\Eqn{FPM.6}
\edn %----------------------------------------------------------------------
が成り立つとき{\bf 単調}であるという.
さらに $(Tu-Tv, u-v) =0$ となるのは $u=v$ のときに限るとき,
{\bf 狭義単調}であるという.
\end{definition} %**********************************************************

\Remark{FPM-4} %************************************************************
複素 Banach 空間の場合は \Eq{FPM.6} を $\Re(Tu-Tv, u-v) \ge 0$
の条件に替える.
また $T$ も一般には多価関数に対して定義するが,
ここではそこまで一般にはしない.
さらに深入理論は田辺に述べてあるが,ここでは Lions に従って
以下の議論を進める.
\end{remark} %**********************************************************


一般にBanach 空間から Banach 空間への写像が任意の有界集合を有界集合に
移すとき,{\bf 有界}であるという.
さらに後で必要になる連続性について述べておこう.


\Definition{FPM-6} %********************************************************
写像 $T\colon V\to V^*$ が線分上弱連続であることを,
すべての $u$, $v$, $w\in V$ に対し
\bdn %----------------------------------------------------------------------
[0,1]\ni t\mapsto (T((1-t)u+tv),w)
\Eqn{FPM.8}
\edn %----------------------------------------------------------------------
が連続であることと定義する.
\end{definition} %**********************************************************

さて単調写像の連続性に関して次が成り立つ.

\Proposition{FPM-10} %******************************************************
$V$ を反射的な Banach 空間とする.
$T\colon V\to V^*$ が有界単調写像で線分上弱連続であるとき,
任意の $v\in V$ に対して $u\mapsto (Tu,v)$ は連続である.
即ち $T$ は $V$(強位相) から $V^*$(弱位相) への連続な写像である.
\end{proposition} %*********************************************************

\Proof
$u_n$ が $u$ に強収束しているとする.
$\{Tu_n\}$ は有界集合だから弱収束する部分列を持つ.
部分列のとり方によらず極限が $Tu$ であることを示せばよい.
簡単のために $\{Tu_n\}$ が $f$ に弱収束しているとする.

さて,単調性から任意の $w$ に対し
\bdm %----------------------------------------------------------------------
(Tu_n - Tw, u_n-w)
\ge 0
\edm %----------------------------------------------------------------------
が成り立つ.
$w=(1-t)u+tv$ $t\in(0,1)$ ととると.
\bdm %----------------------------------------------------------------------
0
&\le (Tu_n - Tw, u_n-w)
= (Tu_n - Tw, u_n-(1-t)u -tv) \\
&= (Tu_n - Tw,u_n-u) + t(Tu_n - Tw, u-v).
\edm %----------------------------------------------------------------------
ここで $n\to\infty$ として
\bdm %----------------------------------------------------------------------
0
\le  t(f - Tw, u-v).
\edm %----------------------------------------------------------------------
よって $t$ で割って
\bdm %----------------------------------------------------------------------
(f, u-v)
\ge (Tw, u-v).
\edm %----------------------------------------------------------------------
さらに $t\to0$ とすると,右辺は $T$ の線分上弱連続性から
$(Tu, u-v)$ に収束する.
よって
\bdm %----------------------------------------------------------------------
(f, u-v)
\ge (Tu, u-v).
\edm %----------------------------------------------------------------------
これが任意の $v$ について成り立つので $Tu=f$ が従う.
これが示すべきことであった.
\QED %======================================================================

\hide
P.-L.~Lions \cite{Lions69} p.~179, Propposition~2.2.5 ではもう
少し弱い擬単調性からこのことを導いている.
さらに田辺 \cite{Tanabe75} では第6章 \S 5 で単調作用素のことが
より一般の設定で議論されている.
そこまで一般的にする必要はないので,強い条件の下で証明しておいた.
\endhide

最後にもう一つ定義をしておく.

\Definition{FPM-12} %*******************************************************
$T$ を Banach 空間 $V$ からその双対空間 $V^*$ への写像とする.
ある $u_0\in V$ が存在して
\bdn %----------------------------------------------------------------------
\lim_{\|u\|\to\infty} \frac{(Tu,u-u_0)}{\|u\|} = \infty
\Eqn{FPM.10}
\edn %----------------------------------------------------------------------
がなりたつとき $T$ は統御的 (coercive) であるという.
\end{definition} %**********************************************************

さて,ここでは単調作用素の全射性について調べたいのであるが,
そのためにまず有限次元の結果から述べる.

\Proposition{FPM-16} %******************************************************
$P$ を $\R^n$ から $\R^n$ への連続写像で,ある $\rho>0$ が存在して
\bdn %----------------------------------------------------------------------
(P(\xi),\xi) \ge 0, \quad \forall \xi \text{ with $|\xi|=\rho$} 
\Eqn{FPM.12}
\edn %----------------------------------------------------------------------
が成り立っている.
このときある $\xi$, $|\xi| \le \rho$ が存在して $P(\xi)=0$ とできる.
\end{proposition} %*********************************************************

\Proof
背理法で示す.
$K=\{\xi ; \|\xi|\le \rho\}$ で $P(\xi)\not=0$ とする.
さらに
\bdm %----------------------------------------------------------------------
\xi \mapsto -P(\xi)\frac{\rho}{|P(\xi)|}
\edm %----------------------------------------------------------------------
という写像を考えるとこれは $K$ から $K$ への連続写像である.
すると Brouwer の不動点定理から不動点 $\xi$ が存在する.
即ち
\bdm %----------------------------------------------------------------------
\xi =  -P(\xi)\frac{\rho}{|P(\xi)|}.
\edm %----------------------------------------------------------------------
この $\xi$ は $|\xi|=\rho$ をみたし
\bdm %----------------------------------------------------------------------
(P(\xi),\xi)
= \biggl(P(\xi), -P(\xi)\frac{\rho}{|P(\xi)|}\biggr)
= -\rho |P(\xi)|
<0
\edm %----------------------------------------------------------------------
となり,\Eq{FPM.12} に矛盾する.
\QED %======================================================================

\hide
これは Lions \cite{Lions69} p.53, Lemme~1.4.3 に述べてある.
確かに証明は簡単だが.
\endhide

さて,これを使って本題の単調関数に関する全射性を導く定理を示す.

\Theorem{FPM-18} %**********************************************************
$V$ を反射的可分 Banach 空間,
$T\colon V \to V^*$ が有界単調写像で統御的であるとする.
このとき $T$ は全射である.
さらに $T$ が狭義単調であれば $T$ は単射でもある.
\end{theorem} %*************************************************************

\Proof
統御的の条件 \Eq{FPM.10} は $u_0=0$ とすることが出来る.
実際 $Tu$ の代わりに $T(u+u_0)$ を考えればよいからである.
以下 \Eq{FPM.10} が $u_0=0$ として成り立っているとする.

さて示すべきは任意に $f\in V^*$ を与えて $Tu=f$ となる
$u$ の存在を示すことである.
$V$ は可分だから $w_1$, $w_2,\dots$ を1次独立で,一次結合が  $V$ で
稠密なものをとる.
$m$ を固定して $\{w_1,\dots,w_m\}$ で張られる線型空間を
$V_m$ とする.
$V_m$ から $\R^m$ への写像 $P$ を
\bdm %----------------------------------------------------------------------
Pu
= ((Tu-f,w_1),\dots,(Tu-f,w_m))
\edm %----------------------------------------------------------------------
で定める.
$\vph^j$ を $w_i$ の dual basis とする.
即ち,
\bdm %----------------------------------------------------------------------
(\vph^i,w_j)
= \d^i_j
\edm %----------------------------------------------------------------------
を満たすものとする.
$V_m$ は次の対応で $\R^m$ と同一視する:
$u\mapsto ((\vph^1,u),\dots,(\vph^m,u))$.
また $u=\sum_j(\vph^j) w_j$ だから
\bdm %----------------------------------------------------------------------
(Tu-f,u)
= \sum_j (Tu-f,w_j)(\vph^j,u)
= (Pu,u)_{\R^n}
\edm %----------------------------------------------------------------------
となる.
さらに
\bdm %----------------------------------------------------------------------
(Pu,u)
= (Tu-f,u)
\ge (Tu,u) - \|f\|\,\|u\|
\edm %----------------------------------------------------------------------
であり,統御的の条件から $\|u\|$ が十分大きいと
\bdm %----------------------------------------------------------------------
(Tu,u)
\ge \|f\|\,\|u\|
\edm %----------------------------------------------------------------------
とできる.
また $V_m$ は有限次元だからノルム $\|u\|$ は $\sqrt{\sum_j (\vph^j,u)^2}$
と同値になる(定数倍で上下から抑えられる).
よって $\rho$ を十分大きくとると $\sqrt{\sum_j (\vph^j,u)^2}=\rho$
のとき $Pu,u)\ge 0$ が成り立つ.
また \Prop{FPM-10} から $P$ は $V_m$ 上で連続である.
よって \Prop{FPM-16} から $Pu_m=0$ となる$u_m\in V_m$ が存在する.
これは
\bdm %----------------------------------------------------------------------
(Tu_m,w_j)
= (f,w_j), \quad j=1,2,\dots,m
\edm %----------------------------------------------------------------------
を意味している.
これから
\bdm %----------------------------------------------------------------------
(Tu_m,u_m)
= (f,u_m)
\le \|f\| \, \|u_m\|
\edm %----------------------------------------------------------------------
であるから
\bdm %----------------------------------------------------------------------
\frac{(Tu_m,u_m)}{\|u_m\|}
\le \|f\|
\edm %----------------------------------------------------------------------
が成り立つ.

ここから $m$ を動かして考えよう.
統御的の条件から $\|u_m\|$ は有界である.
さらに $T$ の有界性を使えば $Tu_m$ も有界である.
そこで部分列 $\{u_\mu\}$ を
\bg %----------------------------------------------------------------------
u_\mu \to u \text{ weakly in $V$} \\
Tu_\mu \to \chi \text{ weakly in $V^*$}
\eg %----------------------------------------------------------------------
ととる.
$j$ を固定すると $(Tu_\mu,w_j) = (f,w_j)$ が十分大きな $\mu$ に対して
成り立っているから $\mu\to\infty$ として
\bdm %----------------------------------------------------------------------
(\chi,w_j) = (f,w_j).
\edm %----------------------------------------------------------------------
これが任意の $j$ について成り立つから $\chi=f$ である.
一方
\bdm %----------------------------------------------------------------------
(Tu_\mu, u_\mu) = (f,u_\mu) \to (f,u)
\edm %----------------------------------------------------------------------
より
\bdn %----------------------------------------------------------------------
\lim_{\mu\to\infty} (Tu_\mu, u_\mu) = (f,u).
\Eqn{FPM.14}
\edn %----------------------------------------------------------------------

あとは $Tu=\chi$ を示せばよい.
そこで単調性から
\bdm %----------------------------------------------------------------------
(Tu_\mu-Tv,u_\mu-v) \ge 0 \quad \forall v\in V
\edm %----------------------------------------------------------------------
即ち
\bdm %----------------------------------------------------------------------
(Tu_\mu, u_\mu) - (Tu_\mu, v) - (Tv,u_\mu-v) \ge 0.
\edm %----------------------------------------------------------------------
ここで \Eq{FPM.14} に注意して $\mu\to\infty$ とすれば
\bdm %----------------------------------------------------------------------
(f,u) - (\chi, v) - (Tv,u-v) \ge 0.
\edm %----------------------------------------------------------------------
従って
\bdm %----------------------------------------------------------------------
(\chi-Tv,u-v) \ge 0.
\edm %----------------------------------------------------------------------
ここで $v=(1-t)u + tw$ $t\in(0,1)$ とすると
\bdm %----------------------------------------------------------------------
0
\le (\chi-Tv,u-(1-t)u - tv)
= t(\chi-Tv,u+w).
\edm %----------------------------------------------------------------------
$t$ で割って
\bdm %----------------------------------------------------------------------
(\chi-Tv,u+w) \ge 0.
\edm %----------------------------------------------------------------------
ここで $t\to 0$ とすると $T$ の線分上弱連続性から
\bdm %----------------------------------------------------------------------
(\chi-Tu,u+w) \ge 0.
\edm %----------------------------------------------------------------------
$w$ は任意にとれるから $Tu=\chi$ が示せる.

狭義単調性があれば $Tu=Tv$ ならば
\bdm %----------------------------------------------------------------------
(Tu-Tv,u-v) = 0
\edm %----------------------------------------------------------------------
なので $u=v$ が成り立ち $T$ は単射である.
\QED %======================================================================

\subsec{Notes}
\begin{itemize}
\item 単調写像の全射性について纏めた.
P.L.~Lions \cite{Lions69} に従っている.
[2011年2月24日]
\end{itemize}

