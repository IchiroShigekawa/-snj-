%%%%%%%%%%%%%%%%%%%%%%%%%%%%%%%%%%%%%%%%%%%%%%%%%
%                                               %
%        =======  Program  No.4  =======        %
%                                               %
%===============================================%
%%%%%%%%%%%%%%%%%%%%%%%%%%%%%%%%%%%%%%%%%%%%%%%%%
%
\hide
\vspace{-4mm}
\begin{itemize} \itemsep=-2mm \parsep=0mm
\item Total file name: snj01 snj02 $\dots $ snj?, snj\_bibliography
\item File name: snj04.tex \hfill コンパイル日: \today \ \now
\end{itemize}
\endhide

\SS{NSS}{正規作用素} %//////////////////////////////////////////////////////
% Non-symmetric semigroup
\hide
論文としてまとめようと思いながら,とうとう2011年となってしまった.
前に進むのが相変わらず遅い.
\hfill 2011年1月3日(月)
\endhide

正規作用素を考察しよう
作用素 $A$ が正規作用素とは,$A$ が次の条件をみたすときをいう.
\bdn %----------------------------------------------------------------------
A^*A
= AA^*.
\Eqn{NSS.20}
\edn %----------------------------------------------------------------------
自己共役作用素は正規作用素であるが,一般に作用素の正規性を確かめるのは
難しい問題である.
応用上現れる作用素は,定義域を特徴付けることが難しく,
ある作用素の閉包,というような形で定義されることが多いからである.
この問題はまた後で考えることにする.
$A$ が正規作用素なら複素数 $\lm$ に対し $\lm+A$ も正規であることは
容易に分かる.
実際
\bdm %----------------------------------------------------------------------
(\lm+A)^* (\lm+A)
&= (\ol{\lm} +A^*) (\lm+A)
= \lm^2 + \ol{\lm}A + \lm A^* + A^*A \\
&= \lm^2 + \ol{\lm}A + \lm A^* + AA^*
= (\lm+A) (\lm+A)^*.
\edm %----------------------------------------------------------------------
正規作用素に $A$ 対しては スペクトル分解定理が成立する.
即ち,単位の分解 $E(dz)$ が存在し
\bdn %----------------------------------------------------------------------
A
= \int_\C z E(dz)
\Eqn{NSS.22}
\edn %----------------------------------------------------------------------
と表現される.
$E$ は射影作用素に値をとる $\C$ 上の測度である.
$\s(A)$ を $A$ のスペクトル集合とすると $E(dz)$ の台は $\s(A)$ と一致する.
ここで
\bdm %----------------------------------------------------------------------
(Au,u)
= \int_\C z (E(dz)u,u).
\edm %----------------------------------------------------------------------
が成立する.
右辺の積分は $\C$ 全体ではなく $\s(A)$ に制限してよい.
$|u|=1$ ならば $(E(dz)u,u)$ は 確率測度である.
つまり $(Au,u)$ は $\s(A)$ の convex combination で表されていることになる.
$\s(A)$ の閉凸包を $\co(\s(A))$ で表すと numerical range $\Theta(A)$
に対しては $\Theta(A) \subseteq \co(\s(A))$ が成り立つ.
一方 $\ol{\Theta(A)}$ はスペクトルを含む閉凸集合であった.
従って
\bdn %----------------------------------------------------------------------
\ol{\Theta(A)} = \co(\s(A))
\Eqn{NSS.24}
\edn %----------------------------------------------------------------------
が得られた.
これで sectorial かどうかはスペクトル $\s(A)$ を見れば
完全に分かることになる.

さて,後で縮小半群を生成する作用素を考えたいので,
ここで扱う正規作用素は次の増大性を仮定する
\bdn %----------------------------------------------------------------------
\Re (Au,u)
\ge 0, \quad \forall u\in \Dom(A).
\Eqn{NSS.26}
\edn %----------------------------------------------------------------------
従って縮小半群との対応を考えるときは $-A$ が縮小半群の生成作用素となる.
\Eq{NSS.22} から $A^*$ に対しては
\bdn %----------------------------------------------------------------------
A^*
= \int_\C \ol{z} E(dz)
\Eqn{NSS.28}
\edn %----------------------------------------------------------------------
が成り立つ.
これは
\bdm %----------------------------------------------------------------------
(A^*u,v)
= (u,Av)
= (u,\int_\C z E(dz)v)
= \int_\C \ol{z} (u,E(dz)v)
= \int_\C \ol{z} (E(dz)u,v)
= (\int_\C \ol{z}E(dz)u,v)
\edm %----------------------------------------------------------------------
から分かる.


さらに $A$ はスペクトル分解を持つから作用素の平方根 $\sqrt{A}$ が
次で定義できる.
\bdn %----------------------------------------------------------------------
\sqrt{A}
= \int_\C \sqrt{z} E(dz).
\Eqn{NSS.30}
\edn %----------------------------------------------------------------------
ここで $\sqrt{z}$ の分岐は実軸上で $\sqrt{z}$ となるものをとる.
この作用素の定義域は $|\sqrt{z}|^2 = |z|$ であるから
\bdn %----------------------------------------------------------------------
\Dom(\sqrt{A})
= \{u\in H;\, \int_\C |z| (u,E(dz)u) < \infty \}
\Eqn{NSS.32}
\edn %----------------------------------------------------------------------
である. 
同様に $A^*$ に対しても
\bdn %----------------------------------------------------------------------
\sqrt{A^*}
= \int_\C \sqrt{\ol{z}} E(dz)
\Eqn{NSS.34}
\edn %----------------------------------------------------------------------
および
\bdn %----------------------------------------------------------------------
\Dom(\sqrt{A^*})
= \{u\in H;\, \int_\C |z| (u,E(dz)u) < \infty \}
\Eqn{NSS.36}
\edn %----------------------------------------------------------------------
が成り立つ.
両者から正規作用素に対しては 
\bdn %----------------------------------------------------------------------
\Dom(\sqrt{A})
= \Dom(\sqrt{A^*})
\Eqn{NSS.38}
\edn %----------------------------------------------------------------------
が常に成り立っていることが分かる.
これはある意味で Kato の平方根問題が肯定的に解けていることを意味する.
ここでは sesquilinear form との関連を調べよう.

さて, $A$ に対応する sesquilinear form $a$ を定義するなら
\bdn %----------------------------------------------------------------------
a(u,v)
= (Au,v), \quad u,v\in\Dom(A)
\Eqn{NSS.40}
\edn %----------------------------------------------------------------------
とするのが自然である.
さらにこの $a$ の実部 (対称な部分) $b$ は
\bdn %----------------------------------------------------------------------
b(u,v)
= \frac{(Au,v) + (A^*u,v)}{2}, \quad u,v\in\Dom(A)
\Eqn{NSS.42}
\edn %----------------------------------------------------------------------
で定義される.
増大性の仮定から 
\bdm %----------------------------------------------------------------------
b(u,u)
= \frac{(Au,u) + (A^*u,u)}{2}
= \Re(Au,u)
\ge 0
\edm %----------------------------------------------------------------------
が成り立つ.
さらにスペクトル分解を用いて
\bdm %----------------------------------------------------------------------
b(u,v)
= \int_\C \Re{z} (u,E(dz)v)
\edm %----------------------------------------------------------------------
が成り立つ.
これから定義域を次で定めると $b$ は閉になる:
\bdn %----------------------------------------------------------------------
\Dom(b)
= \{u\in H;\, \int_\C \Re{z} (u,E(dz)u) < \infty \}.
\Eqn{NSS.44}
\edn %----------------------------------------------------------------------

これから $\Dom(\sqrt{A}) \subseteq \Dom(b)$ であることは明らかである.
二つが一致するための必要十分条件を与えよう.
$0<\theta<\pi$ に対し複素平面内の角領域 $S_\theta$ を
\bdn %----------------------------------------------------------------------
S_\theta
= \{z\in\C;, |\arg z| \le \theta \}
\Eqn{NSS.50}
\edn %----------------------------------------------------------------------
で定義する.
このとき次が成り立つ.
\Theorem{NSS-4} %***********************************************************
$\Dom(\sqrt{A}) = \Dom(b)$ が成り立つための必要十分条件は
ある $\theta\in (0,\pi/2)$ が存在して $1+\s(A) \subseteq S_\theta$
となることである.
\end{theorem} %*************************************************************

\Proof
ある $\theta\in(0,\pi/2)$ が存在して $1+\s(A) \subseteq S_\theta$
が成立しているとしよう.
すると $u\in \Dom(\sDiri)$ ならば
$z=x+iy\in S_\theta-1$ ならば $|y| \le (x+1)\tan\theta$ だから
\bdm %----------------------------------------------------------------------
\int_\C |z| (E(dz)u,u)
&= \int_{S_\theta-1} |z| (E(dz)u,u) \\
&\le \int_{S_\theta-1} (x+ |y|) (E(dz)u,u) \\
&\le \int_{S_\theta-1} (x+ (x+1)\tan\theta) (E(dz)u,u) \\
&\le (1+\tan\theta) \int_{S_\theta-1} x (E(dz)u,u)
    + \tan\theta \int_{S_\theta-1} (E(dz)u,u)
< \infty
\edm %----------------------------------------------------------------------
となり $u\in \Dom(\sqrt{A})$ が従う.

逆に,いくら $\theta$ を $\frac{\pi}{2}$ に近くとっても
$1+\s(A) \subseteq S_\theta$ とならないときは,任意の $n\in\N$ に対し
$z_n=x_n+iy_n\in \s(A)$ を
\bdm %----------------------------------------------------------------------
|y_n|
\ge n(x_n+1) + 1
\edm %----------------------------------------------------------------------
が成り立つように出来る.
\memo{条件からは $|y_n| \ge n(x_n+1)$ だけど,これから $|y_n|-1 \ge (n-1)(x_n+1) +1$ が成立するから,番号を一つずらせばよい.}
さらに $|z_n-z_m|>2$ ($n\not=m$)としてよい.
$B_n$ を $z_n$ を中心とする半径 $1$ の閉円板とする.
そして $u_n\in\Ran E(B_n)$ を $|u_n|=1$ となるようにとる.
とり方から $\{u_n\}$ は正規直交系になる.
これを用いて $u = \sum_n \frac{u_n}{n\sqrt{x_n+1}}$ と定める.
$u\in H$ は明らかである.
また 
\bdm %----------------------------------------------------------------------
\int_\C \Re z(E(dz)u,u)
&=   \sum_n \int_\C \Re z
     (E(dz)\frac{u_n}{n\sqrt{x_n+1}},\frac{u_n}{n\sqrt{x_n+1}}) \\
&=   \sum_n \int_{B_n} \Re z
     (E(dz)\frac{u_n}{n\sqrt{x_n+1}},\frac{u_n}{n\sqrt{x_n+1}}) \\
&\le \sum_n \int_{B_n} (x_n+1) \frac{1}{n^2(x_n+1)} (E(dz)u_n,u_n) \\
&= \sum_n \frac{1}{n^2}
<  \infty
\edm %----------------------------------------------------------------------
より $u\in\Dom(\sDiri)$ である.

一方
\bdm %----------------------------------------------------------------------
\int_\C |z|(E(dz)u,u)
&=   \sum_n \int_\C |z|
     (E(dz)\frac{u_n}{n\sqrt{x_n+1}},\frac{u_n}{n\sqrt{x_n+1}}) \\
&=   \sum_n \int_{B_n} |z|
     (E(dz)\frac{u_n}{n\sqrt{x_n+1}},\frac{u_n}{n\sqrt{x_n+1}}) \\
&\ge \sum_n \int_{B_n} (|y_n|-1) \frac{1}{n^2(x_n+1)} (E(dz)u_n,u_n) \\
&\ge \sum_n \frac{n(x_n+1)}{n^2(x_n+1)}
=    \infty
\edm %----------------------------------------------------------------------
となるから $u\not\in\Dom(\sqrt{A})$ である.
これで主張が示せた.
\QED %======================================================================

最後に実作用素の場合の注意を与えておこう.
正規作用素 $A$ が実作用素であるとする.
単位の分解 $E(dz)$ が存在して
\bdm %----------------------------------------------------------------------
A
= \int_\C z E(dz)
\edm %----------------------------------------------------------------------
と表される.
従って
\bdm %----------------------------------------------------------------------
\J A \J
= \J \int_\C z E(dz) \J
=  \int_\C \ol{z} \J E(dz)\J
=  \int_\C z \J E(d\ol{z})\J. \quad (\because \text{ 変数変換})
\edm %----------------------------------------------------------------------
$\J E(d\ol{z})\J$ も一つの単位の分解を与えている.
ここで $A=\J A \J$ だから,分解の一意性から二つの単位の分解は一致する.
従って
\bdn %----------------------------------------------------------------------
\J E(d\ol{z})\J
=  E(dz)
\Eqn{NSS.58}
\edn %----------------------------------------------------------------------
あるいは
\bdn %----------------------------------------------------------------------
E(d\ol{z})
= \J  E(dz)\J
\Eqn{NSS.60}
\edn %----------------------------------------------------------------------
成立していることが示せた.



\subsec{Notes}
\begin{itemize}
\item
\end{itemize}


