%%%%%%%%%%%%%%%%%%%%%%%%%%%%%%%%%%%%%%%%%%%%%%%%%
%                                               %
%         ====== Program  No.18 =======          %
%                                               %
%             file name snj18.tex               %
%                                               %
%===============================================%
%===================  for  =====================%
%===============================================%
%
\hide
\vspace{-4mm}
\begin{itemize} \itemsep=-2mm \parsep=0mm
\item Total file name: snj01 snj02 $\dots $ snj?, snj\_bibliography
\item File name: snj18.tex \hfill 印刷日: \today \ \now
\item 正規作用素の例を挙げる.
さらにそのスペクトルを調べる.
[2011年8月16日(火)]
\end{itemize}
\endhide
\SS{BWD}{ドリフトのある1次元ブラウン運動} %===================================
% On-dimensional brownian motion with a drift
ここからいくつか正規作用素の例を与える.
またスペクトルを完全な形で求めることを目標とする.

$\R$ 上で作用素 $\gen = \frac{d^2}{dx^2} - c\frac{d}{dx}$ を考える.
但しここでは基礎の測度を
\bdn %----------------------------------------------------------------------
\nu_1(dx)
= e^{-cx}dx
\Eqn{BWD.4}
\edn %----------------------------------------------------------------------
にとる.
このとき
\bdm %----------------------------------------------------------------------
\int_\R (\frac{d^2}{dx^2}f - c\frac{d}{dx}f)g(x) e^{-cx}\,dx
&= \int_\R \frac{d}{dx}(\frac{d}{dx}fe^{-cx})\, g(x)\,dx
&= - \int_\R (\frac{d}{dx}f e^{-cx}) \frac{d}{dx}g(x)\,dx
\edm %----------------------------------------------------------------------
であるから,$\gen$ は自己共役作用素となる.
実際この作用素は次の対称双線型形式 $\Diri$ に対応する自己共役作用素である:
\bdn %----------------------------------------------------------------------
\Diri(f,g)
= \int_\R \frac{df}{dx} \frac{dg}{dx} e^{-cx}\,dx
\Eqn{BWD.6}
\edn %----------------------------------------------------------------------
また $\nu_1$ に関する $\frac{d}{dx}$ の共役は
\bdn %----------------------------------------------------------------------
\Bigl(\frac{d}{dx}\Bigr)_{\nu_1}^*
= -\frac{d}{dx} + c
\Eqn{BWD.8}
\edn %----------------------------------------------------------------------
であり,生成作用素 $\gen$ は
\bdm %----------------------------------------------------------------------
\gen
= \Bigl(\frac{d}{dx}\Bigr)_{\nu_1}^* \frac{d}{dx}
\edm %----------------------------------------------------------------------
の形でも表される.
$\gen$ のスペクトルを調べるには変換したほうがよい.
次の写像 $I\colon L^2(\nu_1)\longrightarrow L^2(dx)$ は等距離作用素である:
\bdn %----------------------------------------------------------------------
If(x)=e^{-cx/2}f(x).
\Eqn{BWD.12}
\edn %----------------------------------------------------------------------
ここで
\bdm %----------------------------------------------------------------------
I\circ \gen \circ I^{-1} f
&= e^{-cx/2}(\frac{d^2}{dx^2} - c\frac{d}{dx})(e^{cx/2}f) \\
&= e^{-cx/2}(\frac{c^2}{4} e^{cx/2} f + c e^{cx/2}\frac{df}{dx}
    + e^{cx/2} \frac{d^2f}{dx^2} - c\frac{1}{2}ce^{cx/2} f
    - ce^{cx/2}\frac{df}{dx}) \\
&= - \frac{c^2}{4} f + \frac{d^2f}{dx^2}
\edm %----------------------------------------------------------------------
であるから次の図式が可換となる:
\bdn %----------------------------------------------------------------------
\begin{CD}
L^2(\nu_1)	@>{\gen}>>	L^2(\nu_1)		\\
@V{I}VV             @VV{I}V	\\
L^2(dx)		@>{\frac{d^2}{dx^2} - \frac{c^2}{4}}>>	L^2(dx)
\end{CD}
\Eqn{BWD.14}
\edn %----------------------------------------------------------------------
従ってこの場合のスペクトル集合を $\s_1$ とすると
\bdn %----------------------------------------------------------------------
\s_1
= [\frac{c^2}{4},\infty)
\Eqn{BWD.16}
\edn %----------------------------------------------------------------------
となる.

さて,作用素 $\gen$ に摂動を加えて,スペクトルの変化を見ていく.
ベクトル場 $b$ を
\bdn %----------------------------------------------------------------------
b
= k\frac{d}{dx}
\Eqn{BWD.18}
\edn %----------------------------------------------------------------------
で定める.
すると $\nu_1$ に関する発散は
\bdm %----------------------------------------------------------------------
\div_{\nu_1} b
= -ck
\edm %----------------------------------------------------------------------
となる.
従って\Thm{CNO-16} から $\gen+b$ は正規作用素になる.
このスペクトルを求めよう.
この場合もやはり $I$ で変換して $L^2(dx)$ の話に持っていく.
\bdm %----------------------------------------------------------------------
I\circ b \circ I^{-1}f
&= e^{-cx/2}(k\frac{d}{dx})(e^{cx/2}f) \\
&= e^{-cx/2} (k\frac{1}{2}ce^{cx/2} f + ke^{cx/2}\frac{df}{dx}) \\
&= \frac{kc}{2} f + k \frac{df}{dx}
\edm %----------------------------------------------------------------------
であるから
\bdm %----------------------------------------------------------------------
I\circ (\gen+ b) \circ I^{-1}f
&= (\frac{d^2}{dx^2} - \frac{c^2}{4} + \frac{kc}{2} + k \frac{d}{dx})f \\
&= (\frac{d^2}{dx^2} + k \frac{d}{dx} - \frac{c(c-2k)}{4} )f.
\edm %----------------------------------------------------------------------
よって $\frac{d^2}{dx^2} + k \frac{d}{dx}$ の
スペクトルを調べればよいことになる. 
そのために Fourier 変換を利用する.
Fourier 変換は次で定義される.
\bdm %----------------------------------------------------------------------
\hat{f}(\xi)
= \frac{1}{\sqrt{2\pi}}\int_\R f(x) e^{-\xi x}\,dx.
\edm %----------------------------------------------------------------------
これは $L^2(dx)$ から $L^2(d\xi)$ への等長変換を与える.
さらに
\bdm %----------------------------------------------------------------------
\int_\R (\frac{d^2}{dx^2} + k\frac{d}{dx})f(x)\ol{g(x)}\,dx
= \int_\R (-\xi^2 + ik\xi) \hat{f}(\xi) \ol{\hat{g}(\xi)}\,d\xi
\edm %----------------------------------------------------------------------
が成り立つ.
即ち,Fourier 変換した空間では $\frac{d^2}{dx^2} + k\frac{d}{dx}$ に対応する
写像は $-\xi^2 + ik\xi$ をかけるという掛け算作用素である.
よって $\frac{d^2}{dx^2} + k\frac{d}{dx}$ のスペクトルは
\bdm %----------------------------------------------------------------------
\{ - \xi^2 + ik\xi; \xi\in\R\}
\edm %----------------------------------------------------------------------
という放物線である.

元に戻って,$L^2(\nu_1)$ での
$-\gen -b = -\frac{d^2}{dx^2} + c \frac{d}{dx} - k\frac{d}{dx} - $
のスペクトルは
\bdm %----------------------------------------------------------------------
 \{ \frac{c(c-k)}{2} +  \xi^2 + ik\xi;\, \xi\in\R\}
\edm %----------------------------------------------------------------------
である.


\begin{center}
\includePdfEps{}{snj_BM_drift1}
\includePdfEps{}{snj_BM_drift2}
\end{center}




\begin{comment}
\begin{center}
\begin{pspicture}(-1,-4)(5,3) %\showgrid
%\psset{arcangle=8, linewidth=.8pt}  % default
%\psset{arrowsize=1.5pt 2} % default
\psset{arrowsize=3pt 2, arrowlength=1.5}
\psset{xunit=8mm, yunit=12mm}
\psaxes[labels=none,showorigin=true,linewidth=.2pt,ticks=none]{->}(0,0)(-1,-2)(4.5,2)
\psset{linewidth=1pt,arrows=c-c,linecolor=red}
\psline(.5,0)(4.2,0)

%\psecurve(9,3)(4,2)(1,1)(0,0)(1,-1)(4,-2)(9,-3)
%\psecurve(9,3)(4,2)(0,0)(4,-2)(9,-3)

\uput[90](.5,0){$\ds \frac{c^2}{4}$}
\uput{5pt}[45](1,-3){$-\gen$}
%\showgrid
\end{pspicture} 
%
\begin{pspicture}(-1,-4)(5,3) %\showgrid
%\psset{arcangle=8, linewidth=.8pt}  % default
%\psset{arrowsize=1.5pt} % default
\psset{arrowsize=3pt 2, arrowlength=1.5}
\psset{xunit=8mm, yunit=12mm}
\psaxes[labels=y,showorigin=true,linewidth=.2pt,ticks=none]{->}(0,0)(-1,-2)(4.5,2)
\psset{linewidth=1pt,arrows=c-c,linecolor=red}
%\psecurve(6.25,2.5)(4,2)(2.25,1.5)(1,1)(.25,.5)(0,0)(1,-1)(2.25,-1.5)(4,-2)(6.25,-2.5)
\pscurve(5,2)(1,0)(5,-2)
\uput{5pt}[90](1,0){$\frac{c(c-2k)}{4}$}
\uput{5pt}[45](1,-3){$-\gen - k\frac{d}{dx}$}
\end{pspicture} 
\end{center}
\end{comment}


これは $k$ とともにスペクトルが連続的に変化している様を表している.
これを違った観点から見てみよう.
作用素を $\gen = \frac{d^2}{dx^2} - c\frac{d}{dx}$ とする.
測度を $\nu_0=dx$, $\nu_1=e^{-cx}dx$ として $L^2(\nu_0)$ でのスペクトルと
$L^2(\nu_1)$ でのスペクトルを比べてみよう.
同じ作用素を異なった空間で考えていることになる.
上の計算は $-\gen$ のスペクトルは $L^2(\nu_0)$ では $\{\xi^2 - ic\xi\}$
であり $L^2(\nu_1)$ では $[\frac{c^2}{4},\infty)$ となっていることが分かる.


\begin{center}
\includePdfEps{}{snj_BM_exp1}
\includePdfEps{}{snj_BM_exp2}
\end{center}


\begin{comment}
\begin{center}
\begin{pspicture}(-1,-4)(5,3) %\showgrid
%\psset{arcangle=8, linewidth=.8pt}  % default
%\psset{arrowsize=1.5pt} % default
\psset{arrowsize=3pt 2, arrowlength=1.5}
\psset{xunit=8mm, yunit=12mm}
\psaxes[labels=y,showorigin=true,linewidth=.2pt,ticks=none]{->}(0,0)(-1,-2)(4.5,2)
\psset{linewidth=1pt,arrows=c-c,linecolor=red}
%\psecurve(6.25,2.5)(4,2)(2.25,1.5)(1,1)(.25,.5)(0,0)(1,-1)(2.25,-1.5)(4,-2)(6.25,-2.5)
\pscurve(4,2)(0,0)(4,-2)
\uput{5pt}[45](1,-3){w.r.t. $\nu_0=dx$}

%\showgrid
\end{pspicture} 
%
\begin{pspicture}(-1,-4)(5,3) %\showgrid
%\psset{arcangle=8, linewidth=.8pt}  % default
%\psset{arrowsize=1.5pt 2} % default
\psset{arrowsize=3pt 2, arrowlength=1.5}
\psset{xunit=8mm, yunit=12mm}
\psaxes[labels=none,showorigin=true,linewidth=.2pt,ticks=none]{->}(0,0)(-1,-2)(4.5,2)
\psset{linewidth=1pt,arrows=c-c,linecolor=red}
\psline(.5,0)(4.2,0)

%\psecurve(9,3)(4,2)(1,1)(0,0)(1,-1)(4,-2)(9,-3)
%\psecurve(9,3)(4,2)(0,0)(4,-2)(9,-3)

\uput[90](.5,0){$\ds \frac{c^2}{4}$}
\uput{5pt}[45](1,-3){w.r.t. $\nu_1=e^{-cx}dx$}
%\showgrid
\end{pspicture} 
\end{center}
\end{comment}


今度は測度を連続的に動かしてスペクトルの変化を見よう.
$\theta\in[0,1]$ に対し測度 $\nu_\theta$ を
\bdn %----------------------------------------------------------------------
\nu_\theta(dx) = (1-\theta)dx  + \theta e^{-cx} dx
\Eqn{BWD.22}
\edn %----------------------------------------------------------------------
で定める.
$L^2(\nu_\theta)$ でのスペクトルはどのように変化するであろうか.
次のように連続的に変化することも考えられる.

\begin{center}
\includePdfEps{}{snj_BM_measure1}
\end{center}

\begin{comment}
\begin{center}
\begin{pspicture}(-1,-4)(5,3) %\showgrid
%\psset{arcangle=8, linewidth=.8pt}  % default
%\psset{arrowsize=1.5pt} % default
\psset{arrowsize=3pt 2, arrowlength=1.5}
\psset{xunit=8mm, yunit=12mm}
\psaxes[labels=y,showorigin=true,linewidth=.2pt,ticks=none]{->}(0,0)(-1,-2)(4.5,2)
%\psset{linewidth=1pt,arrows=c-c}
\psset{linewidth=1pt,arrows=c-c,linecolor=red}
%\psecurve(6.25,2.5)(4,2)(2.25,1.5)(1,1)(.25,.5)(0,0)(1,-1)(2.25,-1.5)(4,-2)(6.25,-2.5)
\psline(.5,0)(4.2,0)
\pscurve(4,2)(0,0)(4,-2)
\pscurve[linecolor=blue](4,.9)(.25,0)(4,-.9)
\psline[linecolor=black]{->}(2.3,1.2)(3.1,.3)
\uput{5pt}[45](1,-3){w.r.t. $\nu_\theta$}
\uput[45](2.5,1){?}

%\showgrid
\end{pspicture} 
\end{center}
\end{comment}

しかし,これは正しくない.
実際は二つの和集合になる.
\Theorem{BWD-4} %**********************************************************
$\theta\in (0,1)$ のとき $-\gen$ の $L^2(\nu_\theta)$ におけるスペクトルは
\bdm %----------------------------------------------------------------------
\{\xi^2 - ik\xi; \xi\in\R\} \cup [\frac{c^2}{4},\infty)
\edm %----------------------------------------------------------------------
である.
\end{theorem} %*************************************************************



\begin{center}
\includePdfEps{}{snj_BM_measure2}
\end{center}

\begin{comment}
\begin{center}
\begin{pspicture}(-1,-4)(5,3) %\showgrid
%\psset{arcangle=8, linewidth=.8pt}  % default
%\psset{arrowsize=1.5pt} % default
\psset{arrowsize=3pt 2, arrowlength=1.5}
\psset{xunit=8mm, yunit=12mm}
\psaxes[labels=y,showorigin=true,linewidth=.2pt,ticks=none]{->}(0,0)(-1,-2)(4.5,2)
\psset{linewidth=1pt,arrows=c-c,linecolor=red}
%\psecurve(6.25,2.5)(4,2)(2.25,1.5)(1,1)(.25,.5)(0,0)(1,-1)(2.25,-1.5)(4,-2)(6.25,-2.5)
\psline(.5,0)(4.2,0)
\pscurve(4,2)(0,0)(4,-2)
\uput{5pt}[45](1,-3){w.r.t. $\nu_\theta$}
%\showgrid
\end{pspicture} 
\end{center}
\end{comment}















\subsec{Notes}
\begin{itemize}
\item 
\end{itemize}

