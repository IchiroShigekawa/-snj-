%%%%%%%%%%%%%%%%%%%%%%%%%%%%%%%%%%%%%%%%%%%%%%%%%
%                                               %
%         ====== Program  No.19 =======          %
%                                               %
%             file name snj19.tex               %
%                                               %
%===============================================%
%===================  for  =====================%
%===============================================%
%
\hide
\vspace{-4mm}
\begin{itemize} \itemsep=-2mm \parsep=0mm
\item Total file name: snj01 snj02 $\dots $ snj?, snj\_bibliography
\item File name: snj19.tex \hfill 印刷日: \today \ \now
\item 球面上で正規作用素の例を挙げる.
\end{itemize}
\endhide
\SS{LBS}{Laplace-Beltrami 作用素に回転を加えた作用素} %=====================
% Laplace-Beltrami operator with rotation on a sphier
幾何学的な構造が反映するような作用素の例を与えよう.
球面は最も単純な例である.

\subsec{$S^2$ 上の正規作用素} 
球面 $S^2$ 上の Laplace-Berltrami は次のように表される.
\bdm %----------------------------------------------------------------------
\Laplace
= \frac{1}{\sin \theta} \frac{\del}{\del \theta}
  \biggl(\sin \theta \frac{\del}{\del \theta}\biggr)
  + \frac{1}{\sin^2 \theta} \frac{\del^2}{\del \vph^2}.
\edm %----------------------------------------------------------------------
ここで $(\theta,\vph)$ は次のような球面極座標である.

\begin{figure}[h]
\begin{center}
\includePdfEps{}{snj_polar_coordinate}
\caption{球面極座標}
\end{center}
\end{figure}

Laplace-Berltrami 作用素の固有値はよく知られているように
$n(n+1)$, $n=0,1,2,\dots$ である.

対応する固有関数を記述するには次のものが必要である.
\begin{itemize}
\item Legendre 多項式:
\bdm %----------------------------------------------------------------------
P_n(x)
= \frac{(-1)^n}{2^n n!} \frac{d^n}{dx^n}(1-x^2)^n.
\edm %----------------------------------------------------------------------

\item Legendre 多項式の満たす微分方程式
\bdm %----------------------------------------------------------------------
(1-x^2)P_n'' - 2x P_n' = -n(n+1) P_n.
\edm %----------------------------------------------------------------------
\item Legendre の陪関数:
\bdm %----------------------------------------------------------------------
P_n^m(x)
= (-1)^m(1-x^2)^{m/2}\frac{d^m}{dx^m} P_n(x).
\edm %----------------------------------------------------------------------
\item Legendre の陪関数の満たす微分方程式
\bdm %----------------------------------------------------------------------
(1-x^2)\frac{d^2}{dx^2} P_n^m(x) - 2x \frac{d}{dx} P_n^m(x) 
  + \biggl[ n(n+1)-\frac{m^2}{1-x^2} \biggr] P_n^m(x)
= 0.
\edm %----------------------------------------------------------------------
\end{itemize}

これらの関数を用いると,Laplace-Berltrami 作用素の固有関数は
\bdm %----------------------------------------------------------------------
P_n^m(\cos \theta) e^{im\vph}, \quad
n=0,1,\dots, \ m=-n, -n+1, \dots, -1, 0,1,\dots, n
\edm %----------------------------------------------------------------------
で与えられる.

これらはよく知られた結果であるが,Laplace-Beltrami 作用素に回転 $\frac{\del}{\del\vph}$ が加わったものもこれが固有関数になる.
実際
\bdm %----------------------------------------------------------------------
\frac{\del}{\del \vph} [ P_n^m(\cos \theta) e^{im\vph}]
= im P_n^m(\cos \theta) e^{im\vph}
\edm %----------------------------------------------------------------------
に注意すれば,$-\Laplace-\frac{\del}{\del \vph}$ の固有値は $n(n+1)+im$ で
対応する固有関数は 
\bdm %----------------------------------------------------------------------
P_n^m(\cos \theta) e^{im\vph}
\edm %----------------------------------------------------------------------
であることが容易に分かる.
図示すれば次のようになる.

\begin{figure}[h]
\begin{center}
\includePdfEps{}{snj_LB}
\includePdfEps{}{snj_LB_rotate}
\caption{回転を加えた Laplace-Beltrami 作用素のスペクトル}
\end{center}
\end{figure}

この作用素は正規作用素で実の作用素でもあるので,固有値が実軸に対して
対称に配置されていることが読み取れるだろう.










\subsec{Notes}
\begin{itemize}
\item 
\end{itemize}

