%%%%%%%%%%%%%%%%%%%%%%%%%%%%%%%%%%%%%%%%%%%%%%%%%
%                                               %
%        =======  Program  No.2  =======        %
%                                               %
%===============================================%
%%%%%%%%%%%%%%%%%%%%%%%%%%%%%%%%%%%%%%%%%%%%%%%%%
%
\hide
\vspace{-4mm}
\begin{itemize} \itemsep=-2mm \parsep=0mm
\item Total file name: snj01 snj02 $\dots $ snj?, snj\_bibliography
\item File name: snj02.tex \hfill コンパイル日: \today \ \now
\end{itemize}
\endhide

\SS{SFH}{Hilbert 空間の上の sesquilinear form} %////////////////////////////
% Sesquilinear forms in a Hilbert space

Hilbert 空間上の sesquilinear form ついて,後で必要となることをまとめておく.$\Dom(a)\times\Dom(a)$ 上の関数 $a(u,v)$ で $u$ に関して線型,
$v$ に関して共役線型な関数を sesquilinear form という.
さらに $a(u)=a(u,u)$ を quadratic form という.
$a(u,v)$ は次の関係で $a(u)$ から一意的に定まる.
\bdn %----------------------------------------------------------------------
a(u,v)
= \frac{1}{4}(a(u+v) - a(u-v) + ia(u+iv) - ia(u-iv)).
\Eqn{SFH.4}
\edn %----------------------------------------------------------------------
sesquilinear form $a$ が
\bdm %----------------------------------------------------------------------
a(u,v) = \ol{a(v,u)}
\edm %----------------------------------------------------------------------
をみたすとき,対称であるという.
$a$ が対称であるための必要十分条件は,
対応する quadratic form が実数値であることである.

一般の sesquilinear form $a$ に対して adjoint form を
\bdn %----------------------------------------------------------------------
a^*(u,v)
= \ol{a(v,u)}
\Eqn{SFH.6}
\edn %----------------------------------------------------------------------
で定義する.
$a$ が対称であることは $a=a^*$ が成り立つことに他ならない.
また sesquilinear form $a$ に対して
\bdn %----------------------------------------------------------------------
b
= \frac{1}{2}(a+a^*), \quad
c
= \frac{1}{2i}(a-a^*),
\Eqn{SFH.8}
\edn %----------------------------------------------------------------------
とおくと
\bdn %----------------------------------------------------------------------
a
= b + ic
\Eqn{SFH.10}
\edn %----------------------------------------------------------------------
が成り立つ.
$b$ を実部,$c$ を虚部と呼ぶ.
これは quadratic form で考えると
\bdn %----------------------------------------------------------------------
b(u)
= \Re a(u), \quad
c(u)
= \Im a(u)
\Eqn{SFH.12}
\edn %----------------------------------------------------------------------
の関係が成り立っている.

$a$ がsymmetric のとき
\bdn %----------------------------------------------------------------------
a(u)
\ge 0, \quad \forall u\in\Dom(a)
\Eqn{SFH.14}
\edn %----------------------------------------------------------------------
が成り立つとき,非負であるという.
正値性 から次の Schwarz の不等式が成り立つ:
\bdn %----------------------------------------------------------------------
a(u,v)
\le a(u)^{1/2}a(v)^{1/2}.
\Eqn{SFH.16}
\edn %----------------------------------------------------------------------
さて,$a$ を非負として,$t$ を一般の symmetric form とする.
このとき $\Dom(a) \subseteq \Dom(t) $ で quadratic form として
\bdm %----------------------------------------------------------------------
|t(u)| \le K a(u), \quad \forall u\in\Dom(a)
\edm %----------------------------------------------------------------------
が成り立つとき,
\bdm %----------------------------------------------------------------------
|t(u,v)| \le K a(u)^{1/2} a(v)^{1/2}, \quad \forall u,v\in\Dom(a)
\edm %----------------------------------------------------------------------
が成立する.
これを見るには,$u$ の代わりに $e^{i\theta}u$ を考えることにより
$t(u,v)$ は実数としてよい.
すると \Eq{SFH.4} の表示は
\bdm %----------------------------------------------------------------------
t(u,v)
= \frac{1}{4}(t(u+v) - t(u-v))
\edm %----------------------------------------------------------------------
となる.
ここで quadratic form $t$ が実数値であることを使っている.
よって仮定から
\bdm %----------------------------------------------------------------------
|t(u,v)|
\le \frac{K}{4}(a(u+v) + a(u-v))
=   \frac{K}{2}(a(u) + a(v)).
\edm %----------------------------------------------------------------------
ここで $u$, $v$ の代わりに $\e u$, $\e^{-1}v$ をとれば
\bdm %----------------------------------------------------------------------
|t(u,v)|
\le \frac{K}{2}(\e a(u) + \e^{-1} a(v)).
\edm %----------------------------------------------------------------------
$a(u)=0$ ならば $\e\to\infty$ として $t(u,v)=0$ を得る.
$a(u)\not=0$ のときは $\e=\frac{a(v)^1/2}{a(u)^{1/2}}$ と取れば
\bdm %----------------------------------------------------------------------
|t(u,v)|
\le K a(u)^{1/2} a(v)^{1/2}
\edm %----------------------------------------------------------------------
が成り立つことが分かる.

さて $a$ を sesquilinear form とする.  
$a = b+ic$ を実部と虚部に分ける.
$b$ が非負であるとき $a$ も非負 (または accretive) という.
\bdm %----------------------------------------------------------------------
\Re a(u,u) \ge 0, \quad \forall u\in\Dom(a)
\edm %----------------------------------------------------------------------
また $a$ の numerical range を
\bdn %----------------------------------------------------------------------
\Theta(a)
= \{ a(u,u);\, u\in\Dom(a),\ |u|=1\}
\Eqn{SFH.20}
\edn %----------------------------------------------------------------------
で定義する.
$a$ が非負であることは
$\Theta(a) \subseteq \{\zeta\in\C;\, \Re \zeta \ge 0\}$ と同値である.
さらに $a$ が扇形条件をみたす(sectorial)ということを
\bdm %----------------------------------------------------------------------
\Theta(a)
\subseteq S_\theta
\edm %----------------------------------------------------------------------
で定義する.
ここで $S\theta$ 次で定まる角領域である:
\bdn %----------------------------------------------------------------------
S_\theta
= \{z\in\C;\, |\arg z| \le \theta \}.
\Eqn{SFH.22}
\edn %----------------------------------------------------------------------
但し $\theta$ は $0\le \theta<\frac{\pi}{2}$ にとる.

扇形条件が成り立つと,実部 $b$ は非負である.
この条件は $a(u,u)= b(u)+ic(u)$ に注意すれば
$c(u) \le \tan\theta b(u)$ とも同値である.
さらにこの条件は次の条件とも同値になる:
$b$ が非負で
ある定数 $K \ge 0$ が存在して
\bdn %----------------------------------------------------------------------
|a(u,v)|
\le (1+K) b(u)^{1/2} b(v)^{1/2}
\Eqn{SFH.24}
\edn %----------------------------------------------------------------------
が成り立つ.
このことを証明しておこう.

\Proposition{SFH-6} %*******************************************************
$a$ が \Eq{SFH.24} をみたすことと,
ある $\theta\in [0,\frac{\pi}{2})$ が存在して
$\Theta(a) \subseteq S_\theta$ となることは同値である.
\end{proposition} %*********************************************************

\Proof
まず \Eq{SFH.24} が成り立っているとする.
すると $v=u$ として両辺を2乗して
\bdm %----------------------------------------------------------------------
|a(u,u)|^2
\le (1+K)^2 b(u)^2.
\edm %----------------------------------------------------------------------
ところで,$a(u,u) = b(u) + ic(u)$ だから左辺は $b(u)^2 + c(u)^2$ であり
\bdm %----------------------------------------------------------------------
c(u)^2 + b(u)^2
\le (1+2K + K^2) b(u)^2
\edm %----------------------------------------------------------------------
から
\bdm %----------------------------------------------------------------------
c(u) \le \sqrt{K(2 + K)} b(u)
\edm %----------------------------------------------------------------------
が成り立つ.
即ち $\theta=\arctan \sqrt{K(2 + K)}$ として
$\Theta(a) \subseteq S_\theta$ が成り立つ.

逆に $\Theta(a) \subseteq S_\theta$ が成り立てば
$c(u) \le \tan\theta b(u)$ なので
\bdm %----------------------------------------------------------------------
|c(u,v)| \le \tan\theta b(u)^{1/2}b(v)^{1/2}
\edm %----------------------------------------------------------------------
が成り立つ.
従って
\bdm %----------------------------------------------------------------------
|a(u,v)|
&= |b(u,v) + ic(u,v)| \\
&\le |b(u,v)| + |c(u,v)| \\
&\le b(u)^{1/2}b(v)^{1/2} + \tan\theta b(u)^{1/2}b(v)^{1/2} \\
&= (1+\tan \theta) b(u)^{1/2}b(v)^{1/2}.
\edm %----------------------------------------------------------------------
これで \Eq{SFH.24} が $K=\tan\theta$ として成立することが分かった.
\QED %======================================================================

非負性はもう少し緩めることが出来る.
$a$ が下に有界であることを,ある $\gm\in\R$ が存在して
\bdn %----------------------------------------------------------------------
S_\theta
\subseteq \{\zeta \in\C;\, \Re \zeta \ge - \gm \}.
\Eqn{SFH.28}
\edn %----------------------------------------------------------------------
さらに扇形的に下に有界 (sectorially bounded from below) であることを
ある $\gm\in\R$ が存在して
\bdm %----------------------------------------------------------------------
\Theta(a)
\subseteq S_\theta - \gm 
\edm %----------------------------------------------------------------------
が成り立つと定義する.
これは $a_\gm$ を
\bdm %----------------------------------------------------------------------
a_\gm(u,v)
= a(u,v) + \gm(u,v)
\edm %----------------------------------------------------------------------
で定義したとき $a_\gm$ が 扇形条件をみたすことに他ならない.


\subsec{Notes}
\begin{itemize}
\item ここの話は Kato \cite{Kato76} Chapter VI, \S1,\S2 に元づく.
扇形条件という言葉をどう定義するかは迷うところである.
ここでは numerical range を用いて定義した.
Ma-R\"ockner とは用語法がずれる.
[2011年1月15日]
\end{itemize}


