%%%%%%%%%%%%%%%%%%%%%%%%%%%%%%%%%%%%%%%%%%%%%%%%%
%                                               %
%        =======  Program  No.3  =======        %
%                                               %
%===============================================%
%%%%%%%%%%%%%%%%%%%%%%%%%%%%%%%%%%%%%%%%%%%%%%%%%
%
\hide
\vspace{-4mm}
\begin{itemize} \itemsep=-2mm \parsep=0mm
\item Total file name: snj01 snj02 $\dots $ snj?, snj\_bibliography
\item File name: snj03.tex \hfill コンパイル日: \today \ \now
\end{itemize}
\endhide

\SS{OHS}{Hilbert 空間の上の作用素} %////////////////////////////////////////
% Operators on a Hilbert spaces

Hilbert 空間上の作用素のついて,後で必要となることをまとめておく.
$T$ をHilbert 空間 $H$ 上の作用素とする.
$T$ の numerical range $\Theta(T)$ を
\bdn %----------------------------------------------------------------------
\Theta(T)
:= \{(Tu,u);\, u\in\Dom(T)\}. 
\Eqn{OHS.4}
\edn %----------------------------------------------------------------------
ここでは証明しないが $\Theta(T)$ は凸集合であることが知られている
(Stone \cite{Stone32} を参照).
$\Delta=\C\setminus \ol{\Theta(T)}$ とする.
$\Theta(T)$ は凸集合だから,$\Delta$ の連結成分は高々二つである.
連結成分が2つある場合は $\Delta_1$, $\Delta_2$ とかく.
  

\Theorem{OHS-4} %***********************************************************
$T$ を閉作用素とする.
$\zeta\in\Delta$ のとき $T-\zeta$ の像は閉集合となり,
$\Ker(T-\zeta)=\{0\}$ で $T-\zeta$ の指数は $\Delta$ の各連結成分で
一定である.
もし $\Delta$ ($\Delta_1$ or $\Delta_2$) で $T-\zeta$ の指数が $0$
であるならば $\Delta$ ($\Delta_1$ or $\Delta_2$) はレゾルベント集合
$\rho(T)$ に含まれる.
\end{theorem} %*************************************************************

\Proof
まず $u\in\Dom(T)$, $|u|=1$ のとき
\bdm %----------------------------------------------------------------------
|(Tu,u)-\zeta|
= |((T-\zeta u),u)-\zeta|
\le |(T-\zeta)u|
\edm %----------------------------------------------------------------------
が成り立つことに注意しよう.
これから $\zeta\in\Delta$ のとき $\d=d(\zeta, \ol{\Theta(T)})>0$ とおくと
$|(T-\zeta)u|\ge \d$, $|u|=1$ だから
\bdm %----------------------------------------------------------------------
|(T-\zeta)u|
\ge \d |u|, \quad \forall u\in\Dom(T).
\edm %----------------------------------------------------------------------
これから $\Ker(T-\zeta)=0$ で $\Ran(T-\zeta)$ が閉集合であることが従う.
また各連結成分で $T-\zeta$ の指数は指数の安定性から一定である.
特に指数が 0 のときは $\Ran(T-\zeta)=H$ であるから有界な逆が存在する.
従って $\rho(T)$ に含まれていることが分かる.
\QED %======================================================================

さて閉作用素 $A$ が accretive であることを
$\Re (Au,u)\ge 0$, $\forall u\in\Dom(A)$
で定義する.
これは $\Theta(A)$ が右半平面に含まれることを意味する.
さらにある $\Re \zeta<0$ となる $\zeta$ が存在して $\Ran(A-\zeta)=H$
が成り立っているとき,$\gen$ を $m$-accretive と呼ぶ.
このとき $\Re \lm<0$ となる任意の $\lm$ に対して $\lm\in\rho(A)$ で
\bdm %----------------------------------------------------------------------
\|(A-\lm)^{-1}\| \le |\Re \lm|^{-1}
\edm %----------------------------------------------------------------------
が成り立っている.
$A$ の定義域が稠密であることは $m$-accretive の条件から必然的に従う.
実際,$((A-\lm)^{-1}u,v)$, $\forall u\in H$ が成り立てば
$u-v$, $w=(A-\lm)^{-1}v$ として
\bdm %----------------------------------------------------------------------
0
= \Re((A-\lm)^{-1}v,v)
= \Re(w,(A-\lm)w)
\ge -\Re\lm |w|^2
\edm %----------------------------------------------------------------------
から $w=0$ が従う.

さて $A$ が accretive のとき
$\Theta(A) \subseteq S_\theta$, $\theta\in[0,\frac{\pi}{2})$
であるとき,扇形条件が成り立つ (sectorial) という.
ここで $S_\theta$ は次で定まる角領域である:
$S_\theta = \{z\in\C;\, |\arg z| \le \theta \}$.
これらの概念は sesquilinear form のときに既に出てきている.
実際 sesquilinear form $a$ を
\bdn %----------------------------------------------------------------------
a(u,v)
= (Au,v)
\Eqn{OHS.8}
\edn %----------------------------------------------------------------------
で定めればよい.
定義域は $\Dom(A)\times \Dom(A)$ とする.
$a$ の実部,虚部を $b$, $c$ とすれば
\bdmn %---------------------------------------------------------------------
b(u,v)
&= \frac{1}{2}\{ (Au,v) + (u,Av)\}
\Eqn{OHS.10} \\
c(u,v)
&= \frac{1}{2i}\{ (Au,v) - (u,Av)\}
\Eqn{OHS.12}
\edmn %---------------------------------------------------------------------
\Prop{SFH-6} から扇形条件は次の条件と同値である:
ある定数 $K$ が存在して
\bdn %----------------------------------------------------------------------
|(Au,v)|
\le K (\Re(Au,u))^{1/2} (\Re(Av,v))^{1/2}.
\Eqn{OHS.16}
\edn %----------------------------------------------------------------------

$T$ を閉作用素とする.
ある実数 $\gm$ が存在して $T+\gm$ が accretive のとき
 $T$ を quasi-accretive という.
また $T+\gm$ が扇形条件をみたすとき,quasi-sectorial という.
さらに $T+\gm$ が $m$-accretive のとき quasi-$m$-accretive といい,
これに加えて扇形条件をみたすとき quasi-$m$-sectorial であるという.


\bigskip
今後 Hilbert 空間上の作用素のスペクトルのことを問題にしていくわけだが,
そのためには複素Hilbert 空間で調べることが自然である.
そこで,実 Hilbert 空間の複素化の話をまずまとめておく.

$H$ を実Hilbert 空間とする.
複素化を $H\oplus i H$ で定める.
$H\oplus i H$ の元は $u+iv$ ($u$,$v\in H$) と表される.
スカラー倍は $x+iy\in\C$ に対し
\bdn %----------------------------------------------------------------------
(x+iy)(u+iv)
= (x u -y v) + i(x v + y u) 
\Eqn{OHS.20}
\edn %----------------------------------------------------------------------
で定め,内積は
\bdn %----------------------------------------------------------------------
(u+iv, r+is)
=(u,r) + (v,s) + i\{(v,r) - (u,s)\}
\Eqn{OHS.22}
\edn %----------------------------------------------------------------------
で定める.
また $H\oplus i H$ には自然な共役作用素 $\J$ が次で定義される.
\bdn %----------------------------------------------------------------------
\J(u+iv)
= u-iv.
\Eqn{OHS.24}
\edn %----------------------------------------------------------------------
$\J$ は次の関係をみたす.
\bdmn %---------------------------------------------------------------------
\J^2(f)
&= f,
\Eqn{OHS.26} \\
\J(\al f + \be g)
&= \ol{\al} \J(f) + \ol{\be}\J(g).
\Eqn{OHS.28}
\edmn %---------------------------------------------------------------------
複素数に対しては $\ol{\phantom{f}}$ は 共役複素数を表す.
従って $\J$ は共役線型であることに注意しよう.
また内積については
\bdn %----------------------------------------------------------------------
(\J(f), \J(g))
= \ol{(f,g)}
\Eqn{OHS.30}
\edn %----------------------------------------------------------------------
が成り立っている.
以後 $\J(f)$ を $\ol{f}$ ともかく.

逆に複素 Hilbert 空間 $H$ に,\Eq{OHS.26}, \Eq{OHS.28}, \Eq{OHS.30}
をみたす共役作用素 $\J$ が与えられているとき,
\bdm %----------------------------------------------------------------------
H_r
= \{u\in H;\, \J(u)= u\}
\edm %----------------------------------------------------------------------
で定めると,$H$ は $H_r$ の複素化になる.
$H$ の作用素 $A$ が
\bdm %----------------------------------------------------------------------
A\J
= \J A
\edm %----------------------------------------------------------------------
をみたすとき実作用素と呼ぶ.
$A$ が実作用素のとき,$A$ は $H_r$ の作用素になる.
もう少し正確に言うと $f\in \Dom(A)\cap H_r$ ならば $Af\in H_r$ で,
さらに $\Dom(A) = (\Dom(A)\cap H_r)\oplus i(\Dom(A)\cap H_r)$ が成り立つ.
逆に実 Hilbert 空間 $H_r$ の作用素は,複素化すると実作用素になる.

さて $A$ を実作用素とする.
$\J$ の定義から $\J (A-\zeta)\J=A-\ol{\zeta}$ である.
これから $\Ker(A-\zeta)=\{0\}$ $\Leftrightarrow$ $\Ker(A-\ol{\zeta})=\{0\}$
であり $\Ran(A-\zeta)=H$ $\Leftrightarrow$ $\Ran(A-\ol{\zeta})=H$ である.
これから $(A-\zeta)^{-1}$ が $H$ 全体で定義された有界作用素になるときには
$(A-\ol{\zeta})^{-1}$ も $H$ 全体で定義された有界作用素になる.
従ってレゾルベント集合は共役をとる操作で不変になる.
スペクトル集合も同じ性質を持つことが分かる.

実の Hilbert 空間で双一次形式 $a$ が与えられたとき,
その複素化について考えよう.
$a$ は自然に複素化された Hilbert 空間で sesquilinear form に拡張できる.
従って $a$ は第2変数について共役線型だから
\bdm %----------------------------------------------------------------------
a(u+iv,u+iv)
&= a(u,u)+a(v,v) -i a(u,v) + ia(v,u) \\
&= a(u,u)+a(v,v) + i \{a(v,u) - a(u,v)\}
\edm %----------------------------------------------------------------------
が成り立つ.
従って $a$ の実部,虚部を $b$, $c$ とすると
\bdm %----------------------------------------------------------------------
b(u+iv) = a(u,u)+a(v,v), \quad
c(u+iv) = a(v,u)-a(u,v)
\edm %----------------------------------------------------------------------
これから扇形条件 $|c| \le \tan\theta b$ は
$|a(v,u)-a(u,v)| \le \tan\theta\{a(u,u)+a(v,v)\}$ となる.
$u$ の代わりに $\e u$, $v$ の代わりに $\e^{-1} v$ をとって
\bdm %----------------------------------------------------------------------
|a(v,u)-a(u,v)| \le \tan\theta\{\e^2 a(u,u)+\e^{-2} a(v,v)\}
\edm %----------------------------------------------------------------------
特に $\e^2= \frac{a(v,v)^{1/2}}{a(u,u)^{1/2}}$ をとれば
\bdn %----------------------------------------------------------------------
|a(v,u)-a(u,v)| \le 2 \tan\theta a(u,u)^{1/2} a(v,v)^{1/2}
\Eqn{OHS.32}
\edn %----------------------------------------------------------------------
が成り立つ.
逆にこの関係から
\bdm %----------------------------------------------------------------------
|a(v,u)-a(u,v)| \le 2 \tan\theta a(u,u)^{1/2} a(v,v)^{1/2}
\le  \tan\theta \{a(u,u) + a(v,v)\}
\edm %----------------------------------------------------------------------
が成り立つ.
従って扇形条件は上の式と同値になる.

\subsec{Notes}
\begin{itemize}
\item accretive な作用素についてのまとめ.
\item 実 Hilbert 空間の複素化と実作用素.
\end{itemize}



