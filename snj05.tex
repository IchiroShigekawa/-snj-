%%%%%%%%%%%%%%%%%%%%%%%%%%%%%%%%%%%%%%%%%%%%%%%%%
%                                               %
%         ====== Program  No.5 =======          %
%                                               %
%             file name snj05.tex               %
%                                               %
%===============================================%
%===================  for  =====================%
%===============================================%
%
\hide
\vspace{-4mm}
\begin{itemize} \itemsep=-2mm \parsep=0mm
\item Total file name: snj01 snj02 $\dots $ snj?, snj\_bibliography
\item File name: snj05.tex \hfill 印刷日: \today \ \now
\item 生成作用素が正規であるための条件を求める.
[2009年8月19日]
\item この節は 初め lnj にあったのを移動させてきた.
Riemannian manifold の上の作用素も一般的な形で論じてある.
単に趣味的なだけではあるが.
[2011年1月5日]
\end{itemize}
\endhide
\SS{CNO}{正規作用素であるための条件} %======================================
% Condition for normal operator
生成作用素が正規であるための条件を求めよう.
まず一般的な枠組みを準備し,その後で Riemannian manifold の場合を考える.

\Theorem{CNO-1} %**********************************************************
$H$ を Hilbert 空間,$A$, $B$ を $\sD$ を定義域とする消散作用素,
$\ol{A}$, $\ol{B}$ をその閉方とし,
$\ol{A}$, $\ol{B}$ が $m$-dissipative であることを仮定する.
さらに $A\sD\subseteq \sD$, $B\sD\subseteq \sD$ で
\bdmn %---------------------------------------------------------------------
AB &= BA \quad \text{on $\sD$}
\Eqn{CNO.2} \\
(Au,v) &= (u,Bv), \quad  u,v\in\sD
\Eqn{CNO.4}
\edmn %---------------------------------------------------------------------
が成り立っているとする.
このとき $\ol{A}$ は正規作用素で $\ol{A}^*=\ol{B}$ である.
\end{theorem} %*************************************************************

\Proof
条件 \Eq{CNO.2}, \Eq{CNO.4} から $u$, $v\in\sD$ のとき
\bdn %----------------------------------------------------------------------
(Au, Av)
= (u,BAv)
= (u,ABv)
= (Bu,Bv)
\Eqn{CNO.6}
\edn %----------------------------------------------------------------------
が成り立つから $|Au|=|Bu|$ が任意の $u\in\sD$ に対して成立する.
従って $u\in\Dom(\ol{A})$ とすると $\{u_n\}\subseteq \sD$ で $u_n\to u$,
$\{Au_n\}$ は Cauchy 列となるものが存在する.
従って $Bu_n$ も Cauchy 列となり,$\Dom(\ol{A})=\Dom(\ol{B})$ が成り立つ.
$\sE=\Dom(\ol{A})$ とおく.
さらに \Eq{CNO.4} で極限をとることにより
\bdm %----------------------------------------------------------------------
(\ol{A} u, v)
= (u,\ol{B}v), \quad u,v\in \sE
\edm %----------------------------------------------------------------------
が成り立つ.
従って $\ol{A}^* \supseteq \ol{B}$ となる.
ところで $\ol{A}$ は $m$-dissipative だから $\ol{A}^*$ も dissipative
になる.
$\ol{B}$ が $m$-dissipative
であるから極大性から $\ol{A}^* = \ol{B}$ となる.
\hide
ここで $\ol{A}$ が $m$- dissipative であることも使っている.
実際 $\ol{A}$ が $m$-dissipative だと 
$\ol{A}^*$ も dissipative になることが田辺の定理 2.1.3 に書いてある.
このことがなければ $\ol{A}^*$ が dissipative かどうか分からなくなるので,
$\ol{A}^* = \ol{B}$ を出せなくなる.
$m$-dissipative の仮定は片方だけでよいと思ったが,やはり両方必要である.
\hfill [2011年1月8日]
\endhide

一方上の  \Eq{CNO.6} で極限をとることにより
\bdm %----------------------------------------------------------------------
(\ol{A} u, \ol{A}v)
= (\ol{B}u,\ol{B}v), \quad u,v\in \sE
\edm %----------------------------------------------------------------------
が成り立つ.
さて $u\in\Dom(\ol{A}^*\, \ol{A})$ をとる.
つまり $u\in \Dom(\ol{A})$ かつ $\ol{A}u\in\Dom(\ol{A}^*)=\sE$
が成り立っているとする.
すると上の式から
\bdm %----------------------------------------------------------------------
(\ol{A}^*\, \ol{A} u, v)
= (\ol{A}u, \ol{A}v)
= (\ol{B}u,\ol{B}v), \quad g\in \sE
\edm %----------------------------------------------------------------------
が成り立つ.
これは $\ol{B}u\in \Dom(\ol{B}^*)$ で
 $\ol{B}^*\,\ol{B}u=\ol{A}^*\, \ol{A}u$ を意味する.
同様に $u\in\Dom(\ol{B}^*\, \ol{B})$ をとると $\ol{B}u\in \Dom(\ol{A}^*)$
で  $\ol{A}^*\,\ol{A}u=\ol{B}^* \,\ol{B}u$ が成り立っていることを意味する.
従って $\ol{A}^*\,\ol{A} = \ol{B}^*\, \ol{B}$ であるが,
$\ol{A}^*=\ol{B}$ であったから $\ol{A}^* \,\ol{A} = \ol{A}\, \ol{A}^*$
が成立する.
これは即ち $\ol{A}$ が正規作用素であることを意味している.
\QED %======================================================================

この結果を使って Riemannian manifold の場合を考えよう.
$M$ を Riemannian manifold として,完備性を常に仮定しておく.
作用素の正規性を言うには可換性を調べる必要があるが,
そのためには Killing vector field
の概念が必要になるので,まずそのことの準備をする.
ベクトル場 $X$ が Killing field であることは $L_X g = 0$ が
成り立つことであった.
このとき $X$ は等距離変換群を定義する.
$L_X$ は Lie 微分を表す.

\Proposition{CNO-2} %*******************************************************
$X$ が Killing vector field であるための必要十分条件は
$v\mapsto \nabla_v X$ が歪対称であること,即ち
\bdn %----------------------------------------------------------------------
g(\nabla_v X, w) + g(\nabla_w X, v)
= 0, \quad \forall v,\, w
\Eqn{CNO.12}
\edn %----------------------------------------------------------------------
が成り立つことである.
\end{proposition} %*********************************************************

\Proof
$\xi=X^\flat$ とするとき,$V$, $W$ を任意のベクトル場として
\bdn %----------------------------------------------------------------------
d\xi(V,W) + (L_X g)(V, W)
= 2g(\nabla_V X, W)
\Eqn{CNO.14}
\edn %----------------------------------------------------------------------
が成り立っている.
これは \cite{Petersen06} の Chapter~2, \S 1 に出ている.
\memo{むしろ Petersen ではこれを定義にして議論を進めている.}
これから $L_Xg=0$ ならば $(V,W) \mapsto g(\nabla_V X, W)$ が歪対称になる.
逆に $(V,W) \mapsto g(\nabla_V X, W)$ が歪対称ならば,$L_X g$ が
歪対称になるが,$L_X g$ は対称でもあるので $L_Xg=0$ となる.
\QED %======================================================================

\Theorem{CNO-6} %***********************************************************
$X$ が Killing vector fileld とする.
$\xi=X^\flat$ とおくと次が成り立つ:
\bdmn %---------------------------------------------------------------------
\nabla^* \nabla \xi
&= \Ric(\xi),
\Eqn{CNO.18} \\
\nabla^* \xi
&= 0.
\Eqn{CNO.20}
\edmn %----------------------------------------------------------------------
\end{theorem} %*************************************************************

\Proof
\Eq{CNO.14} の関係式から $d\xi= 2\nabla\xi$ である.
一方
\bdm %----------------------------------------------------------------------
\nabla^* \xi
= - \tr \nabla\xi
= -\frac{1}{2} \tr d\xi
\edm %----------------------------------------------------------------------
で,$d\xi$ は skew symmetric だから $\nabla^* \xi=0$ が成り立つ.
さらに,一般に微分形式に対して $d^*\w = \nabla^* \w$ が成り立つので
\bdm %----------------------------------------------------------------------
2 \nabla^* \nabla \xi
&= 2 d^* \nabla \xi \\
&= d^* d \xi \quad (\because\ 2\nabla\xi=d\xi) \\
&= (d^* d + d d^*) \xi \quad (\because\ d^*\xi=0) \\
&= (\nabla^*\nabla + \Ric) \theta.
\edm %----------------------------------------------------------------------
よって $\nabla^* \nabla \xi = \Ric(\theta)$ が成り立つ.
\QED %======================================================================

多様体がコンパクトだと上の逆が成立する.
これを示すには更に準備が必要になる.
次の結果は Petersen \cite{Petersen06} のp.~230 に述べてある.
証明がないのでつけておく.

\Proposition{CNO-8} %*******************************************************
$M$ がコンパクトな多様体であるとき,ベクトル場 $X$ に対して
次の等式が成立する.
\bdn %----------------------------------------------------------------------
\int_M (\Ric(X),X) + \tr\{(\nabla X)^2\} - (\div X)^2) d\m
= 0.
\Eqn{CNO.22}
\edn %----------------------------------------------------------------------
ここで $\nabla X$ は $\Hom(T(M))$ の元とみている.
$\tr$ もその意味である.
\end{proposition} %*********************************************************

\Proof
いくつか予備的な等式が必要である.
まず $\div \nabla_X X$ を計算する.
\bdm %----------------------------------------------------------------------
\div(\nabla_X X)
&= \div(X^i \nabla_i X) \\
&= - \la \nabla_j(X^i \nabla_i X), dx^j\ra \\
&= - \la \nabla_j X^i \nabla_i X, dx^j \ra 
   - \la X^i  \nabla_j\nabla_i X, dx^j\ra \\
&= - \nabla_j X^i \la \nabla_i X, dx^j \ra 
   - X^i \la \nabla_j\nabla_i X, dx^j\ra \\
&= - \nabla_j X^i \la \nabla_i X, dx^j \ra 
   - X^i \la (\nabla_j\nabla_i - \nabla_i\nabla_j) X, dx^j\ra
   - X^i \la \nabla_i\nabla_j X, dx^j\ra \\
&= - \nabla_j X^i \la \nabla_i X, dx^j \ra 
   - X^i \la R(\del_j,\del_i)X, dx^j\ra
   - X^i \la \nabla_i\nabla_j X, dx^j\ra \\
&= - \nabla_j X^i \la \nabla_i X, dx^j \ra 
   - \Ric(X,X) - X^i \la \nabla_i\nabla_j X, dx^j\ra.
\edm %----------------------------------------------------------------------
次に $\div((\div X)X)$ を計算する.
\bdm %----------------------------------------------------------------------
\div((\div X)X)
&= - \la \nabla_i((\div X)X), dx^i\ra \\
&= \la \nabla_i((\la \nabla_j X, dx^j\ra X), dx^i\ra \\
&= \la \del_i \la \nabla_j X, dx^j\ra X, dx^i\ra
   + \la \la \nabla_j X, dx^j\ra \nabla_i X, dx^i\ra \\
&= \la (\la \nabla_i \nabla_j X
         + \la \nabla_j X,\nabla_i dx^j \ra) X, dx^i\ra
   + \la \nabla_j X, dx^j\ra \la \nabla_i X, dx^i\ra \\
&= X^i \la \nabla_i \nabla_j X dx^j \ra
         + X^i \la \nabla_j X,\nabla_i dx^j \ra + (\div X)^2.
\edm %----------------------------------------------------------------------
ここで
\bdm %----------------------------------------------------------------------
\nabla_j X^i = \la \nabla_j X,dx^i\ra + \la X, \nabla_j dx^i\ra
\edm %----------------------------------------------------------------------
であることに注意して
\bdm %----------------------------------------------------------------------
\div(\nabla_X X) + \div((\div X)X)
&= -(\la \nabla_j X, dx^i\ra
   + \la X, \nabla_j dx^i\ra) \la\nabla_i X,dx^j\ra
   - \Ric(X,X) \\
&\squad
   + X^i \la \nabla_j X,\nabla_i dx^j \ra + (\div X)^2 \\
&= - \la \nabla_j X, dx^i\ra \la\nabla_i X,dx^j\ra - \Ric(X,X) \\
&\squad
   + (\div X)^2 - \la X, \nabla_j dx^i\ra \la \nabla_i X,dx^j\ra
   + X^i \la \nabla_j X, \nabla_i dx^j\ra.
\edm %----------------------------------------------------------------------
ここで上の最後の2項が $0$ になることを示せば,求める結果に成る.
最後の2項は
\bdm %----------------------------------------------------------------------
- \la X, \nabla_j dx^i\ra \la \nabla_i X,dx^j\ra
  + X^i \la \nabla_j X, \nabla_i dx^j\ra
&= \la X, \Gm_{jk}^i dx^k\ra \la \nabla_i X, dx^j\ra
  - X^i\la\nabla_j X, \Gm_{ik}^j dx^k\ra \\
&= X^k \Gm_{jk}^i \la \nabla_i X, dx^j\ra
  - X^i \Gm_{ik}^j \la\nabla_j X,  dx^k\ra \\
&=0. 
\edm %----------------------------------------------------------------------
これで証明できた.
\QED %======================================================================
\hide
$\div((\div X)X)$ の方は $\nabla^*(\nabla^*\xi,\xi))$ であることに注意して
\bdm %----------------------------------------------------------------------
\nabla^*(\nabla^*\xi,\xi))
&= - \la \nabla\nabla^* \xi,\xi\ra + (\nabla^*\xi)^2 \\
&= - \la dd^* \xi,\xi\ra + (\nabla^*\xi)^2 \\
&= - \la (dd^* + d^*d)\xi,\xi\ra + \la d^*d\xi,\xi\ra
   + (\nabla^*\xi)^2
\edm %----------------------------------------------------------------------
と表現できる.
他方も同じような考えで変形できないか.
そうすれば計算が簡単になるのではないか.
\endhide

\hide
Kobayashi \cite{Kobayashi95}(こちらは \cite{Kobayashi72} のreprint版である) の p.~155 に対応する証明が載っているが,follow 出来ない.
間違っているように思うのだが.
Yono-Bochner \cite{YB53} p.~57 にはテンソル計算できちんと示してある.
\endhide

$X \in \Gm(T(M))$  に対して $\nabla X\in \Gm(\Hom(T(M),T(M))$
と見ているわけであるが,この転置を ${}^t\nabla X\in \Gm(\Hom(T^*(M),T^*(M))$
と表すことにする.
$\nabla X$ と ${}^t\nabla X$ の合成を考えたいが,
このままでは意味を成さないので,同型
$\sharp\colon T^*(M) \to T(M)$,
$\flat\colon T(M) \to T^*(M)$ を用いて $T(M)$ と $T^*(M)$ を同一視して
${}^t\nabla X$ を $\Gm(\Hom(T(M),T(M))$ のように考える.
より正確に言えば,${}^t\nabla X$ の代わりに,次の合成
\bdm %----------------------------------------------------------------------
\begin{CD}
T(M) @>\flat>> T^*(M) @>{{}^t\nabla X}>> T^*(M) @>\sharp>> T(M)
\end{CD}
\edm %----------------------------------------------------------------------
$\sharp\maru {}^t\nabla X \maru \flat$ を考えることである.
煩瑣ではあるが,数学的に厳密な $\sharp\maru {}^t\nabla X \maru \flat$ 
の方を用いることにして,${}^t\nabla X$ は本来の $\Gm(\Hom(T^*(M),T^*(M))$
の元を表すものとする.
basis として $\del_i$, $dx^j$ を取って成分表示すると
\bdm %----------------------------------------------------------------------
\flat_{ij} = g_{ij},\quad
\sharp^{ij} = g^{ij},\quad
(\nabla X)_i^j = \la \nabla_i X, dx^j\ra
\edm %----------------------------------------------------------------------
であるから
\bdm %----------------------------------------------------------------------
(\sharp\maru {}^t\nabla X\maru \flat)_i^l
= g_{ij} (\nabla X)_k^j g^{kl}
\edm %----------------------------------------------------------------------
という成分表示が得られる.

\Proposition{CNO-10} %******************************************************
ベクトル場 $X$, $Y$ に対して次が成立する:
\bdn %----------------------------------------------------------------------
(\nabla X, \nabla Y)
= \tr(\nabla X\maru \sharp \maru {}^t\nabla Y \maru \flat)
\Eqn{CNO.24}
\edn %----------------------------------------------------------------------
\end{proposition} %*********************************************************

\Proof
$(\nabla X)_i^j = \la \nabla_i X,dx^j\ra$ であったから
\bdm %----------------------------------------------------------------------
(\nabla X, \nabla Y)
&= ((\nabla X)_i^j dx^i \otimes \del_j,
    (\nabla Y)_k^l dx^k \otimes \del_l) \\
&= (\nabla X)_i^j (\nabla Y)_k^l g^{ik} g_{jl} \\
&= (\nabla X)_i^j (\sharp \maru \nabla Y \maru \flat)_j^i \\
&= \tr( \nabla X \maru \sharp \maru \nabla Y \maru \flat).
\edm %----------------------------------------------------------------------
これが示すべきことである.
\QED %======================================================================

以上で逆を示す準備が出来た.

\Theorem{CNO-12} %**********************************************************
$M$ をコンパクトな Riemannian manifold で,ベクトル場 $X$ に対して
$\xi=X^\flat$ とおくとき次が成り立っているとする.
\bdmn %---------------------------------------------------------------------
\nabla^* \nabla X
&= \Ric(X), 
\Eqn{CNO.30} \\
\div X
&= 0.
\Eqn{CNO.32}
\edmn %----------------------------------------------------------------------
このとき $X$ は Killing vector field になる.
\end{theorem} %*************************************************************

\Proof
\Prop{CNO-8} から
\bdm %----------------------------------------------------------------------
\int_M (\Ric(X),X) + \tr\{(\nabla X)^2\} - (\div X)^2) d\m
= 0
\edm %----------------------------------------------------------------------
が成り立つ.
ここで \Eq{CNO.30} を使うと
\bdm %----------------------------------------------------------------------
0
&= \int_M (\nabla^*\nabla X, X) + \tr\{(\nabla X)^2\} \,d\m \\
&= \int_M (\nabla X, \nabla^X) + \tr\{(\nabla X)^2\} \,d\m.
\edm %----------------------------------------------------------------------
さらに \Eq{CNO.32} を使えば
\bdm %----------------------------------------------------------------------
0
&= \int_M [\tr( \nabla X\maru \sharp\maru {}^t\nabla X\maru\flat
           + \tr\{(\nabla X)^2\}] \,d\m \\
&= \frac{1}{2} \int_M
   \tr\{ (\nabla X + \sharp\maru {}^t\nabla X\maru\flat)^2\}\,d\m.
\edm %----------------------------------------------------------------------
ここで $\nabla X + \sharp\maru {}^t\nabla X\maru\flat$ が
対称であることに注意しよう.
実際
\bdm %----------------------------------------------------------------------
g(\sharp\maru {}^t\nabla X\maru\flat(v), w)
= \la  {}^t\nabla X\maru\flat(v), w\ra
= \la  \flat(v), \nabla_w X \ra
=g(v,\nabla_w X)
\edm %----------------------------------------------------------------------
から対称であることが分かる.
従って2乗すれば非定値となり,積分して $0$ になることが上で示せているので
$\tr\{ (\nabla X + \sharp\maru {}^t\nabla X\maru\flat)^2\}=0$,
さらには
\bdm %----------------------------------------------------------------------
\nabla X + \sharp\maru {}^t\nabla X\maru\flat = 0
\edm %----------------------------------------------------------------------
が示せる.
これから上の対称性と合わせて
$(v,w) \mapsto g(\nabla_v X,w)$ が skew-symmetric
であることが従う.
よって \Prop{CNO-2} から $X$ は Killing vector field であることが分かる.
\QED %======================================================================

$M$ がコンパクトでなければこの定理は成り立たない.
$\R^2$ の場合に簡単に見ておこう.
$\R^2$ の座標を $(x,y)$ とかく.
$\C=\R^2$ とみて, $\C$ で正則な関数 $f$ をとり,$f=u+iv$ と
実部と虚部に分解しておく.
Cauchy-Riemann の関係式から $u_x=v_y$, $u_y=-v_x$ が成り立つ.
そこで ベクトル場 $X$ を $X= u\frac{\del}{\del x}- v\frac{\del}{\del y}$
と定めると,$\div X=0$ はすぐに分かる.
また
$\nabla^*\nabla X
= - \Laplace u\frac{\del}{\del x} + \Laplace v\frac{\del}{\del y}=0$
もすぐに分かる.
これで \Eq{CNO.30}, \Eq{CNO.32} を満たすが,Killing vector field
でないものが容易に作れる.

\bigskip
さて,正規作用素の話に戻ろう.
この場合に必要となるのは \Prop{CNO-2} と \Thm{CNO-6} だけである.
\memo{その意味では,やや余分の話をしたことになる.}

% \tb を \tilde{b} としていたがここでは特に tilde をつける意味はないから
% はずしておく.別の文脈ではつけたほうが自然であろうから,そこでも使える
% ようにしておく.
\renewcommand{\tb}{b}


\bigskip
$M$ を完備なリーマン多様体とする.
$\m$ を Riemannian volume として,測度 $\nu=e^{-U}\m$ のもとで
 $\gen = -\frac{1}{2} \nabla_\nu^* \nabla + \tb$
という作用素を $L^2(\nu)$ で考える.
$\nabla_\nu^*$ は,測度 $\nu$ に対する $\nabla$ の共役で
Riemannian volume に対する共役 $\nabla^*$ を用いて
$\nabla_\nu^* = \nabla^* + (\nabla U, \cdot)$ と表される.
この作用素が正規である条件を求めたいわけである.

\hide
$\div_\nu\tb=0$ を始めは仮定していた.
それで以下ではこの条件をつけないでやってみたが,条件が綺麗にならない.
$\div_\nu\tb=0$ が出てくると思ったのだが.
この条件は $\gen^*(e^{-U}) = 0$ と同値であった.
これと正規性と関連付けるとかできないのだろうか.
とにかく今のところうまく行っていない.
\hfill [2011年1月6日]
\endhide
$L^2(\nu)$ での(形式的な)共役作用素は
$\gen_\nu^*= -\frac{1}{2} \nabla_\nu^* \nabla - \tb -\div_\nu\tb$ となる.
$\gen_\nu^*$ と $\nu$ を添え字につけたのは
 $\nu$ に関する共役という意味である.
$\div_\nu$ も $\nu$ に対する共役という意味である.
$\nu$ をつけないときは Riemannian volume $\m$ に対する共役を表すものとする.
\bdn %----------------------------------------------------------------------
\Laplace_\nu
= - \nabla_\nu^*\nabla
= - \nabla^*\nabla - \nabla U \cdot \nabla
\Eqn{CNO.66}
\edn %----------------------------------------------------------------------
とおくと,
\bdmn %---------------------------------------------------------------------
\gen
&= \frac{1}{2} \Laplace_\nu + \nabla_\tb,
\Eqn{CNO.68} \\
\gen_\nu^*
&= \frac{1}{2} \Laplace_\nu - \nabla_\tb - \div_\nu\tb.
\Eqn{CNO.70}
\edmn %---------------------------------------------------------------------
であるから,
\bdm %----------------------------------------------------------------------
\gen_\nu^* \gen - \gen \gen_\nu^* 
&= (\frac{1}{2} \Laplace_\nu - \nabla_\tb -\div_\nu\tb)
  (\frac{1}{2} \Laplace_\nu + \nabla_\tb)
  -(\frac{1}{2} \Laplace_\nu + \nabla_\tb)
   (\frac{1}{2} \Laplace_\nu - \nabla_\tb -\div_\nu\tb) \\
&= \Laplace_\nu \nabla_\tb - \nabla_\tb \Laplace_\nu
   + [\frac{1}{2}\Laplace_\nu + \nabla_\tb, \div_\nu\tb] \\
&= [\Laplace_\nu, \nabla_\tb]
   + [\frac{1}{2}\Laplace_\nu+\nabla_\tb, \div_\nu\tb].
\edm %----------------------------------------------------------------------
従って可換である条件は
$[\Laplace_\nu,\nabla_\tb]+[\frac{1}{2}\Laplace_\nu+\nabla_\tb, \div_\nu\tb]=0$
である.
このための条件を求めればよい.
\Theorem{CNO-10} %**********************************************************
$\gen$ と $\gen_\nu^*$ が可換であるための必要十分条件は $\tb$ が
Killing vector field であり,次の等式が成立することである.
\bdmn %---------------------------------------------------------------------
(\frac{1}{2}\Laplace_\nu + \nabla_\tb)\div_\nu \tb
= 0,
\Eqn{CNO.74} \\
[(\nabla U)^\sharp,\tb] + \nabla\div_\nu\tb
= 0.
\Eqn{CNO.76}
\edmn %---------------------------------------------------------------------
\end{theorem} %*************************************************************

\Proof
$\tb$ に対応する 1-form を $\tw$ とする.

まず一般のテンソル $\theta$, $\eta$ に対し
\bdm %----------------------------------------------------------------------
\Laplace(\theta,\eta)
= 2(\nabla \theta, \nabla \eta) - (\nabla^*\nabla\theta, \eta)
-(\theta, \nabla^*\nabla\eta)
\edm %----------------------------------------------------------------------
が成り立つことに注意しよう
(例えば,\cite{Petersen06} の Chapter 7, \S 3 か同じ章の \S 7, Exercise 13
を見よ.)
 特に 1-form の場合,$\theta=\tw$, $\eta=\nabla f$ として
\bdm %----------------------------------------------------------------------
\Laplace(\tw, \nabla f)
= 2(\nabla \tw, \nabla^2 f) - (\nabla^*\nabla\tw, \nabla f)
-(\tw, \nabla^*\nabla\nabla f).
\edm %----------------------------------------------------------------------
ここで 1-form に対して $dd^*+d^*d = \nabla^*\nabla + \Ric$ であることと
\bdm %----------------------------------------------------------------------
(dd^*+d^*d)\nabla f = \nabla (dd^*+d^*d)f = \nabla \nabla^*\nabla
\edm %----------------------------------------------------------------------
を使うと
\bdm %----------------------------------------------------------------------
\Laplace(\tw, \nabla f)
&= 2(\nabla \tw, \nabla^2 f) - (\nabla^*\nabla\tw, \nabla f)
  -(\tw, (dd^*+d^*d)\nabla f) +(\tw, \Ric(\nabla f)) \\
&= 2(\nabla \tw, \nabla^2 f) - (\nabla^*\nabla\tw, \nabla f)
  -(\tw, \nabla \nabla^*\nabla f) +(\tw, \Ric(\nabla f))
\edm %----------------------------------------------------------------------
$(\tw, \nabla f) = \nabla_\tb f$ であるから
\bdm %----------------------------------------------------------------------
[\Laplace, \nabla_\tb]f
= 2(\nabla \tw, \nabla^2 f) - (\nabla^*\nabla\tw - \Ric(\tw), \nabla f).
\edm %----------------------------------------------------------------------
$\Laplace_\nu = \Laplace + \nabla U\cdot\nabla$ だから
\bdm %----------------------------------------------------------------------
[\Laplace_\nu, \nabla_\tb]f
&= [\Laplace + \nabla U\cdot\nabla, \nabla_\tb]f \\
&= [\Laplace, \nabla_\tb]f
  + \nabla_{[\nabla U^{\sharp}, \tb]}f \\
&= 2 (\nabla\tw, \nabla^2 f)
  + (-\nabla^*\nabla\tw + \Ric(\tw) + [\nabla U^\sharp,\tb]^\flat, \nabla f)
\edm %----------------------------------------------------------------------
また $[\frac{1}{2}\Laplace_\nu+\nabla_\tb, \div_\nu\tb]$ の方は
\bdm %----------------------------------------------------------------------
[\frac{1}{2}\Laplace_\nu+\nabla_\tb, \div_\nu\tb]f
&= (\frac{1}{2}\Laplace_\nu+\nabla_\tb)(\div_\nu\tb f)
  - \div_\nu\tb(\frac{1}{2}\Laplace_\nu+\nabla_\tb)f \\
&= \frac{1}{2}\{ (\Laplace_\nu\div_\nu\tb)f+2\nabla\div_\nu\tb\cdot\nabla f
   + \div_\nu \tb\Laplace_\nu f\} \\
&\squad
   + (\nabla_\tb\div_\nu \tb)f + \div_\nu \tb \nabla_\tb f)
   - \div_\nu\tb(\frac{1}{2}\Laplace_\nu+\nabla_\tb)f \\
&= (\frac{1}{2}\Laplace_\nu\div_\nu\tb + \nabla_\tb\div_\nu \tb)f
   + \nabla\div_\nu\tb\cdot\nabla f.
\edm %----------------------------------------------------------------------
両者をあわせて
\bdm %----------------------------------------------------------------------
[\Laplace_\nu, \nabla_\tb]f
 + [\frac{1}{2}\Laplace_\nu+\nabla_\tb, \div_\nu\tb]f
&= 2 (\nabla\tw, \nabla^2 f) \\
&\squad
  + (-\nabla^*\nabla\tw + \Ric(\tw) + [\nabla U^\sharp,\tb]^\flat
      + \nabla\div_\nu\tb, \nabla f) \\
&\squad
  + (\frac{1}{2}\Laplace_\nu\div_\nu\tb + \nabla_\tb\div_\nu \tb)f.
\edm %----------------------------------------------------------------------
これが全ての $f$ に対して 0 であればよいことが,可換性の必要十分条件である.
$f=1$ として 
\bdn %----------------------------------------------------------------------
(\frac{1}{2}\Laplace_\nu + \nabla_\tb)\div_\nu \tb
= 0
\Eqn{CNO.80}
\edn %----------------------------------------------------------------------
が従う.
さらに,点 $x$ をとめれば,$\nabla f(x)=0$ で $\nabla^2 f$ が
任意の対称行列を取るように出来る.
従って可換性の条件は \Eq{CNO.70} と
\bdmn %---------------------------------------------------------------------
&\widehat{\nabla\tw}= 0,
\Eqn{CNO.82} \\
&-\nabla^*\nabla\tw + \Ric(\tw) + [\nabla U^\sharp,\tb]^\flat
 + \nabla\div_\nu\tb= 0
\Eqn{CNO.84}
\edmn %---------------------------------------------------------------------
である.
ここで $\widehat{\nabla\tw}$ は $\nabla\tw$ の対称部分を表す.
まず $\widehat{\nabla\tw}= 0$ は $\nabla\tw$ が skew symmetric
であることを意味している.
従って \Prop{CNO-2} から $\tb$ は Killing field である.
ここで Killing field の性質 \Eq{CNO.18} と条件 \Eq{CNO.84} を使えば
\bdn %----------------------------------------------------------------------
[(\nabla U)^\sharp,\tb] + \nabla\div_\nu\tb
= 0
\Eqn{CNO.86}
\edn %----------------------------------------------------------------------
が従う.

逆に $\tb$ が Killing vector fileld であり,
\Eq{CNO.80}, \Eq{CNO.86} が成り立っているとする. 
すると,$\tb$ がKilling であることから $\nabla\tw$ がskey symmetricとなり
また \Eq{CNO.18} が成り立っている.
よって \Eq{CNO.84} が成り立っていることが分かる.
先の計算により $\gen$ と $\gen_\nu^*$ が可換になる.
\QED %======================================================================

$M$ がコンパクトのときは
$\gen 1=0$, $\gen_\nu^* 1=-\div_\nu\tb$ であるが,可換性が成り立てば 
$\|\gen 1\|_2=\|\gen_\nu^*1\|_2$ が成り立つから $\div_\nu\tb=0$ が成立する.
従って \Eq{CNO.76} の条件は $[(\nabla U)^\sharp, \tb]=0$ と少し簡単になる.
\hide
non-compact の場合も $\div_\nu\tb=0$ が出せるといいのだが.
$\nu$ が有限測度の場合も同じだと思ったが,$1\in L^2(\nu)$
は成り立つが $1\in\Dom(\gen)$ は成り立つのだろうか.
成り立つとしても $\gen 1=0$ となるのだろうか.
定義としては $C_0^\infty(M)$ の元で近似することになるから
$\gen 1=0$ も決して自明ではないことになる.
$1$ に収束する近似列を作らなければならない.
そうすると $\tb$ の増大条件を課さなければならなくなるのではないか.
結局あまり単純な条件ではなくなるような気がする.[2011年1月6日]
\endhide

non-compact の場合は必ずしも $\div_\nu\tb=0$ は成り立たない.
$M=\R$, $U=cx$, $\nu(dx) = e^{-cx}dx$ の場合を考えよう.
このとき
\bdm %----------------------------------------------------------------------
(\frac{d}{dx})_\nu^*
= -\frac{d}{dx} + c
\edm %----------------------------------------------------------------------
である.
$\gen = -\frac{1}{2}(\frac{d}{dx})_\nu^* \frac{d}{dx}+ k\frac{d}{dx}$
とする.
$b= k\frac{d}{dx}$ で $\div_\nu b = \div b - bU = -kc$
また
\bdm %----------------------------------------------------------------------
[\nabla U^{\sharp},b]
= [c\frac{d}{dx}, k\frac{d}{dx}]
= 0
\edm %----------------------------------------------------------------------
だから \Thm{CNO-10} の条件はすべてみたしている.
従ってこの作用素は(閉包をとれば)正規作用素になる.
しかし $\div_\nu b = 0$ は成立していない.
後の例でこの作用素のスペクトルを求めてみる.

\bigskip
上の定理は計算が自由に行えるところでの話である.
例えば滑らかな関数のクラス $C_0^\infty(M)$ では正当化できる.
\Thm{CNO-1} によれば,正規性を示すには $C_0^\infty(M)$ で定義された
$\gen$ の閉包が $m$-dissipative を示さなければならない.
以下ではこの条件を付加して $m$-dissipative であることを示そう.

\Theorem{CNO-14} %**********************************************************
$\tb$ を Killing vector field で $\div_\nu\tb$ が下に有界を仮定する.
このとき $C_0^\infty(M)$ を定義域とする $\gen$, $\gen_\nu^*$ の閉包は
$m$-dissipative である.
\end{theorem} %*************************************************************

\Proof
十分条件が Shigekawa \cite{Shigekawa10} に与えられているからそれを確かめる.
基準の点 $x_0\in M$ を取り固定する.
また $\rho(x)=d(x,x_0)$ を $x_0$ からの距離とする.
ベクトル場 $\tb$ で生成される1-パラメーター変換群を $\vph_t$ で表す.
$\vph_t$ は次の微分方程式をみたす.
\bdm %----------------------------------------------------------------------
\frac{d}{dt}\vph_t(x) = \tb(\vph_t(x)).
\edm %----------------------------------------------------------------------
$|t| \le 1$ で考えると $\tb(\vph_t(x_0))$ は有界であるから
\bdm %----------------------------------------------------------------------
|d(\vph_t(x_0),x_0)|
\le \int_0^t |\tb(\vph_s(x_0))|ds
\le K|t|
\edm %----------------------------------------------------------------------
をみたす $K$ が存在する.
これから
\bdm %----------------------------------------------------------------------
\rho(\vph_t(x))
&= d(\vph_t(x),x_0)
\le d(\vph_t(x),\vph_t(x_0)) + d(\vph_t(x_0),x_0) \\
&\le d(x,x_0) + K|t|
\le \rho(x) + K|t|.
\edm %----------------------------------------------------------------------
同様に $\rho(\vph_t(x)) \ge \rho(x) - K|t|$ も示せるから
\bdm %----------------------------------------------------------------------
|\rho(\vph_t(x)) - \rho(x)|
\le K|t|.
\edm %----------------------------------------------------------------------
ここで 
\bdm %----------------------------------------------------------------------
\tb\rho(x)
= \lim_{t\to 0} \frac{\rho(\vph_t(x)) - \rho(x)}{t}
\edm %----------------------------------------------------------------------
であるから $|\tb\rho(x)|\le K$ が成立する.

さて,閉包が $m$-dissipative であるための十分条件が \cite{Shigekawa10}
で次のように与えられている.
非増加関数 $\kappa:[0,\infty) \to [0,1]$ で
$\int_0^\infty \kappa(x) = \infty$ をみたす関数 $\kappa$ が存在して
$\kappa(\rho(x))\tb\rho(x) \ge -1$ をみたす.

ここでは $\kappa=\frac{1}{K}$ とすればよい.
$\gen_\nu^*$ のときは $-\tb$ を考えればいいので同様に示せる.
\QED %======================================================================

以上纏めれば次が得られた.

\Theorem{CNO-16} %**********************************************************
$\gen$ は $\tb$ が Killing vector field で $\div_\nu\tb$ は下に有界であるとする.
このとき,$\gen$ が正規であるための必要十分条件は $\tb$ が
Killing vector field であり \Eq{CNO.74}, \Eq{CNO.76} が成り立つことである.
\end{theorem} %*************************************************************


最後に簡単な正規作用素の例を挙げておこう.
$\R^2$ で $\nu = \frac{1}{2\pi} e^{-(x^2+y^2)/2} dxdy$
とし,$\tb = c(y\frac{\del}{\del x} - x\frac{\del}{\del y})$
とする.
すると $\tb$ はKilling vector field である.
$U=\frac{x^2+y^2}{2} + \log 2\pi$ とおくと $\nu=e^{-U}dxdy$ である.
このとき $\div\tb=0$ で $\tb U = c(yx - xy)=0$ であるから
$\div_\nu\tb = 0$ である.
さらに $\nabla U^\sharp = x\frac{\del}{\del x} + y \frac{\del}{\del y}$
であるから,
\bdm %----------------------------------------------------------------------
[\tb, \nabla U^\sharp]
&= c[y\frac{\del}{\del x} - x\frac{\del}{\del y},
    x\frac{\del}{\del x} + y \frac{\del}{\del y}] \\
&= c[y\frac{\del}{\del x},x\frac{\del}{\del x}]
  + c[y\frac{\del}{\del x},y \frac{\del}{\del y}]
  - c[x\frac{\del}{\del y},x\frac{\del}{\del x}]
  - c[x\frac{\del}{\del y},y\frac{\del}{\del y}] \\
&= cy\frac{\del}{\del x} - cy\frac{\del}{\del x}
  + cx\frac{\del}{\del y} - cx\frac{\del}{\del y}
= 0.
\edm %----------------------------------------------------------------------
よって,$\gen = -\frac{1}{2} \nabla_\nu^*\nabla + \tb$ は
$L^2(\nu)$ での正規作用素である.
\hide
では,スペクトルはどうなっているのかというのが次の問題になる.
この問題は池野君に解いてもらった.
複素 Hermite 多項式を用いるとスペクトルが完全に決定できる.
それはとても面白かった.また Laguerre 多項式との繋がりも分かってきたし.
[2011年1月5日]
\endhide

\subsec{Notes}
\begin{itemize}
\item 正規作用素の例では $\div_\nu\tb = 0$ の条件を付加して
考えたことになるが,この条件がなければどうなるのだろう.
なんとか $\div_\nu\tb = 0$ の条件を出してきたいのだが.
[2011年1月6日]
\item 本文中に追加してあるが $\div_\nu\tb = 0$ が成り立たない
正規作用素の例が存在する.
但しまだ $\div_\nu\tb = {\rm const.}$ である.
constant 以外の例は思いつかない.
[2011年1月15日]
\end{itemize}

