%%%%%%%%%%%%%%%%%%%%%%%%%%%%%%%%%%%%%%%%%%%%%%%%%
%                                               %
%         ====== Program  No.7 =======          %
%                                               %
%             file name snj07.tex               %
%                                               %
%===============================================%
%===================  for  =====================%
%===============================================%
%
\StartNewSection
\hide
\vspace{-4mm}
\begin{itemize} \itemsep=-2mm \parsep=0mm
\item Total file name: snj01 snj02 $\dots $ snj?, snj\_bibliography
\item File name: snj07.tex \hfill 印刷日: \today \ \now
\item 正規作用素が Generalized Dirichlet form の枠組みに入ることを証明する.
2008年の日独のとき R\"ockner が別の話で Generalized Dirichlet form
の枠組みに入るのではと言っていたが,正規作用素だったらいいように思う.
何か感じたんでしょうね,彼は.
[2011年1月16日]
\end{itemize}
\endhide
\SS{GDF}{正規作用素と Generalized Dirichlet form} %=========================
% Generalized Dirichlet form
正規作用素で生成される Markov 半群は Generalized Diriclet form
の枠組みに入る.
そのことを示していく.
\hide
昔 Stanatt がやっていたことだが,僕は彼の話を全然理解していなかった.
かれの話は実は結構使えるわけだ.
[2011年1月16日]
\endhide
$(M,\m)$ を $\s$-有限な測度空間とし,$H=L^2(\m)$ として,以下 Hilbert 空間
$H$ で考える.
$\gen$ を正規作用素とする.
$m$-dissipative を仮定する.
spectral represntation により
\bdn %----------------------------------------------------------------------
- \gen
= \int_\C z E(dz)
\Eqn{GDF.6}
\edn %----------------------------------------------------------------------
と表現できる.
対称部分 $(\gen+\gen^*)/2$ は
\bdn %----------------------------------------------------------------------
\frac{\gen+\gen^*}{2}
= \int_\C \Re z E(dz)
\Eqn{GDF.8}
\edn %----------------------------------------------------------------------
歪対称部分 $(\gen-\gen^*)/2$ は
\bdn %----------------------------------------------------------------------
\frac{\gen-\gen^*}{2}
= \int_\C i \Im z E(dz)
\Eqn{GDF.10}
\edn %----------------------------------------------------------------------
で表される.
これはユニタリーグループを生成する.
ここで
\bdn %----------------------------------------------------------------------
L
= \int_\C \Re z E(dz),\quad
\Lm
= \int_\C i \Im z E(dz)
\Eqn{GDF.14}
\edn %----------------------------------------------------------------------
とおく.
$L$ は self-adjoint で $\Lm$ は $i\Lm$ が self-adjointになる.
また定義域は
\bdn %----------------------------------------------------------------------
\Dom(L)
= \{ f;\, \int_\C |\Re z|^2 (f,E(dz)f) <\infty \},\quad
\Dom(\Lm)
= \{ f;\, \int_\C |\Im z|^2 (f,E(dz)f) <\infty \}
\Eqn{GDF.16}
\edn %----------------------------------------------------------------------
となる.
$L$ は $m$-dissipative を仮定したので,半群を生成する.
また $L$ から対称な双線型形式 $\tDiri$ が定まる.
実際 $\tDiri$ は次で定義される:
\bdn %----------------------------------------------------------------------
\tDiri(f,g)
= \int_\C \Re z (f,E(dz)g).
\Eqn{GDF.18}
\edn %----------------------------------------------------------------------
またその定義域も
\bdn %----------------------------------------------------------------------
\Dom(\tDiri)
= \{f;\, \int_\C |\Re z| (f,E(dz)f) <\infty\}
\Eqn{GDF.20}
\edn %----------------------------------------------------------------------
で与えられる.
$\sV =\Dom(\tDiri)$ とおく.

さて,$\Lm$ も半群を生成する.
実際は 1変数 unitary 群を生成するが,半群の部分だけ使う.
この半群を $\{U_t\}_{t\ge0}$ とかく.
このとき次が成り立つ.


\Proposition{GDF-4} %*******************************************************
$\{U_t\}$ は $\sV$ の $C_0$-縮小半群である.
(実際は1変数 unitary 群である.)
\end{proposition} %*********************************************************

さて $\Lm\colon \Dom(\Lm)\cap \sV \to \sV'$ を $\sV$ から $\sV'$
への作用素とみて,そのの閉包を $(\Lm,\sF)$ とする.
この閉包が存在することは Stanatt Lemma 2.3 にある.

\Proposition{GDF-6} %*******************************************************
$f\in \sF$ であるための必要十分条件は
\bdn %----------------------------------------------------------------------
\int_\C (\frac{|\Im(z)|^2}{\Re z+1} + \Re z) \,(f,E(dz)f) < \infty
\Eqn{GDF.22}
\edn %----------------------------------------------------------------------
である.
\end{proposition} %*********************************************************

\Proof
まず $\sV$ のノルムとして
\bdm %----------------------------------------------------------------------
\|f\|_\sV^2
= \int_C (\Re z + 1)(f,E(dz)f)
\edm %----------------------------------------------------------------------
を取ることができる.
\bdm %----------------------------------------------------------------------
|(\Lm f,g)_H|
&=   |\int_\C \Im z (f,E(dz)g)| \\
&=   |\int_\C \frac{\Im z}{\sqrt{\Re z + 1}} \sqrt{\Re z + 1} (f,E(dz)g)| \\
&\le \biggl\{\int_\C \frac{|\Im z|^2}{\Re z + 1} (f,E(dz)f)\biggr\}^{1/2}
     \biggl\{\int_\C (\Re z + 1) (g,E(dz)g) \biggl\}^{1/2} \\
&\le \biggl\{\int_\C \frac{|\Im z|^2}{\Re z + 1} (f,E(dz)f)\biggr\}^{1/2}
     \|g\|_\sV
\edm %----------------------------------------------------------------------
これから
\bdm %----------------------------------------------------------------------
\|\Lm f\|_{\sV'}
\le \biggl\{\int_\C \frac{|\Im z|^2}{\Re z + 1} (f,E(dz)f)\biggr\}^{1/2}
\edm %----------------------------------------------------------------------
が示せる.
逆向きの不等式も $g$ をうまくとれば成立することが分かる.
よって
\bdm %----------------------------------------------------------------------
\|\Lm f\|_{\sV'}
= \biggl\{\int_\C \frac{|\Im z|^2}{\Re z + 1} (f,E(dz)f)\biggr\}^{1/2}
\edm %----------------------------------------------------------------------
また
\bdm %----------------------------------------------------------------------
\|f\|_\sF^2
= \|f\|_\sV^2 + \|\Lm f\|_{\sV'}
\edm %----------------------------------------------------------------------
であるから求める結果を得る.
\QED %======================================================================

\Remark{GDF-8} %************************************************************
弱扇形条件が成り立つときはスペクトルが扇形領域に含まれるから
そこでは $|\Im(z)| \le C(\Re z+1)$ となる定数 $C$ が存在する.
従って,$\sF$ の定義域は $\sV$ と同じになる.
\end{remark} %**************************************************************

同様の議論を $U_t$ の双対半群 $\hat{U}_t$ についても行って
$\hat{\sF}$ を定める.
$\hat{U}_t$ の生成作用素は
\bdn %----------------------------------------------------------------------
\hat{\Lm}
= - \int_\C i \Im z E(dz)
\Eqn{GDF.26}
\edn %----------------------------------------------------------------------

さて generalized Dirichlet form の枠組みでは双線型形式を
\bdn %----------------------------------------------------------------------
\Diri(f,g)
=
\begin{cases}
\tDiri(f,g)-\la \Lm f,g\ra, \quad \text{if $f\in\sF$, $g\in \sV$} \\
\tDiri(f,g)-\la \hat{\Lm} g, f\ra, \quad \text{if $f\in\sV$, $g\in \hat{\sF}$}
\end{cases}
\Eqn{GDF.24}
\edn %----------------------------------------------------------------------
で定める.

容量の概念も定義され,次の定理を適用することにより確率過程の存在が保証される.

\Theorem{GDF-10} %**********************************************************
{\bf (Stanatt 1994) }
条件 (D3) のもとで,擬正則一般 Dirichlet 形式に対して,
$\m$-thght special standard 過程が存在する.
\end{theorem} %*************************************************************

\begin{description}
\item[(D3)]
ある線型部分空間 $\sY \subseteq L^2(\m)\cap L^\infty(\m)$ で
$\sY\cap \sF$ が $\sF$ 稠密であり,
$\lim_{\al\to\infty} e_{\al G_\al u-u}=0$ が $u\in\sY$ に対して成り立つ.
さらに
$\sY$ の $L^\infty$ での閉包を $\ol{\sY}$ とすると
$u\wedge \al\in \ol{\sY}$ が $u\in\sY$, $\al>0$ に対して成立する.
\end{description}







\subsec{Notes}
\begin{itemize}
\item 
\end{itemize}

