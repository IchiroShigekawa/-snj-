%%%%%%%%%%%%%%%%%%%%%%%%%%%%%%%%%%%%%%%%%%%%%%%%%
%                                               %
%         ====== Program  No.16 =======          %
%                                               %
%             file name snj16.tex               %
%                                               %
%===============================================%
%===================  for  =====================%
%===============================================%
%
\hide
\vspace{-4mm}
\begin{itemize} \itemsep=-2mm \parsep=0mm
\item Total file name: snj01 snj02 $\dots $ snj?, snj\_bibliography
\item File name: snj16.tex \hfill 印刷日: \today \ \now
\item Ornstein-Uhlenbeck に回転を加えた作用素のスペクトルを調べる.
[2011年8月26日(金)]
\end{itemize}
\endhide
\SS{OUR}{$\R^2$ 上の回転を加えた Ornstein-Uhlenbeck 作用素} %==================
% Ornstein-Uhlenbeck operator with rotation
この節では Ornstein-Uhlenbeck に回転を加えた作用素のスペクトルを調べる.
$\al\in\R$ に対し $L_{\al}$ を
\bdn %----------------------------------------------------------------------
L_\al
= \frac{\del^2}{\del x^2} + \frac{\del }{\del y^2} 
  - x \frac{\del}{\del x} - y \frac{\del }{\del y}
  \alpha \biggl( x \frac{\del }{\del y} - y \frac{\del }{\del x} \biggr)
\Eqn{OUR.6}
\edn %----------------------------------------------------------------------
で定め,$L^2(\R^2, \mu)$ 上の作用素としてスペクトルを調べる.
ここで測度 $\mu$ は
\bdn %----------------------------------------------------------------------
\mu
= \frac{1}{2\pi} e^{-(x^2+y^2)/2}\,dx\,dy
\Eqn{OUR.8}
\edn %----------------------------------------------------------------------
で定まるGauss 測度である.

通常の Ornstein-Uhlenbeck 作用素 $L_0$ のスペクトルは $\{0,-1,-2,\dots\}$
であることはよく知られている.
固有関数は,まず Hermite 多項式を
\bdn %----------------------------------------------------------------------
H_n(x)
= (-1)^n e^{x^2/2}\frac{d^n}{dx^n}e^{-x^2/2}.
\Eqn{OUR.10}
\edn %----------------------------------------------------------------------
で定めると,
\bdm %----------------------------------------------------------------------
L_0 H_k(x)H_{n-k}(y)
= - n H_k(x)H_{n-k}(y)
\edm %----------------------------------------------------------------------
が成り立つ.
回転を加えた場合は,固有値は複素 Hermite 多項式になる.

\subsec{複素 Hermite 多項式}
複素 Hermite 多項式について復習しておく.
$\R^2$ は $\C$ と同一視,$z=x+iy$ とする.
複素 Hermite 多項式は $p$, $q\in\plus{\Z}$ に対し
\bdn %----------------------------------------------------------------------
H_{p,q}(z , \bar{z})
= (-1)^{p+q} e^{z\bar{z}/2} \biggl( \frac{\del}{\del\bar{z}} \biggr)^p
 \biggl(\frac{\del }{\del z} \biggr)^q e^{-z\bar{z}/2}
\Eqn{OUR.14}
\edn %----------------------------------------------------------------------
で定義される.
通常の定義とは定数が異なっていることに注意しよう.
これは Gauss 測度 $\mu$ の取り方合わせたためである.
ここで
\bdm %----------------------------------------------------------------------
\frac{\del}{\del z}
= \frac{1}{2}\biggl( \frac{\del}{\del x} - i \frac{\del}{\del y}\biggr), \quad
\frac{\del}{\del \bar{z}}
= \frac{1}{2}\biggl( \frac{\del}{\del x} + i \frac{\del}{\del y}\biggr)
\edm %----------------------------------------------------------------------
である.
以下では
\bdm %----------------------------------------------------------------------
\del
= \frac{\del}{\del z}, \quad
\delb
= \frac{\del}{\del \bar{z}}, \quad
\edm %----------------------------------------------------------------------
と略記する.
また
\bdn %----------------------------------------------------------------------
\del^*
= -\del + \frac{\bar{z}}{2}, \quad
\delb^*
= -\delb  + \frac{\bar{z}}{2}
\Eqn{OUR.18}
\edn %----------------------------------------------------------------------
が成り立っている.

\Proposition{OUR-2} %*******************************************************
上の \Eq{OUR.18} の作用素には次の交換関係が成立している.
\bdmn %---------------------------------------------------------------------
\del H_{p,q}
&= \frac{p}{2} H_{p-1,q}, 
\Eqn{OUR.20} \\
\delb H_{p,q}
&= \frac{q}{2} H_{p,q-1}, 
\Eqn{OUR.22} \\
\del^* H_{p,q}
&= H_{p+1,q}, 
\Eqn{OUR.24} \\
\delb^* H_{p,q}
&= H_{p,q+1}, 
\Eqn{OUR.26} \\
2\del\delb H_{p,q} - z\del H_{p,q}
&= - p H_{p,q}
\Eqn{OUR.28} \\
2\del\delb H_{p,q} - \bar{z}\delb H_{p,q}
&= - q H_{p,q}
\Eqn{OUR.30} \\
(z\del - \bar{z}\delb)H_{p,q}
&= (p-q) H_{p,q}
\Eqn{OUR.32}
\edmn %----------------------------------------------------------------------
\end{proposition} %*********************************************************

\Proof
\QED %======================================================================

さて $L_\al$ のスペクトルを求めるために
$L_\al$ を $\del$, $\delb$ を用いて表そう.
まず
\bdm %----------------------------------------------------------------------
\del \delb
&= \frac{1}{4} \biggl(\frac{\del^2}{\del x^2}+\frac{\del}{\del y^2}\biggr), \\
z\del
&=  x\frac{\del}{\del x} + y\frac{\del}{\del y}
    -i\biggl(  x \frac{\del}{\del y} - y \frac{\del }{\del x}\biggr), \\
\bar{z}\delb
&=  x \frac{\del}{\del x} + y \frac{\del }{\del y}
    +i\biggl(  x \frac{\del}{\del y} - y \frac{\del }{\del x}\biggr)
\edm %----------------------------------------------------------------------
が成り立つので
\bdm %----------------------------------------------------------------------
L_\al
&= 4 \del\delb - z\del - \bar{z}\delb +\al i (z\del - \bar{z}\delb) \\
&= (2\del\delb - z\del) + (2\del\delb- \bar{z}\delb)
   + \al i (z\del - \bar{z}\delb).
\edm %----------------------------------------------------------------------
従って
\bdm %----------------------------------------------------------------------
L_\al H_{p,q}
&= (2\del\delb - z\del)  H_{p,q} + (2\del\delb- \bar{z}\delb)H_{p,q}
   + \al i (z\del - \bar{z}\delb)H_{p,q} \\
&= - p H_{p,q} + q H_{p,q} + (p-q) \al i H_{p,q} \\
&= (-p-q + (p-q) \al i) H_{p,q}.
\edm %----------------------------------------------------------------------

よって次の定理が得られた.
\Theorem{OUR-6} %**********************************************************
$-L\al$ の固有値は $\{ - (p+q) + (p-q) \alpha i \}_{p,q = 0}^{\infty}$
であり対応する固有関数は $H_{p,q}$ である.
\end{theorem} %*************************************************************


\begin{center}
\includePdfEps{}{snj_OU_spectrum}
\includePdfEps{}{snj_OU_rotate_spectrum}
\end{center}


\bdm %----------------------------------------------------------------------
V_n := \{ L_0 f = n f\}
\edm %----------------------------------------------------------------------
とすれば直交分解
\bdm %----------------------------------------------------------------------
L^2(\C,\mu)
= \bigoplus_{n=0}^\infty V_n
\edm %----------------------------------------------------------------------
が成り立つ.
$V_n$ をさらに
\bdm %----------------------------------------------------------------------
V_n = \bigoplus_{p + q = n} {\bf{C}}H_{p,q}
\edm %----------------------------------------------------------------------
と分解したことになる.
これは回転群 $U(1)$ の規約分解を与えている.
固有値 $2n$ ($n\in\plus{\Z}$)に対応する固有関数は回転方向の微分
$(z\del - \bar{z}\delb)$ が $0$ になるので,半径方向の関数であり
\bdm %----------------------------------------------------------------------
\biggl(\frac{d^2}{d r^2} + \frac{1}{r}\frac{d}{d r}
       - r \frac{d}{d r} \biggr) H_{n,n}
= -2n H_{n,n}
\edm %----------------------------------------------------------------------
が成り立っている.
ここで $r=\sqrt{2u}$ と変数変換すると
\bdm %----------------------------------------------------------------------
\frac{d^2}{d r^2} + \frac{1}{r}\frac{d}{d r}
       - r \frac{d}{d r}
= 2u \frac{d^2}{du^2} + 2(1-u) \frac{d}{du}
\edm %----------------------------------------------------------------------
となる.
$F(u)= H_{n,n}(r)$ は微分方程式
\bdm %----------------------------------------------------------------------
2u \frac{d^2}{du^2}F + 2(1-u) \frac{d}{du}F +nF=0
\edm %----------------------------------------------------------------------
をみたしている.
これは Laguerre 多項式が満たす微分方程式である.
ここで Laguerre 多項式は
\bdn %----------------------------------------------------------------------
L_{n} = \frac{e^x}{n!} \frac{d^n}{dx^n} (e^{-x}x^n)
\Eqn{OUR.40}
\edn %----------------------------------------------------------------------
で定義される.
実際次が成立する.

\Theorem{OUR-10} %**********************************************************
複素 Hermite polynomials $H_{n,n}$ は次をみたす.
\bdn %----------------------------------------------------------------------
H_{n,n}(z,\bar{z}) = c L_n \left(\frac{|z|^2}{2} \right),
\Eqn{OUR.42}
\edn %----------------------------------------------------------------------
$c$ は定数である.
\end{theorem} %*************************************************************

\hide
上の定数 $c$ がはっきりしないのは問題である.

\endhide

さらに複素 Hermite 多項式は,実 Hermite 多項式を用いて
\bdn %----------------------------------------------------------------------
H_{n,n}(z,\bar{z})
\frac{1}{4^n} \sum_{k=0}^n H_{2k}(x)H_{2n-k}(y)
\Eqn{OUR.44}
\edn %----------------------------------------------------------------------
と表されるから次を得る.

\Corollary{OUR-12} %********************************************************
Laguerre 多項式は Hermite 多項式と次の関係で結ばれている.
\bdn %----------------------------------------------------------------------
L_n(\frac{x^2+y^2}{2})
= \frac{c}{4^n} \sum_{k=0}^n H_{2k}(x)H_{2n-k}(y).
\Eqn{OUR.46}
\edn %----------------------------------------------------------------------
\end{corollary} %***********************************************************

\subsec{Notes}
\begin{itemize}
\item 2011年9月12日 に Jan Van Neerven からメールが来た.OU のスペクトルの研究をしている.Bonn のISSAA で会って,話を聞いた.この節で論じたスペクトルの話に関連したことをやっている.
\item 2012年7月に中国に行ったとき,同じ結果を得ていることを聞いた.
さっさと書かないからこういうことになる・
\end{itemize}

